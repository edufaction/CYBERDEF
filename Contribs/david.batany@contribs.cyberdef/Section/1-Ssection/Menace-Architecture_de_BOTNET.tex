\subsection{Les type de menaces}
Les botnets représentent les outils de diffusion des attaques.
Cet outil permet aux cybercriminels de disposer d'un grand nombre de services développé dans un environnement collaboratif. Vendus sur le web, ils instrumentalisent l'attaque quelque soit le but recherché.
\newline \textbf{Liste des menaces possibles:}
\begin{itemize}
	\item Relayer du spam pour du commerce illégal ou pour de la manipulation d'information (par exemple des cours de bourse) 
  \item Réaliser des opérations d'hameçonnage 
  \item Identifier et infecter d’autres machines par diffusion de virus et de programmes malveillants (malwares) 
  \item Participer à des attaques groupées de déni de service (DDoS)
  \item Générer de façon abusive des clics sur un lien publicitaire au sein d’une page web (fraude au clic) 
  \item Capturer de l’information sur les machines compromises (vol puis revente d'information) ;
  \item Exploiter la puissance de calcul des machines ou effectuer des opérations de calcul distribué notamment pour cassage de mots de passe 
  \item Voler des sessions utilisateurs par credential stuffing ;
  \item Mener des opérations de commerce illicite en gérant l'accès à des sites de ventes de produits interdits ou de contrefaçons via des techniques de fast flux, simple ou double-flux ou RockPhish 
  \item Miner des cryptomonnaies, telles que le bitcoin1.
\end{itemize}