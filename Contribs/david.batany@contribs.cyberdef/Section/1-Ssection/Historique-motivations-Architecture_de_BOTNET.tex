\subsection{Définition}
\par
Le terme \textbf{botnet}, contraction de l'anglais \textbf{robot+net}, se définit par l'ensemble des programmes, machines, serveurs connectés à internet ayant un ou plusieurs processus commun de communication.
Placé sous le contrôle d'un opérateur humain, appelé botmaster, le botnet recrute des machines en exploitant les vulnérabilités, failles, infections afin d'étendre son réseau à travers l'utilisation de canaux de Command and Control(C\&C).
\newline Avec l'IoT\footnote{Internet of Things}, et ses appareils connectés, le réseau s'étend de plus en plus au sein de notre société. L'actuelle faiblesse en terme de sécurité lié aux objets connectés représente une menace majeure et croissante dans notre environnement.
\subsection{Historique}
\par
Le concept, inventé en 1988 à l'université de Oulu en Finlande, fut développé à l'origine pour gérer les services associés au protocole IRC\footnote{Internet Relay Chat, un protocole de communication textuel}.
\newline
Le premier bot "<GM"> assistait ainsi l'utilisateur dans la gestion des connections IRC.
Cette gestion automatisée, permettant via un accès à distance, de contrôler et de réaliser des opérations a très vite montré un haut pouvoir malveillant.
\newline
En Mai 1999, Pretty Park, un malware de forme trojan horse se propageant sur le net permettait de voler les mots de passe.
\newline
Les premières dérives furent notamment l'affrontement de botnet IRC (\href{https://www.eggheads.org/}{Eggdrop} en décembre 1993, puis GTbot en avril 1998).
\subsection{Motivations liées à la menace botnet}
\begin{enumerate}
	\item L'aspect lucratif représente l'intérêt majeur pour l'utilisateur de botnet. L'automatisation d'une tâche contrôlée à distance permettant de rapporter facilement des revenus (revente d'information, fraude au clic, spam), surtout si celle-ci est réalisée de manière anonyme(réseau TOR\footnote{The Onion Routing, un réseau d'anonymisation}).
	\item La motivation idéologique, comme par exemple, lors du conflit entre la Georgie et la Russie en 2008 ou de nombreux sites étatiques faisait l'objet de cyberattaques massives paralysaient les infrastructures.
	\item La motivation personnelle, à travers la vengeance ou le chantage, est également une finalité grâce notamment au caractère anonyme de l'attaque.
\end{enumerate}