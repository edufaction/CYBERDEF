\section*{Structure du document}
L'ensemble du contenu se trouve dans le dossier Section.
\paragraph {La section 1 comprend 4 fichiers reprenant les thèmes de:}
\begin{itemize}
	\item Historique et motivations liés à la menace botnet
  \item Les Le type de menace
  \item Le Le cycle de vie d'une attaque
  \item Les Les attaques 
\end{itemize}
\paragraph {La section 2 comprend 4 fichiers reprenant les architectures:}
\begin{itemize}
	\item Architecture centralisée
  \item Architecture décentralisée
  \item Architecture hybride
  \item Architecture aléatoire
\end{itemize}
Un dossier exemples inclut les caractéristiques de certains botnet saisis dans un fichier bloc-notes enregistré en .csv .
Le package csvsimple avec l'instruction csvautotabular permet de mettre en forme ces données dans un tableau.
La virgule sert de séparateur et permet de créer une colonne dans ce tableau.
\paragraph {La section 3 comprend 4 fichiers sur les méthodes de lutte:}
\begin{itemize}
	\item La détection
  \item L'analyse
	\item La défense et le blocage
  \item Le démantèlement
\end{itemize}
Le dossier Annexe inclut le fichier "nom-du-botnet.csv" servant de modèle pour les exemples et le fichier de mise en forme du modèle.
Enfin une conclusion se trouve dans le dossier associé.