\subsection{La défense et le blocage}
Au même titre que la protection contre les malwares, les recommandations en terme de SSI\href{https://www.ssi.gouv.fr/administration/bonnes-pratiques/}{recommandations ANSSI}, applicables localement, doivent s'inscrire dans nos habitudes et augmentent ainsi la probabilité de bloquer l'étape initiale de l'infection (spam, navigation non-sécurisée,etc.).
\newline Les mises à jour logicielles et système sont essentielles pour bloquer l'exploitation de CVE\footnote{Common Vulnerabilities and Exposures} \href{https://www.cvedetails.com}{Common Vulnerabilities and Exposures}.
\newline Au niveau du FAI, les notifications en cas de connexions malveillantes et la surveillance des adresses IP sont un frein à l'extension du botnet.
\newline Enfin la détection d'un appel de fonctions anormal par l'antivirus, l'autorisation et l’identification des flux sortants par le firewall permettent le blocage de l'activité malveillante.
Les fonctionnalités recherchées de l'antivirus dans ce cadre sont un firewall bidirectionnel, une protection contre le phishing, la vérification de la certification, la lutte contre le tracking, la vérification du téléchargement, le blocage des pop-ups et pages WEB malveillantes,etc. 
