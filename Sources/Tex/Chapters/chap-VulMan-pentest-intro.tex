%-------------------------------------
% Chapitre
% Vulnerability Management
% Pentest Intro
% File : chap-Vulman-pentest-intro.tex
%--------------------------------------
\uchap{\jobname}


\section{RECHERCHER des vulnérabilités}

\subsection{Les tests d'intrusion}

Il y existe de nombreuses activités d'évaluation des vulnérabilités. La littérature identifie par ailleurs des noms de métiers ou de compétences d'activité.

% Begin PRZ ===========================
\begin{frame}
\frametitle<presentation>{Les types de tests}
% end header PRZ =======================
\upicture{Tex/Pictures/img-vul-typetests}{Les types de tests de vulnérabilités}{0.5}{lblteamstart1}
\end{frame}
% end PRZ ===========================


% Begin PRZ ===========================
\begin{frame}
\frametitle<presentation>{Les branches du test}
% end header PRZ =======================
\upicture{Tex/Pictures/img-teamstar}{Les branches du test}{0.5}{lblteamstart}
\end{frame}
% end PRZ ===========================

%https://danielmiessler.com/study/red-blue-purple-teams/

%https://www.bulletproof.co.uk/blog/what-is-penetration-testing

Un des cadres les plus courant pour conduire des audits est l'audit technique qui comprend les tests d'intrusion : PENTEST.

\subsection{Généralités}

Le terme PENTEST est devenu tellement courant, que l'on oublie quelques fois qu'il est l'abréviation de : \g{Penetration Testing}, qui  veut littéralement dire tests d'intrusion.  L'expression  \g{test de pénétration} est parfois rencontrée, mais les professionnels du PENTEST n'apprécient par trop cette expression. 

Il est toujours un peu complexe de catégoriser l'activité de pentests. Il aura assez rapidement des détracteurs pour soulever le fait que la catégorie n'est pas la bonne, qu'elle n'est pas représentative du métier. 
Je vais donc faire rentrer cette activité dans plusieurs catégories (métier de l'audit, métiers d'expertise, métiers du tests).
Dans le cadre de ce cours, je propose de relier cette activité métier comme un des outils du processus de gestion des vulnérabilités (\emph{Vulnerability Management}).  Mais il est plus courant de  classer les activités de PENTESTS dans les activités d'audit.

\subsection{Le métier de Pentesteur}

Le métier du PENTEST est  lié aux métiers techniques de l'informatique et télécom. Les origines des Pentesteur sont très variées.
Ce sont des métiers qu'il est possible d'exercer avec différents niveaux de formation. 

\subsubsection{ Ethical Hackers}
Etre Ethical Hacker fait partie du mythe de la cybersécurité.

\subsubsection{Peut-on faire confiance à des pentesteurs ?}
Parmi les grandes questions que se posent les \g{commanditaires} de tests d'intrusion se trouve celle de la confiance.
En effet, le principe des tests d'intrusion est d'ouvrir un peu les portes de ses systèmes informatiques à des \g{intrus} qui vont certainement découvrir des fragilités.
Certains, peut être plus paranoïaques que d'autres, peuvent se poser la question de savoir ce que vont devenir ces informations sensibles dans \g{les mains} de Hackers.
Parmi les commanditaires, on trouve bien entendu les RSSI, mais aussi les chefs de projet d'application ou de produits embarquant des technologies de informatique ou de communication (Objets intelligents, connectés).

\subsection{Les sociétés de confiance}

Le niveau de sécurité du système d'information et des  outils permettant de réaliser les audits sont vérifiés et validés par une société de certification (LSTI, AMOSSYS ...). A l'issue de certification, l'ANSSI prononce la qualification de la société d'audit au titre de ce PASSI. Il existe une extension pour les audits liés à la loi de programmation militaire (LPM). c'est à dire pour les audits sur les SIIV (Système d'information d'importance vitale) des Opérateurs d'importance Vitale. (Voir chapitre sur la Cyberdefense)

\subsubsection{Formation des Pentests}

\subsection {Certifications professionnelles}

% Begin PRZ ===========================
\begin{frame}
\frametitle<presentation>{2 Certifications}
\framesubtitle<presentation>{reconnues pour les auditeurs techniques}
% end header PRZ =======================
\includer{inc-certifs-hacking.tex}
\end{frame}
% end PRZ ===========================

\subsection {Les rapport d'audits, et cadre méthodologique}

Au coeur de l'activité \g{professionnelle} des pentesteurs se situe le rapport d'audit.
En effet, si l'objectif est bien d'identifier des fragilités (des vulnérabilités), et les scénarios qui permettent de les exploiter à des fins concrètes et relevant d'une menace présentant un risque pour l'entreprise.
il n'en demeure pas moins important de \g{rendre compte} de ce qui a été trouvé. Ce rapport doit aussi contenir des préconisations, car il est important face à une ou des fragilité(s) de proposer des solutions. 




