%gouvernance

%https://www.ssi.gouv.fr/uploads/IMG/pdf/tdbssi-memento-2004-02-05.pdf

%la maturité c'est la capacité de se mesurer et d'agir de manière mesuré.

% le modèle des entreprises c'est le pilotage de l'AUDIT ...

\section{Gouvernance de la sécurité}

Du risque aux implémentations de sécurité avec des composants de sécurité configurés, pilotés, et controlés.


\subsection{Politique de sécurité}

Une politique de sécurité est un ensemble de règles et de procédures destinées à protéger les actifs (matériels, logiciels, informations) d'une entreprise ou d'une organisation contre les menaces internes et externes. Elle définit les responsabilités et les autorisations en matière de sécurité, les moyens de prévention et de détection des incidents de sécurité, ainsi que les mesures à prendre en cas de violation de la sécurité.

Elle peut couvrir différents domaines, tels que la sécurité physique (accès aux locaux, utilisation des clés, protection contre les intrusions), la sécurité informatique (mot de passe, pare-feu, antivirus), la sécurité des données (sauvegarde, chiffrement), la sécurité des communications (confidentialité, intégrité), la sécurité des processus (gestion des droits d'accès, respect de la vie privée).

Une politique de sécurité adaptée est nécessaire afin de protéger au juste besoin les actifs de l'entreprise contre les menaces externes et internes, telles que les cyberattaques, les fuites de données, les actes de sabotage, etc. Une politique de sécurité efficace se doit d'être régulièrement mise à jour et adaptée aux évolutions de l'environnement de l'entreprise et des menaces pesant sur elle.
Elle doit donc être non seulement être tournée sur les enjeux internes mais ne doit pas oublier de deployer ses moyens de perception sur les menaces potentielles avec des activités d'intelligence économique et stratégique.