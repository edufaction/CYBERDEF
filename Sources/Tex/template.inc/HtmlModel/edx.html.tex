\newcommand{\usechapterimagefalse}{}

\newcommand{\graphicxextension}{}


% XCOLOR Package
\definecolor{ocre}{RGB}{160,0,0}  
\definecolor{grey}{RGB}{50,80,80}
\definecolor{cnam}{RGB}{128, 0, 32}


\newcommand{\includer}[1]
{
\input{Tex/Chapters/#1}
}

\newcommand{\picframe}[4]
{
\begin{figure}[!hbtp] % !hbtp ou !h 
\begin{center}
	\includegraphics[width=#3\textwidth]{#1}
\end{center}
\caption{#2}
\label{#4}
\end{figure}
}

%\newcommand{\picframe}[4]{\includegraphics{#1}}
%\newcommand{\picframe}[4]{Picframe #1}

%\label{#4}
\newcommand{\picframeS}[2]
{
\begin{figure}[!hbtp] % !hbtp ou !h 
\begin{center}
\includegraphics[width=0.9\textwidth]{\picpath/img-#1}	
\end{center}
\caption{#2}\label{lbl#1}
\end{figure}
}

%\newcommand{\picframeS}[2]{PICFRAME Simplified : #1 #2}

%\newcommand{\upicture}[4]{UPICTURE : #1 #2 #3 #4 \\ }

\newcommand{\upicture}[4]
{
\begin{figure}[!hbtp] %h !hbtp
  \begin{center}
	 \includegraphics[width=#3\textwidth]{#1}
  \end{center}
\caption{#2}\label{#4}
\end{figure}
}

 
\newcommand{\g}[1]{\og #1 \fg{}}

\newcommand{\uindex}[1]
{
#1 \index{#1}\xspace
}
\newcommand{\ulindex}[1]{\index{#1}}


\newcommand{\UKword}[1]{\textit{#1}}
\newcommand{\textUK}[1]{\textit{#1}}
\newcommand{\ugls}[1]{\gls{#1}}

%\gls{#1}

\newcommand{\tb}[1]{#1}

\newcommand{\head}[1]{#1}

\newcommand{\ulink}[2]{\href{#1}{#2}}

%\newcommand{\ucite}[1]{\cite{#1}}

\newcommand{\ucite}[1]{\cite{#1}}


\newcommand{\texframe}[1]{#1}
\newcommand{\uchap}[1]{UCHAP #1}
%Original BOOK
\newcommand{\ubg}[1]{
\textbf{\g{\uppercase{#1}}}
}

\newcommand{\uref}[2]{~#1~(\ref{#2} page~ \pageref{#2})}


%\newcommand{\uref}[2]{(REF #1 #2)}

\newcommand{\utodo}{** rédaction réservée **}
\newcommand{\utocomplete}{*** partie à compléter ***}

\newcommand{\edxdico}[2] {\textbf{#1}:\textit{#2}}

% BEAMER Package
\newcommand{\frametitle}[1]{}
\newcommand{\framesubtitle}[1]{}
% BEAMER Null Proxy
\newcommand{\mode}{.}

\newcommand{\wikipedia}[1]{href{https://fr.wikipedia.org/wiki/#1}{#1 (Wikipédia)}}


\newenvironment{techworkbox}[1]
{------techworbox #1----------}
{------END techworkbox------}	


\newenvironment{notebox}[1]
{------notebox #1----------}
{------END notebox------}	

\newenvironment{toolsbox}[1]
{------toolbox #1----------}
{------END toolbox------}	


\newenvironment{nota}[1]
{------NOTA #1----------}
{------END NOTA------}	




