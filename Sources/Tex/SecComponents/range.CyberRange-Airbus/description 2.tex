La CyberRange est utilisée par ses utilisateurs (administrateur, intégrateur, testeur, formateur) pour concevoir des réseaux virtualisés ou hybrides, émuler des activités unitaires comme des communications entre deux machines ou encore pour lancer des scenarios complexes reproduisant une activité réaliste (échange de fichier, email, trafic web et potentiellement de véritable cyber-attaques).

CyberRange est disponible dans un caisson mobile, dans une baie ou accessible depuis un cloud.