La plate-forme s’adresse aux promotions cybersécurité des écoles et université et plus largement à l’ensemble des promotions européennes en IT qui veulent accélérer le partage de jeux de données de qualité. L’approche pédagogique encourage d’abord l’émulation via la gamification et l’implication de chaque élève dans la sécurisation de son institution. Elle ouvre surtout des perspectives aux futurs développeurs vers des spécialisations porteuses telles que DevSecOps, Data Scientist, Security Analyst, etc. Enfin, YesWeHack EDU facilite la mise en place de projets collaboratifs et d’initiatives transversales entre les institutions académiques et le secteur privé.

Disponible partout en Europe, la plateforme YesWeHack EDU s’inscrit dans la ligne de l’initiative du consortium SPARTA, dont YesWeHack est un des membres fondateurs, qui vise à renforcer l’innovation et la recherche en matière de cybersécurité au niveau européen.