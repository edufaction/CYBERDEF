% ======================================
%            CyberEdu QUIZZ
%            VOLUME 1
%            Mise en forme Moodle QUIZZ
%            (c) eric.dupuis@lecnam.net
% ======================================
\documentclass[12pt]{article}

%  \usepackage[draft]{moodle}
   \usepackage{moodle}
   \begin{document}

% =====================================
   \begin{quiz}{CYBERDEF101-Cyberedu-QUIZZ 1}
% =====================================


% Q.........................................
\begin{multi}[multiple=true]{Cyberedu-QUIZZ-ID-1.1}
	Entourer les exemples d'enjeu de la cybers\'ecurit\'e ? (Slide 7 - Les enjeux de la s\'ecurit\'e des S.I.)
\item 	Augmenter les risques pesant sur le Syst\`eme d'information
\item 	R\'ev\'eler les secrets
\item	Rendre difficile la vie des utilisateurs en ajoutant plusieurs contraintes comme les mots de passe longs et complexes
\item* 	Prot\'eger le syst\`eme d'information
\end{multi}

% Q.........................................
\begin{multi}[multiple=true]{Cyberedu-QUIZZ-ID-1.2}
	Impacts sur la vie priv\'ee  ?
\item* 	Impact sur l'image / le caract\`ere / la vie priv\'ee : Diffamation de caract\`ere , Divulgation d'informations personnelles (photos d\'enud\'ees) Harc\`element
\item*	Impact sur l'identit\'e: Usurpation d'identit\'e
 \end{multi}

% Q.........................................
\begin{multi}[multiple=true]{Cyberedu-QUIZZ-ID-1.3}
	Quels sont les trois principaux besoins de s\'ecurit\'e (Voir slide 23 - 24)
\item* 	D: Disponibilit\'e
\item* 	I : Int\'egrit\'e
\item*	C : Confidentialit\'e
\item  P : Prouvabilit\'e
\item  A : InputAbilit\'e
\end{multi}


% Q.........................................
\begin{multi}[multiple=true]{Cyberedu-QUIZZ-ID-1.4}
	Entourer la (ou les) phrase(s) correcte(s)
\item 	Le chiffrement permet de garantir que la donn\'ee sera toujours disponible/accessible
\item* 	La s\'ecurit\'e physique permet d'assurer la disponibilit\'e des \'equipements et des donn\'ees
\item 	La signature \'electronique permet de garantir la confidentialit\'e de la donn\'ee
\item*	Les d\'enis de service distribu\'es (DDoS) portent atteinte\`a la disponibilit\'e des donn\'ees
\end{multi}



% Q.........................................
\begin{multi}[multiple=true]{Cyberedu-QUIZZ-ID-1.5}
	Vous d\'eveloppez un site web www.asso-etudiants-touristes.org  pour une association qui regroupe les \'etudiants souhaitant effectuer des voyages ensemble\`a l'\'etranger. Sur ce site on retrouve les informations concernant les voyages propos\'es telles que : le pays, les villes\`a visiter, le prix du transport, les conditions d'h\'ebergement, les dates potentielles du voyage. Ces informations ont un besoin en confidentialit\'e :
\item* 	Faible
\item 	Fort
\end{multi}

% Q.........................................
\begin{multi}[multiple=true]{Cyberedu-QUIZZ-ID-1.6}
	Je viens de d\'evelopper un site web pour une association qui regroupe les \'etudiants souhaitant effectuer des voyages en groupe\`a l'\'etranger. Les informations relatives aux \'etudiants inscrits sur le site (login et mot de passe, nom, pr\'enom, num\'ero de t\'el\'ephone, adresse), ont un besoin en confidentialit\'e :
\item 	Faible
\item* 	Fort
\end{multi}

% Q.........................................
\begin{multi}[multiple=true]{Cyberedu-QUIZZ-ID-1.6}
	Je peux r\'eussir une attaque sur un bien qui n'a aucune vuln\'erabilit\'e (voir Slide 34 - Notions de vuln\'erabilit\'e, menace, attaque - attaque):
\item 	Vrai
\item* 	Faux
\end{multi}

% Q.........................................
\begin{multi}[multiple=true]{Cyberedu-QUIZZ-ID-1.7}
	Toutes les organisations et tous les individus font face aux m\^emes menaces (voir slide 40 - Exemples de sources de menaces):
\item 	Vrai
\item* 	Faux
\end{multi}

% Q.........................................
\begin{multi}[multiple=true]{Cyberedu-QUIZZ-ID-1.8}
	Entourer les attaques g\'en\'eralement de type  cibl\'ee (slide 42 - 52: Panorama de quelques menaces):
\item 	Phishing ou hame\c(c)onnage
\item 	Ransomware ou ran\c(c)ongiciel
\item* 	Social engineering ou ing\'enierie sociale
\item* 	Spear phishing ou l'arnaque au pr\'esident
\end{multi}

% Q.........................................
\begin{multi}[multiple=true]{Cyberedu-QUIZZ-ID-1.9}
	Entourer les attaques g\'en\'eralement de type non  cibl\'ee  (slide 42 - 52: Panorama de quelques menaces):
\item 	Intrusion informatique
\item* 	Virus informatique
\item 	D\'eni de service distribu\'e
\item* 	Phishing ou hame\c{c}onnage
\end{multi}

% Q.........................................
\begin{multi}[multiple=true]{Cyberedu-QUIZZ-ID-1.10}
	Quels sont les \'el\'ements facilitateurs de fraudes internes (Slide 47 - Panorama de quelques menaces : Fraude interne)
\item* 	Des comptes utilisateurs partag\'es entre plusieurs personnes
\item 	L'existence de proc\'edures de contr\^ole interne
\item* 	Peu ou pas de supervision des actions internes
\item 	Une gestion stricte et revue des habilitations
\end{multi}

% Q.........................................
\begin{multi}[multiple=true]{Cyberedu-QUIZZ-ID-1.11}
	Entourer les \'el\'ements qui peuvent r\'eduire ou emp\^echer des fraudes internes
\item* 	Une gestion stricte et une revue des habilitations
\item* 	Une s\'eparation des rôles des utilisateurs
\item 	Peu ou pas de surveillance interne
\item* 	Des comptes utilisateurs individuels pour chacun
\end{multi}

% Q.........................................
\begin{multi}[multiple=true]{Cyberedu-QUIZZ-ID-1.12}
	Citer les vecteurs d'infection de virus 
\item* 	Une pi\`ece jointe attach\'e\`a un message \'electronique
\item* 	Un support amovible infect\'e par exemple une cl\'e USB
\item* 	Un site web malveillant ou ayant des pages web corrompues
\item* 	Un partage r\'eseau ouvert
\item* 	Un syst\`eme vuln\'erable
\end{multi}

% Q.........................................
\begin{multi}[multiple=true]{Cyberedu-QUIZZ-ID-1.13}
	Qu'est-ce qu'un botnet? (Slide -  Panorama de quelques menaces : D\'eni de service distribu\'e)
\item*  un r\'eseau d'ordinateurs infect\'es et contrôl\'es par une personne malveillante.
\item un logiciel maveillant s'autorepliquant sur internet
\item un système controlé\`a distance par un logiciel malveillant
\end{multi}

% Q.........................................
\begin{multi}[multiple=true]{Cyberedu-QUIZZ-ID-1.14}
	Vous devez syst\'ematiquement donner votre accord avant de faire partir d'un r\'eseau de botnets? (Slide 52 - Panorama de quelques menaces : D\'eni de service distribu\'e - illustration d'un botnet)
\item 	Vrai
\item* 	Faux
\end{multi}

% Q.........................................
\begin{multi}[multiple=true]{Cyberedu-QUIZZ-ID-1.15}
	En France, la cybers\'ecurit\'e ne concerne que les entreprises du secteur priv\'e et les individus (Slide 54 : L'organisation de la s\'ecurit\'e en France)
\item 	Vrai
\item* 	Faux
\end{multi}

% Q.........................................
\begin{multi}[multiple=true]{Cyberedu-QUIZZ-ID-1.16}
	L'usage d'outils pour obtenir les cl\'es Wifi et acc\'eder au r\'eseau  Wifi du voisin tombe sous le coup de la loi (Slide 58 - Dispositif juridique fran\c{c}ais de lutte contre la cybercriminalit\'e):
\item 	Vigipirate
\item* 	Godfrain
\item 	Hadopi
\item 	Patriot act
\end{multi}

% Q.........................................
\begin{multi}[multiple=true]{Cyberedu-QUIZZ-ID-1.17}
	Mon r\'eseau wifi personnel est mal s\'ecuris\'e, par exemple par l'usage d'une cl\'e Wifi faible (exemple: 12345678). Une personne (intrus) se connecte\`a mon r\'eseau  pour effectuer des actions malveillantes comme attaquer un site gouvernemental :
\item 	J'encours des sanctions
\item 	Seul l'intrus encourt des sanctions
\item* 	L'intrus et moi encourons des sanctions.
\item 	Aucune sanction n'est encourue
\end{multi}

% Q.........................................
\begin{multi}[multiple=true]{Cyberedu-QUIZZ-ID-1.18}
	 Donn\'ees\`a caract\`ere personnel lesquels ?
\item*	Nom, pr\'enom
\item* 	Nom, t\'el\'ephone
\item* 	Date de naissance et commune
\item* 	Lieu de naissance
\item* 	Nationalit\'e ou pays de naissance des parents ou des grands parents
\item* 	Adresse
\item* 	No carte d'identit\'e / No de passeport / No de permis de conduire, …
\item* 	Empreinte digitale
\end{multi}

% Q.........................................
\begin{multi}[multiple=true]{Cyberedu-QUIZZ-ID-1.20}
	Lors de la cr\'eation du site Web de notre association \'etudiante, si vous stockez les informations suivantes pour chaque membre : nom, pr\'enom, adresse, adresse email. Aupr\`es de quel organisme devez-vous faire une d\'eclaration (Slide 60 - 64 : Droit de protection des donn\'ees\`a caract\`ere personnel)?
\item 	Gendarmerie
\item 	Universit\'e
\item* 	CNIL
\item 	Hadopi
\end{multi}
  \end{quiz}
   \end{document}
