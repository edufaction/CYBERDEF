% ======================================
%            CyberEdu QUIZZ
%            VOLUME 1
%            Mise en forme Moodle QUIZZ
%            (c) eric.dupuis@lecnam.net
% ======================================
\documentclass[12pt]{article}
%  \usepackage[draft]{moodle}
   \usepackage{moodle}
   \begin{document}
% =====================================
   \begin{quiz}{CYBERDEF101-Cyberedu-QUIZZ 2}
% =====================================


% Q.........................................
\begin{multi}[multiple=true]{Cyberedu-QUIZZ-ID-2.1}
	Donner 2 exemples de donn\'ees \'electroniques sensibles pour un \'etudiant :
\item Adresse postale
\item* Nom et num\'ero de s\'ecurit\'e sociale
\item* Num\'ero de carte bancaire
\item Nom de famille
\end{multi}

% Q.........................................
\begin{multi}[multiple=true]{Cyberedu-QUIZZ-ID-2.2}
	Donner 2 exemples de donn\'ees \'electroniques sensibles pour une universit\'e/\'ecole :
\item Le nom et l'origine de l'universit\'e
\item Les noms des professeurs
\item* Les brevets d\'epos\'es
\item* Les \'epreuves d'examens \`{a} venir (non encore pass\'es)
\end{multi}

% Q.........................................
\begin{multi}[multiple=true]{Cyberedu-QUIZZ-ID-2.3}
	Dans un r\'eseau, qu'est-ce qu'on entend par une zone de confiance?
\item Le hotspot wifi offert aux visiteurs, exemple \`{a} la gare SNCF
\item* Le r\'eseau interne (où sont h\'eberg\'es les postes des utilisateurs et les serveurs)
\item Le r\'eseau Internet
\item Une zone d\'emilitaris\'ee (DMZ)
\end{multi}

% Q.........................................
\begin{multi}[multiple=true]{Cyberedu-QUIZZ-ID-2.4}
	Quand parle-t-on d'une authentification mutuelle entre deux entit\'es?
\item Lorsque des deux entit\'es sont administr\'ees par la m\^eme personne
\item* Lorsque chacune des entit\'es doit s'authentifier vis-\`{a}-vis de l'autre
\item Lorsque la communication entre les deux entit\'es est chiffr\'ee
\item Lorsque les deux entit\'es sont situ\'ees sur le m\^eme r\'eseau
\end{multi}

% Q.........................................
\begin{multi}[multiple=true]{Cyberedu-QUIZZ-ID-2.5}
	Dans un r\'eseau, l'usage du BYOD peut entrainer (choisir la (ou les) proposition(s) vraie(s)) :
\item Une restriction du p\'erim\`etre \`{a} s\'ecuriser
\item* La propagation de codes malveillants
\item* La fuite de donn\'ees de l'entreprise
\item Une meilleure s\'ecurit\'e du SI
\end{multi}

% Q.........................................
\begin{multi}[multiple=true]{Cyberedu-QUIZZ-ID-2.6}
	Quel est le principe c\'el\`ebre en mati\`ere de gestion de flux sur un r\'eseau?
\item* Tout ce qui n'est pas autoris\'e est interdit
\item Tout ce qui est autoris\'e n'est pas interdit
\item Tout ce qui est interdit est interdit
\end{multi}

% Q.........................................
\begin{multi}[multiple=true]{Cyberedu-QUIZZ-ID-2.7}
	Un  pare-feu  peut \^etre aussi bien mat\'eriel (appliance d\'edi\'ee) que logiciel?
\item* Vrai
\item Faux
\end{multi}

% Q.........................................
\begin{multi}[multiple=true]{Cyberedu-QUIZZ-ID-2.8}
	Entourer la (ou les) proposition(s) vraie(s) qui peut (ou peuvent) servir de mesure de s\'ecurisation des acc\`es distants \`{a} un r\'eseau :
\item* Utiliser un serveur d'authentification centralis\'e comme TACACS+
\item Utiliser Internet
\item Utiliser un protocole s\'ecuris\'e tel que telnet ou ftp
\item* Utiliser un VPN
\end{multi}

% Q.........................................
\begin{multi}[multiple=true]{Cyberedu-QUIZZ-ID-2.9}
	Entourer la (ou les) bonne(s) mesure(s) de s\'ecurisation de l'administration
\item Rendre les interfaces d'administration disponibles \`{a} tous depuis Internet
\item Tous les administrateurs doivent utiliser le m\^eme compte pour se connecter
\item* Utiliser un r\'eseau d\'edi\'e pour l'administration
\item* Authentifier mutuellement les postes des administrateurs et les serveurs \`{a} administrer.
\end{multi}

% Q.........................................
\begin{multi}[multiple=true]{Cyberedu-QUIZZ-ID-2.10}
	Quelle est la technologie la plus appropri\'ee pour s\'ecuriser son acc\`es Wifi:
\item WEP
\item WPA
\item WPS
\item* WPA2
\end{multi}

% Q.........................................
\begin{multi}[multiple=true]{Cyberedu-QUIZZ-ID-2.11}
	Entourer la (ou les) proposition(s) vraie(s) lors de l'usage d'un hotspot Wifi?
\item* Il peut s'agir d'un faux point d'acc\`es ;
\item* Les autres personnes connect\'ees peuvent voir mes communications
\item Je suis prot\'eg\'e des personnes malveillantes
\item Je suis sur un r\'eseau de confiance, je peux d\'esactiver mon pare-feu.
\end{multi}

% Q.........................................
\begin{multi}[multiple=true]{Cyberedu-QUIZZ-ID-2.12}
	Pourquoi v\'erifier l'int\'egrit\'e d'un logiciel?
\item Pour m'assurer qu'il ne contient pas de virus
\item* Pour m'assurer que le logiciel que je t\'el\'echarge n'a pas \'et\'e corrompu
\item Pour m'assurer que le logiciel fonctionne bien comme promis
\item Pour m'assurer qu'il est gratuit
\end{multi}

% Q.........................................
\begin{multi}[multiple=true]{Cyberedu-QUIZZ-ID-2.13}
	Laquelle (ou lesquelles) des expressions suivantes est (sont) vraie(s) pour un logiciel t\'el\'echargeable?
\item toujours gratuit
\item* Peut \^etre  open source
\item* Peut contenir des logiciels espions
\item* Peut \^etre un programme malveillant
\end{multi}

% Q.........................................
\begin{multi}[multiple=true]{Cyberedu-QUIZZ-ID-2.14}
	Citer une bonne pratique de configuration de son antivirus
\item Avoir un antivirus d'un \'editeur connu
\item Avoir un jour install\'e un antivirus
\item* Tenir son antivirus \`{a} jour (mise \`{a} jour des signatures et du moteur)
\item Interdire l'analyse antivirale \`{a} certains r\'epertoires ou p\'eriph\'eriques.
\end{multi}

% Q.........................................
\begin{multi}[multiple=true]{Cyberedu-QUIZZ-ID-2.15}
	S\'electionner la (ou les) proposition(s) vraie(s) parmi les suivantes. Un antivirus:
\item peut d\'etecter tous les virus et programmes malveillants, y compris ceux non d\'ecouverts
\item prot\`ege de toutes les menaces
\item* ne peut d\'etecter que les virus qui sont connus dans sa base de signatures
\item* doit \^etre actif, et \`{a} jour pour \^etre utile
\end{multi}

% Q.........................................
\begin{multi}[multiple=true]{Cyberedu-QUIZZ-ID-2.16}
	Choisir un (ou des) sympt\^ome(s) potentiel(s) d'infection par un code malveillant
\item* Mon antivirus est d\'esactiv\'e
\item* Mon ordinateur fonctionne plus lentement
\item  J'ai plusieurs pages Web  qui s'ouvrent toutes seules
\item Des fichiers ou des r\'epertoires sont cr\'e\'es automatiquement sur mon poste
\end{multi}

% Q.........................................
\begin{multi}[multiple=true]{Cyberedu-QUIZZ-ID-2.17}
	Les mises \`{a} jour logicielles servent \`{a} am\'eliorer les logiciels et \`{a} corriger les failles de s\'ecurit\'e
\item* Vrai
\item Faux
\end{multi}

% Q.........................................
\begin{multi}[multiple=true]{Cyberedu-QUIZZ-ID-2.18}
	Vous pouvez prot\'eger la confidentialit\'e vos donn\'ees en :
\item* Les chiffrant
\item En calculant leur empreinte de mani\`ere \`{a} v\'erifier leur int\'egrit\'e
\item En les envoyant vers des supports externes ou vers le Cloud
\item En  les publiant sur Internet
\end{multi}

% Q.........................................
\begin{multi}[multiple=true]{Cyberedu-QUIZZ-ID-2.19}
	S\'electionner le (ou les) moyen(s) de durcissement d'une configuration
\item* Modifier les mots de passe par d\'efaut
\item* D\'esinstaller les logiciels inutiles
\item Activer le mode  d\'ebogage USB  sur les t\'el\'ephones
\item* S\'ecuriser le BIOS \`{a} l'aide d'un mot de passe
\end{multi}

% Q.........................................
\begin{multi}[multiple=true]{Cyberedu-QUIZZ-ID-2.20}
	S\'electionner le (ou les) principes(s) \`{a} prendre en compte lors de l'attribution de privil\`eges utilisateurs
\item  Tout ce qui n'est pas interdit, est autoris\'e
\item*  Moindre privil\`ege
\item*  Besoin d'en connaitre
\item  Droit administrateur pour tous
\end{multi}

% Q.........................................
\begin{multi}[multiple=true]{Cyberedu-QUIZZ-ID-2.21}
	Entourer la (ou les) mauvaise(s) pratique(s) pour les mots de passe
\item* Je cr\'ee un mot de passe tr\`es long et tr\`es complexe, dont je ne me souviens pas
\item* Ma date de naissance me sert de mot de passe
\item* Je stocke mes mots de passe en clair dans un fichier texte
\item* Mon mot de passe doit avoir au plus 7 caract\`eres
\end{multi}

% Q.........................................
\begin{multi}[multiple=true]{Cyberedu-QUIZZ-ID-2.22}
	Entourer la (ou les) bonne(s) pratique(s) pour les mots de passe
\item J'enregistre mes mots de passe sur chaque navigateur Internet
\item* Je cr\'ee un mot de passe long et complexe dont je peux me souvenir * facilement
\item J'\'ecris mon mot de passe sur un post-it que je cache sous mon clavier/PC
\item* J'utilise un porte-cl\'es de mots de passe
\end{multi}

% Q.........................................
\begin{multi}[multiple=true]{Cyberedu-QUIZZ-ID-2.23}
	Entourer la (ou les) bonne(s) pratique(s) de navigation sur Internet
\item Je suis victime de ransomware, je paye la ran\c{c}on
\item* J'\'evite de communiquer avec des inconnus
\item J'accepte toutes les demandes sur les m\'edias sociaux
\item Je donne mon mot de passe de messagerie \`{a}  l'administrateur   lorsqu'il me le demande.
\end{multi}

% Q.........................................
\begin{multi}[multiple=true]{Cyberedu-QUIZZ-ID-2.24}
	Citer deux moyens de s\'ecurisation physique des biens/\'equipements
\item Mettre les \'equipements sensibles dans une salle sans contr\^ole d'acc\`es
\item* Attacher les \'equipements sensibles avec des c\`{a}bles de s\'ecurit\'e
\item Nommer tous les \'equipements de la m\^eme fa\c{c}on
\item* Utiliser des filtres de confidentialit\'e pour les \'ecrans
\end{multi}

% Q.........................................
\begin{multi}[multiple=true]{Cyberedu-QUIZZ-ID-2.25}
	Choisir l' (ou les) exemple(s) d'incidents de s\'ecurit\'e
\item* Le vol d'un \'equipement/terminal
\item La cr\'eation d'un compte utilisateur pour un nouvel \'etudiant
\item* La pr\'esence d'un code malveillant sur un poste
\item* La divulgation sur un forum des noms, pr\'enoms, et num\'eros de s\'ecurit\'e sociale des \'etudiants
\end{multi}

% Q.........................................
\begin{multi}[multiple=true]{Cyberedu-QUIZZ-ID-2.26}
	Choisir la (ou les) bonne(s) r\'eaction(s) face \`{a} un incident de s\'ecurit\'e :
\item D\'esactiver/d\'esinstaller son antivirus
\item* Appliquer les r\`egles/consignes re\c{c}ues par exemple dans la charte informatique
\item* Chercher \`{a} identifier la cause de l'incident
\item D\'esactiver son pare-feu (personnel par exemple)
\end{multi}

% Q.........................................
\begin{multi}[multiple=true]{Cyberedu-QUIZZ-ID-2.27}
	S\'electionner la (ou les) raison(s) pour laquelle (ou lesquelles) les audits de s\'ecurit\'e peuvent \^etre effectu\'es :
\item* Pour obtenir une certification ou un agr\'ement
\item* Pour trouver des faiblesses et les corriger
\item* Pour \'evaluer le niveau de s\'ecurit\'e
\item Provoquer des incidents de s\'ecurit\'e
\end{multi}

  \end{quiz}
   \end{document}
