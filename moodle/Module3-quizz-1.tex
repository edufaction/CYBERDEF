% ======================================
% CYBERDEF 101
% MODULE 3 QUIZZ
% VOLUME 1
% Mise en forme Moodle QUIZZ
% (\item eric.dupuis@lecnam.net
% ======================================
\documentclass[12pt]{article}

% \usepackage[draft]{moodle}
 \usepackage{moodle}
 \begin{document}

% =====================================
 \begin{quiz}{CYBERDEF101-MODULE 3-QUIZZ}
% =====================================


% Q.........................................
\begin{multi}[multiple=true]{CD101-Mod3-Q1} 
Quand on parle de gestion de vulnérabilités dans l'entreprise on parle ?
\item de répertorier les vulnérabilités informatiques sur internet
\item* de constituer une base de données de ses propres fragilités pour plus tard
\item* d'identifier ses vulnérabilités et de les corriger au plus tôt	
\end{multi}


% Q.........................................
\begin{multi}[multiple=true]{CD101-Mod3-Q2}
Dans la gestion de l'évènement de sécurité d'un SIEM, donner le premier et le dernier des processus de gestion de l'évènement :
\item* détection 
\item corrélation
\item collecte
\item* alerte
\end{multi}

 % Q.........................................
\begin{multi}[multiple=true]{CD101-Mod3-Q3}
La norme ISO 27035 traite en particulier de :
\item gestion des vulnérabilités	
\item* gestion des incidents	
\item gestion des menaces informatiques
\item gestion de crise
\end{multi}	

% Q.........................................
\begin{multi}[multiple=true]{CD101-Mod3-Q4} 
Le Forensic dans la réponse à incident comprend  :
\item* l'analyse de traces dans les réseaux	
\item* l'analyse de documents cachés dans une machine
\item* la recherche d'indices de compromission
\item* l'identification de l'attaquant
\end{multi}

% Q.........................................
\begin{multi}[multiple=true]{CD101-Mod3-Q5} 
La sécurité opérationnelle comporte  : 
\item* le maintien en condition de sécurité au sein de la DSI
\item* le pilotage les audits techniques (pentests) sur l'environnement	
\item* les opérations du SOC : Security Operation Center	
\item la gestion de la continuité d'activité de l'entreprise
\item la construction des politiques de sécurité
\item la supervision du fonctionnement du système d'information
\end{multi}


% Q.........................................
\begin{multi}[multiple=true]{CD101-Mod3-Q6} 
Quelle la signification dans le domaine des systèmes d'information du terme MTTR 
\item le temps moyen entre deux attaques
\item le temps moyen entre deux bugs	
\item* le temps moyen pour réparer
\item la gestion de la continuité d'activité
\end{multi}

% Q.........................................
\begin{multi}[multiple=true]{CD101-Mod3-Q7} 
Un CERT privé peu fournir moyennant paiement des services sur :
\item* des vulnérabilités pour faire des tests d'entreprise
\item des bases de données d'applications sures 
\item des attaques informatiques rendues inoffensives pour les pentests	
\item* des alertes liées à la sécurité informatique
\end{multi}


% Q.........................................
\begin{multi}[multiple=true]{CD101-Mod3-Q8}
A quoi peut-on associer le terme : WannaCry 
\item* Un ransonware
\item Un spyware
\item Un antivirus
\item Un réseau social
\end{multi}

% Q.........................................
\begin{multi}[multiple=true]{CD101-Mod3-Q9}
Un plan de cyberdéfense comporte
\item Le périmètre de certification ISO 27001
\item* La liste des actifs critiques
\item* Les évènements les plus redoutés
\item* Un annuaire de crise
\item La liste des politiques de sécurité de l’entreprise 
\end{multi}


% Q.........................................
\begin{multi}[multiple=true]{CD101-Mod3-Q10}
Un SIEM intègre les évènements de détection de
\item* IPS / IDS
\item AD
\item* WAF
\item* Proxy
\end{multi}

% Q.........................................
\begin{multi}[multiple=true]{CD101-Mod3-Q11}
L’orchestration dans la gestion de vulnérabilités comporte
\item L’automatisation des patchs 
\item* La gestion et le suivi des vulnérabilités des actifs
\item* La gestion des scans de vulnérabilités
\end{multi}

% Q.........................................
\begin{multi}[multiple=true]{CD101-Mod3-Q12}
La recherche récurrente hebdo de vulnérabilités passe principalement par
\item* Le scan de vulnérabilités
\item Les tests d’intrusion (Pentest)
\item Le bug-bounty
\item* L’analyse statique de code source
\end{multi}

% Q.........................................
\begin{multi}[multiple=true]{CD101-Mod3-Q13}
 Quelles sont les deux priorités d’un RSSI en SECOPS
\item* Déployer la gestion de vulnérabilités
\item* Mettre en place une fonction de détection
\item Mettre à jour ses Firewalls
\item Déployer un VPN pour l’accès à distance
\end{multi}

% Q.........................................
\begin{multi}[multiple=true]{CD101-Mod3-Q14}
Que signifie l’acronyme APT ?
\item Array Processor Tampering
\item* Advanced Persistent Threat
\item Advanced Programming Theory
\item Agnostic Programming Threat
\end{multi}

% Q.........................................
\begin{multi}[multiple=true]{CD101-Mod3-Q15}
Une faille 0 day est une faille ?
\item Qui a été corrigée depuis moins de 24h
\item Qui a été découverte depuis moins de 24h
\item* Qui n'a pas encore de correctif de sécurité disponible
\item Qui n'a pas encore été exploitée
\end{multi}

% Q.........................................
\begin{multi}[multiple=true]{CD101-Mod3-Q16}
Qu’est qu’un outil SIEM ne fait pas ou n’est pas dans son scope  à ce jour
\item Identification de la menace
\item Enregistrement de l’incident
\item Classement de l’incident
\item Escalade d’un incident
\item* Diagnostic d’impact
\item* Résolution et rétablissement du service
\end{multi}

% Q.........................................
\begin{multi}[multiple=true]{CD101-Mod3-Q17}
En cybersécurité, le terme  SOAR veut dire
\item Strength Orchestration, for Automation and Response 
\item Strengths, Opportunities , Aspirations, Results
\item*  Security Orchestration, Automation and Response
\item Security Organisation, Automation and Reaction
\end{multi}

% Q.........................................
\begin{multi}[multiple=true]{CD101-Mod3-Q18}
Une équipe CSIRT conduit dans 
\item* L’investigation sur incident
\item La remédiation sur le système attaqué
\item L’anticipation par recherche de vulnérabilité
\item* Le hunting (chasse à la menace)
\end{multi}

% Q.........................................
\begin{multi}[multiple=true]{CD101-Mod3-Q19}
La norme 22301 doit être utilisée dans
\item* L’organisation de la gestion de crise
\item La gestion des incidents non critiques
\item La mise en place du SMSI (Système de management système de gestion de la sécurité)
\end{multi}

% Q.........................................
\begin{multi}[multiple=true]{CD101-Mod3-Q20}
 Dans la gestion des incidents de sécurité, l'équipe SECOPS Incidents doit gérer
\item* La détection et l’enregistrement des incidents
\item L'exploitation de l’IT
\item* La classification et l’aide initiale
\item* L’enquête et le diagnostic
\item La restauration des données
\item* La cloture de l’incident
\item La communication de crise
\end{multi}

% =====================================
 \end{quiz}
 % =====================================
 
 \end{document}
