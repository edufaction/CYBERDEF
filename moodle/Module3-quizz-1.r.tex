% ======================================
% CYBERDEF 101
% MODULE 3 QUIZZ
% VOLUME 1
% Mise en forme Moodle QUIZZ
% (\item eric.dupuis@lecnam.net
% ======================================
\documentclass[12pt]{article}

% \usepackage[draft]{moodle}
 \usepackage{moodle}
 \begin{document}

% =====================================
 \begin{quiz}{CYBERDEF101-MODULE 3-QUIZZ}
% =====================================


% Q.........................................
\begin{multi}[multiple=true]{CD101-Mod3-Q1} 
Quand on parle de gestion de vuln\'erabilit\'es dans l'entreprise on parle ?
\item de r\'epertorier les vuln\'erabilit\'es informatiques sur internet
\item* de constituer une base de donn\'ees de ses propres fragilit\'es pour plus tard
\item* d'identifier ses vuln\'erabilit\'es et de les corriger au plus t\^ot	
\end{multi}


% Q.........................................
\begin{multi}[multiple=true]{CD101-Mod3-Q2}
Dans la gestion de l'\'ev\`enement de s\'ecurit\'e d'un SIEM, donner le premier et le dernier des processus de gestion de l'\'ev\`enement :
\item* d\'etection 
\item corr\'elation
\item collecte
\item* alerte
\end{multi}

 % Q.........................................
\begin{multi}[multiple=true]{CD101-Mod3-Q3}
La norme ISO 27035 traite en particulier de :
\item gestion des vuln\'erabilit\'es	
\item* gestion des incidents	
\item gestion des menaces informatiques
\item gestion de crise
\end{multi}	

% Q.........................................
\begin{multi}[multiple=true]{CD101-Mod3-Q4} 
Le Forensic dans la r\'eponse \`a incident comprend  :
\item* l'analyse de traces dans les r\'eseaux	
\item* l'analyse de documents cach\'es dans une machine
\item* la recherche d'indices de compromission
\item* l'identification de l'attaquant
\end{multi}

% Q.........................................
\begin{multi}[multiple=true]{CD101-Mod3-Q5} 
La s\'ecurit\'e op\'erationnelle comporte  : 
\item* le maintien en condition de s\'ecurit\'e au sein de la DSI
\item* le pilotage les audits techniques (pentests) sur l'environnement	
\item* les op\'erations du SOC : Security Operation Center	
\item la gestion de la continuit\'e d'activit\'e de l'entreprise
\item la construction des politiques de s\'ecurit\'e
\item la supervision du fonctionnement du syst\`eme d'information
\end{multi}


% Q.........................................
\begin{multi}[multiple=true]{CD101-Mod3-Q6} 
Quelle la signification dans le domaine des syst\`emes d'information du terme MTTR 
\item le temps moyen entre deux attaques
\item le temps moyen entre deux bugs	
\item* le temps moyen pour r\'eparer
\item la gestion de la continuit\'e d'activit\'e
\end{multi}

% Q.........................................
\begin{multi}[multiple=true]{CD101-Mod3-Q7} 
Un CERT priv\'e peu fournir moyennant paiement des services sur :
\item* des vuln\'erabilit\'es pour faire des tests d'entreprise
\item des bases de donn\'ees d'applications sures 
\item des attaques informatiques rendues inoffensives pour les pentests	
\item* des alertes li\'ees \`a la s\'ecurit\'e informatique
\end{multi}


% Q.........................................
\begin{multi}[multiple=true]{CD101-Mod3-Q8}
A quoi peut-on associer le terme : WannaCry 
\item* Un ransonware
\item Un spyware
\item Un antivirus
\item Un r\'eseau social
\end{multi}

% Q.........................................
\begin{multi}[multiple=true]{CD101-Mod3-Q9}
Un plan de cyberd\'efense comporte
\item Le p\'erim\`etre de certification ISO 27001
\item* La liste des actifs critiques
\item* Les \'ev\`enements les plus redout\'es
\item* Un annuaire de crise
\item La liste des politiques de s\'ecurit\'e de l'entreprise 
\end{multi}


% Q.........................................
\begin{multi}[multiple=true]{CD101-Mod3-Q10}
Un SIEM int\`egre les \'ev\`enements de d\'etection de
\item* IPS / IDS
\item AD
\item* WAF
\item* Proxy
\end{multi}

% Q.........................................
\begin{multi}[multiple=true]{CD101-Mod3-Q11}
L'orchestration dans la gestion de vuln\'erabilit\'es comporte
\item L'automatisation des patchs 
\item* La gestion et le suivi des vuln\'erabilit\'es des actifs
\item* La gestion des scans de vuln\'erabilit\'es
\end{multi}

% Q.........................................
\begin{multi}[multiple=true]{CD101-Mod3-Q12}
La recherche r\'ecurrente hebdo de vuln\'erabilit\'es passe principalement par
\item* Le scan de vuln\'erabilit\'es
\item Les tests d'intrusion (Pentest)
\item Le bug-bounty
\item* L'analyse statique de code source
\end{multi}

% Q.........................................
\begin{multi}[multiple=true]{CD101-Mod3-Q13}
 Quelles sont les deux priorit\'es d'un RSSI en SECOPS
\item* D\'eployer la gestion de vuln\'erabilit\'es
\item* Mettre en place une fonction de d\'etection
\item Mettre \`a jour ses Firewalls
\item D\'eployer un VPN pour l'acc\`es \`a distance
\end{multi}

% Q.........................................
\begin{multi}[multiple=true]{CD101-Mod3-Q14}
Que signifie l'acronyme APT ?
\item Array Processor Tampering
\item* Advanced Persistent Threat
\item Advanced Programming Theory
\item Agnostic Programming Threat
\end{multi}

% Q.........................................
\begin{multi}[multiple=true]{CD101-Mod3-Q15}
Une faille 0 day est une faille ?
\item Qui a \'et\'e corrig\'ee depuis moins de 24h
\item Qui a \'et\'e d\'ecouverte depuis moins de 24h
\item* Qui n'a pas encore de correctif de s\'ecurit\'e disponible
\item Qui n'a pas encore \'et\'e exploit\'ee
\end{multi}

% Q.........................................
\begin{multi}[multiple=true]{CD101-Mod3-Q16}
Qu'est qu'un outil SIEM ne fait pas ou n'est pas dans son scope  \`a ce jour
\item Identification de la menace
\item Enregistrement de l'incident
\item Classement de l'incident
\item Escalade d'un incident
\item* Diagnostic d'impact
\item* R\'esolution et r\'etablissement du service
\end{multi}

% Q.........................................
\begin{multi}[multiple=true]{CD101-Mod3-Q17}
En cybers\'ecurit\'e, le terme  SOAR veut dire
\item Strength Orchestration, for Automation and Response 
\item Strengths, Opportunities , Aspirations, Results
\item*  Security Orchestration, Automation and Response
\item Security Organisation, Automation and Reaction
\end{multi}

% Q.........................................
\begin{multi}[multiple=true]{CD101-Mod3-Q18}
Une \'equipe CSIRT conduit dans 
\item* L'investigation sur incident
\item La rem\'ediation sur le syst\`eme attaqu\'e
\item L'anticipation par recherche de vuln\'erabilit\'e
\item* Le hunting (chasse \`a la menace)
\end{multi}

% Q.........................................
\begin{multi}[multiple=true]{CD101-Mod3-Q19}
La norme 22301 doit \^etre utilis\'ee dans
\item* L'organisation de la gestion de crise
\item La gestion des incidents non critiques
\item La mise en place du SMSI (Syst\`eme de management syst\`eme de gestion de la s\'ecurit\'e)
\end{multi}

% Q.........................................
\begin{multi}[multiple=true]{CD101-Mod3-Q20}
 Dans la gestion des incidents de s\'ecurit\'e, l'\'equipe SECOPS Incidents doit g\'erer
\item* La d\'etection et l'enregistrement des incidents
\item L'exploitation de l'IT
\item* La classification et l'aide initiale
\item* L'enqu\^ete et le diagnostic
\item La restauration des donn\'ees
\item* La cloture de l'incident
\item La communication de crise
\end{multi}

% =====================================
 \end{quiz}
 % =====================================
 
 \end{document}
