% ======================================
%                               CyberEdu QUIZZ
%                                    VOLUME 4 
%                   Mise en forme Moodle QUIZZ
%                       (c) eric.dupuis@lecnam.net
% ======================================

\documentclass[12pt]{article}

%  \usepackage[draft]{moodle}
   \usepackage{moodle}
   \begin{document}

% ===================================== 
   \begin{quiz}{CYBERDEF101-Cyberedu-QUIZZ 4}
% =====================================       
  
% Q..................................................................................
\begin{multi}[multiple=true]{Cyberedu-QUIZZ-ID-4.1}
De quelle famille de normes internationales, une organisation peut-elle s'inspirer pour int\'egrer la s\'ecurit\'e en son sein ?
\item*27000
\item 9000
\item 14000
\end{multi}
% Q..................................................................................
\begin{multi}[multiple=true]{Cyberedu-QUIZZ-ID-4.2}
	Citer un exemple repr\'esentatif d'une organisation devant avoir recours \`a une certification de s\'ecurit\'e. 
\item* pour respecter une r\'eglementation
\item* pour s'am\'eliorer
\item pour se prot\'eger contre des menaces sp\'ecifiques
\end{multi}

% Q..................................................................................
\begin{multi}[multiple=true]{Cyberedu-QUIZZ-ID-4.3}
Tr\`es souvent dans les entreprises, les informations ont toutes le m\^eme niveau de confidentialit\'e  toutes non confidentielles  
\item  Vrai 
\item* Faux.
\end{multi}

% Q..................................................................................
\begin{multi}[multiple=true]{Cyberedu-QUIZZ-ID-4.4}
Pour une bonne int\'egration de la s\'ecurit\'e dans l'organisation, le personnel doit \^etre sensibilis\'e \`a la s\'ecurit\'e conform\'ement \`a leurs fonctions 
\item* Vrai 
\item Faux.
\end{multi}


% Q..................................................................................
\begin{multi}[multiple=true]{Cyberedu-QUIZZ-ID-4.5}
Citer les proc\'edures de gestion des d\'eparts du personnel indispensables impactant la s\'ecurit\'e 
\item* Retrait des acc\`es 
\item* Restitution du mat\'eriel fourni (badge, ordinateur, …)
\item Solde de tout compte
\end{multi}


% Q..................................................................................
\begin{multi}[multiple=true]{Cyberedu-QUIZZ-ID-4.6}
La s\'ecurit\'e c'est comme la cerise sur le g\^ateau  elle doit \^etre prise en compte \`a la fin d'un projet
\item	Vrai 
\item*	Faux.
\end{multi}
% Q..................................................................................
\begin{multi}[multiple=true]{Cyberedu-QUIZZ-ID-4.7}
Le but d'une analyse de risques est de d\'eterminer pour un p\'erim\`etre donn\'e (projet par exemple), les risques qui peuvent porter sur les biens non sensibles 
\item	Vrai 
\item*	Faux
\end{multi}

% Q..................................................................................
\begin{multi}[multiple=true]{Cyberedu-QUIZZ-ID-4.8}
S\'electionner la phase qui r\'esume le plus la d\'emarche d'analyse de risques
\item Identifier les agents menacants et les neutraliser 
\item Identifier les acteurs importants du projet 
\item Inventorier les biens
\item* D\'eterminer les risques et les traiter
\end{multi}

\item Identifier les agents mena\c{c}ants et les neutraliser 
\item Identifier les acteurs importants du projet 
\item Inventorier les biens
\item* D\'eterminer les risques et les traiter

% Q..................................................................................
\begin{multi}[multiple=true]{Cyberedu-QUIZZ-ID-4.9}
Est-ce que tous les risques issus d'une analyse de risques doivent-ils \^etre trait\'es par une mesure de r\'eduction des risques ?
\item* Non seuls ceux dont le niveau de criticit\'e est sup\'erieur au seuil de tol\'erance
\item Oui tous les risques doivent \^etre couverts
\end{multi}
% Q..................................................................................
\begin{multi}[multiple=true]{Cyberedu-QUIZZ-ID-4.10}
Choisir la (les) proposition(s) correcte(s). Au cours de l'analyse de risques, les contre-mesures sont des mesures de r\'eduction de risque qui peuvent \^etre 
\item*	techniques et organisationnelles 
\item	techniques uniquement 
\item	organisationnelles uniquement 
\item* d\'eclin\'ees des objectifs de s\'ecurit\'e d\'efinis.
\end{multi}
% Q..................................................................................
\begin{multi}[multiple=true]{Cyberedu-QUIZZ-ID-4.11}
Choisir la (les) proposition(s) correcte(s) 
\item*	Il est plus facile d'attaquer un syst\`eme que de le rendre invuln\'erable 
\item	Il est facile de cr\'eer un syst\`eme sans aucune vuln\'erabilit\'e 
\item	Pour d\'efendre un syst\`eme, il suffit de le prot\'eger de mani\`ere p\'erim\'etrique 
\item*	La d\'efense en profondeur peut \^etre appliqu\'ee pour prot\'eger un syst\`eme.
\end{multi}


% Q..................................................................................
\begin{multi}[multiple=true]{Cyberedu-QUIZZ-ID-4.12}
Choisir la (les) proposition(s) correcte(s). La d\'efense en profondeur est un principe d'origine militaire qui consiste \`a avoir plusieurs lignes de d\'efense constituant des barri\`eres autonomes pour d\'efense un syst\`eme 
\item*	Vrai
\item	Faux.
\end{multi}
% Q..................................................................................
\begin{multi}[multiple=true]{Cyberedu-QUIZZ-ID-4.13}
Choisir la (les) proposition(s) correcte(s). Pour une organisation, l'usage des services du Cloud doit prendre en compte 
\item*	les exigences l\'egales relatives aux donn\'ees h\'eberg\'ees
\item*	les m\'ecanismes de s\'ecurit\'e tels que le chiffrement des donn\'ees stock\'ees propos\'es par le fournisseur du service
\item* le devenir des donn\'ees h\'eberg\'ees \`a la fin du contrat
\item*	les certifications dont dispose le fournisseur du service Cloud.
\end{multi}
% Q..................................................................................
\begin{multi}[multiple=true]{Cyberedu-QUIZZ-ID-4.14}
L'une des difficult\'es de l'int\'egration de la s\'ecurit\'e dans une organisation, est celle des choix \'eclair\'es en mati\`ere de produits de confiance 
\item*	Vrai 
\item	Faux.
\end{multi}
% Q..................................................................................
\begin{multi}[multiple=true]{Cyberedu-QUIZZ-ID-4.15}
Dans une organisation, la s\'ecurit\'e est critique. Elle doit \^etre impos\'ee \`a tous sans consultation
\item	Vrai 
\item*	Faux.
\end{multi}
% Q..................................................................................
\begin{multi}[multiple=true]{Cyberedu-QUIZZ-ID-4.16}
Le  Shadow IT  ou  Shadow Cloud  est une pratique qui consiste pour les utilisateurs \`a souscrire directement aux services Cloud sans la consultation et aval de leur DSI et en souvent en d\'epit de la politique de s\'ecurit\'e 
\item*	Vrai 
\item	Faux
\end{multi}
% Q..................................................................................
\begin{multi}[multiple=true]{Cyberedu-QUIZZ-ID-4.17}
Choisir la (les) proposition(s) correcte(s). Le Big Data peut constituer une opportunit\'e en s\'ecurit\'e car il peut permettre de 
\item	d'envoyer les donn\'ees sensibles de l'organisation en clair vers le Cloud 
\item*	d'utiliser une capacit\'e de traitement de mani\`ere \`a effectuer l'analyse d'\'ev\`enements de s\'ecurit\'e en temps r\'eel 
\item*	de corr\'eler les traces provenant de diff\'erents \'equipements r\'eseau pour d\'etecter des menaces persistantes avanc\'ees (APT) 
\item*	de surveiller le trafic r\'eseau en temps r\'eel pour d\'etecter les botnets.
\end{multi}
% Q..................................................................................
\begin{multi}[multiple=true]{Cyberedu-QUIZZ-ID-4.18}
Citer un m\'etier, avec une principale activit\'e associ\'ee, sollicit\'e dans chaque phase d'un cycle d'un projet 
\item*	Expression de besoin  Chef de projet/consultant MOA pour d\'efinir les exigences de s\'ecurit\'e issues d'une analyse de risque
\item*	D\'eveloppement  Chef de projet/consultant MOE, architecte, concepteur/d\'eveloppeur  pour sp\'ecifier, concevoir/d\'evelopper les mesures de s\'ecurit\'e
\item*	Validation  auditeur technique ou organisationnel pour contrôler la conformit\'e et l'efficacit\'e des mesures de s\'ecurit\'e
\item*	Exploitation  technicien ou administrateur pour maintenir en condition de s\'ecurit\'e (mise \`a jour des patchs et des bases de signature), analyste pour faire la veille sur les vuln\'erabilit\'es ou d\'etecter des incidents de s\'ecurit\'e
\end{multi}

% Q..................................................................................
\begin{multi}[multiple=true]{Cyberedu-QUIZZ-ID-4.19}	
Les comp\'etences recherch\'ees en cybers\'ecurit\'e sont uniquement techniques 
\item Vrai 
\item* Faux.
\end{multi}

% Q..................................................................................
\begin{multi}[multiple=true]{Cyberedu-QUIZZ-ID-4.20}
La cybers\'ecurit\'e est un secteur ayant peu de perspective d'embauche 
\item	Vrai 
\item*	Faux.
\end{multi}




  \end{quiz} 
   \end{document}
