%===========================
% EDX COMMAND
%===========================

\newcommand\UKword[1]{\emph{#1}}
\newcommand\Ucenter[1]{\begin{center}{#1}\end{center}}

\newcommand\head[1]{\textbf{#1}}
\newcommand\tb[1]{\textbf{#1}}
\newcommand{\g}[1]{\og #1 \fg}
\newcommand{\uload}[1] {\input{../Tex/#1}}
\newcommand{\ucurcolor}{black}


%-----------------------------------------------------
%										 	\upspicture 
%---------------------------------------------------
\newcommand{\upspicture}[3]
{
\renewcommand{\ucurcolor}{#3}
\resizebox{#2\textwidth}{!}{\input{#1}}
}

%-----------------------------------------------------
%											\upicture 
%---------------------------------------------------

\newcommand{\upicture}[4]{
\begin{figure}[] %h
  \begin{center}
	 \includegraphics[scale=#3]{#1.pdf}
  \end{center}
\caption{#2 \label{#4}}
\end{figure}
}

%-----------------------------------------------------
%											\updfimage 
%---------------------------------------------------

\newcommand{\updfimage}[3]
{
\begin{wrapfigure}{R}{#3\textwidth}
  \begin{center}
	 \resizebox{#3\textwidth}{!}{\includegraphics{#1.pdf}}
  \end{center}
\caption{#2}
\end{wrapfigure}
}



%************************PACKAGE***

\usepackage{tikz}% Required for drawing custom shapes
\usetikzlibrary{shadows}


%-----------------------------------------------------
%											 		\rem 
% insertion d'une remarque avec R
%---------------------------------------------------
\newcommand\rem[1]{%
   \marginpar{%
   \tikzpicture[baseline={(title.base)}]
      \node[inner sep=5pt,text width=4cm,drop shadow={shadow yshift=-5pt,shadow xshift=5pt,ocre},fill=white] (box) {\vskip5pt \nointerlineskip #1};
      \node[right=10pt,inner sep=0pt,outer sep=10pt] (title) at (box.north west) {\bfseries\color{ocre}Remarque};
      \draw[draw=ocre,very thick](title.west)--(box.north west)--(box.south west)--(box.south east)--(box.north east)--(title.east);
      \fill[ocre]([yshift=-10pt]box.north west)--+(-5pt,-5pt)--+(0pt,-10pt);
   \endtikzpicture}%
}
%-----------------------------------------------------
%											 \utikzimage 
% insertion d'une image sur fichier tkz.tex
%---------------------------------------------------

\newcommand{\utikzimage}[3]
{
\begin{wrapfigure}{R}{#3\textwidth}
  \begin{center}
	 \resizebox{#3\textwidth}{!}{\input{#1.tkz.tex}}
  \end{center}
\caption{#2}
\end{wrapfigure}
}


%************************PACKAGE***
 \usepackage{fontawesome}
  \usepackage{awesomebox}   
%-----------------------------------------------------
%											 \videobox 
% Video BOX lien
%----------------------------------------------------
  
%\newcommand{\videobox}[1]{%
%\awesomebox{\faYoutubePlay}{\aweboxrulewidth}{url}{#1}}

%-----------------------------------------------------
%														\link 
% Url et footnote du lien
%-----------------------------------------------------
  
  \newcommand{\link}[2]
  {
  \href{#1}
  {
  #2~\raisebox{-0.2ex}{\faExternalLink}\footnote
  	{
 #1
  	}
  }
  }
  
  
  
  
