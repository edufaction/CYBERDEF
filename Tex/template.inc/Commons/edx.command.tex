%===========================
% EDX COMMAND STANDART
%===========================

\newcommand\UKword[1]{\emph{#1}}
\newcommand\Ucenter[1]{\begin{center}{#1}\end{center}}

\newcommand\head[1]{\textbf{#1}}
\newcommand\tb[1]{\textbf{#1}}
\newcommand{\g}[1]{\og #1 \fg{}}
\newcommand{\uload}[1] {\input{../Tex/#1}}
\newcommand{\ucurcolor}{black}

\newcommand{\mak}[1]{\faicon{\aArrowCircleORight} #1}
%-----------------------------------------------------
%										 	\upspicture 
%---------------------------------------------------
\newcommand{\upspicture}[3]
{
\renewcommand{\ucurcolor}{#3}
\resizebox{#2\textwidth}{!}{\input{#1}}
}

\newcommand{\uchap}[1]
{\textbf{\Ucenter{--- #1 ---}}}

%-----------------------------------------------------
%											\uexpand
%---------------------------------------------------
\newcommand {\uexpand}[1]{
\mode<all>\input{../Tex/Chapters/#1}
}

%-----------------------------------------------------
%											\upicture 
%---------------------------------------------------

\newcommand{\upicture}[4]{
\begin{figure}[!h] %h
  \begin{center}
	 \includegraphics[width=#3\textwidth]{#1.pdf}
  \end{center}
\caption{#2 \label{#4}}
\end{figure}
}


%-----------------------------------------------------
%											\uref 
%---------------------------------------------------
\newcommand{\uref}[1]{(Voir~\cref{#1} page~\pageref{#1})}



%-----------------------------------------------------
%											\updfimage 
%---------------------------------------------------

\newcommand{\updfimage}[3]
{
\begin{wrapfigure}{R}{#3\textwidth}
  \begin{center}
	 \resizebox{#3\textwidth}{!}{\includegraphics{#1.pdf}}
  \end{center}
\caption{#2}
\end{wrapfigure}
}



%************************PACKAGE***

%\usepackage{tikz}% Required for drawing custom shapes
\usetikzlibrary{shadows}


%-----------------------------------------------------
%											 		\rem 
% insertion d'une remarque avec R
%---------------------------------------------------
\newcommand\rem[1]{%
   \marginpar{%
   \tikzpicture[baseline={(title.base)}]
      \node[inner sep=5pt,text width=4cm,drop shadow={shadow yshift=-5pt,shadow xshift=5pt,ocre},fill=white] (box) {\vskip5pt \nointerlineskip #1};
      \node[right=10pt,inner sep=0pt,outer sep=10pt] (title) at (box.north west) {\bfseries\color{ocre}Remarque};
      \draw[draw=ocre,very thick](title.west)--(box.north west)--(box.south west)--(box.south east)--(box.north east)--(title.east);
      \fill[ocre]([yshift=-10pt]box.north west)--+(-5pt,-5pt)--+(0pt,-10pt);
   \endtikzpicture}%
}
%-----------------------------------------------------
%											 \utikzimage 
% insertion d'une image sur fichier tkz.tex
%---------------------------------------------------

\newcommand{\utikzimage}[3]
{
\begin{wrapfigure}{R}{#3\textwidth}
  \begin{center}
	 \resizebox{#3\textwidth}{!}{\input{#1.tkz.tex}}
  \end{center}
\caption{#2}
\end{wrapfigure}
}

%-----------------------------------------------------
%														\link 
% Url et footnote du lien
%-----------------------------------------------------
  
%  \newcommand{\link}[2]
%  {
%  \href{#1}
%  {
%  #2~\raisebox{-0.2ex}{\footnote
%  	{
% #1
%  	}
%  }
%  }
%  }
  
  
    \newcommand{\link}[2]
  {
  \href{#1}
  {
  #2~\raisebox{-0.2ex}{\faExternalLink}\footnote
  	{
 #1
  	}
  }
  }
  

  %-----------------------------------------------------
%														\Pframe 
% Beamer Frame
%-----------------------------------------------------




%\newenvironment{Pframe}[1]{\mode<all>{\begin{frame}<presentation>\frametitle{#1}}{\end{frame}}}
%
%\newenvironment{Tframe}[1]{{\begin{frame}<presentation>\frametitle{#1}}{\end{frame}}
%
%
%\newenvironment{Aframe}{\begin{frame}}{\end{frame}}
%

%\newcommand{\Aframetitle}[1]
%{
%\mode<presentation>{\frametittle{#1}}
%}
%












