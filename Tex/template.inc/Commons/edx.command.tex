%-------------------------------------------------------------
%               FR CYBERDEF SECOPS COURSE
%                                  EDX PACKAGES
%.    Commons COMMANDS for BOOK, NOTES (.doc) 
%                            and BEAMER (.prz)
%                              2020 eduf@ction
%-------------------------------------------------------------

\newcommand\UKword[1]{\emph{#1}}
\newcommand\textUK[1]{\textit{\textbf{#1}}}
\newcommand\Ucenter[1]{\begin{center}{#1}\end{center}}

\newcommand\head[1]{\textbf{#1}}
\newcommand\tb[1]{\textbf{#1}}
\newcommand{\g}[1]{\og #1 \fg{}}
\newcommand{\uload}[1] {\input{../Tex/#1}}
\newcommand{\ucurcolor}{black}

\newcommand{\mak}[1]{\faicon{\aArrowCircleORight} #1}


%-----------------------------------------------------
%										 	\INFOGitHub 
%---------------------------------------------------

\newcommand{\safeqrcode}[2][]{%
\qrcode[#1]{\detokenize{#2}} }
\newcommand{\GITfilename}{https://github.com/edufaction/CYBERDEF/raw/master/Builder/\jobname.pdf}

\newcommand{\INFOGitHub}
{
\begin{center} 
\setstretch{2}
 {Vérifiez la disponibilité d'une version plus récente de} \\
 {\link{\GITfilename}{\textbf{\jobname.pdf }sur GITHUB CYBERDEF} }   \\ 
{{\huge\ccbyncndeu}}  \\  
{2020 eduf@ction Publication en Creative Common BY-NC-ND }    \\  % Copyright notice
{\safeqrcode[padding]{\GITfilename}}  
\end{center} 
}



%-----------------------------------------------------
%										 	\upspicture 
%---------------------------------------------------
\newcommand{\upspicture}[3]
{
\renewcommand{\ucurcolor}{#3}
\resizebox{#2\textwidth}{!}{\input{#1}}
}


%-----------------------------------------------------
%										 	\uindex 
%---------------------------------------------------


\newcommand{\uindex}[1]
{
#1\index{#1}
}

%-----------------------------------------------------
%										 	\includer 
%---------------------------------------------------

\newcommand{\includer}[1]
{
\input{../Tex/Chapters/#1}
}

%-----------------------------------------------------
%										 	\edxdico 
%---------------------------------------------------

\newcommand{\edxdico}[2] {\textbf{#1} :\textit{ #2} \index{#1}}

%-----------------------------------------------------
%  command DEF : InTextx
%-----------------------------------------------------
\newcommand\InTexta{no text}
\newcommand\InTextb{no text}
\newcommand\InTextc{no text}
\newcommand\rnc{\renewcommand}


%-----------------------------------------------------
%										 	ENV warningbox
%---------------------------------------------------

\newtcolorbox{warningbox}[2][]
{
  colframe = ocre!25,
  colback  = black!05,
  coltitle = ocre!20!black,
  fonttitle = \bfseries, 
  title    = #2,
  #1,
}


\newtcolorbox{techworkbox}[2][]
{
sharp corners, 
  colframe = ocre!25,
  colback  = black!05,
  coltitle = ocre!20!black,
  fonttitle = \bfseries, 
  title    =  {\textcolor{ocre}{\faFile~\small {Fiche TECHNO}}} : #2,
  #1,
}




\newtcolorbox{notebox2}[2][]
{
sharp corners, 
colback = ocre!5!white, 
colframe = ocre!75!black,
fonttitle = \bfseries, 
colbacktitle= ocre!85!black,
title=#2,#1
}

\newtcolorbox{notebox}[2][]{colback=ocre!5!white,
colframe=ocre!75!black,fonttitle=\bfseries,
colbacktitle=ocre!85!black,enhanced,
attach boxed title to top right={yshift=+1mm},
title=#2,#1}


\newtcolorbox{toolsboxin}[2][]{sharp corners, colback=ocre!0!white,
colframe= black!25,fonttitle=\bfseries,
colbacktitle=black!60,enhanced,
attach boxed title to top left={yshift=+1mm},
title=#2,#1}


\newtcolorbox{techworkin}[2][]{sharp corners, colback=ocre!5!white,
colframe= ocre!100, fonttitle=\bfseries,
colbacktitle=black!5, enhanced,
 coltitle=black!50,
attach boxed title to top right ={yshift=+1mm},
title=#2,#1}



%-----------------------------------------------------
%										 	ENV ToolsBox
%---------------------------------------------------


\newcommand\toolseditor{No EDITOR in Datas.tex}
\newcommand\toolsurl{No URL INSERTED in Datas.tex}
\renewcommand\toolclass{No EDITOR in Datas.tex}
\renewcommand\toolname{No NAME in Datas.tex}
\renewcommand\toolanalyst{No EDITOR in Datas.tex}

\newcommand{\toolsbox}[2]
{
\renewcommand\tooleditor{RSA}
\renewcommand\toolurl{https://www.rsa.com/fr-fr/products/threat-detection-response/security-automation-orchestration}
\renewcommand\toolclass{SOAR}
\renewcommand\toolname{Netwitness Orchestrator}
\renewcommand\toolanalyst{eduf@ction} 
\begin{toolsboxin}[sidebyside,righthand width=.7\textwidth]{#1}
\includegraphics[width=0.8\textwidth]{../SecTools/#2.#1/icon.png} \\
\tcblower
Une plateforme de renseignement sur les menaces pour partager, stocker et corréler les indicateurs de compromissions et d'attaques ciblées, des renseignements sur les menaces, des informations sur la fraude financière, des informations sur la vulnérabilité ou même des informations contre le terrorisme.  Pour stocker, partager, collaborer sur les indicateurs de cybersécurité, l'analyse des logiciels malveillants, mais aussi pour utiliser les IoC et les informations pour détecter et prévenir les attaques, les fraudes ou les menaces contre les infrastructures TIC, les organisations ou les personnes
\end{toolsboxin}
\noindent \small {\faCogs~Classe : \uppercase{\textbf{#2}}, Site de référence : \link{\toolsurl}{#1}} \\
\small {\faGlobe~Editeur : \textbf{\toolseditor}}
\small {\faGlobe~\toolsurl}
\small {\faEye~\toolsurl}
}

%\renewcommand\toolsclass{SIEM}
%\renewcommand\toolsname{RSA Netwitness}
%\renewcommand\toolseditor{RSA}
%\renewcommand\toolsurl{https://www.rsa.com/fr-fr/products/threat-detection-response}

%-----------------------------------------------------
%										 	ENV  TechWork
%---------------------------------------------------


\newcommand{\techwork}[2]
{
\begin{techworkin} [righthand width=.8\textwidth]{\faCogs~\small {Fiche TECHNO} : #1}#2
\end{techworkin}
}

%-----------------------------------------------------
%										 	\uchap
%---------------------------------------------------
\ifdefined\INTERTITLE
\newcommand{\uchap}[1]
	{
	\begin{LARGE}
	\textbf{\Ucenter{#1 \\ -oOo-}}	
	\end{LARGE}
	}
\else
	\newcommand{\uchap}[1]{}
\fi

%-----------------------------------------------------
%											\uexpand
%---------------------------------------------------
\newcommand {\uexpand}[1]{
\mode<all>\input{../Tex/Chapters/#1}
}

%-----------------------------------------------------
%											\upicture 
%---------------------------------------------------


\ifdefined\PRZMODE

\newcommand{\upicture}[4]{
\framesubtitle{#2}
%\begin{figure}[!h] %h
  \begin{center}
	 \includegraphics[height=0.95\textheight, width=0.95\textwidth, keepaspectratio ]{#1.pdf}
  \end{center}
%  \end{figure}
}

\else


\newcommand{\upicture}[4]{
\begin{figure}[!h] %h
  \begin{center}
	 \includegraphics[width=#3\textwidth]{#1.pdf}
  \end{center}
\caption{#2 \label{#4}}
\end{figure}
}
	
\fi

%\newcommand{\upicture}[4]{\ulpicture{#1.pdf}{#2}{#3}{#4}}



\ifdefined\PRZMODE

\newcommand{\uwpicture}[4]{
\framesubtitle{#2}
%\begin{figure}[!h] %h
  \begin{center}
	 \includegraphics[height=0.95\textheight, width=0.95\textwidth, keepaspectratio ]{#1.pdf}
  \end{center}
%  \end{figure}
}
\else
\newcommand{\uwpicture}[4]
{
\begin{wrapfigure}{O}{#3\textwidth} % inside / Outside
  \begin{center}
	 \includegraphics[width=#3\textwidth]{#1.pdf}
\end{center}
\caption{#2 \label{#4}}
\end{wrapfigure}
}
\fi


\ifdefined\PRZMODE

\newcommand{\picframe}[4]
{
\begin{frame}
\frametitle<presentation>{#2}
  \begin{center}
	 \includegraphics[height=0.95\textheight, width=0.95\textwidth, keepaspectratio ]{#1.pdf}
  \end{center}
\end{frame}

}
\else
\newcommand{\picframe}[4]
{
\begin{figure}[!hbtp] % !hbtp ou !h 
\centering \includegraphics[width=#3\textwidth]{#1.pdf}
\caption{#2 \label{#4}}
\end{figure}
}
\fi

%{figure}[!h]
%{wrapfigure}{O}{#3\textwidth}
%-----------------------------------------------------
%	FRAME										\wpicframe 
%---------------------------------------------------

\ifdefined\PRZMODE

\newcommand{\wpicframe}[4]
{
\begin{frame}
\frametitle<presentation>{#2}
  \begin{center}
	 \includegraphics[height=0.95\textheight, width=0.95\textwidth, keepaspectratio ]{#1.pdf}
  \end{center}
\end{frame}

}
\else
\newcommand{\wpicframe}[4]
{
\begin{wrapfigure}{O}{#3\textwidth}
\centering \includegraphics[width=#3\textwidth]{#1.pdf}
\caption{#2 \label{#4}}
\end{wrapfigure}
}
\fi

%-----------------------------------------------------
%	FRAME										\texframe 
%---------------------------------------------------
\ifdefined\PRZMODE

\newcommand{\texframe}[3]
{
\begin{frame}
\frametitle<presentation>{#1}
\framesubtitle<presentation>{#2}
#3
\end{frame}
}
\else
\newcommand{\texframe}[3]
{
#3
}

\fi

%-----------------------------------------------------
%											\uref 
%---------------------------------------------------
\newcommand{\uref}[2]{(Voir~#1~\cref{#2} page~\pageref{#2})}


%-----------------------------------------------------
%											\ubg
%---------------------------------------------------
\newcommand{\ubg}[1]{
\textbf{\g{\uppercase{#1}}}
}
%-----------------------------------------------------
%											\updfimage 
%---------------------------------------------------

\newcommand{\updfimage}[3]
{
\begin{wrapfigure}{R}{#3\textwidth}
  \begin{center}
	 \resizebox{#3\textwidth}{!}{\includegraphics{#1.pdf}}
  \end{center}
\caption{#2}
\end{wrapfigure}
}



%************************PACKAGE***

%\usepackage{tikz}% Required for drawing custom shapes
\usetikzlibrary{shadows}


%-----------------------------------------------------
%											 		\rem 
% insertion d'une remarque avec R
%---------------------------------------------------
\newcommand\rem[1]{%
   \marginpar{%
   \tikzpicture[baseline={(title.base)}]
      \node[inner sep=5pt,text width=4cm,drop shadow={shadow yshift=-5pt,shadow xshift=5pt,ocre},fill=white] (box) {\vskip5pt \nointerlineskip #1};
      \node[right=10pt,inner sep=0pt,outer sep=10pt] (title) at (box.north west) {\bfseries\color{ocre}Remarque};
      \draw[draw=ocre,very thick](title.west)--(box.north west)--(box.south west)--(box.south east)--(box.north east)--(title.east);
      \fill[ocre]([yshift=-10pt]box.north west)--+(-5pt,-5pt)--+(0pt,-10pt);
   \endtikzpicture}%
}
%-----------------------------------------------------
%											 \utikzimage 
% insertion d'une image sur fichier tkz.tex
%---------------------------------------------------

\newcommand{\utikzimage}[3]
{
\begin{wrapfigure}{R}{#3\textwidth}
  \begin{center}
	 \resizebox{#3\textwidth}{!}{\input{#1.tkz.tex}}
  \end{center}
\caption{#2}
\end{wrapfigure}
}

%-----------------------------------------------------
%														\link 
% Url et footnote du lien
%-----------------------------------------------------
  
%  \newcommand{\link}[2]
%  {
%  \href{#1}
%  {
%  #2~\raisebox{-0.2ex}{\footnote
%  	{
% #1
%  	}
%  }
%  }
%  }
  
  
  \newcommand{\link}[2]
  {\href{#1}
  {#2~\raisebox{-0.2ex}{\faExternalLink}\footnote{\url{#1}}}
  }
  

  %-----------------------------------------------------
%														\Pframe 
% Beamer Frame
%-----------------------------------------------------




%\newenvironment{Pframe}[1]{\mode<all>{\begin{frame}<presentation>\frametitle{#1}}{\end{frame}}}
%
%\newenvironment{Tframe}[1]{{\begin{frame}<presentation>\frametitle{#1}}{\end{frame}}
%
%
%\newenvironment{Aframe}{\begin{frame}}{\end{frame}}
%

%\newcommand{\Aframetitle}[1]
%{
%\mode<presentation>{\frametittle{#1}}
%}
%












