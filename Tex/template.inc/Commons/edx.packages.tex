%-------------------------------------------------------------
%               FR CYBERDEF SECOPS COURSE
%                                  EDX PACKAGES
%.    Commons packages for BOOK, NOTES (.doc) 
%                            and BEAMER (.prz)
%                              2020 eduf@ction
%-------------------------------------------------------------

\usepackage {url}

\usepackage[utf8]{inputenc}
\usepackage[french]{babel}
\usepackage[T1]{fontenc}

\usepackage{tikz} % Required for drawing custom shapes
\usetikzlibrary{tikzmark} 
\usetikzlibrary{mindmap,trees, backgrounds}
\usepackage{verbatim}
\usepackage{wrapfig}
\usepackage{xpatch}

\usepackage{hologo}

\usepackage[figure]{totalcount}

\usepackage{fontawesome}

\usepackage[skins]{tcolorbox} % don't forget SKINS option

\usepackage{listings}
\usepackage{upquote}

\lstset{
upquote=true,columns=flexible, basicstyle=\ttfamily,
language=HTML, 
frameround=tttt,
commentstyle=\color{gray},
identifierstyle=\color{blue},
keywordstyle=\color{ocre}\bfseries,
xleftmargin=2em,
xrightmargin=2em,
aboveskip=\topsep,
belowskip=\topsep, 
frame=single,
rulecolor=\color{ocre},
backgroundcolor=\color{ocre!5},
breaklines,
breakindent=1.5em,
showspaces=false,
showstringspaces=false,
showtabs=false,
xleftmargin=2em,
xrightmargin=2em,
aboveskip=1em,
belowskip=1em,
}


\usepackage{graphicx} % Required for including pictures
\graphicspath{{Pictures/}} % Specifies the directory where pictures are stored

\usepackage{booktabs} % Required for nicer horizontal rules in tables

\usepackage{setspace}
\setstretch{1,2}

\usepackage{csquotes}

\usepackage{fancyhdr}

\usepackage{makeidx} 

\usepackage{xspace} 

\usepackage{floatflt} 

\usepackage{hyperref}

\hypersetup{
pdftitle={\utitle},
pdfauthor={\uauthor},
%pdfdisplaydoctitle={\utitle}, Attention le titre ne doit pas contenir de saut de ligne
hidelinks,
backref=true,
pagebackref=true,
hyperindex=true,
pdfcenterwindow = true,
colorlinks=false,
breaklinks=true,
urlcolor=ocre,
bookmarks=true,
pdfsubject = {\uCoursetheme}
pdfcreator = {\uauthorwriter TEXDEV}
bookmarksopen=true}


%\pdfstringdefDisableCommands{%
%  \def\\{}%
%  \def\texttt#1{<#1>}%
%}

\usepackage[style=numeric,citestyle=numeric,sorting=nyt,sortcites=true,autopunct=true,hyperref=true,abbreviate=false,backref=true,backend=biber]{biblatex}

\usepackage{cleveref}

\usepackage{multicol}

%--------------------------------------------------------
% pour la génération du QRCODE en page de garde d'article
\usepackage{qrcode} 

%--------------------------------------------------------
% POur la date d'impression en premiere page d'article
\usepackage[useregional]{datetime2}

%--------------------------------------------------------
% Gestion des glossaires et acronymes
% Un fichier acroymes.tex centralisé
\usepackage[acronym]{glossaries}
%\usepackage{glossaries}
\makeglossaries

%\renewcommand{\bibname}{R\'ef\'erences}

%--------------------------------------------------------
% Pour exclure des "environnement", ici techworkbox par exemple, travaux pour étudiants, non affiché dans le BOOK.
\usepackage{comment}


%--------------------------------------------------------
% pour filigramme sur les pages
\usepackage{eso-pic}

