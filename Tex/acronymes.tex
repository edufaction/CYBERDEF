%-------------------------------------------------------------
%               FR CYBERDEF SECOPS COURSE
%
%                 DEFINITION DES ACRONYMES

%                      $File :  acronymes.tex
%                             2020 eduf@ction
%-------------------------------------------------------------
% Require : glossaries package
%-------------------------------------------------------------
%\newacronym{XXX}{XXX}{XXX}
%-----------------------------------------------------------

%  Organismes

\newacronym{aCNIL}{CNIL}{Commission National Informatique et Liberté}
\newacronym{aOTAN}{OTAN}{Organisation du Traité de l'Atlantique Nord}
\newacronym{aANSSI}{ANSSI}{Agence Nationale de la Sécurité des systèmes d'information}
\newacronym{aCERT}{CERT}{Computer Emergency Response Team}
\newacronym{aAFNOR}{AFNOR}{Association Française de Normalisation}
\newacronym{aSIEM}{SIEM}{System Incident and Event Management}
\newacronym{aSIM}{SIM}{System Incident Management}
\newacronym{aSEM}{SEM}{System Event Management}
\newacronym{aIRP}{IRP}{Incident Response Plan, Plan de Réponse à Incident}
\newacronym{aISO}{ISO}{International Organization for Standardization}
\newacronym{aDICT}{DICT}{Disponibilité, Intégrité, Confidentialité, Traçabilité}
\newacronym{aPDCA}{PDCA}{Plan, Do, Check, Act (Roue de Deming)}
\newacronym{aGDPR}{GDPR}{General Data Protection Regulation ou RGPD Règlement général sur la protection des données}

%-----------------------------------------------------------

% ROLES ET METIERS

\newacronym{aOS}{OS}{Officier de sécurité}
\newacronym{aRSSI}{RSSI}{Responsable de la sécurité des systèmes d'information}
\newacronym{aDPO}{DPO}{Data Protection Officier}
\newacronym{aCIL}{CIL}{ Correspondant Informatique et Liberté}
\newacronym{aSSI}{SSI}{Sécurité des Systèmes d’information}
\newacronym{aGRC}{GRC}{Gouvernance Risques et Conformité}
\newacronym{aDTF}{DTF}{Defend The Flag}
\newacronym{aCTF}{CTF}{Capture The Flag}
\newacronym{aOIV}{OIV}{Opérateur d'Infrastructures Vitales}
\newacronym{aOSE}{OSE}{Opérateur de Services Essentiels}
\newacronym{aLPM}{LPM}{Loi de Programmation Militaire}
\newacronym{aEBIOS}{EBIOS}{Expression des Besoins et Identification des Objectifs de Sécurité}
\newacronym{aDLL}{DLL}{Dynamic Loads Library}
\newacronym{aACID}{ACID}{Authentification, Confidentialité, Intégrité, Disponibilité}

%-------------------------------------------------------
% ACRONYMES DATABASE XLS Generation (cf. RESSOURCES DIR)
%-------------------------------------------------------
% DO NOT MODIFY DIRECTLY, USE XLS File (BEGIN)

\newacronym{a2FA}{2FA}{\textit{Two-Factor Authentication} : Authentification à deux facteurs}
\newacronym{aAAA}{AAA}{\textit{Authentication, Authorization, Accounting} : Authentification, autorisation, comptabilisation}
\newacronym{aAC}{AC}{\textit{Access Control} : Accès conditionnel}
\newacronym{aAES}{AES}{\textit{Advanced Encryption Standard} : Algorithme de chiffrement avancé}
\newacronym{aAPT}{APT}{\textit{Advanced Persistent Threat Group} : Groupe de cyber-espionnage avancé}
\newacronym{aASLR}{ASLR}{\textit{Address Space Layout Randomization} : Randomisation de l'espace d'adressage}
\newacronym{aAV}{AV}{\textit{Antivirus Software} : Logiciel antivirus}
\newacronym{aBOT}{BOT}{\textit{Robot (Malicious)} : Robot (malveillant)}
\newacronym{aCC}{C\&C}{\textit{Command And Control} : Commande et contrôle}
\newacronym{aCSRF}{CSRF}{\textit{Cross-Site Request Forgery Attack} : Attaque de confiance de site à site}
\newacronym{aDDOS}{DDOS}{\textit{Distributed Denial Of Service Attack} : Attaque de déni de service distribué}
\newacronym{aDES}{DES}{\textit{Data Encryption Standard} : Algorithme de chiffrement à clé de 64 bits}
\newacronym{aDLP}{DLP}{\textit{Data Loss Prevention} : Gestion de la protection des données sensibles}
\newacronym{aEDR}{EDR}{\textit{Endpoint Detection And Response} : Détection et réponse aux comportements anormaux d'un terminal}
\newacronym{aEK}{EK}{\textit{Exploit Kit} : Kit d'exploit}
\newacronym{aFTP}{FTP}{\textit{File Transfer Protocol} : Protocole de transfert de fichier}
\newacronym{aHTTPS}{HTTPS}{\textit{Protocol For Securing Communications On The Internet} : Protocole de sécurisation HTTP}
\newacronym{aIAM}{IAM}{\textit{Identity And Access Management} : Identité et Gestion des Accés}
\newacronym{aIDS}{IDS}{\textit{Intrusion Detection System} : Système de détection d'intrusion}
\newacronym{aIIS}{IIS}{\textit{Internet Information Services Server}}
\newacronym{aIPS}{IPS}{\textit{Intrusion Prevention System} : Système de prévention d'incident}
\newacronym{aISA}{ISA}{\textit{Internet Security And Acceleration Server} : Serveur de sécurité et d'accélération Internet (Microsoft)}
\newacronym{aLDAP}{LDAP}{\textit{Lightweight Directory Access Protocol} : Ptotocole de gestion d'annuaire léger}
\newacronym{aMD5}{MD5}{\textit{Message-Digest Algorithm 5}}
\newacronym{aMFA}{MFA}{\textit{Multi-Factor Authentication} : Authentification multi-facteurs}
\newacronym{aNAC}{NAC}{\textit{Network Access Control}}
\newacronym{aOAuth}{OAuth}{\textit{Open Authorization Protocol}}
\newacronym{aPAM}{PAM}{\textit{Privilege Access Management}}
\newacronym{aPKI}{PKI}{\textit{Public Key Infrastructure} : Infrastructure de clefs publiques}
\newacronym{aRSA}{RSA}{\textit{Rivest-Shamir-Adleman Encryption Algorithm}}
\newacronym{aSAML}{SAML}{\textit{Security Assertion Markup Language}}
\newacronym{aSMS}{SMS}{\textit{Short Message Service}}
\newacronym{aSQLI}{SQLI}{\textit{Sql Injection} : Injection de script de commande SQL}
\newacronym{aSSH}{SSH}{\textit{Secure Shell Protocol}}
\newacronym{aSSL}{SSL}{\textit{Protocol For Securing Communications On The Internet} : Protocole de sécurisation des communications sur Internet}
\newacronym{aTKIP}{TKIP}{\textit{Temporal Key Integrity Protocol}}
\newacronym{aTLS}{TLS}{\textit{Transport Layer Security}}
\newacronym{aURL}{URL}{\textit{Uniform Resource Locator}}
\newacronym{aVPN}{VPN}{\textit{Virtual Private Network} : Réseau privé virtuel}
\newacronym{aWAF}{WAF}{\textit{Web Application Firewall}}
\newacronym{aWPA}{WPA}{\textit{Wifi Protected Access}}
\newacronym{aXSS}{XSS}{\textit{Cross-Site Scripting Attack}}

% DO NOT MODIFY (END)
%-------------------------------------------------------



