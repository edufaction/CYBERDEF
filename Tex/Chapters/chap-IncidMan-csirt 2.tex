%-------------------------------------------------------------
%               FR CYBERDEF SECOPS COURSE
%                      INCIDENT MANAGEMENT
%
%                                    CERT & CSIRT
%
%                              2020 eduf@ction
%-------------------------------------------------------------

\section{CERT et CSIRT}

Idéalement, les activités de réponse aux incidents sont menées par l' équipe de réponse aux incidents de sécurité informatique de l'organisation (CSIRT), un groupe qui a été précédemment sélectionné pour inclure la sécurité de l'information et le personnel informatique général . L'équipe peut également comprendre des représentants des services juridique, des ressources humaines et de la communication. L'équipe de réponse aux incidents suit le plan de réponse aux incidents (Cyber Incident Plan) de l'organisation, qui est un ensemble d'instructions écrites qui décrivent la réponse de l'organisation aux événements du réseau, aux incidents de sécurité et aux violations confirmées.

\subsubsection{Un peu d'histoire}
Un peu d'histoire, dans les années 80, au coeur du réseau IP les plus célèbre ARPANET, un étudiant de CORNELL UNIVERSITY implanta, sur le réseau, un ver  qui se propageait, se répliquait et exploitait les failles de sécurité UNIX de l’époque. Afin d’exterminer ce ver internet, une équipe d’analyse, en compagnie d’experts MIT, a été créée pour identifier et corriger les failles d’une part, et d’autre part développer des solutions d’éradication. A la suite de cet incident, le DARPA (Defense Advanced Research Projects Agency), maitrise d’ouvrage d’ARPANET, décida la mise en place d’une structure dédiée, le CERT coordination Center, pour résoudre tous types d’incidents sécurité. Le terme CERT est le plus utilisé et le plus connu mais il s’agit d’une marque américaine qui appartient à l’université Carnegie Mellon. Les CSIRT peuvent demander l’autorisation d’utiliser le nom de « CERT ». En 2019, environ 80 CSIRT sont affiliés et autorisés à utiliser cette marque CERT.

Ces CSIRTs peuvent être internes à l'entreprise ou externes de type publique ou commercial.

Dans bien des entreprises, il y a des moments où il est indispensable de faire intervenir des équipes experts externes généralement nécessaires pour faire face à une crise évoluant rapidement. Ces équipes fournissent une assistance d'expertise en fonction de l’étendue et de la gravité de l’incident et de la charge nécessaire à sa remédiation.

Ces équipes de type CSIRT peuvent rapidement apporter les ressources et l’expertise adapté au contexte de l'incident.

 Mais il y a beaucoup à faire par retirer tous les bénéfices de ce type de services. Et cela commence par une compréhension claire de la manière dont fonctionne le processus de réponse aux incidents, et ce que l’on attend d’une équipe externe dans une telle situation.
 
 % Parler de la reconstrucgtion d'AD) 
 
 Un CSIRT est généralement une équipe de sécurité opérationnelle, composée d’experts de différents domaines (malwares, test d’intrusion, veille, lutte contre la cybercriminalité, forensics...). Elle est chargée de prévenir et de réagir en cas d’incidents de sécurité informatique.
\begin{itemize}
  \item En \textbf{prévention}, elle assure notamment une veille sécurité (les nouvelles attaques, les nouveaux logiciels malveillants, les dernières vulnérabilités) pour « connaître » l’état de la menace et évaluer les propres vulnérabilités de son organisation.
  \item En \textbf{réaction}, elle analyse et traite les incidents de sécurité en aidant à leur résolution.
\end{itemize}

Le travail de fond d'un CSIRT est de centraliser la réponse à incident toutefois il sert de relais vers l’intérieur de l’organisation (pour prévenir les menaces en informant et sensibilisant) et surtout vers l’extérieur à destination des autres CSIRT et CERT mondiaux et de la communauté cybersécurité.
 
  \subsubsection{Les missions d’un CSIRT,}
 Les missions d’un CSIRT sont nombreuses, mais il est intéressants de prendre les 5 principales définies par le CERTA (CERT de l'Administration Française).

\begin{itemize}
  \item Centralisation des demandes d’assistance suite aux incidents de sécurité (attaques) sur les réseaux et les systèmes d’informations : réception des demandes, analyse des symptômes et éventuelle corrélation des incidents ;
  \item Traitement des alertes et réaction aux attaques informatiques : analyse technique, échange d’informations avec d’autres CSIRT, contribution à des études techniques spécifiques ;
  \item Etablissement et maintenance d’une base de données des vulnérabilités ;
  \item Prévention par diffusion d’informations sur les précautions à prendre pour minimiser les risques d’incident ou au pire leurs conséquences ;
  \item Coordination éventuelle avec les autres entités (hors du domaine d’action) : centres de compétence réseaux, opérateurs et fournisseurs d’accès à Internet, CSIRT nationaux et internationaux.
\end{itemize}

Il existe plusieurs types de CSIRT :

\begin{itemize}
  \item  Le CSIRT interne (d’entreprise ou d’une administration / université / Etat...) que l'on trouve dans de grandes entreprises bancaires comme par exemple (Le monde bancaire ayant été précurseurs dans l'utilisation de CERT). Le CSIRT a alors un rôle d’«alerte» et de « cyber pompier », prêt à intervenir pour aider et conseiller l’ensemble du groupe, ses filiales voire ses clients en cas d’incident de sécurité.
  \item  Les CSIRT «commerciaux». Ce sont des équipes d’experts appartenant à des prestataires de services qui  proposent des offres de veille, des réponses à incidents à leurs clients, cela peut être  considéré  un CSIRT externalisé et mutualisé.
\end{itemize}
 
 \subsection{Faire intervenir un CSIRT commercial}
 
 \subsubsection{Phase d'analyse}

La première chose à laquelle souhaite accéder un CSIRT est un présentation de situation la plus clair et concise  que possible.

 Les organisations qui sont la cible d’une attaque par logiciel malveillant à plusieurs couches, ou d’une intrusion réseau, n’ont souvent pas une idée complète de l’origine ou de l’étendue du problème. Pourtant, il est vital de pouvoir fournir autant de détails que possible. « Normalement, lorsqu’un client entre en contact avec un spécialiste de la réponse à incident, la première chose que l’on veut savoir, est ce qui est en train de se passer », relève ainsi Bob Shaker, directeur des opérations stratégiques, préparation cyber et réponse, chez Symantec.

Un spécialiste de la réponse à incident va vouloir des informations sur ce qui conduit son client à penser qu’il a été compromis, quand et comment il l’a découvert, et encore s’il l’a fait grâce à une source interne ou externe, des autorités locales, par exemple, ou encore un émetteur de cartes bancaires.

Durant la phase d’établissement de l’étendue du problème, il est vital de disposer de personnes internes à l’organisation sachant quelles informations il est possible de fournir au prestataire, qu’il s’agisse de journaux d’activité ou de tout autre élément de preuve.

Le prestataire va utiliser l’information qui lui est fournie pour évaluer l’étendue des dégâts et déterminer le type de ressources , y compris les experts à dépêcher sur place.

\subsubsection{La phase de contractualisation}

Une fois que le prestataire a eu l’opportunité d’évaluer la situation, il pourra fournir une estimation de ce dont il aura besoin pour son intervention. Le contrat proposer doit généralement contenir des explications détaillées des services qui seront apportés, précisant au passage s’il aidera effectivement à remédier à l’incident, ou s’il ne fera que fournir les informations permettant à son client d’assurer seul la remédiation.

Dans cette phase, il est important de bien comprendre quelle documentation, accès et savoirs seront nécessaires au prestataire.  Les entreprises utilisant des applications et des services en mode Cloud sont parfois limitées dans le choix des tiers d’investigation qu’elles peuvent solliciter. Il est donc important d’examiner ces points avant de signer le contrat.

En outre, il est vital d’identifier les compétences que peut apporter le prestataire, ainsi que ses ressources technologiques, ses outils ou encore ses renseignements sur les menaces.

Signer pour un engagement de longue durée avec un prestataire, avant le premier incident, peut s’avérer profitable : ainsi, il n’est pas nécessaire de consacrer un temps critique en plein incident aux détails du processus de contractualisation, ou d’expliquer ses processus internes de réponse aux incident au milieu d’une crise. 

\subsubsection{Enquêter sur l’incident}

Le CSIRT aura besoin de toute l’information possible : logs systèmes et réseau, diagrammes de topologie réseau, images systèmes, rapports d’analyse du trafic réseau, etc.

Souvent, il est tentant de céder à la panique et d’arrêter les systèmes dans la précipitation. Mais pour Shaker, c’est une mauvaise idée :  la première chose importante est de ne pas éteindre les systèmes. Une fois qu’un système est éteint, une quantité considérable d’éléments de preuve peuvent être effacés, en particulier tout ce qui réside en mémoire vive.

Les équipes d’investigation utilisent les informations fournies par les entreprises clients, ainsi que celles qu’elles collectent elles-mêmes sur leurs points de terminaison et d’autres sources, via des outils propriétaires, pour identifier des indicateurs de compromission.

C’est après cela que le prestataire est généralement en mesure d’informer son client sur ce qui s’est passé, sur la manière dont l’intrusion est susceptible d’avoir commencé, ou comment le logiciel malveillant a été introduit sur le réseau, et sur quoi faire pour contenir l’incident.

\subsubsection{Contrôle et remédiation}

L’équipe responsable de la remédiation travaille souvent en tandem avec l’équipe chargée de l’investigation, selon Aldridge. « Nous avons deux flux de travail. Le premier est lié à l’enquête et vise à identifier quels systèmes, comptes et données ont été compromis ; le second touche à la remédiation ». Et dès que la première équipe trouve des éléments relatifs à l’incident, elle les transmet à la seconde qui travaille avec le client à la mise en œuvre des mesures correctrices.


\subsection{Création de son équipe CSIRT}

Quelles sont les motivations pour créer un CSIRT dans son entreprise : 

\begin{itemize}
  \item Une augmentation exponentielle du nombre d’incidents sécurité;
  \item Une augmentation du nombre et type d’organisations affectées par des incidents sécurité;
  \item Un focus de la part des entreprises sur le besoin de politiques sécurité dans le cadre de leur management du risque
  \item Nouvelles lois et régulations impactant les entreprises en terme de protection des données;
  \item Réaliser que les administrateurs systèmes et réseaux ne peuvent pas protéger l’entreprise à eux seuls.
\end{itemize}

Un CSIRT est composé de plusieurs experts dans différents domaines de la sécurité (intrusions, forensics, malwares, crytpo, etc..) qui préviennent mais surtout réagissent en cas d’incident. Ces experts sont en constante mise à jour des nouveaux vecteurs d’attaques (nouveaux malwares, nouvelles vulnérabilités), tout ceci afin de traiter les incidents de la manière la plus aboutie qui soit. Une véritable équipe CSIRT dans une entreprise à un coût non négligeable, il convient d'en étudier les modalités de de fonctionnement et de couverture.


\begin{techworkbox}{CSIRT}
La création d'un CSIRT dans son entreprise, n'est pas chose facile, c'est un vrai sujet de RSSI dans le sens ou des choix sont a faire tant sur les compétences, les moyens, les procédures de travail. C'est un vrai sujet de mémoire pour \fichetech.
\end{techworkbox}
