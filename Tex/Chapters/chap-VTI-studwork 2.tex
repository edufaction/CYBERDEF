%-------------------------------------------------------------
%               FR CYBERDEF SECOPS COURSE
%                            
%                           TRAVAUX ETUDIANTS
%
%                              2020 eduf@ction
%-------------------------------------------------------------

\section{Travaux personnels}

\subsection{généralités}


\mode<presentation>{\texframe{Fiche TECHNO}{Généralités}
{
\centering\Huge{Le travail demandé}
}} 


% Begin PRZ ===========================
\begin{frame}
\frametitle<presentation>{Votre travail}
Dans le cadre de ce cours, un seul travail  est demandé. C'est un travail personnel, dont l'objectif est de vous faire travailler sur un sujet que vous souhaitez étudier dans le but de le présenter aux autres. Vous pouvez donc choisir un sujet que vous maitrisez ou un sujet que vous ferez découvrir avec un regard de béotien.
Ce travail se concrétise par un document à remettre dénommé : \textbf{FICHE TECHNO} .
\end{frame}
% end PRZ ========================

En résumé votre travail devra être : 

% Begin PRZ ===========================
\begin{frame}
\frametitle<presentation>{Votre travail}
% end header PRZ =======================
\begin{itemize}
  \item  1 document de moins 30 pages (Conseil de 10 à 15 pages)
  \item  Sur un produit, un concept, une méthodologie du monde de la Sécurité Opérationnelle (Vulnérabilités, menaces, incidents, crises, attaques ...)
  \item  Un travail de votre expérience, ou simplement sur une recherche sur internet pour un produit à choisir … 
\end{itemize}


\end{frame}
% end PRZ ========================

Votre  analyse sera étayée et critique sur un élément de la sécurité opérationnelle. La notion d'élément SECOPS regroupe de nombreuses thématiques :
% Begin PRZ ===========================
\begin{frame}
\frametitle<presentation>{Thèmes}
% end header PRZ =======================
\begin{itemize}
  \item Méthodologique;
  \item Technologique ou technique;
  \item Conceptuel;
  \item Juridique...
\end{itemize}

\end{frame}
% end PRZ ========================

Votre rédaction doit faire apparaitre les sources, vous devez surtout développer votre propre vision ou retour d'expérience.
Sur ces thématiques, il est important que votre sujet de FICHE TECHNO reste dans le domaine de la sécurité opérationnelle :

% Begin PRZ ===========================
\begin{frame}
\frametitle<presentation>{Domaines}
% end header PRZ =======================
\begin{itemize}
  \item \textbf{VEILLE/AUDIT} : Des produits/services de veille et de scan de vulnérabilités informatiques (Qualys, nessus, nmap, checkmarx, appscan … et bien d’autres …) 
  \item \textbf{SURVEILLE/ALERTE} Des produits/services de gestion d’événement, de supervision et d’alerte (Log, SIEM : Qradar, ArcSight, LogPoint, splunk … et bien d’autres …)
  \item \textbf{ANALYSE/REPONSE} : Des produits/services d’analyse post-mortem, et de forensique (Forensic Toolkit, encase … et bien d’autres …)
\end{itemize}

\end{frame}
% end PRZ ========================

Votre travail est à rendre en fin de session, sous forme informatique (OpenDoc, Formats Microsoft, PDF, Latex...)

\subsection{Méthode de notation}

\mode<presentation>{\texframe{Fiche TECHNO}{Généralités}
{
\centering\Huge{l'évaluation et la notation}
}} 

Votre travail est noté sur différents critères ci-dessous.

Chaque critère est évalué suivant les valeurs suivantes

% Begin PRZ ===========================
\begin{frame}
\frametitle<presentation>{Critères}
% end header PRZ =======================

\begin{itemize}
  \item Qualité du positionnement du problème ou du sujet
  \item Qualité de la conclusion, dont l'ouverture vers d'autres points
  \item Présence et affichage de votre point de vue : Apports personnels (apports liés à votre propre expérience, ou aux découvertes  faites lors de la rédaction de ce travail)
\end{itemize}

\end{frame}
% end PRZ ========================
Les valeurs d'évaluation de ces critères sont :

% Begin PRZ ===========================
\begin{frame}
\frametitle<presentation>{Evaluation}
% end header PRZ =======================

\begin{itemize}
  \item 0 - Travaux trop simpliste et sans valeur d'apport personnel;
  \item 1 - Travaux simples ou sans apport personnel;
  \item 2 - Apport étayé et présentation claire;
  \item 4 - Apport didactique;
  \item 5 - Apport personnel étayé.
\end{itemize}
\end{frame}
% end PRZ ========================

\subsection{Format}

Si le format n'est pas imposé, il est demandé toutefois de suivre un plan permettant de suivre votre démarche et permettant d'être le plus pédagogique possible.
Vous pouvez utiliser le modèle de document mis à votre disposition.

\begin{itemize}
  \item WORD : SEC101-Part3-Modele-Fiche-Techno-VxRy
  \item Latex :  MemModel sur GITHUB (Cyberdef101)
\end{itemize}

\subsection{Remise de Fiches}

Vous devez remettre vos documents via l'outils de dépôts et d'analyse de plagiat du CNAM, via le site \link{https://interface.compilatio.net/dossier/q8anf}{COMPILATIO.NET}
Lors du dépôt, le système analysera les similitudes avec des sources ouvertes. Je vous engage donc à citer vos sources.

\section{Sujets}

\mode<presentation>{\texframe{Fiche TECHNO}{Généralités}
{
\centering\Huge{Le sujet et le contenu de vos travaux}
}} 

\subsection{Sélection des sujets}

% Begin PRZ ===========================
\begin{frame}
\frametitle<presentation>{Travaux à valider}
Avant de vous lancer dans vos travaux, il est demander de faire valider votre sujet par l'enseignant. Pour cela simplement envoyer un mail avec votre sujet et vos justificatif de choix.

Vous trouverez ci après quelques différentes thématiques avec des idées de sujet. Chaque sujet est constitué d'un thème, et d'un descriptif optionnel.
Ces sujets sont donnés à titre indicatif. Il vous revient d'en proposer un si aucun de ceux présentés vous intéressent.

Votre travail est \textbf{à rendre} en fin de session

\end{frame}
% end PRZ ========================


\subsection{Thématique produits et services}

Pour les fiches sur les produits et services,  vous devez livrer votre FICHE TECHNO, avec un petit dossier  supplémentaire (Fichier ZIP au nom du produit contenant)  quelques lignes et contenant l'icône du produit nommé \textbf{icon.png} . Les deux petit fichiers textes  :

\begin{itemize}

  \item  \textbf{description.tex } : fichier contenant un description succincte d'un maximum de 5000 caractères  du produit. (Les commandes simples latex sont autorisés dans ce document (mise en forme, tableau, itemize).
  \item  \textbf{datas.tex} : Fichier contenant quelques éléments descriptifs du produit ou système. Les éléments descriptifs sont à saisir :
\begin{itemize}
 \item \verb|\toolsclass{classe de l'outil}(Facultatif)|
 \item \verb|\toolsname{Nom du produit ou de l'outil}(Facultatif : nom du  répertoire ou ZIP = produit)|
 \item \verb|\toolseditor{Nom de l'éditeur}|
\item \verb|\toolsurl{Url de l'outil}|
\item \ \verb|\toolversion{Version du produit}(Facultatif)|
\end{itemize}
\end{itemize}

Le fichier  exemple (téléchargeable)  \link{https://github.com/edufaction/CYBERDEF/raw/master/SecTools/classe.tool.example/siem.Qradar.zip}{\textbf{Qradar.zip}} doit donc contenir au moins pour le produit \g{Qradar} de la classe \g{SIEM} : 

\begin{lstlisting} [language={[LaTeX]TeX},morekeywords={renewcommand,toolseditor,toolsurl}]
\renewcommand\toolseditor{IBM}
\renewcommand\toolsurl{https://www.ibm.com/fr-fr/security/security-intelligence/qradar}
\end{lstlisting}

Ces fichiers de présentation seront mis en forme dans les supports de cours et d'exemple de produits sous la forme :

\toolsbox {Qradar}{siem}

Ces éléments permettent de présenter les produits de manière plus accessible sur le site du CNAM et dans les illustrations du cours

\subsection{Les thématiques des fiches TECHNO}

Les thématiques des fiches TECHNO et METHODO, dénommées par  simplification fiches TECHNO sont classées en deux catégories :

% FRAME beamer PRZ ------------------------------------
\mode<all>{\texframe{Les thèmes}{produits et concepts}
%. . . . . . . . . . . . . . . . . . . . . . . . . . . . . . . . . . . . . . . . . . . .
{
\begin{itemize}
  \item Les\textbf{ fiches produits, outils et services }pour chacune des trois thématiques présentées dans le cours, et pour lequel l'outil peut illustrer l'usage d'une technologie ou d'un service  de cyberdéfense;
  \item Les fiches thématiques  \textbf{méthodologies,  et concepts } dont le sujet peut être choisis dans la liste fournie.
\end{itemize}
}} % end FRAME.........................................................


.

%*********************************************
\subsection{Thématique : Vulnerability Management}
%----------------------------------------------------
\subsubsection{Exemples étayés de vulnérabilités}
\subsubsection{BugBounty}
\subsubsection{Outils au service des tests d'intrusion}
\subsubsection{Essentiels 27001 et vulnérabilités}


%*********************************************
\subsection{Thématique : Threat Management}
%----------------------------------------------------
\subsubsection{Description d'une attaque virale}
\subsubsection{Architecture d'un BOTNET}
\subsubsection{Organisation des bugbounty}
\subsubsection{Description attaque DDOS}
\subsubsection{Description du fonctionnement d'un ransomware}
\subsubsection{Technique de recherche de LEAK dans le darkweb}

\subsubsection{Sondes de sécurité}

%*********************************************
\subsection{Thématique : Exemples attaques et traitement}
%----------------------------------------------------
Description d'une attaque dans le monde réel avec les mécanismes, stratégies de l'attaquant, le retour d'expérience de l'attaqué et le traitement par les médias.
%*********************************************
\subsection{Thématique : Incident Management}
%----------------------------------------------------
\subsubsection{Description d'une contre attaque DDOS}
\subsubsection{Description d'une  contre attaque de ransomware} 
\subsubsection{Description d'une recherche d'APT}
\subsubsection{Essentiels 27035}
\subsubsection{Stratégie d'enquête avec des HoneyPots}
\subsubsection{Deceptive defense}
\subsubsection{Essentiels 27001 et incidents}
\subsubsection{ITIL et gestion des incidents de sécurité}

%*********************************************
\subsection{Thématique : Crisis Management}
%----------------------------------------------------
\subsubsection{Essentiels ISO 22301}
\subsubsection{Annuaire de crise}
\subsubsection{Comment Gérer une crise ransomware }

%*********************************************
\subsection{Thématique : Gouvernance CyberDef}
%----------------------------------------------------
\subsubsection{Architecture d'un SOC}
\subsubsection{Tableau de Bord Vulnerability Management}
\subsubsection{Tableau de Bord Incident Management}
\subsubsection{Tableau de Bord SIEM et SIC}


%*********************************************
\subsection{Thématique : Stratégies CyberDef}
%----------------------------------------------------
\subsubsection{Concepts de Deceptive cyberdefense}
\subsubsection{Utilisation des Honeypots dans la réaction }


%*********************************************
\subsection{Thématiques Transverses}
%----------------------------------------------------
Pour faire évoluer et être au. plus prêt de l'actualité
\subsubsection{Bibliographie et book de références par chapitre}
\subsubsection{Compilation avec abstract d'éléments de lectures}

