%********************************************
% $Chapitre : Threat Management
% $theme : Threat Intelligence
% $File : chap-Vulman-threatintel.tex
%
%********************************************
\uchap{chap-Vulman-threatintel.tex}
%********************************************

\section{Threat Intelligence}

\subsection{Surveiller et anticiper}

% Begin PRZ ===========================
\begin{frame}
\frametitle<presentation>{Threat Intelligence}
% end header PRZ =======================
La surveillance et le renseignement de la menace au sens général du terme (Threat Intelligence) devrait contenir les 2 niveaux :

\begin{itemize}
  \item Le renseignement à vocation cyber qui comprend toutes les analyses et information permettant d'anticiper et de caractériser une menace qui pourrait s'exprimer dans le monde numérique,
  \item Le renseignement d'origine Cyber, dont les données techniques liées à des attaques, menaces qui permettent de configurer des systèmes de détection et de réponse.
\end{itemize}

\end{frame}
% end PRZ ===========================

Il est vrai qu'encore aujourd'hui parler de \g{threat intelligence} nous dirige systématiquement sur la deuxième assertion


% Begin PRZ ===========================
\begin{frame}
\frametitle<presentation>{Veille}
% end header PRZ =======================
\upicture{../Tex/Pictures/img-veillecyber}{Veille cyber, une veille sur les risques}{1}{lbl-veillecyber}
\end{frame}
% end PRZ ===========================




% Begin PRZ ===========================
\begin{frame}
\frametitle<presentation>{Veiller et surveiller}
\framesubtitle<presentation>{2 axes}
% end header PRZ =======================

Veiller et surveiller les menaces, détecter les attaques  nécessite d'analyser  deux axes : 

\begin{itemize}
  \item Les menaces génériques, ou ciblant un domaine  particulier (Santé, Industrie, Banque ...) que l'on trouve généralement en utilisant des technologies de \g{threat Intelligence} permettant 
  \item Les menaces ciblées, dont les indices d'émergence peuvent être détecter en analysant la menace ou  en recherchant des indices de compromissions quand ces menaces sont actives dans le périmètre de l'entreprise. \g{threat Detection, Hunting ...} 
\end{itemize}
\end{frame}
% end PRZ ===========================



% Begin PRZ ===========================
\begin{frame}
\frametitle<presentation>{Veiller et surveiller}
\framesubtitle<presentation>{2 manières}
% end header PRZ =======================

et ceci de deux manières :

\begin{itemize}
  \item Surveillance de l'écosystème de la menace (IOC,  DarkWeb, Threat Intelligence...) 
  \item Recherche de compromission, ou d'infection (Threat Hunting, ...) 
\end{itemize}
\end{frame}
% end PRZ ===========================

Ce sont des sujets que nous aborderons dans le processus de gestion de la menace.
La surveillance des menaces génériques relève d'action de veille comme cela est fait pour les vulnérabilité. Les scénarios de message sont vu comme des éléments de signature d'une attaque ou d'une tentative ou de préparation  d'attaque.

Je vous propose de présenter  la gestion de la menace sous la forme de 3 thèmes  (\uref{Gestion de la menace}{lbl-logman}).

\begin{itemize}
  \item Log Management
  \item Threat Intelligence (au sens renseignement)
  \item Threat Detection
\end{itemize}


% Begin PRZ ===========================
\begin{frame}
\frametitle<presentation>{Les sources}
% end header PRZ =======================
\upicture{../Tex/Pictures/img-threatvigil}{Les sources}{1}{lbl-threatvigil}
\end{frame}
% end PRZ ===========================

%TODO Tester les SIEM et les SOC et le deuxième niveau de maturité

%TODO Niveau de maturité dans les processus

% TODO le renseignement


%*************************************
\subsubsection{Surveillance de la compromission}

Un des domaine de la surveillance est donc celui de la compromission. C'est à dire la surveillance dans le fameux Darkweb de l'emergence de données volées, "\g{perdues}" par une entreprise ou par un particulier.

% Mettre un exemple : adresse mail compromisse (avec le site PWND)


%*************************************
\subsubsection{Surveillance du ciblage}

% Threat Hunting 

La surveillance du ciblage, que les anglo-saxons appelle le TARGETING est aussi un élément d'anticipation. 
En effet, ces éléments sont souvent les premiers signaux d'un préparation d'un évènement "\g{cyber}" qui pourrait toucher l'entreprise.

On y trouve l'émergence de la collecte d'information sur une cible donnée. La mise en oeuvre dans les code malveillant de targetting d'IP spécifique, etc...

% Begin PRZ ===========================
\begin{frame}
\frametitle<presentation>{Surveillance du ciblage}
\framesubtitle<presentation>{l'outillage du "targetting"}
% end header PRZ =======================
Il y a deux types d'outils pour ce se faire : 

\begin{itemize}
	\item La surveillance classique du web de type \g{cyberveille}, qui permet de découvrir des éléments appartenant à l'entreprise compromis (soient les données, soient des informations permettant de déduire que l'entreprise a été compromise).
	\item L'analyse en temps réel des codes malveillants qui peut permettre en regardant de manière détaillée l'évolution du code pour comprendre et connaitre les modalités des attaques et les nouvelles cibles.
\end{itemize}
\end{frame}
% end PRZ ===========================


%*************************************
\subsubsection{Que faire des ces informations}

Disposer des fragilités de l'entreprise, et connaitre les scénarios potentiels permets d'évaluer un niveau de risque. 







