%-------------------------------------------------------------
%               FR CYBERDEF SECOPS COURSE
%              $Chapitre : Threat Management
%                         $theme : Threat Intelligence
%        $File :  chap-ThreatMan-threatintel.tex
%                             2020 eduf@ction
%-------------------------------------------------------------
\uchap{chap-Vulman-threatintel.tex}
%-------------------------------------------------------------

\section{ANTICIPER  les menaces}

\subsection{Surveiller et anticiper : Cyber-Threat Intelligence (CTI)}

Une menace informatique ou cyber se matérialise par la combinaison de trois facteurs : 

\begin{itemize}
  \item une intention de nuire, 
  \item une capacité d’attaque et 
  \item une opportunité à exploiter, c’est-à-dire une vulnérabilité de nature technique ou humaine. 
\end{itemize}


Les motivations sont différentes et susceptibles de cibler tous types d’organisations : des hacktivistes poussés par une idéologie, les cybercriminels motivés par l’appât du gain ou  des groupes sponsorisés par un État. On peux se référer à des \link{https://attack.mitre.org/groups/}{origines d'APT (Mitre)} avec les groupes à l'origine de ces menaces.



La «threat intelligence» est un service de renseignement concernant les cyber-menaces. Les solutions SIEM par exemple, possèdent leurs propres sources, il n'y a pas que les SIEM qui peuvent utiliser ces sources,  il est possible de connecter de nombreux outils de détections à d’autres sources externes de «threat intelligence». De manière non exhaustive un service de «threat intelligence»  fournit des éléments comme :
\begin{itemize}
  \item \textUK{Malicious / Phishing IP / URL} : une liste d’URL utilisée pour  délivrer un fichier malicieux ou procéder à une attaque par hameçonnage;
  \item 	\textUK{Botnet C\&C URL}» : une liste d’URL utilisée pour héberger des serveurs de commande et contrôle de logiciel malveillant ou de réseaux de machine zombie;
  \item \textUK{Malicious Hash} : liste d’empreintes de logiciels malveillants connus et ayant déjà été analysés;
  \item \textUK{IP Reputation} : liste d’adresses IP suspectées dans des attaques informatiques ou en lien avec une cyber-menace (pouvant être utilisée en Black List sur les systèmes de filtrage, comme historiquement).
\end{itemize}


La Cyber Threat Intelligence est une activité possédant un double objectif  :  l’étude et la surveillance de ces cybermenaces. Pour ce faire, la majorité des fournisseurs  de flux d’information de type CTI se basent sur deux approches :


\begin{itemize}
  \item La première approche est l’analyse des attaques passées qui permet de caractériser ces dernières par ces marqueurs techniques. Le but pour le défenseur est de se prémunir et de bloquer au plus vite une campagne qui se renouvellerait via l’utilisation de ces marqueurs techniques. (Voir MISP)
  \item La deuxième approche est la surveillance directe des attaquants. L’objectif est de se placer en amont d’une attaque et de détecter des éléments permettant d’identifier sa préparation (en sources ouverts, ou en infiltration dans le darkweb) : émergence de signaux faibles, objectifs , mode opératoire,  capacités, organisation de l'attaque. Cette approche nécessite un dynamique d’anticipation face à ces menaces,
\end{itemize}


% Begin PRZ ===========================
\begin{frame}
\frametitle<presentation>{Threat Intelligence}
% end header PRZ =======================
La surveillance et le renseignement de la menace au sens général du terme (Threat Intelligence) devrait contenir les 2 niveaux :

\begin{itemize}
  \item Le renseignement à \textbf{vocation} cyber qui comprend toutes les analyses et information permettant d'anticiper et de caractériser une menace qui pourrait s'exprimer dans le monde numérique;
  \item Le renseignement \textbf{d'origine} Cyber, dont les données techniques liées à des attaques, menaces qui permettent de configurer des systèmes de détection et de réponse.
\end{itemize}
\end{frame}

% end PRZ ===========================

Il est vrai qu'encore aujourd'hui parler de \g{threat intelligence} nous dirige systématiquement sur la deuxième assertion.


\mode<all>{\picframe{../Tex/Pictures/img-veillecyber}{Veille cyber, une veille sur les risques}{0.7}{lbl-veillecyber}}

% Begin PRZ ===========================
\begin{frame}
\frametitle<presentation>{Veiller et surveiller}
\framesubtitle<presentation>{2 axes}
% end header PRZ =======================

Veiller et surveiller les menaces, détecter les attaques  nécessite d'analyser deux axes : 

\begin{itemize}
  \item Les menaces génériques, ou ciblant un domaine  particulier (Santé, Industrie, Banque ...) que l'on trouve généralement en utilisant des technologies de \g{threat Intelligence}; 
  \item Les menaces ciblées, dont les indices d'émergence peuvent être détecter en analysant la menace ou  en recherchant des indices de compromissions quand ces menaces sont actives dans le périmètre de l'entreprise. \g{threat Detection, Hunting ...} 
\end{itemize}
\end{frame}

% end PRZ ===========================



% Begin PRZ ===========================
\begin{frame}
\frametitle<presentation>{Veiller et surveiller}
\framesubtitle<presentation>{2 manières}
% end header PRZ =======================

et ceci de deux manières :

\begin{itemize}
  \item Surveillance de l'écosystème de la menace (IOC,  DarkWeb, Threat Intelligence...) 
  \item Recherche de compromission, ou d'infection (Threat Hunting, ...) 
\end{itemize}
\end{frame}
% end PRZ ===========================

Ce sont des sujets que nous aborderons dans le processus de gestion de la menace.
La surveillance des menaces génériques relève d'action de veille comme cela est fait pour les vulnérabilités. Les scénarios de message sont vus comme des éléments de signature d'une attaque ou d'une tentative ou de préparation  d'attaque.

Je vous propose de présenter  la gestion de la menace sous la forme de ces trois thèmes  (\uref{Gestion de la menace}{lbl-logman}).

\begin{itemize}
  \item Log Management;
  \item Threat Detection;
 \item  et en dernier lieu des élémens Threat Intelligence (au sens renseignement);
\end{itemize}

\mode<all>{\picframe{../Tex/Pictures/img-threatvigil}{Les sources}{0.7}{lbl-threatvigil}}

\subsubsection{Quelques producteurs de CTI}
\toolsbox{FireEye}{cti}

\subsubsection{Quelques outils de CTI}
\toolsbox{MISP}{cti}
\toolsbox{OpenCTI}{cti}

%TODO Tester les SIEM et les SOC et le deuxième niveau de maturité

%TODO Niveau de maturité dans les processus

% TODO le renseignement


%*************************************

\subsection{Les surveillances} 


\subsubsection{Surveillance de la compromission}

Un des domaines de la surveillance est donc celui de la compromission. C'est à dire la surveillance dans le fameux Darkweb de l'émergence de données volées, \g{perdues} par une entreprise ou par un particulier.

Ce domaine, dénommé par certains \g{Leak Intelligence} ou \g{Leak Management}, correspond à la gestion des fuites de données, au sens de leur détection et la recherche de \g{la source} de fuite. C'est souvent dans cette dynamique de découverte d'information \g{interne} dans des espaces \g{malveillants} ou pas que l'on découvre des compromissions techniques ou non techniques issues d'attaques.

La compromission la plus connue reste encore de nos jours  la fuite du couple Utilisateur/mot de passe sur des sites hackés. Il existe des bases de données publiques qui publient ces données \g{compromises} comme par exemple : \link{https://haveibeenpwned.com}{Have i been pwned}, qui permettent à partir d'une adresse mail ayant servie d'identifiant de savoir si on a été compromis sur un site qui aurait été piraté.

\subsubsection{Surveillance des fragilités}

Comme nous l'avons vu dans le chapitre sur les vulnérabilités,  des \g{scans} de vulnérabilités sur des plages d'IP de l'entreprise permet de déterminer les fragilités de services ouverts ou accessibles. Généralement organisées dans une dynamique d'audit, ces évaluations de sécurité sont conduites avec un cadre contractuel et légal. Il existe pourtant des entreprises qui fournissent des informations de fragilités sur des entreprises ou des plages d'IP. On trouvera pas exemple sur le site de \link{https://www.shodan.io}{SHODAN} des informations intéressantes (et payantes) sur des fragilités de systèmes dont une grande partie de systèmes d'objets connectés.

\begin{techworkbox}{Base de données de fragilitées publiques}
Les techniques et les sites qui propose des outils ou l'accès à des bases de données de \g{sites} vulnérables. C'est un sujet intéressant pour \fichetech de synthèse sur ce qui existe sur le marché. %sujets : https://alternativeto.net/software/shodan/
\end{techworkbox}


\toolsbox{Oval}{vmt}
\toolsbox{OpenSCAP}{vmt}


Les approches des analyses du renseignement peuvent se classer à différents niveaux :

\begin{itemize}
  \item \textbf{Niveau Stratégique} : ce sont en général des analyses de très haut niveau mais peu techniques et destinées à des décideurs. Cela peut-être par exemple des rapports d'analyses géopolitiques sur des adversaires qui ciblent un secteur donné.
  \item \textbf{Niveau Tactique} : ce sont souvent des documents qui donnent des informations sur les outils et méthodologies utilisés par les menaces (analyse de malware, contournement d’anti-virus, outils utilisés pour mener des attaques DDoS, etc.).
  \item \textbf{Niveau Opérationnel} : l’objectif est d’anticiper l'attaque en étant au plus près de l’attaquant (Ecoute, infiltration  des hacktivistes, cybercriminels ...).
  \item \textbf{Niveau Technique} : il est composé d’indicateurs de compromission (IPs, URLs, noms de domaine, listes de hashs, etc.) qui permettent d’identifier et donc de bloquer directement une attaque.
\end{itemize}


%*************************************
\subsubsection{Surveillance du ciblage}

% Threat Hunting 

La surveillance du ciblage, que les anglo-saxons appelle le \g{TARGETING} est aussi un élément d'anticipation. 
En effet, ces éléments sont souvent les premiers signaux d'un préparation d'un évènement \g{cyber} qui pourrait toucher l'entreprise.

On y trouve l'émergence de la collecte d'information sur une cible donnée. La mise en oeuvre dans les codes  malveillants de ciblage d'IP spécifique, etc...

% Begin PRZ ===========================
\begin{frame}
\frametitle<presentation>{Surveillance du ciblage}
\framesubtitle<presentation>{l'outillage du \g{targetting}}
% end header PRZ =======================
Il y a deux types d'outils pour ce se faire : 

\begin{itemize}
	\item La surveillance classique du web de type \g{cyberveille}, qui permet de découvrir des éléments compromis appartenant à l'entreprise (soient les données, soient des informations permettant de déduire que l'entreprise a été compromise).
	\item L'analyse en temps réel des codes malveillants qui peut permettre en regardant de manière détaillée l'évolution du code pour comprendre et connaitre les modalités des attaques et les nouvelles cibles.
\end{itemize}
\end{frame}
% end PRZ ===========================


%*************************************
\subsubsection{Que faire des ces informations}

Disposer des fragilités de l'entreprise, et connaitre les scénarios potentiels permets d'évaluer un niveau de risque. 

\subsection{de l'outillage sur la menace}

\subsubsection{la gestion des menaces CTI}

Le projet OpenCTI (Open Cyber Threat Intelligence), développé par l’ANSSI en partenariat avec le CERT-EU, est un outil de gestion et de partage de la connaissance en matière d’analyse de la cybermenace (Threat Intelligence). Initialement conçue pour structurer les informations de l’agence relatives à la menace informatique, la plateforme facilite aussi les interactions entre l’ANSSI et ses partenaires.

L’outil, intégralement libre, est  disponible à l’usage de l’ensemble des acteurs de la « threat intelligence ». L’application permet ainsi de stocker, organiser, visualiser et partager leurs propres connaissances en la matière. L'outil est structuré autour de des méodèle de description STIX 2 et utilise un base hypergraphe.

Le projet \link{https://www.ssi.gouv.fr/actualite/opencti-la-solution-libre-pour-traiter-et-partager-la-connaissance-de-la-cybermenace/}{OpenCTI} a été initié en septembre 2018 par l’ANSSI et co-développé avec le CERT-EU en l’absence de solutions complètement appropriées pour structurer, stocker, organiser, visualiser et partager la connaissance de l’ANSSI en matière de cybermenace, à tous les niveaux.

OpenCTI peut être vu comme un premier outil au centre d'un \g{fusion center}.

\toolsbox{OpenCTI}{cti}


\subsubsection{STIX et TAXII}

%The structuration of the data is performed using a knowledge schema based on the STIX2 standards. It has been designed as a modern web application including a GraphQL API and an UX oriented frontend. Also, OpenCTI can be integrated with other tools and applications such as MISP, TheHive, MITRE ATT&CK, etc.

Les \textbf{modélisations} des attaques est un large champs de recherche et d'outillage, elles sont au coeur de la compréhension des attaques mais surtout au coeur de la détection de celle-ci. Juste à titre d'illustration, nous pouvons parler d'un modèle comme STIX™ (Structured Threat Information Expression) langage et format de donnée permettant de modéliser et échanger des informations techniques sur les processus d'attaque cyber. 
Je vous propose d'explorer pour cela sur le site \link{https://oasis-open.github.io/cti-documentation/}{STIX sur GitHub}.

Le premier type de standard à avoir été proposé est le format de modélisation des informations. L’objectif est qu’un émetteur puisse communiquer à l’ensemble de ses destinataires les données collectées à partir d’un support et dans un format défini. Le plus connu d’entre eux, OpenIOC, standard historique créé par l'ex société MANDIANT  pour échanger des IOC. À la base, OpenIOC est un format (en XML) utilisé par les outils de la société mais qui a été rendu open-source pour permettre un usage par tous. Un fichier OpenIOC décrit les symptômes à rechercher pour identifier une menace.
D’autres formats comme comme SNORT  ou encore YARA qui se focalise sur les caractéristiques intrinsèques d’un fichier et est particulièrement pertinent pour identifier des souches de familles de code malveillant.
Ces formats de modélisation rencontrent cependant des limites car centrés sur l'événement et n'offre pas de vision globale de la cybermenace à l’origine de l’attaque. 
STIX (Structured Threat Information Expression) et TAXII (Trusted Automated eXchange of Indicator Information), et auparavant CybOX (Cyber Observable eXpression), ont été développés dès 2012 aux États-Unis par le CERT-US (United States Computer Emergency Readiness Team) du Department of Homeland Security  puis repris par les organisations MITRE et OASIS. Ces langages tentent de combler les lacunes.

STIX est un langage normalisé et structuré permettant de représenter les informations sur les menaces cyber, et en particulier des indicateurs de compromission (IOC) issu des analyse de Cyber Threat Intelligence (CTI).

TAXII est quand à lui un protocole conçu pour améliorer la qualité des échanges des informations de type CTI, formatées dans le langage STIX.

Les utilisateurs STIX peuvent en particulier décrire et modéliser les concepts  :

\begin{itemize}
  \item \textbf{Attack Pattern} : mode opératoire de l’attaquant ;
  \item \textbf{Threat Actor }: individus, groupes ou organisations agissant avec une intention malveillante ;
  \item \textbf{Malware} : logiciel malveillant utilisé pour compromettre la confidentialité, l'intégrité ou la disponibilité du système d’information de la cible (voir \link{https://www.virustotal.com/gui/}{VirusTotal};
  \item \textbf{Tool} : logiciel légitime utilisé par les cyber-menaces dans le cadre de leurs attaques ;
  \item \textbf{Vulnerability} : vulnérabilité présente dans un logiciel qui est exploitée directement par un attaquant afin de compromettre un système d’information ;
  \item  \textbf{Indicator} : indicateur utilisé pour détecter et bloquer une activité suspecte ou malveillante sur le système d’information.
\end{itemize}


\subsubsection{MISP}

La distribution d'informations caractérisant les attaques nécessite au delà de la modélisation des attaques des indicateurs des IOC (\UKword{Indice of compromission}) facilement détectables dans les plateformes techniques de détection que nous verront par la suite.

MISP (Malware Information Sharing Platform and Threat Sharing) est une solution open-source permettant la collecte, le stockage, la distribution et le partage d’IOC liés aux malware. Cet outil permet à divers organismes de partager les indicateurs de compromission identifiés lors des activités de SOC et campagnes de réponses à incidents (CERT et CSIRT).  D’un point de vue technique, MISP est une plateforme d’échange d’IOC où chaque acteur peut entrer et organiser ses IOC afin de les publier pour les partager aux autres acteurs présents sur ce portail.  MISP fournit des fonctionnalités pour faciliter les échanges d'informations, mais aussi l’intégration de l'information par les IDS (Intrusion Detection System) et les outils de défense comme ceux d'analyse de logs et les SIEMs (Security Information and Event Management).


\subsubsection{Vérifier une donnée par rapport à de la  CTI}

Disposer de Treat Intelligence donc de renseignement  est complémentaire de la connaissance des scénario issus des analyses de risques, mais enrichissent aussi les analyses de risques par les modèles d'attaques existants.

\begin{itemize}
  \item Tester un fichier, une url (\textbf{Malveillance}) :  \link{https://www.virustotal.com/gui/}{VirusTotal};
  \item Tester une adresse email (\textbf{Compromission})  : \link{https://haveibeenpwned.com}{haveibeenpwned};
  \item Tester sans action une adresse IP publique ou un domaine (\textbf{Vulnérabilités Internet)} : \link{https://www.shodan.io}{Shodan};
\end{itemize}



