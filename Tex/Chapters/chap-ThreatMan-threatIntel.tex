


\section{Gestion de la menace}

Nous avons évoqué dans le chapitre sur l'anticipation, la veille sur la menace. Opérer la détection d'attaques ou de menaces dormante dans l'environnement de l'entreprise nécessite une connaissance précise des mécanismes d'exécution oui d'opération de ces menaces.
La connaissance de ces mécanismes  d'action, de protection, de déploiement, de réplication, de survivabilité, de déplacement des codes malveillants par exemple est la base de leur détection. Il es en de même sur les scénario mixant des actions sur les réseaux ou sur les systèmes informatiques ou numériques.
Ces connaissances sont généralement structurés dans des base de connaissances dont les sources sont gratuites ou payantes.


\subsection{Cibles de menaces}

\subsection{Sources de menaces}
Nous parlerons ici de sources de menaces comme les indicateurs permettant d'identifier l'origine technique d'une menace. Cela peut être une adresse mail, un serveur/service de mail , une adresse IP de provenance d'un code malveillant, d'une attaque, ou d'un comportement anormal.

On peut citer par exemple 
Une adresse mail connue pour envoyer un code malveillant.
le blacklistage d'adresse IP ou de d'adresse de serveur Mail pour Spam


En face, il des attaquants qui bien entendue vont changer leur position pour émettre ou attaquer d'ailleurs, ou avec une autre forme (furtivité)

%TODO ne pas oublier de parler des bases de donnes qui ne sont plus valable. USURE de l'information ...


Les sources de menace dans l'environnement internet			


\subsection{Data Lake}


\subsection{Threats Huntings}

La chasse à la menaces dormantes ou aux compromissions  mais aussi le maintien du contact entre la défense et les attaquants.

La chasse aux menaces est une tactique permettant de connaître avec plus d'acuité l'environnement de la menance et donc  le degré de risque de cyberattaques auquel est soumise une entreprise. 


La terminologie threat hunting regroupe plusieurs type de d'action et la définition de n'est pas totalement stabilisée. Globalement on y trouve deux grande classes de threat hunting

Celle travaillant autour de l'environnent, de la surface d'attaque et qui oriente ses actions sur des méthodes de "recherches" permettant de débusquer des menaces latentes ou des menaces dormantes et les réveiller et de les suivre de les comprendre et Pour établir le contact avec l'attaquant.

Et un autres plus active ou proactive dont l'objectif est de rester, conserver le contact avec l'attaquant lors d'une reaction à une alerte.
 
\subsubsection{Etablir le contact}

Quand on parle d'établir contact, nous parlons d'aller au contact au sens martial du terme. c'est dire en direct de suivre, caractériser ka sources de la menace et jouer avec elle.

La méthode de "hunter" consiste en premier à dresser un portrait global de la surface d’attaque, tout en identifiant les attaquants potentiels, leurs motifs et leurs façons de faire. Plus précisément, le « threat hunting » consiste en une analyse détaillée de :

\begin{itemize}
	\item La position de de l’entreprise,  notoriété, popularité sur internet, en analysant en particulier  les médias traditionnels et les médias sociaux;
	\item l'environnement économique de l’entreprise dont ses fournisseurs, ses clients, ses partenaires, ses employés;
	\item le corpus technologiques et physique de l'entreprise, dont les architectures techniques et les mécanismes informatique avec l'environnement économique ainsi l'environnement sécuritaire de ses relations.
\end{itemize}


Sur la base de cette analyse globale, des SPOF (Sigle Point Of Failure) peuvent être trouvés. 

grace à la visualisation  globale des lien il sera possible comprendre où, comment, pourquoi et potentiellement par qui (hactivistes, anciens employés, fournisseurs, etc.) la prochaine attaque pourrait être perpétrée. 
Les « threat hunters », ne sont pas simplement en attente de répondre aux alertes du système de défense, ils cherchent activement des menaces dans leurs propres réseaux afin de prévenir ou de minimiser les dommages. Cette méthode s’avère l’une des plus proactives. 


