%-------------------------------------------------------------
%               FR CYBERDEF SECOPS COURSE
%              $Chapitre : Threat Management
%                         $theme : Threat Intelligence
%        $File :  chap-ThreatMan-threatintel.tex
%                             2020 eduf@ction
%-------------------------------------------------------------
\uchap{chap-Vulman-threatintel.tex}
%-------------------------------------------------------------

\section{Anticiper : Threat Intelligence}

\subsection{Surveiller et anticiper}

La «threat intelligence» est un service de renseignement concernant les cyber-menaces. Les solutions SIEM par exemple, possèdent leurs propres sources, il n'y a pas que les SIEM qui peuvent utiliser ces sources,  il est possible de connecter de nombreux outils de détections à d’autres sources externes de «threat intelligence». De manière non exhaustive un service de «threat intelligence»  fournit des éléments comme :
\begin{itemize}
  \item \textUK{Malicious / Phishing IP / URL} : une liste d’URL utilisée pour  délivrer un fichier malicieux ou procéder à une attaque par hameçonnage;
  \item 	\textUK{Botnet C\&C URL}» : une liste d’URL utilisée pour héberger des serveurs de commande et contrôle de logiciel malveillant ou de réseaux de machine zombie;
  \item \textUK{Malicious Hash} : liste d’empreintes de logiciels malveillants connus et ayant déjà été analysés;
  \item \textUK{IP Reputation} : liste d’adresses IP suspectées dans des attaques informatiques ou en lien avec une cyber-menace (pouvant être utilisée en Black List sur les système sde filtrage, comme historiquement).
\end{itemize}


% Begin PRZ ===========================
\begin{frame}
\frametitle<presentation>{Threat Intelligence}
% end header PRZ =======================
La surveillance et le renseignement de la menace au sens général du terme (Threat Intelligence) devrait contenir les 2 niveaux :

\begin{itemize}
  \item Le renseignement à \textbf{vocation} cyber qui comprend toutes les analyses et information permettant d'anticiper et de caractériser une menace qui pourrait s'exprimer dans le monde numérique;
  \item Le renseignement \textbf{d'origine} Cyber, dont les données techniques liées à des attaques, menaces qui permettent de configurer des systèmes de détection et de réponse.
\end{itemize}
\end{frame}

% end PRZ ===========================

Il est vrai qu'encore aujourd'hui parler de \g{threat intelligence} nous dirige systématiquement sur la deuxième assertion.


\mode<all>{\picframe{../Tex/Pictures/img-veillecyber}{Veille cyber, une veille sur les risques}{0.7}{lbl-veillecyber}}

% Begin PRZ ===========================
\begin{frame}
\frametitle<presentation>{Veiller et surveiller}
\framesubtitle<presentation>{2 axes}
% end header PRZ =======================

Veiller et surveiller les menaces, détecter les attaques  nécessite d'analyser deux axes : 

\begin{itemize}
  \item Les menaces génériques, ou ciblant un domaine  particulier (Santé, Industrie, Banque ...) que l'on trouve généralement en utilisant des technologies de \g{threat Intelligence}; 
  \item Les menaces ciblées, dont les indices d'émergence peuvent être détecter en analysant la menace ou  en recherchant des indices de compromissions quand ces menaces sont actives dans le périmètre de l'entreprise. \g{threat Detection, Hunting ...} 
\end{itemize}
\end{frame}

% end PRZ ===========================



% Begin PRZ ===========================
\begin{frame}
\frametitle<presentation>{Veiller et surveiller}
\framesubtitle<presentation>{2 manières}
% end header PRZ =======================

et ceci de deux manières :

\begin{itemize}
  \item Surveillance de l'écosystème de la menace (IOC,  DarkWeb, Threat Intelligence...) 
  \item Recherche de compromission, ou d'infection (Threat Hunting, ...) 
\end{itemize}
\end{frame}
% end PRZ ===========================

Ce sont des sujets que nous aborderons dans le processus de gestion de la menace.
La surveillance des menaces génériques relève d'action de veille comme cela est fait pour les vulnérabilité. Les scénarios de message sont vus comme des éléments de signature d'une attaque ou d'une tentative ou de préparation  d'attaque.

Je vous propose de présenter  la gestion de la menace sous la forme de 3 thèmes  (\uref{Gestion de la menace}{lbl-logman}).

\begin{itemize}
  \item Log Management;
  \item Threat Intelligence (au sens renseignement);
  \item Threat Detection.
\end{itemize}

\mode<all>{\picframe{../Tex/Pictures/img-threatvigil}{Les sources}{0.7}{lbl-threatvigil}}


%TODO Tester les SIEM et les SOC et le deuxième niveau de maturité

%TODO Niveau de maturité dans les processus

% TODO le renseignement


%*************************************
\subsubsection{Surveillance de la compromission}

Un des domaines de la surveillance est donc celui de la compromission. C'est à dire la surveillance dans le fameux Darkweb de l'émergence de données volées, \g{perdues} par une entreprise ou par un particulier.

Ce domaine, dénommé par certains "Leak Intelligence" ou "Leak Management", correspond à la gestion des fuites de données, au sens de leur détection et la recherche de "la source" de fuite. C'est souvent dans cette dynamique de découverte d'information \g{interne} dans des espaces \g{malveillants} ou pas que l'on découvre des compromissions techniques ou non techniques issues d'attaques.

La compromission la plus connue reste encore de nos jours  la fuite du couple Utilisateur/mot de passe sur des sites hackés. Il existe des bases de données publiques qui publient ces données "compromises" comme par exemple : \link{https://haveibeenpwned.com}{Have i been pwned}, qui permettent à partir d'une adresse mail ayant servie d'identifiant de savoir si on a été compromis sur un site qui aurait été piraté.

\subsubsection{Surveillance des fragilités}

Comme nous l'avons vu dans le chapitre sur les vulnérabilités,  des \g{scans} de vulnérabilités sur des plages d'IP de l'entreprise permet de déterminer les fragilités de services ouverts ou accessibles. Généralement organisées dans une dynamique d'audit, ces évaluations de sécurité sont conduites avec un cadre contractuel et légal. Il existe pourtant des entreprises qui fournissent des informations de fragilités sur des entreprises ou des plages d'IP. On trouvera pas exemple sur le site de \link{https://www.shodan.io}{SHODAN} des informations intéressantes (et payantes) sur des fragilités de systèmes dont une grande partie de systèmes d'objets connectés.

\begin{warningbox}{Base de données de fragilitées publiques}
Les techniques et les sites qui propose des outils ou l'accès à des bases de données de \g{sites} vulnérables. C'est un sujet intéressant pour une fiche TECHNO de synthèse sur ce qui existe sur le marché. %sujets : https://alternativeto.net/software/shodan/
\end{warningbox}


%*************************************
\subsubsection{Surveillance du ciblage}

% Threat Hunting 

La surveillance du ciblage, que les anglo-saxons appelle le \g{TARGETING} est aussi un élément d'anticipation. 
En effet, ces éléments sont souvent les premiers signaux d'un préparation d'un évènement \g{cyber} qui pourrait toucher l'entreprise.

On y trouve l'émergence de la collecte d'information sur une cible donnée. La mise en oeuvre dans les codes  malveillants de ciblage d'IP spécifique, etc...

% Begin PRZ ===========================
\begin{frame}
\frametitle<presentation>{Surveillance du ciblage}
\framesubtitle<presentation>{l'outillage du \g{targetting}}
% end header PRZ =======================
Il y a deux types d'outils pour ce se faire : 

\begin{itemize}
	\item La surveillance classique du web de type \g{cyberveille}, qui permet de découvrir des éléments compromis appartenant à l'entreprise (soient les données, soient des informations permettant de déduire que l'entreprise a été compromise).
	\item L'analyse en temps réel des codes malveillants qui peut permettre en regardant de manière détaillée l'évolution du code pour comprendre et connaitre les modalités des attaques et les nouvelles cibles.
\end{itemize}
\end{frame}
% end PRZ ===========================


%*************************************
\subsubsection{Que faire des ces informations}

Disposer des fragilités de l'entreprise, et connaitre les scénarios potentiels permets d'évaluer un niveau de risque. 

\subsection{de l'outillage sur la menace}

\subsubsection{la gestion des menaces CTI}

Le projet OpenCTI (Open Cyber Threat Intelligence), développé par l’ANSSI en partenariat avec le CERT-EU, est un outil de gestion et de partage de la connaissance en matière d’analyse de la cybermenace (Threat Intelligence). Initialement conçue pour structurer les informations de l’agence relatives à la menace informatique, la plateforme facilite aussi les interactions entre l’ANSSI et ses partenaires.

L’outil, intégralement libre, est  disponible à l’usage de l’ensemble des acteurs de la « threat intelligence ». L’application permet ainsi de stocker, organiser, visualiser et partager leurs propres connaissances en la matière.

Le projet \link{https://www.ssi.gouv.fr/actualite/opencti-la-solution-libre-pour-traiter-et-partager-la-connaissance-de-la-cybermenace/}{OpenCTI} a été initié en septembre 2018 par l’ANSSI et co-développé avec le CERT-EU en l’absence de solutions complètement appropriées pour structurer, stocker, organiser, visualiser et partager la connaissance de l’ANSSI en matière de cybermenace, à tous les niveaux.


\subsubsection{STIX}

%The structuration of the data is performed using a knowledge schema based on the STIX2 standards. It has been designed as a modern web application including a GraphQL API and an UX oriented frontend. Also, OpenCTI can be integrated with other tools and applications such as MISP, TheHive, MITRE ATT&CK, etc.

Les modélisations des attaques est un large champs de recherche et d'outillage. Juste à titre d'illustration, nous pouvons parler d'un modèle comme STIX™ (Structured Threat Information Expression) langage et format de donnée permettant de modéliser et échanger des informations techniques sur les processus d'attaque cyber. 
Je vous propose d'explorer pour cela sur le site \link{https://github.com/OpenCTI-Platform/opencti}{STIX sur GitHub}.

