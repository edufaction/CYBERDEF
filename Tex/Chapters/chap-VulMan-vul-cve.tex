\section{ CVE}

De Nombreuses vulnérabilités sont découvertes chaque année. Les informations techniques sur ces vulnérabilités permet de les détecter, et de les caractériser. il était important dans le monde des technologies de l'information qu'elles puissent être identifié de manière unique.  


L’objectif est de constituer un dictionnaire qui recense toutes les failles avec une description succincte de la vulnérabilité, ainsi qu’un ensemble de liens que les utilisateurs peuvent consulter pour plus d’informations. Cette base  est proposée pour consultation et reste maintenue par le Mitre Corporation.  Cet organisme à but non lucratif américaine a pour  l'objectif est de travailler dans des domaines technologique comme l'ingénierie des systèmes, les technologies de l'information, la sécurité. 
 
\UKword{ Common Vulnerabilities and Exposures} ou CVE est une base de données (Dictionnaire) des informations publiques relatives aux vulnérabilités de sécurité. Le dictionnaire est maintenu par l'organisme MITRE.


Pour consulter les CVE il suffit de se rendre sur \link {https://www.cve.mitre.org}{CVE.mitre.org}

\subsection{Identifiant CVE}

Les identifiants CVE sont des références de la forme CVE-AAAA-NNNN 

\subsection{CVSS}

Bien entendu, disposer d'un identifiant d'une vulnérabilité
Common Vulnerability Scoring System (CVSS)
Common Weakness Enumeration (CWE)