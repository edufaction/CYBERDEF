%-------------------------------------------------------------
%               FR CYBERDEF SECOPS COURSE
%                                        SECOPS
%.                    Vulnerability/Threat/Incident
%
%                           Introduction Cyberdefense
%                            chap-Cyberdef-intro.tex
%
%                              2020 eduf@ction
%-------------------------------------------------------------
\uchap{Chapitre introductif de la Partie 3 du cours SEC 101 - CYBERDEF}

% ***** CHAPTER CYBERDEF


\section{SECOPS et cyberdéfense}

Mettre en place des stratégies de cyberdéfense, c'est partir du principe que l'entreprise sera attaquée et que l'enjeu des équipes est de se préparer à des attaques pouvant violemment impacter l'entreprise. Pour cela, l'entreprise doit anticiper, détecter à temps et réagir vite pour réduire l'impact.

	\subsection{Des opérations de cyberdéfense d'entreprise}
	
C'est dans ce contexte de sécurité  que doit s'organiser cette cyberdéfense opérationnelle et se structurer autours des axes qui caractérise une posture de cyberdéfense d'entreprise : 
	
\begin{itemize}
  \item Le renseignement : 
  				\begin{itemize}
  					\item sur les menaces  : les attaquant et leurs intentions, leurs techniques et outils, les sources compromises,
  					\item sur les vulnérabilités : des logiciels, et des structures organisationnelles,
			\end{itemize}
  \item La détection d'attaques ou de menaces dormantes ou cachées;
  \item La mise en alerte, ou la réponse à incidents pour aller à la gestion de crise;
  \item La neutralisation de sources malveillantes.
\end{itemize}
	
\subsection{Veille et renseignement}

Au coeur des opérations de cyberdéfense, le renseignement reste le moteur de l'anticipation.  Une grande partie des attaques exploitent des failles ou des vulnérabilités. Être au plus tôt au courant de l'existence d'une vulnérabilité sur un système utilisé par l'entreprise est le premier stade d'une veille pour ANTICIPER une attaque potentielle basée sur cette faille.

Toutes les attaques ne sont pas liées à l'utilisation d'une faille technique, il existe d'autres marquants ou indicateurs qui peuvent être surveiller pour évaluer les risques. Nous verrons dans le chapitre sur la détection, que les marquants des menaces sont souvent des données techniques qui peuvent caractériser l'attaque. l'attaquant, les techniques utilisées...
Par ailleurs le renseignement permet aussi de détecter des fuites de données, ou des informations sensibles compromises en surveillant les sites spécialisées.
On peut citer le célèbre \textbf{\link{https://haveibeenpwned.com}{';--have i been pwned?}} qui indique si une adresse électronique utilisée comme identifiant de compte a été compromise lors d'un vol de données sur un site. 

L'ensemble des informations liées aux menaces s'appelle de la \UKword{Threat Intelligence ou Cyber-Threat Intelligence (CTI)}, et celles liées aux vulnérabilités  classées dans la dynamique \UKword{Vulnerability Intelligence}.

\upicture{Tex/Pictures/img-intelligence}{Les axes du renseignement cyber}{0.5}{lbl-intelligence}

\subsection{VTI}

\upicture{Tex/Pictures/img-VTIsimple}{Les axes du renseignement cyber}{0.7}{lbl-VTIsimple}

Le triptyque VTI (Vulnérabilités/Failles, Menaces/Attaques, Incidents/Alertes), est souvent présenté comme le coeur des activités de sécurité opérationnelle. 

La sécurité opérationnelle c’est l’ensemble des processus opérationnels qu’il faut mettre en place et évaluer au quotidien afin réduire la surface d’exposition du système d’information aux risques, mais aussi réduire l'impact en cas d'attaque. 

On y trouve en particulier : 

\begin{itemize}
  \item Les audits techniques, pentest, scan : pour identifier, mais aussi la cartographie des actifs, des acteurs ...,
  \item Les systèmes de détection comme le SIEM, de chasse comme le Threat Hunting ...,
  \item Les mécanismes de remédiation, les techniques et outils de forensique.
\end{itemize}

Tous ces actions génèrent au sein de l'entreprise de l'information, souvent à destination d'acteurs différents mais participant globalement à l'objectif de cyberdéfense.

\subsection{Fusion Center}

Les préoccupations des responsables de cette sécurité opérationnelle restent encore l'accès et le partage de l'information car dans le cycle de gestion du risque et de réduction de la surface d'attaque. Le terme \UKword{Fusion Center} est historiquement liée au attentats au 11 septembre 2001, qui avaient montré les lacunes de partage de l'information entre les services de police et renseignements américains face à cette attaque. Repris par les experts de la cybersécurité, pour gérer les menaces 
La sécurité opérationnelle est gérée généralement par deux principales structures : le SOC, en charge de la détection, de la qualification et de la gestion des incidents ; et le CSIRT, responsable de la gestion de crise, de l’investigation numérique, de la veille et de la Threat Intelligence.


\upicture{Tex/Pictures/img-fusioncenter}{Fusion Center et CSOC}{0.7}{lbl-fusioncenter}

\begin{itemize}
  \item Automatiser la \textbf{création de règles} basées sur des menaces avérées  et détecter de manière avancée avec le \UKword{Machine Learning}
  \item Automatiser le \textbf{processus de management }des incidents avec en  particulier les SOAR (Security Orchestration, Automation et Response) outils d’aide et d’automatisation de la réaction aux incidents de sécurité.
  \item Adapter \textbf{l’organisation} à cette automatisation
  \item et surtout automatiser la \textbf{collecte et la distribution} de l'information dans les différents services grace à un \g{Centre de Fusion de l'information et du renseignement}.
\end{itemize}


\subsection{Entrainement à la cyberdéfense}

En matière de cybersécurité, les organisations sont confrontées à la persistance des cyberattaques, combinées à la montée en puissance de nouvelles menaces, ne sont pas suffisamment préparées à anticiper les incidents et à y apporter les réponses adéquates. De nombreuses études continuent à montrer  que la formation du personnel demeure le défi premier, se traduisant par un investissement conséquent de la part des organisations.

L’entrainement répond à différents objectifs de l'entreprise  :
\begin{itemize}
  \item Sensibilisation des acteurs (salariés, décideurs, managers) aux risques « cyber ». Ces entrainements, relativement brefs et regroupant qu'une partie des acteurs  permettent d’illustrer concrètement les menaces. Il permet en particulier de l'illustrer concrètement les impacts des cyberattaques sur une entreprise notamment pour les niveaux stratégiques;
  \item   Evaluation du dispositif de sécurité et de gestion de crise en particulier pour les évaluations autour des travaux de PCA/PRA avec l'ISO 22301. L’objectif est d'évaluer toute la chaîne opérationnelle, des équipes techniques informatiques et sécurité, aux top management sur l'ensemble de structures de la résilience;
  \item Entraînement des équipes techniques, par exemple au sein des SOC (Security Operation Center). Cela suppose de s’appuyer sur des environnements de simulation informatique représentatif de la réalité pour s'adapter à l'évolution des attaques (Outils de Range en particulier);
\end{itemize}


Pour répondre à ces besoins, trois niveaux d’entraînement doivent donc être distingués :

\begin{itemize}
  \item Des entraînements techniques au niveau des opérations. Ces entraînement peuvent intervenir dans le cadre de formations des équipes;
  \item Des entraînements managériales ou niveau des managers de proximité ou des COMEX. La durée de ces entraînements dépend des objectifs de monté en compétence. Pour une  sensibilisation de COMEX,  la durée doit être adaptée et courte (de l'ordre d'une demi journée). Les entraînements  ou exercices peuvent doivent être réguliers, une fréquence d'exercice annuelle est recommandée.
  \item Des entraînements mixtes associant des populations techniques et non techniques, afin de traiter l'inter-dépendance des différentes structures engagées.
\end{itemize}



\subsection{Gestion de crise}

Les opérations de sécurité opérationnelle doivent embarquer le processus de gestion de crise de l'entreprise, on trouvera dans le chapitre de gestion des incidents la manière donc les opérations de cybersécurité peuvent s'intégrer à des processus de gestion de crise existants ou comment créer des processus spécifiques à la cyber-crise. En effet, une des particularités des cyber-crises c'est que le systèmes d'information et de communication peut lui aussi être compromis, et que la confiance dans les systèmes informatiques de gestion peut être altérée, ou ces systèmes totalement inopérants.  

\section{Range et Simulation}

Les cyber-ranges, environnements de simulation informatique, dédiés non seulement à l’expérimentation des technologies, mais également à la formation par la pratique et à l’entraînement du personnel, constituent une réponse à l'enjeu de vision globale.  Initialement développés dans un cadre militaire,  les  cyber-ranges intéressent aujourd'hui l’ensemble de l’écosystème de cybersécurité. Ils offrent des conditions d’entrainement proches du réel, tant dans les topologies réseau reconstituées que dans les technologies de sécurité déployées. Ils fournissent un environnement d’affrontement informatique permettant une nouvelle approche de la formation axée sur l’opérationnel.

Il existe de nombreux outils d'entraînements. La majorité sont construits sur des technologies de virtualisation.
L’environnement de simulation technique est donc basé sur un « bac à sable » numérique (une sorte de paillasse numérique) offrant notamment des capacités de :
\begin{itemize}
  \item Virtualisation de postes de travail et de serveurs  et de tout autres composants informatique ou réseaux. l'environnement de simulation informatique peut être  « hybride » lorsqu’il permet également de connecter des équipements physiques (routeur, sonde, automate industriel...) ou des appliances ;
  \item Virtualisation des différents couches de transports, de stockage et de traitement ( couche réseau pour les liens et les équipements, couches base de données et infrastructures de base : Annuaire, Infrastructure de gestion de clefs, ...) ; 
  \item  Création de scénarios pédagogiques ou de situation tactiques pour fournir des contextes d'entraînement et de formation ;
  \item  Simulation et Génération de trafic pour donner de la vie au système.

\end{itemize}

L'entraînement est une solution pour dynamiser les activités de cyberdéfense.


\toolsbox{CyberRange-Airbus}{range}
\toolsbox{Hynesim}{range}
\toolsbox{EDUYesWeHack}{range}


\upicture{Tex/Pictures/img-range}{Entrainement}{0.8}{lbl-range}

Sur le plan humain,  l'utilisation d'un cyber-range se construit autour de deux équipes que nous explorerons dans les chapitres sur la gestion des vulnérabilités et la gestion des menaces :

\begin{itemize}
  \item La « Red-Team », composée de hackers éthiques professionnels (les PENTESTEURS). Ceux-ci reproduisent des attaques ciblées et d’ampleur et de complexité croissante, de nature à challenger la défense tout au long de l’entraînement ;
  \item  La « Blue-Team », chargée de la défense des réseaux et systèmes d’information (le SOC),, qui est donc constituée des apprenants participant au programme d’entraînement ou de formation.
\end{itemize}


%Vous pouvez trouver une document intéressant sur \link{https://www.defense.gouv.fr/content/download/502443/8528007/file/OBS_Monde cybernétique_201703.pdf}{le site du ministère des armées} 

% https://www.cloudrangecyber.com/ (autre Range)



