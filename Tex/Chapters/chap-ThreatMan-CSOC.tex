% Chapitre ALERT 

\section{Le SOC}

Les SOC est au coeur du système de Veille Alerte et réponse. C'est la tour de contrôle de l'espace Cyber.


\section{le SOC de demain}
On peut par ailleurs s'interroger sur le fait qu'un tel système peut et doit opérer d'autres missions que les missions de sécurité pures. Si la supervision des réseaux a été longtemps au outils au services des techniciens, la supervision de l'environnement digital c'est à dire l'environnement informationnel de l'entreprise est un axe fondamental. Le SOC Security Operation Center peut devenir Cyber Operational Center opérant le suivi des risques digitaux au sens large, incluant les réseaux sociaux et leur cohorte de fausse informations et d'information pouvant être des indicateurs de crise à venir pour l'entreprise.

\subsection{Evaluation d'un SOC}

L'efficacité d'un SOC, et les purples Team.

%https://www.globalsecuritymag.fr/SOC-comment-en-mesurer-l,20151005,56416.html

%Integration du threats hunting dans les SOC.
%Par exemple suivre la création de nom de domaine récent pour regarder les noms utiliser dans des requêtes.


\subsection{Les outils connexes d'un SOC}

au dela des SIEM, il semble important d'ajouter à l'outillage dun SOC un ensemble de système permettant de mesurer et d'évaluer l'impact des attaques.
Echelle de RICHTER d'une attaque.

https://observatoire-fic.com/prendre-la-mesure-des-cyberattaques-peut-on-definir-une-echelle-de-richter-dans-le-cyber/