%-------------------------------------
% Chapitre
% Vulnerability Management
% BugBounty
% File : chap-Vulman-vul-bugbounty.tex
%--------------------------------------

\subsection{Le Bug Bounty}

Un bug bounty est une solution de \g{recherche de bugs et de vulnérabilités} proposée par des entreprises organisatrices qui permet à des personnes de recevoir une prime ou compensation après avoir remonté des bugs, surtout ceux concernant des failles et des exploits associés. Dans le domaine de la cybersécurité, on peut résumer de manière macroscopique cette activité  à un déploiement de chasseurs de prime pour chasser les vulnérabilités \g{techniques} des systèmes informatiques.

La question est de savoir où trouver ces chasseurs de primes et d'organiser cette \g{chasse}. Le principe est d'organiser dans l'entreprise ou la structure un programme ciblé sur un système logiciel.

Un des programmes de Bug Bounty  a été lancé au sein d’une entreprise pionnière de l'Internet, développeur du célèbre navigateur: Netscape. Créée en 1994 et rachetée par AOL en 1998, l’entreprise a disparue en janvier 2003.
A l'époque 90\% des internautes utilisaient ce navigateur et Jarrett Ridlinghafer, un ingénieur du support technique, avait constaté que les membres de la communauté \g{open source} corrigeaient d’eux-mêmes les bugs de l'application sans que l'entreprise ne soit dans la boucle de contrôle validation. C'est ainsi que naquit l'idée d'organiser et piloter cette chasse au bug et d'offrir des goodies au titre de récompenses. 
Aujourd'hui des entreprises intermédiaires proposent de réunir tous les conditions pour réaliser ces chasses au bug.
Elle proposent des plateformes permettant de faire se croiser des \g{hackers} et des \g{éditeurs} autour de chasse au bugs / vulnérabilité.

L'utilisation de ces techniques pour trouver des vulnérabilités dans ses systèmes est conseillée en fin de cycle de recherche de vulnérabilités

\subsection{Quelques entreprises}

\begin{itemize}
    \item HackerOne (\url{https://www.hackerone.com})
    \item YesWeHack (\url{https://www.yeswehack.com})
\end{itemize}


