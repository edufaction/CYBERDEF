%-------------------------------------
% Chapitre
% Vulnerability Management
% CERT
% File : chap-Vulman-CERT.tex
%--------------------------------------
\uchap{\jobname}

\section{Les CERTs}

J'ai donné quelques éléments les CERTs dans le chapitre sur les vulnérabilités, toutefois l'évolution des fonctions et services des CERTs s'est rapidement élargie ces dernières années. Au delà des de la diffusion et alerte sur des vulnérabilités, ils couvrent maintenant avec précision les menaces (analyse et alerte sur codes malveillants, ...) et les incidents. 

\subsection{Les CERTs commerciaux}

\begin{warningbox}{Fiche TECHNO}
	Les CERTs commerciaux est un bon sujet d'exploration des sociétés qui delivrent des services de veille. Vous trouverez 
	\link{https://www.ssi.gouv.fr/agence/cybersecurite/ssi-en-france/les-cert-francais/}{les Certs en France} sur le site de l'ANSSI.
\end{warningbox}

\subsection{La relation avec un CSIRT Interne }
% Analyse de malware. (Un malware utilise des vulnérabilités, et l’analyse d’une attaque donne aussi les éléments pour comber les fragilités

	Une méthodologie efficace de gestion des vulnérabilités comprend une équipe d’intervention en cas d’incident de sécurité informatique (CSIRT). Le CSIRT est responsable de la publication des avis de sécurité, de la tenue d'information régulières pour échanger sur les activités malveillantes et des dernières attaques du jour zéro, de la simplification et de la diffusion des alertes de sécurité et de l’élaboration de directives compréhensibles et efficaces en matière de réaction aux incidents pour tous les salariés. De cette manière, chacun seront en mesure de réagir aux indicateurs de compromis potentiels conformément aux pratiques recommandées par l'équipe CSIRT.