%-------------------------------------
% Chapitre
% Vulnerability Management
% CERT
% File : chap-Vulman-CERT.tex
%--------------------------------------
\uchap{\jobname}

\subsection{Les services de veille en vulnérabilités}

Vous trouverez quelques éléments sur les CERTs (Computer Emergency Response Team) dans le chapitre sur les vulnérabilités, toutefois le périmètre de fonctions et services des CERTs s'est rapidement élargie ces dernières années. Au delà de la diffusion et alertes sur des vulnérabilités, ils couvrent maintenant avec précision les menaces (analyse et alerte sur codes malveillants, ...) et les incidents. Les CERTs restent les acteurs principaux de cette veille et capacité d'alerte.  
Nous trouvons toutefois de nombreux services de veille en vulnérabilités qui ne sont pas des CERTs, mais qui offrent des services dédiés à des typologies de produits, ou des secteurs ....

\subsubsection{Les CERTs commerciaux}

\begin{techworkbox}{Fiche TECHNO}
	Les CERTs commerciaux est un bon sujet d'exploration des sociétés qui délivrent des services de veille. Vous trouverez 
	\link{https://www.ssi.gouv.fr/agence/cybersécurité/ssi-en-france/les-cert-francais/}{les Certs en France} sur le site de l'ANSSI.  Excellent sujet pour \fichetech.
\end{techworkbox}

\subsubsection{La relation avec un CSIRT Interne }
% Analyse de malware. (Un malware utilise des vulnérabilités, et l’analyse d’une attaque donne aussi les éléments pour comber les fragilités

	Une méthodologie efficace de gestion des vulnérabilités comprend une équipe d’intervention en cas d’incident de sécurité informatique (CSIRT). Le CSIRT est responsable de la publication des avis de sécurité, de la tenue d'information régulières pour échanger sur les activités malveillantes et des dernières attaques du jour zéro, de la simplification et de la diffusion des alertes de sécurité et de l’élaboration de directives compréhensibles et efficaces en matière de réaction aux incidents pour tous les salariés. De cette manière, chacun seront en mesure de réagir aux indicateurs de compromis potentiels conformément aux pratiques recommandées par l'équipe CSIRT.