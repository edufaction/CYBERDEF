% xxxxxxxxxxxxxxxxxxxxxxxxxxxxxxxxxxxxxxxxxxxx
%
% CHAPITRE 3.1.1 - VEILLE  et vulnérabilités
%
%xxxxxxxxxxxxxxxxxxxxxxxxxxxxxxxxxxxxxxxxxxxxx
%==============================




%===============================================================
% Recherche de vulnérabilité mais ou
%--------------------------------------------------------------------------------------------------------------
\section{Ds difficultés d'exécution }
%--------------------------------------------------	
% des difficultés de la cartographie 
\subsection{Périmètre sous responsabilité de l’entreprise}

\subsubsection{la notion de responsabilité}

% ajouter ici, un texte qui parle de la notion de périmètre, au sens sois périmètre de l’entreprise soit périmètre de responsabilité pour un projet et/ou pour une sous structure.

\subsubsection{Inventaire des actifs}
%L'une des premières étapes d'un programme de gestion des vulnérabilités est un exercice de définition du périmètre de responsabilité et d'inventaire des actifs associés. En particulier au niveau de l'entreprise, les entreprises ont tendance à passer par une multitude de fusions, d'acquisitions et de nouvelles technologies et doivent donc combiner des systèmes incompatibles de manière native ou changer de personnel. Malheureusement, ces circonstances laissent souvent les entreprises confuses quant à la qualité de leur inventaire et beaucoup sont incapables d'identifier tous leurs actifs nécessitant un niveau de protection adéquat. Trop souvent, les entreprises possèdent une multitude d'actifs inconnus dans leur environnement qui pourraient compromettre leur sécurité sur le long terme.
%
%Selon les meilleures pratiques en matière d’inventaire des actifs, la gestion des actifs doit être confiée à une autorité unique qui consulte des cartes réseau valides, effectue des analyses de découverte pertinentes dans tous les réseaux locaux (LAN), valide régulièrement l’inventaire des actifs et gère la gestion des modifications. cas d’actifs neufs ou d’actifs retirés. Une fonction centralisée d’inventaire des actifs peut aider à clarifier l’inventaire des actifs d’une organisation et à renforcer sa sécurité.
%--------------------------------------------------	
\subsection{L’environnement digital externe}

%--------------------------------------------------	
\subsection{Veille et alerte sur les vulnérabilités}

	%--------------------------------------------------
	\subsubsection{abonnement au CERT}

% s’abonner au Cert pour être informer des vulnérabilités, mais avec la difficultés de recevoir que les vulnérabilités qui nous intéresse. Le lien avec le pacth, et la disposition des solutions de corrections.

	%--------------------------------------------------
	\subsubsection{Le marché de la vulnérabilité}

	CERT,  veille en vulnérabilités
	Une vulnérabilité et sa cotation

%--------------------------------------------------	
\subsection{La chasse aux vulnérabilités}
	Checking versus Pentest


%===============================================================
% Recherche de vulnérabilité en amont du cycle de vie.
Corriger une vulnérabilité au plus tot
%--------------------------------------------------------------------------------------------------------------
\subsection{Le marché de l’insécurité chronique}
	Sécurité applicative
	et le pire avec les DEVOPS et le WEB (Owasp co)


 

