% Chapitre REACT 

\section{Réagir}


\subsection{Threats Hunting}

Mettre place une interaction entre l'attaque et la défense, pour provoquer une continuité de l'attaque avec des objectifs qui peuvent aller du maintien de l'attaque pour découvrir les scénarios

Exemple se mettre en proxy et modifier les fichiers ex-filtrées pour les corrompre et faire en sort que l'attaquant reste plus longtemps.
%introduction sur la surveillance opérationnelle du quotidien. les outils de « visibilités » permettant de voir, percevoir ... et anticiper.
Réagir, gestion de crise, à quel moment géré t’on la crise.

\subsection{HoneyPots}

\subsection{Haqckback}

\upicture{../Tex/Pictures/img-incidents}{Incidents}{0.5}{lbl_incident}

ceci est un texte

\section{de l'alerte à l'incident}`

\section{de l'incident à la crise}

\subsection{PCA/PRA et PCI/PRI}

\subsection{se préparer et s'entrainer}

\section{Remédiation}



% Ajouter des éléments sur la 22301

Une question qui se pose lors d’une reprise d’activité est la confiance que nous avons dans le système. La difficultés après une attaque informatique ou une compromission, ou tout simplement une suspicion c’est la simple question de savoir si nous savons enlever toute la source de l’attaque. Reste-t-il des résidus.

\section{Aspect juridique de la réaction}
`
\subsection{hackback}

\section{Forensic}
