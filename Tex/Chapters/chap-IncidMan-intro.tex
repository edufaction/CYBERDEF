%-------------------------------------------------------------
%               FR CYBERDEF SECOPS COURSE
%                      INCIDENT MANAGEMENT
%
%                                    Introduction
%
%                              2020 eduf@ction
%-------------------------------------------------------------
%==================================================
\section{Réponse à incident}
%--------------------------------------------------------------------------------

% https://searchsecurity.techtarget.com/definition/incident-response
la réponse sur incident de sécurité pose de nombreuses questions sur les aspects aussi bien techniques que juridiques ou organisationnels. Nous allons tenter d'aborder ces éléments dans ce chapitre.

%- - - - - - - - - - - - - - - - - - - - - - - - - - - - - - - - - - - - - - - - - - - - - - - - 
\subsection{Terminologie}

% FRAME beamer PRZ ------------------------------------
\mode<all>{\texframe
{La réponse à incident}
{quelques éléments de définition}
{%. . . . . . . . . . . . . . . . . . . . . . . . . . . . . . . . . . . . .
La réponse à incident est le processus qui permet de déployer les moyens nécessaires pour traiter un événement de sécurité classé comme incident de sécurité.
Un incident de sécurité peut être enregistré en provenance de systèmes de sécurité, de veille ou d'audit. Le besoin d'intervention peut être immédiat comme différé.
La réponse peut nécessiter des équipes de compétences larges comme expertes sur un domaine donné. L'intervention peut nécessité des moyens techniques important ou pas, et mettre en isolation tout ou partie d'un système d'information.
}} % end FRAME.........................................................


La gouvernance de la réponse aux incidents consiste à planifier à l'avance et  de disposer un plan d'opération avant qu'il ne soit nécessaire. Plutôt que d'être un processus axé sur l'informatique, il s'agit d'une fonction  globale qui permet à une organisation de prendre des décisions rapides avec des informations fiables dans un contexte où la continuité d'activité ou l'image de l'entreprise est menacée. Non seulement le personnel technique des services informatiques et de sécurité est impliqué, mais aussi des représentants d'autres aspects clés de l'entreprise. La réponse à incident  interpelle  dans son mode d'opération, la gestion des plans de continuité et de reprise d'activité, la gestion de crise,  l'interaction juridique et contractuelle  ainsi que l'a gestion des relations avec les services de l'état (CNIL, ANSSI, Police et Gendarmerie ...). 

Quelques éléments de terminologies avec la correspondance anglo-saxonne afin de se repérer dans les usages et trouver de l'information pertinentes lors de vos recherches sur Internet.

\begin{itemize}
		\item \edxdico{Investigations Numériques}{Digital Investigation};
		\item\edxdico{Analyse légale}{Forensic (Inforensic)};
		\item \edxdico{CERT}{Computer Emergency Response Team};
		\item \edxdico{CSIRT}{Computer Security Incident Response Team};
		\item \edxdico{Gestion des Incidents}{Incident Management}.
\end{itemize}

\subsection{Définitions}

\begin{notebox}{Incident}
Un incident de sécurité, correspond donc à la conséquence d’un ou plusieurs évènements de sécurité ou un évènement de sécurité majeur. Pour un \textbf{événement}, il n'y a pas de conséquence alors que pour un \textbf{incident} il y a un impact sur l’un des critères de sécurité DICA (Disponibilité, Intégrité, Confidentialité, Auditabilité).
\end{notebox}

Cette distinction a toujours existé, en effet l'ISO/IEC 27001 l'a reprise de l'ISO TR 18044:2004 (aujourd'hui remplacée par l'ISO/IEC 27035) qui l'avait elle-même reprise de l'ISO TR 13335-2:1997. Cela remonte donc à la création de la normalisation en sécurité informatique en 1991.
Concrètement, un événement peut donc être :
\begin{itemize}
  \item soit la découverte d’une vulnérabilité;
  \item  soit la constatation d’une non-conformité;
  \item soit une altération, une perte ou une atteinte à l’information,
  \item   soit une altération ou une perte d’un élément du système d’information, d’un élément de configuration du SI ou d’un actif non-IT.
\end{itemize}

Un événement peut donner lieu à un \textbf{traitement préventif,} dans la mesure où aucun impact n'a été identifié, par exemple la découverte d’une vulnérabilité.
Un incident donne quant à lui obligatoirement lieu à un \g{traitement curatif }car un impact a été identifié.
Ce qui motive la requalification d’un événement en incident doit impérativement être basé sur une décision humaine en fonction d'une estimation de l'impact.
Néanmoins avant de s'engager dans la descriptions des activités liées à la réponse à incident cybersécurité, je souhaitais évoquer les bonnes pratiques ITIL qui donnent des pistes sur l'organisation de la gestion d'incident. Il ne faut en effet pas considérer la réponse à attaque comme une activité que technique bien que l'urgence nécessite le plus souvent de passer outre les processus classiques de traçabilité.

\subsection{Sources Incidents}

Il existe différents types d'incidents de sécurité et des moyens de les classer. Ce qui peut être considéré comme un incident pour une organisation peut ne pas être aussi critique pour une autre. tous les incidents ne proviennent pas de SIEM. En effet le déclenchement d'incident peut avoir différentes sources.

 %-------------------------------------------------------------------------
\mode<presentation>{\picframe{../Tex/Pictures/img-incident-sources}{Les axes de la gestion des cyber-Incidents}{1}{lb:incident-sources}}
%-------------------------------------------------------------------------

 %-------------------------------------------------------------------------
\mode<presentation>{\picframe{../Tex/Pictures/img-incident-cycle}{Les axes de la gestion des cyber-Incidents}{0.7}{lb:incident-cycle}}
%-------------------------------------------------------------------------

Voici quelques exemples d'incidents relativement courants:

\begin{itemize}
  \item Une attaque par déni de service distribué ( DDoS ) contre les services cloud critiques;
  \item  Infection par un logiciel malveillant ou un rançongiciel qui a chiffré des fichiers d'entreprise critiques sur le réseau de l'entreprise;
  \item Une tentative de phishing réussie qui a conduit à la divulgation d'informations personnelles identifiables des clients;
  \item Perte ou vol, d'un ordinateur portable non chiffré avec des informations sensibles;
  \item découverte sur internet (Darkweb) de données sensibles appartenant à l'entreprise.
\end{itemize}

Selon le SANS Institute, la réponse est construit autour de six phases clés d'un plan de réponse aux incidents:

\begin{itemize}
  \item \textbf{Préparation}: préparer les utilisateurs et le personnel informatique à gérer les incidents potentiels en cas de survenance;
  \item \textbf{Identification}: déterminer si un événement peut être qualifié d'incident de sécurité.
  \item \textbf{Confinement}: limiter les dommages de l'incident et isoler les systèmes affectés pour éviter d'autres dommages;
  \item \textbf{Éradication}: recherche de la cause première de l'incident et suppression des systèmes affectés de l'environnement de production;
  \item \textbf{Récupération}: autoriser les systèmes affectés à réintégrer l'environnement de production et garantir qu'aucune menace ne subsiste.;
  \item \textbf{Leçons apprises}: Remplir la documentation de l'incident, effectuer une analyse pour tirer des leçons de l'incident et potentiellement améliorer les efforts d'intervention futurs.
\end{itemize}

%- - - - - - - - - - - - - - - - - - - - - - - - - - - - - - - - - - - - - - - - - - - - - - - - 
\subsection{Parcours}

% Source https://en.m.wikipedia.org/wiki/Computer_security_incident_management

La gestion de l'incident au quotidien 

Réagir 
	Mécanismes et processus
	compétences
	outils

Enquêter

	Mécanismes et processus
	compétences
	outils

Anticiper

Organiser sa réponse (moyens et compétences)

      l'intégration dans ITIL

(aller plus loin) Organiser  processus de réponse : CSIRT







