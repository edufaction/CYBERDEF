%-------------------------------------
% Chapitre
% Vulnerability Management
% Introduction
% File : chap-Vulman-into.tex
%--------------------------------------


\section{Identifier et corriger les fragilités}

La notion de  fragilités de l'entreprise est à prendre au sens large. Elle comprends les vulnérabilités, humaines, organisationnelle et technique mais aussi la sensibilité à des scénarios d'attaques.

\subsection{Détecter les fragilités de l’entreprise}

La première tâche de fond pour une équipe cybersécurité est  d'identifier de fragilités de l'ensemble de l'environnement numérique \footnote{systèmes d'information de l'entreprise, services dans les cloud, réseaux sociaux...} de l'entreprise. Elle s'inscrit dans la dynamique de  l'\textbf{anticipation} avec la recherche de fragilités ou de risques cyber dans l'entreprise et leur correction. Elle précède généralement la \textbf{détection} d'évènement à risque, d'attaques, de déviance dans l'environnement mais aussi à l'extérieur du périmètre de l'entreprise.
Elle se positionne comme une activité  qui peut déclencher des mécanismes de la  \textbf{réaction} aux incidents, de la gestion de crise, par la nécessaire remédiation en cas de vulnérabilité critique.

% Begin PRZ ===========================
\begin{frame}
\frametitle<presentation>{Identifier ses vulnérabilités}
% end header PRZ =======================
On peut distinguer deux grandes typologies d'actions pour identifier ces fragilités :
\begin{itemize}
	\item  l'audit  de sécurité, qui permet de détecter des fragilités exploitables. Ce type d'audit peut se dérouler sous la forme de scénario exécuter par des équipes de "tests d'intrusion" soit sous la forme de campagne exécuter avec des scanners de vulnérabilités.
	\item la veille en vulnérabilités associée à la cartographie de l'environnement technique qui permet de déclencher une alerte de sécurité si une vulnérabilité apparaissait sur un des produits, services ou logiciel surveillé.
\end{itemize}
\end{frame}
% end PRZ ===========================

La difficulté principale de ces activités est de bien définir les périmètres techniques et de responsabilités sur lesquelles elles portent.

L'audit de sécurité permet d'évaluer les fragilités des éléments (composants) de l'entreprise en ce mettant dans la peau de l'attaquant, afin de découvrir les scénarios potentiellement actifs sur l'environnement digital de l'entreprise.

\subsection{Anticiper et surveiller les menaces }

Comme nous l'avons vu, une grande partie des attaques sur l'entreprise est liée à l'exploitation de fragilités de celle-ci. 

% Begin PRZ ===========================
\begin{frame}
\frametitle<presentation>{Exploitation des vulnérabilités}
L'exploitation de ces fragilités, sont de deux grandes natures.
% end header PRZ =======================
\begin{itemize}
	\item attaques exploitant de manière \textbf{opportuniste} des fragilités cataloguées et sans ciblage particulier de l'attaqué;
	\item attaques \textbf{ciblées} exploitant de manière spécifique des fragilités connues mais pas corrigées ou des fragilités non encore connus par les défenseurs.
\end{itemize}
\end{frame}
% end PRZ ===========================

On trouvera dans le chapitre \ref{CHAP_GESTVUL}, une description plus précise de ces notions de vulnérabilités connues et non connues. Les menaces sont généralement des scénarios, des codes malveillants, des mécanismes d'agression  ...
Le principe de gestion de la menace relève de la même dynamique de gestion que celle liée aux vulnérabilités. 


