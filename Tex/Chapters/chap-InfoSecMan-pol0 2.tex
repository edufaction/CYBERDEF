\section{Politiques de sécurité}

Grace aux outils d’analyse de risque, nous avons pu définir les biens essentiels de l’entreprise vu sous différents aspects :
\begin{itemize}
\item Continuité d’activité;
\item Protection du patrimoine informationnel et des données personnels;
\item Protection de l’image et de la réputation;
\end{itemize}

Globalement, ces critères définissent le niveau de confiance que peuvent avoir les clients, et les partenaires dans l’entreprise.

La gouvernance de la sécurité, est un ensemble de responsabilités et de processus permettant de maintenir au quotidien le niveau de sécurité d’une organisation, de réagir au plus vite, mais surtout de construire l’ensemble des mécanisme qui vont permettre à l’entreprise de construire la confiance.

Comme on le dit souvent la confiance ne se décrète pas elle se construit, et il faut tres peu pour la rompre.

un premier élément pour la confiance, s’appelle la certification ou la labélisation. 



