%-------------------------------------------------------------
%               FR CYBERDEF SECOPS COURSE
%                      INCIDENT MANAGEMENT
%
%                                    Investigation
%
%                              2020 eduf@ction
%-------------------------------------------------------------

%========================================================
% 											REAGIR AUX INCIDENTS
%-----------------------------------------------------------------------------------------
%==================================================
\section{ENQUETER}
%--------------------------------------------------------------------------------
%- - - - - - - - - - - - - - - - - - - - - - - - - - - - - - - - - - - - - - - - - - - - - - - - 
\subsection{Analyse de l'attaque}
Les enquêtes sur les intrusions visent à vérifier les modes d’attaque à l’œuvre dans les cyber-incidents, à déterminer
les activités réseau postérieures aux événements, et à détecter les points terminaux et les comptes utilisateurs additionnels qui ont été compromis. Il est essentiel à la tenue d’une enquête sur une intrusion de tenter de comprendre l’étendue potentielle d’un incident.

\subsection{Evaluation détaillée des dommages}

Les évaluations des dommages consistent pour l’essentiel à identifier les données qui ont été infiltrées ou exposées, ainsi qu’à tenter de comprendre les motivations des cyber-adversaires et la suite possible des événements. Les évaluations peuvent mettre en lumière des enjeux qu’il importe de soulever et renseigner votre entreprise sur les conséquences éventuelles de la perte, de la fuite ou de l’exfiltration de données.



\subsection{Se préparer et s'entrainer}


\subsection{Forensic}

\upicture{../Tex/Pictures/img-incidents}{Incidents}{1}{lbl_incident}



