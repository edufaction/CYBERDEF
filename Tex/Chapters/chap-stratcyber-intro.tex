%-------------------------------------------
% Chapitre
% Stratégie Cyber
% Intro
% File : chap-stratcyber-intro.tex
%-------------------------------------------




Pour être prêt à faire face aux cyber-incidents inévitables, il ne suffit pas de se préparer à simplement réagir pour neutraliser une attaque isolée. Cela nécessite la capacité d’intervenir efficacement et de manière répétitive pour planifier pro-activement, défendre énergiquement vos systèmes et vos actifs informationnels vitaux, devancer l’évolution des menaces, et assurer une reprise complète des activités après
les attaques.
Dans un contexte où les cyber-attaques grèvent de plus en plus les résultats financiers et ternissent la réputation des grandes sociétés, la mise sur pied d’une solide capacité d’intervention en cas de cyber-incident (ICC) devient impérieuse pour les entreprises qui tiennent à sauvegarder leur sécurité, leur vigilance et leur résilience. Une solide capacité d’intervention en cas de cyberincident peut aider votre entreprise à faire ce qui suit :
Comprendre rapidement la nature d’une attaque pour mieux faire face aux questions quoi, où, comment et combien, et y répondre
Réduire le plus possible les coûts – en temps, en ressources et en perte de confiance des clients – associés à la perte de données
Instaurer un niveau accru de gestion et de contrôle pour renforcer les TI et les processus opérationnels et, ainsi, pouvoir vous concentrer sur vos activités de base génératrices de valeur

\section{Orchestration}

CyberDefenseMatrix  \cite{dutta2019cyber}

\section{Cyberrange}



\subsection {les grandes fragilités des infrastructures}

La résilience est dépendante de composantes 
\begin{itemize}
  \item Sur l'identité : Les annuaires : (AD ...)
  \item Sur l'infrastructure de routage (DNS, AS ..)
\end{itemize}

\subsection {Dceeptive sécruity}
Mitre Acttack


