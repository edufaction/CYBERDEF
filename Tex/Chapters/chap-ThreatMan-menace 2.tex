%-------------------------------------
% Chapitre
% Threat Management
% Intro
% File : chap-ThreatMan-menace.tex
%--------------------------------------
\uchap{Introduction à la menace, chapitre 3.2}
%_-_-_-_-_-_-_-_-_-_-_-_-_-_-_-_-_-_-_-_-_-_-_-_-_-_-_


\section{Analyser la menace}

% recherche les typologies de menaces

% La veille sur la menace et cela conduit à irriguer les systèmes à base d'iOC pour les systèmes technioques, et les scénario de risques<;

\subsection{les menaçants, sources de menaces}

% lister les menaces, avec les intentionnalités.
% faire une liste des intentionnalités

\subsection{lUser and Entity Behavior Analytics}
Analyses comportementale

\section{Threats Intelligence}


\section{Threats Hunting}

%TODO : https://securityintelligence.com/a-beginners-guide-to-threat-hunting/

Le threathunting, est au \UKword{threat management}  ce que la chasse à la vulnérabilité du bug bounty est au vunérability management

Le SANS Institute a publié en 2017, des études sur la qualité de la chasse aux cybermenaces dans l'entreprise


Initial - At Level 0 maturity, an organization relies primarily on automated reporting and does little or no routine data collection.
Minimal - At Level 1 maturity, an organization incorporates threat intelligence indicator searches. It has a moderate or high level of routine data collection.
Procedural - At Level 2 maturity, an organization follows analysis procedures created by others. It has a high or very high level of routine data collection.
Innovative - At Level 3 maturity, an organization creates new data analysis procedures. It has a high or very high level of routine data collection.
Leading - At Level 4 maturity, automates the majority of successful data analysis procedures. It has a high or very high level of routine data collection.




