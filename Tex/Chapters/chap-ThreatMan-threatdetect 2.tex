%-------------------------------------------------------------
%               FR CYBERDEF SECOPS COURSE
%              $Chapitre : Threat Management
%                         $theme : Threat Detection
%      $File :  chap-ThreatMan-threatdetect.tex
%                             2020 eduf@ction
%-------------------------------------------------------------
\uchap{chap-ThreatMan-threatdetect.tex}
%-------------------------------------------------------------


\section{DETECTER les attaques}

La détection et la surveillance de système d'information d'entreprise, passent souvent pas l'utilisation des informations qui remontent déjà des services de protections périmétriques comme les Firewall (Réseau, Web application...),  proxy,  routeurs, IPS/IDS. Tous les équipements de sécurité installés sur un environnement sont des sources d'évènements qui permettent pour peu qu'elles soient analysés, de détecter des attaques ou le déploiement de menaces variées (comportements déviants, flux d'APT, flux de scans, ...)
Ces sources peuvent être complétées par tout équipement ou système  produisant des journaux d'activité (LOGs).

Par ailleurs, certains équipement dédiés de \g{surveillance} appelés \g{sondes} sont aussi aptes à remonter des évènements de détection sur la base de \g{signatures} ou \g{algorithmie} dont des IOC (\g{indice of compromission}).

\begin{techworkbox}{Sondes souveraines}
La Loi de Programmation Militaire (LMP) appelle les OIV à s’équiper de sondes dites souveraines, qualifiées par l’ANSSI et  disposant d’un niveau de sécurisation élevé. C'est un sujet technique intéressant pour  \fichetech. 
\end{techworkbox}

Dans cette partie nous allons traiter donc de la détection de menaces sur la base  de différentes techniques de détections et de surveillance :

\begin{itemize}
  \item Logs et sources d'évènements; 
  \item Corrélation d'évènement avec les SIEM;
  \item Évènements provenant des terminaux.
\end{itemize}

et nous explorerons quelques éléments d'organisation autour de la détection et de la surveillance.

