%-------------------------------------
% Chapitre
% Vulnerability Management
% Sec By Design
% File : chap-Vulman-secbydesign.tex
%--------------------------------------
\uchap{Vulnérabilités et SEC By DESIGN (VULMAN-INTRO)}

\section{ANTICIPER et construire solide}

Nous n'allons pas développer les concepts de \UKword{Security By design} qui  englobent de nombreuses thématiques de la sécurité applicative.
Les applications en particulier web et mobiles, de par leur complexité et les temps (et parfois budgets) restreints alloués à leur cycle de développement, contiennent souvent un grand nombre de vulnérabilités.

Comme nous l'avons abordé, tester de manière automatisée les vulnérabilités dans du code applicatif développé se décompose en deux typologies de tests :

% Begin PRZ ===========================
\begin{frame}
\frametitle<presentation>{SEC By DESIGN}
% end header PRZ =======================
\begin{itemize}
  \item  \textbf{Audit de code source automatisé} (SAST - Static Application Security Testing). L’audit du code source (SAST) des applications est important si vous souhaitez détecter et corriger leurs vulnérabilités pendant la phase de développement car en effet plus tôt une vulnérabilité est découverte et moins elle sera coûteuse à corriger.
Un audit SAST est non intrusif par nature. Vous pouvez donc scanner en toute sécurité vos applications les plus critiques sans risque d’impacter leur performance.
  \item   \textbf{Audit dynamique automatisé }(DAST - Dynamic Application Security Testing). Un audit dynamique (DAST) consiste à se servir d’un scanner pour interagir avec l’application (avec des requêtes malicieuses vers l’application auditée)  afin d’y trouver des failles connues.
Un scanner de vulnérabilités DAST est plus à même de détecter des erreurs de configuration au serveur web sur lequel est installée l’application.

\end{itemize}
\end{frame}
% end PRZ ===========================

Toutefois parmi celles-ci se trouve la recherche de vulnérabilités au sein des applications dès la phase de conception, et dans les phases de test et de production. Le terme APPSEC (\UKword{Application Security}) est souvent utilisé.

%\section{CORRIGER les vulnérabilités}

\section{Produits de confiance}

\subsection{Les critères communs} \label{chap:ISOCC}

\utodo

\utocomplete

% la correction des vulnérabilités dasn le SEC by design, introduduire les concepts de feeb, de sécurity chapion dans les équipês de DEV. SCRUM master 