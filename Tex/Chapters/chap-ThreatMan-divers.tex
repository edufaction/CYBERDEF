%-------------------------------------------------------------
%               FR CYBERDEF SECOPS COURSE
%              $Chapitre : Threat Management
%                         $theme : Divers
%              $File : chap-ThreatMan-divers.tex
%                             2020 eduf@ction
%-------------------------------------------------------------
\uchap{chap-ThreatMan-divers.tex}
%-------------------------------------------------------------

\section{Stratégie de cyberdéfense et de surveillance}

Ce chapitre en construction donne quelques éléments complémentaires sur les stratégies et techniques de cyberdéfense d'entreprise. 

%\subsection{Techniques de défense cyber}

\subsection{Technique de déception (Deceptive Defense)}

 D'une protection statique en profondeeur encore trop souvent périmétrique, vers une détection résiliente pour désormais envisager des logiques de contre-attaques dynamiques, les stratégie de cybersécurité évoluent. L'échelle de temps entre l'occurrence d'une attaque, sa détection et son élimination est un marquant significatif du risque : de plusieurs mois à quelques jours,  de jours à quelques heures, l'enjeu est  de nos jours  d'agir en temps réel contre l'attaquant. 
 Le leurrage numérique (deceptive security) est au cœur de nouvelle stratégie de  cybersécurité. Il s'agit de retourner la dynamiques  de l'attaquant en cherchant à le tromper pour tenter de le demasquer, ou de le le dissuader de continuer de part le risque d'être découvert qu'il prend. Le leurrage numérique eput relever d'une forme de dissuasion cyber.

\subsubsection{Honeypots}

Les pièges « honeypots » sont un leurre pour les attaquants, en imitant une ressource de calcul réel (par exemple, un service, une application, un système ou des données). Toute entité entrée en connexion à un « honeypot » est alors considérée comme suspecte, et son activité est surveillée pour détecter une malveillance.
L'arsenal du leurrage défensif est historiquement basé sur les pots de miel (honeypots). Ceux-ci reposent sur l'analyse statique d'écart de composants par rapport à un comportement connu et sain. Ces technologies se heurte à deux problèmes : le passage à l'échelle pour couvrir la diversité et la complexité des systèmes numériques, et la génération excessive de faux positifs. Les honeypots évoluent pour devenir des pièges actifs qui sont disséminés dans l'environnement réel pour mieux cerner les stratégies de l'attaquant. Les architectures de déploiement des leurres se spécialisent selon le domaine d'application (systèmes d'information, systèmes industriels, finance, médical...) ou en fonction des composants ciblés par les attaques (serveurs, pare-feux, antivirus...) ou encore par rapport à la charge offensive (malwares...). Les leurres tendent à générer de vrais positifs en temps réel. Leur efficacité repose sur deux propriétés, l'une inhérente aux composants de sécurité, la non-compromission, et l'autre caractéristique de l'attaque : la furtivité.  
Cette nouvelle génération de leurres numériques enrichit les stratégies d'investigations au sein des centres opérationnels de sécurité (SOC).  Ainsi, des logiques de raisonnements déductifs (déterministes) ou inductifs (hypothétiques) se confrontent pour caractériser finement le mode opératoire des attaquants en le resserrant si possible jusqu'à l'attribution de l'attaque. 
Le caractère actif de ces nouvelles technologies de leurre soulève toutefois de nombreuses  interrogations su le plan éthique et  réglementaires (respect de la vie privée en particulier).


Le leurrage numérique devient une composante essentielle d'une lutte informatique défensive et contribue de plus en plus au processus des scénarios de réponses, ripostes et d'escalade.

\begin{warningbox}{\g{Deceptive Defense} en cyberdéfense}
Les techniques de déception en cyberdéfense sont en pleine évolution. C'est un sujet parfait pour des fiches TECHNOs avec les différents thèmes  :
\begin{itemize}
  \item le leurrage numérique : honeypots, leurres, pièges
  \item 	les architectures de déploiement de leurres selon les domaines d'application
  \item 	la spécialisation de leurres pour les services, pour la sécurité, contre les malwares...
  \item 	les propriétés du leurrage numérique :  non-compromission,  furtivité...
  \item 	l'apport à l'investigation numérique : raisonnements déductifs/inductifs, caractérisation des attaques, attribution...
  \item 	la contribution à la lutte informatique défensive : scénarios de ripostes et d'escalade.
  \item 	le positionnement du leurrage dans les modèles d'attaques (MITRE ATT@CK,...), par rapport à la caractérisation des attaques (CAPEC...) et plus généralement son apport à la connaissance du risque cyber (cyber threat intelligence – CTI)
  \item 	le leurrage et la réglementation (NIS, RGPD,...).
\end{itemize}

\end{warningbox}

%\subsection{Data Lake}
%\subsection{La SecOps par secteur industriel}








