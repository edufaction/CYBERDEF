%********************************************
% $Chapitre : Threat Management
% $theme : DIVERS
% $File : chap-Vulman-divers.tex
%********************************************
\uchap{chap-Vulman-divers.tex}
%********************************************



\section{Au delà de la gestion de la menace}

\subsection{Techniques de défense cyber}

\begin{warningbox}{Stratégies de cyberdéfense}
Les techniques et stratégies de combats cyber sont en pleine évolution. Ce sont des sujets adaptés aux fiches TECHNOs.
\end{warningbox}


\subsection{Technique de déception (Deceptive défense)}

\subsubsection{Honeypot}

Les pièges « honeypots » sont un leurre pour les attaquants, en imitant une ressource de calcul réel (par exemple, un service, une application, un système ou des données). Toute entité entrée en connexion à un « honeypot » est alors considérée comme suspecte, et son activité est surveillée pour détecter une malveillance.

%\subsection{Data Lake}

