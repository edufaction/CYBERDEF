 
%----------------------------------------------------------------------
% 		S E C T IO N   PEDAGO
%----------------------------------------------------------------------
\section{Objectifs pédagogiques}
Il me semblait important d'apporter au lecteur un peu d'information autour des éléments pédagogiques de c\ecours. Vous trouverez donc dans ce chapitre quelques éléments sur les compétences, les métiers, le positionnement des activités de la cybersécurité.
En effet, c\ecours tente d'être une introduction à la \etitle permettant à des acteurs du digital n'ayant pas ou peu de connaissance du domaine de repérer dans ce domaine à large spectre d'activités et de métiers.

Nous y abordons aussi les limites de c\ecours ainsi que des recommandations pour aborder le contenu avec plus de facilité pour ceux moins familiers du monde de l'informatique et des réseaux.

\subsection{Les compétences à acquérir}
A l'issue de c\ecours, vous devriez être en mesure de comprendre les mécanismes qui contribuent à la mise en place d'une organisation de cyberdéfense d'entreprise avec les grandes capacités nécessaires.  Pour les réaliser avec efficacité, il est nécessaire de positionner ces activités au sein des fonctions sécurité plus large. Les compétences acquises sont de diverses natures, mais globalement vous devriez être en mesure à un niveau de gouvernance et de pilotage de :  

\begin{itemize}
	\item Analyser les risques numériques pesant sur l'entreprise ou l'organisation ;
	\item Mesurer le niveau de sécurité de de l'environnement ;
	\item Auditer, conseiller, accompagner le changement ;
	\item Mettre en place une gouvernance efficace dans le domaine de la cybersécurité ;
	\item Déployer une politique de sécurité informatique et de cybersécurité et appliquer des méthodologies efficaces de renforcement et d'aguerrissement ;
	\item Comprendre l'intégration des solutions de sécurité suite à l'analyse de risque ;
	\item Gérer des situations d'incident pouvant aller à la crise cyber.
\end{itemize}
La complexité de l'entreprise, sa taille, sa dynamique de prise en compte des enjeux sécurité, sa culture, l'adhérence ou non aux technologies de l'information nécessitent le plus souvent des projets spécifiques adaptés et très contextualisés. Des sociétés de services assistent les entreprises pour auditer, construire, maintenir la sécurité de l'entreprise. Ce document a aussi pour objectif de fournir au lecteur des clefs de lecture pour encadrer et piloter de telles prestations dans le contexte de l'organisation. 


\subsection{Métiers et compétences } 
Il est complexe d'identifier les métiers de la cybersécurité vers lesquels ces compétences peuvent conduire. Il existe plusieurs modèles permettant de classer les métiers de la cybersécurité, et les compétences associées. Pour ma part,  j'ai retenu un modèle que j'ai proposé dans le cadre d'une GPEC (Gestion des emplois et compétence) dans chez un opérateur de services de cybersécurité. Ce modèle est centré sur une classification des outils technologiques utilisés par l'expertise. Issue plutôt de l'expérience, il ne reflète pas les dénominations des différents métiers ou fiche de poste que l'on trouve dans le domaine mais se centre sur les technologies de sécurité vu du côté des opérationnels. Ceci permet de décliner 5 grands domaines d'activité.

\upicture{../Tex/Pictures/img-metiers}{les  grands domaines de métiers}{0.9}{lbl-metiers}

Il y a en effet une grande différence de métier, de compétences entre un spécialiste de la gestion des accès  qui conduira l'intégration de système d'IAM  \footnote{Identity et Access Management} et un ethical hacker qui devra recherche des scenarii d'attaques potentielles sur un système. 

\upicture{../Tex/Pictures/img-metiers-risk2crisis}{les quelques grandes zones de métiers}{0.9}{lbl-risk2crisis}

Au delà de ces grands métiers du service, il est possible de positionner dans le cycle de vie des systèmes différents métiers de la cybersécurité. Les cultures, les objectifs, les technologies utilisées sont différentes mais concourent à la même finalité de protection de l'entreprise.

\upicture{../Tex/Pictures/img-metierslist}{les métiers dans le cycle de vie}{0.9}{lbl-metiersall}


%TODO : Métiers de l'intégration, des opérations, ...

% De l'importance de la TAXONOMIE, POSTE, METIER et COMPETENCE


Si vous souhaitez connaitre avec plus de détails les compétences nécessaires pour les métiers de la sécurité vous pouvez consulter deux grands sites de référence comme celui de l'\head{ANSSI} des 
 \link{https://www.ssi.gouv.fr/particulier/formations/profils-metiers-de-la-cybersécurité/}{métiers de la cybersécurité} ou celui du NIST sur le référentiel  \link{https://www.nist.gov/itl/applied-cybersecurity/nice/resources/nice-cybersecurity-workforce-framework}{NICE Cybersecurity Workforce Framework}


\subsection{Compétences et certifications}

Se former en cybersécurité, c'est pour celui qui travaille avec vous une certaine garantie de compétences. Dans le domaine de la Cybersécurité, la confiance dans les compétences d'un acteur du domaine se base dans le domaine des services en particulier sur la certification. Dans ces certifications, formes de perfectionnement dans un métier, on trouve généralement des  certifications EDITEURS (liés à des produits de sécurité), et des certifications d'associations professionnelles.

Cette dynamique de certification est une manière de compléter les formations initiales et sont  assez différentiante sur un CV dans le monde de l'entreprise en particulier celles qui travaillent dans un environnement international.



\subsection{Certifications éditeurs}

Nous verrons dans le chapitre sur les architectures de sécurité, les produits et services technologiques de sécurité. Une grande partie des fonctions de sécurité techniques est opérée par des produits (Logiciels, Appliances, Services Saas ...). La complexité de ces produits nécessite une formation spécifique pour en exploiter toute la richesse fonctionnelle.
Ces certifications sont par ailleurs souvent obligatoires pour travailler dans les métiers de l'intégration car elle permettent d'accéder au support des éditeurs. A titre d'exemple, nous pouvons citer deux acteurs connus qui disposent de mécanismes et programme de certifications à leurs produits. Ces certifications peuvent par ailleurs être délivrées par des tiers.


\head{Pour CISCO} \link{https://www.cisco.com}{Certifications de carrière CCNA, CCDA} 
%https://www.cisco.com/c/fr_ca/training-events/career-certification


\head{Pour Microsoft} \link{https://www.microsoft.com/fr-fr/learning/certification-overview.aspx}{Certifications }  pour  développeurs, Administrateurs, Architectes Solutions, Consultants.

\subsection{Certifications professionelles} 

La validation d'expertise par des certifications professionnelles est assez répandue dans le milieu de cybersécurité et en particulier dans les pays anglo-saxons. De nombreuses certifications existent, portées par des associations professionnelles, des groupes d'experts ou des entreprises de référence. Ces certifications nécessitent le plus souvent en plus de l'examen des années d'expérience et de pratiques prouvées.


% parler du fait qu'il faut des années d'experience et maintenir par des preuves

\head{\link{https://www.isc2.org/Certifications}{ISC}}  \UKword{the International Information System Security Certification Consortium} délivre des certifications reconnues et d'excellent niveau de reconnaissance.
Les deux principales sont :
\begin{itemize}
  \item CISSP : Certified Information Systems Security Professional
  \item SSCP : Systems Security Certified Practitioner
\end{itemize}


\head{\link{https://www.isaca.org/}{ISICA} } IT Audit, Security, Governance and Risk 

sous le nom de \UKword{Information Systems Audit and Control Association} cette association professionnelle existe depuis 1967, connue pour sont support à COBIT elle propose plusieurs certifications réclamée par les clients. 


\begin{itemize}
  \item CISA: Certified Information Systems Auditor
  \item CISM: Certified Information Security Manager
 \item CGEIT: Certified in the Governance of Enterprise IT
  \item CRISC: Certified in Risk and Information Systems Control
\end{itemize}

\subsection{Certifications Hacking} 

Il nous faut citer deux certifications très utilisées cans les métiers techniques de la cybersécurité et accessible sans expérience professionnelle à prouver.

\head{SANS Institute} (SysAdmin, Audit, Network, Security) et le GIAC \link{https://www.giac.org}{(Global Information Assurance Certification) }

\begin{itemize}
  \item Cyber Defense
  \item Penetration Testing
  \item Incident Response and Forensics
  \item Management, Audit, Legal
  \item Developer
  \item Industrial Control Systems

\end{itemize}


\includer{inc-certifs-hacking.tex}


%----------------------------------------------------------------------
% 				S E C T IO N   Structure 
%----------------------------------------------------------------------
\section {Structure pédagogique du cours}
Nous avons abordé le cours sur cheminement basé sur trois pivots :

\begin{itemize}
\item Pivot \head{RISQUES} : Pour défendre son espace  cyber  c'est-à-dire l'ensemble des produits, services, matériels, données utilisateurs utilisés par l'activité  économique de l'entreprise il faut non seulement que celui-ci soit identifié mais que les risques portant sur les éléments le constituant soient aussi clairement et consciemment pris en compte. C'est sur la base d'analyses des risques que sont construits les objectifs de sécurité d'un système. Il est bien entendu que de nombreux systèmes préexistent à une analyse de risque et que les objectifs de sécurité ayant conduit à la construction sont issus de  la sédimentation dans le temps de choix technologiques qui ne sont, par ailleurs rarement formalisés. Ainsi on remarque, que l'activité du l'évaluation des risques, ce que appelle en anglais  risk management  est porté plutôt par le domaine d'activités dénommé  information Security management  ou INFOSEC dans les pays anglo-saxons, mais que nous pouvons traduire  management de la sécurité de l'information.
\item Pivot  \head{ARCHITECTURE du SI}. Architecture de sécurité, défense en profondeur, politique de sécurité, usage du SI, IAM. L'analyse sera faite à partie des Politiques de sécurité pour construire ou améliorer la cybersécurité de l'entreprise. Définir des objectifs de sécurité relatifs aux risques,  positionner les politiques de contrôle, decfiltrage, et de gestion sur l'environnement  informationnel  de  l'organisation pour garantir la protection et la confiance sur les actifs sensibles. 
\item  Pivot  \head{MAINTIEN EN CONDITION DE sécurité}. Malgré toutes les précautions pour mettre en confiance un système d'information, il est illusoire d'une part de vouloir tout protéger, mais aussi de penser que les mécanismes de  protection résisteront à toutes les agressions. C'est donc en continu qu'il est nécessaire de veiller à la menace, de vérifier que de nouvelles fragilités n'apparaissent pas, de réagir au plus vite en cas de suspicion d'attaque ou de compromission.  Ce sécurité continue, dite dynamique est à la base du maintien en condition de sécurité de l'environnement digital de l'entreprise. 
A titre indicatif, on peut rapidement donner une matrice des classes de métiers associées à chaque pivot. Ceci permettra au lecteur de se focaliser peut être sur un chapitre qui le concerne un peu plus dans son quotidien.
\end{itemize}



%TODO  ajouter une description des classes d'activités 
% Image sur le cycle de vie

\begin{nota}[Les limites de l'exercice]
C\ecours est essentiellement une introduction à la cybersécurité sur son volet gouvernance (politiques et stratégies). Il permet de mettre en perspective les choix techniques, tant de protection et de défense face à une réalité économique, qui nécessite d'adapter protection et défense au niveau de risque. La décomposition sur ces 3 axes est un parti-pris qui évidement ne couvre pas dans le détail, l'ensemble des processus et actions du domaine de la cybersécurité. 
\end{nota}

Vu du côté du responsable sécurité, et donc des compétences acquises : Le RSSI se doit de  maitriser les risques de son SI vis à vis des conditions de sécurité, il est un auditeur en mesure : 

\begin{itemize}
  \item d'analyser les risques à partir de l'analyse des enjeux de l'entreprise, de ses actifs, de son existant, de la menace inhérente ou non à son entreprise ;
  \item de les traiter, les accepter ou pas, 
  \item de proposer les objectifs de sécurité à déployer pour construire les mesures de sécurité.
  \item Ceci conduit à l'objectif professionnel de cette partie :  Savoir comment démarrer la prise en compte de la sécurité des systèmes d'information dans une entreprise .  Il trouvera donc de bons outils théoriques et pratiques dans l'ISO 27005.
\end{itemize}


\begin{nota}[Dynamicité des risques]
Un RSSI ou son équipe conduit les analyses vis à vis de la menace. Il peut être conduit à lancer des audits.  Les mesures issus de ces audits permettent de définir sur les mesures en cours sont faibles, inutiles, vulnérables vis-à-vis des objectifs de sécurité. C'est ainsi qu'il est possible de conduire des analyses de risques sur des systèmes existants et de vérifier si les mesures actives sont compatibles avec les objectifs. On peut aussi constater qu'à ce titre une analyse de risque n'est pas figée dans le temps car les menaces ainsi que la sensibilité des actifs évoluent.
\end{nota}


Le RSSI se doit de maitriser les politiques de sécurité des systèmes d'information,  la PSSI étant le modèle de référence de façon à :
\begin{itemize}
\item planifier et produire ces conditions de sécurité ;
\item les adapter à l'entreprise ;
\item les mettre en œuvre au travers d'une architecture de sécurité propre à l'entreprise ;
\end{itemize}

Le lecteur trouvera un référentiel global dans l'environnement de l'ISO 27001 pour travailler autour du système de management de la sécurité.

Au delà de la gouvernance classique que l'on dit \g{de protection} de la cybersécurité d'entreprise qui se veut un moyen de deployer des mesures de sécurité (préventives, de formation, d'architecture), la sécurité opérationnelle apporte un nouveau lot de mesures et d'outillages liés à l'anticipation, la détection et la réponse aux attaques.

\begin{nota}[sécurité Opérationnelle] Lutte informatique défensive, sécurité dynamique, 
Cyberdéfense : plusieurs terminologies se côtoient pour évoquer des concepts, techniques, mesures, et méthodes souvent proches. 
\end{nota}


\subsection{Structure du cours}

L\ecours est donc organisé en 3 temps. Chaque temps est un module qui structure l'ensemble des éléments présentés dans le programme de l'unité d'enseignement dans une dynamique associée à la forme d'enseignement à distance et structurée autour de 3 cours issus des retours d'expérience d'experts du domaine de la Cybersécurité.  
\begin{itemize}
\item \head{Temps 1} : De l'analyse des risques à la déclinaison des objectifs de sécurité sur les essentiels de l'entreprise;
\item \head{Temps 2} : Des objectifs de sécurité à une politique de sécurité guidant et mesurant une sécurité implémentée (architectures et systèmes  de sécurité et sécurité des architectures et de systèmes d'information) ;
\item \head{Temps 3} : D'un système d'information \textbf{outillé, protégé et défendu} en matière de sécurité à une sécurité opérationnelle \textbf{maintenue, vigilante et  réactive}.
\end{itemize}

C\edoc regroupe de manière plus détaillée les éléments la 3ième partie de l'unité d'enseignement que je nommerai pour la suite dans ce texte   VAR : Veille / Alerte / Réponse ,  les deux premières parties sont toutefois résumés dans deux chapitres préliminaires, permettant de positionner la démarche VAR dans un contexte global.

\subsection{Pour s'engager plus rapidement}

Du point de vue pédagogique, il est important de noter que vous pouvez aller vous initier au domaine de la sécurité des systèmes d'information avec les travaux de l'ANSSI de la \link{https://www.ssi.gouv.fr/administration/formations/cyberedu/contenu-pedagogique-cyberedu}{Mallette CyberEDU}. Cette mallette de cours contient les basics pour aborder la cyberdéfense d'entreprise.

Ces travaux sont issus d'un marché public de réalisation avec l’Université européenne de Bretagne  (qui regroupe 28 établissements d’enseignement supérieur et de recherche) et Orange pour la réalisation de livrables à destination des responsables de formation et/ou des enseignants en informatique.

l’ANSSI met à disposition cette mallette pédagogique qui contient : un guide pédagogique, un cours préparé d’environ 24 heures sur l’enseignement des bases de la sécurité informatique, ainsi que des éléments de cours pour les masters en informatique (réseaux, systèmes d’exploitation et développement).
Ces documents, réalisés par le consortium et l’ANSSI, sont disponibles sur le site de l'ANSSI.
 

\subsubsection {Pour le niveau BAC +3}

Pour ce niveau la mallette contient un syllabus pour le cours de sensibilisation et initiation à la Cybersécurité ainsi que 4 modules de support de cours.

\begin{itemize}
	\item module 1 :  notions de base
	\item module 2  :  hygiène informatique
	\item module 3 :  réseau et applications
	\item module 4 :  gestion de la cybersécurité au sein d’une organisation
\end{itemize}
Un quizz est également à disposition pour permettre d’évaluer les compétences acquises au fur et à mesure de l’avancé des enseignements.

\subsubsection{Pour le niveau Bac + 5} 

Pour ce niveau, des fiches pédagogiques par domaine permettent de découvrir :
\begin{itemize}
	\item la sécurité des réseaux
	\item la sécurité des logiciels
	\item sécurité des systèmes
	\item l’authentification
	\item la cybersécurité au sein des composants électroniques
\end{itemize}


