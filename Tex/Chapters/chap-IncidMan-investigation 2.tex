%-------------------------------------------------------------
%               FR CYBERDEF SECOPS COURSE
%                      INCIDENT MANAGEMENT
%
%                                    Investigation
%
%                              2020 eduf@ction
%-------------------------------------------------------------

%========================================================
% 											ENQUETER
%-----------------------------------------------------------------------------------------
%==================================================
\section{ENQUETER}
%--------------------------------------------------------------------------------
%- - - - - - - - - - - - - - - - - - - - - - - - - - - - - - - - - - - - - - - - - - - - - - - - 
\subsection{Analyse de l'attaque}
Les enquêtes sur les intrusions visent à vérifier les modes d’attaque à l’œuvre dans les cyber-incidents, à déterminer les activités réseau postérieures aux événements, et à détecter les points terminaux et les comptes utilisateurs additionnels qui ont été compromis. Il est essentiel à la tenue d’une enquête sur une intrusion de tenter de comprendre l’étendue potentielle d’un incident.

\subsection{Evaluation détaillée des dommages}

Les évaluations des dommages consistent pour l’essentiel à identifier les données qui ont été infiltrées ou exposées, ainsi qu’à tenter de comprendre les motivations des cyber-adversaires et la suite possible des événements. Les évaluations peuvent mettre en lumière des enjeux qu’il importe de soulever et renseigner votre entreprise sur les conséquences éventuelles de la perte, de la fuite ou de l’exfiltration de données.


%\subsection{Se préparer et s'entrainer}


\subsection{Forensic}

%source : https://www.algosecure.fr/conseil/investigation-forensic

\subsection{Cadre juridique}

L'analyse forensique est une science qui s'intéresse à la recherche de preuves sur des supports numériques pour comprendre un comportement, remédier à un incident et aider à prendre des décisions éclairées. Ces preuves sont des traces, des artéfacts numériques qui fournissent des informations qui, mises bout-à-bout, permettent de dégager un scénario factuel d'évènements et d'apporter des réponses aux questions que peut se poser le demandeur. L'analyse forensique est encore appelée investigation numérique, digital forensics, inforensique ou informatique légale.


Dans une affaire judiciaire impliquant des supports numériques perquisitionnés pour des enquêtes, le juge peut faire appel à un expert judiciaire pour "faire parler" ces supports, pour l'aider dans sa prise de décision. L'expert judiciaire est une personne physique ou morale, professionnelle dans un domaine technique particulier, spécialement habilitée à exercer son expertise dans des dossiers judiciaires sur sollicitation d'un juge. Son avis ne s'impose pas au juge, qui reste libre dans l'appréciation des éléments fournis.


\subsection{Cadre non-juridique}

C'est ce que l'on retrouve le plus souvent avec les entreprises dans un contexte de réponse à incident à la suite d'une attaque du système d'information. Le cas plus fréquent est par exemple l'infection par un ransomware du parc d'une entreprise.

Dans ce type de situation aussi, il y a des sujets sur lesquels se focalisent les analystes en fonction des souhaits du client. Il peut s'agir de trouver qui a fait quoi, comment et quand, et à l'aide de ces éléments, comprendre comment contenir l'incident et surtout comment remédier à la situation au plus vite pour que l'activité reprenne sereinement, si possible.

Tout comme dans le cadre juridique, tout élément trouvé dans les investigations et qui est pénalement repréhensible est à communiquer au commanditaire. Il en est de même de tout écart par rapport à la charte informatique de l'entreprise.

\subsection{Outillage}

Il existe de nombreux outillages pour assister un analyste forensique.

Je engage à aller voir du côté du SANS, avec \link{https://digital-forensics.sans.org}{SIFT Workstation}

\mode<all>{\picframe{../Tex/Pictures/img-incidents}{Incidents}{0.8}{lbl_incident}}

