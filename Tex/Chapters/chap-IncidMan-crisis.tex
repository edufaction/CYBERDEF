%-------------------------------------------------------------
%               FR CYBERDEF SECOPS COURSE
%                      INCIDENT MANAGEMENT
%
%                          CRISIS MANAGEMENT
%
%                              2020 eduf@ction
%-------------------------------------------------------------

%==================================================
\subsection{Gestion de crises}
%--------------------------------------------------------------------------------


Les cyber-attaques sont généralement pilotées par les équipes IT, toutefois la majorité des crises associées à des incidents cyber interpellent toutes les activités de l'entreprise.   
%Or, et l’histoire récente l’a prouvé (NotPetya, GandCrab, Emotet, Clop…), elles impactent en réalité tous les corps de métier ainsi que la réputation et l’image de la structure ciblée. Toute crise cyber est nécessairement transverse et doit être gérée par des acteurs qui n’ont pas l’habitude de se coordonner. Il leur faut alors trouver un langage commun et poursuivre les mêmes intérêts. Ainsi, dans le cadre d’une gestion de crise cyber, aux équipes IT s’ajoutent presque toujours les équipes communication et juridiques, les ressources humaines, et les métiers concernés par l’évènement.
%
%
%L’une des spécificités de la crise cyber est qu’elle peut être le résultat de la détection tardive d’une cyber-attaque ayant pris naissance plusieurs mois auparavant. L’attaquant a progressivement pu disposer ses charges et les activer au moment opportun (congés ou veille de week-end par exemple).  L’une des qualités de l’organisation de crise est alors sa capacité de réaction face à l’évènement. Elle doit être capable de mettre en œuvre des mesures d’urgence visant à limiter les impacts, éviter le sur-accident et ainsi adopter une posture de défense. Elle peut également déployer des mesures conservatoires lui permettant de préserver ses actifs vitaux (Active Directory, sauvegardes…).
%Autre spécificité : lors d’une crise cyber, il demeure impossible, ou imprudent, de se fier aux moyens de communication habituels (Skype, mails…) car ceux-ci peuvent être hors service ou compromis.
%
%De ce fait, toute l’organisation du travail et des échanges doit être revue. Les impacts sont également forts sur les plans de continuité métiers. Télétravail et autres sites de replis utilisateurs n’ont plus beaucoup de sens. De nouvelles manières de faire que les parties prenantes du dispositif de gestion de crise doivent connaître avant son déclenchement.

\subsection{Anticiper les crises}

L’objectif de l’anticipation est de réduire le risque de survenue et de préparer en amont tout ce qui peut l’être. Cela permet de limiter au maximum les impacts négatifs et d’éviter le risque de sur-accident. Cela se met en place avec les mécanismes de PCA/PRA organisés autour par exemple de la norme ISO 22301.

%Ainsi, il est préconisé de réaliser en amont des mises en situation avec les équipes concernées. Elles pourront ainsi déterminer les mesures d’urgence, celles conservatoires et/ou celles de mises en quarantaine qu’elles pourraient actionner en temps voulu afin de réduire l’impact d’une cyberattaque.
%
%La préparation aide aussi à anticiper la tension que peut provoquer une crise, qu’elle soit cyber ou non. Le stress et la fatigue sont réels et parfois décuplés par l’incompréhension. Durant les premières heures, il est souvent difficile de savoir ce qu’il se passe exactement et d’identifier la/les source(s) de l’attaque et son scénario.
%
%La capacité des salariés à garder la tête froide, travailler ensemble et prendre les bonnes décisions rapidement réside dans le fait d’avoir appris en amont les bons gestes à adopter et d’être capables d’appliquer une méthode simple et efficace ; l’improvisation n’étant pas une option.
