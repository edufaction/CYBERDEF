%-------------------------------------------------------------
%               FR CYBERDEF SECOPS COURSE
%                      INCIDENT MANAGEMENT
%
%                                   Anticipation
%
%                              2020 eduf@ction
%-------------------------------------------------------------

%Sources Intressantes
%https://www.societybyte.swiss/2019/04/03/la-gestion-des-cyber-attaques-au-sein-dune-grande-entreprise-quels-en-sont-les-defis/



%==================================================
\section{ANTICIPER}
%--------------------------------------------------------------------------------

\subsection{Les bons reflexes}

Anticiper la réponse à incident, c'est s'organiser pour assurer rapidement un certain nombre de fonction d'action rapide :

\begin{itemize}
  \item Conduire une levée de doute rapide pour s'assurer qu'il ne s'agit pas de faux positifs.   \item Mobiliser une équipe d'investigation à fin de déterminer la cause de l'incident, d'identifier les vecteur d'infection et de propagation de l'attaquant,
  \item Qualifier les impacts immédiat et avenir, construire un plan de défense le cas échéant ;
  \item Mettre en sûreté des systèmes critiques (cœur de confiance serveur de sauvegarde, etc.) qui seront nécessaires pour assurer la survie métier en particulier en cas d'incident a un impact destructeur comme les rançons logiciel activer les plans de contournement et des procédures en mode dégradé 
  \item et enfin mobiliser surtout les bonnes compétences notifier les agences et les régulateurs déclencher le cas échéant les assurances Cyber.
\end{itemize}

Cela ne s'improvise pas, et nécessite d'avoir structurer sa réponse à incident.

%==================================================
% 											MANAGEMENT DES INCIDENTS
%--------------------------------------------------------------------------------
\subsection{Etablir un processus de management} 
%--------------------------------------------------------------------------------


% https://csrc.nist.gov/publications/detail/sp/800-61/rev-2/final

\subsubsection{Résilience}
 
 On ne peut pas parler de réponse à incident dans un contexte d'attaque informatique sans parler de résilience.
 
La cyber-résilience ou la résilience numérique est la capacité d’un système d’information à résister à une panne ou une cyber-attaque et à revenir à son état initial après l’incident, ou bien comme la faculté d’une structure quelconque à retrouver ses propriétés initiales après une altération significative. La notion peut s’appliquer aussi bien à un système physique ou un système d'information, qu’à un individu ou une organisation. 

Elle se traduit pour cette organisation par sa capacité de continuer à fonctionner et de résister à des agressions internes comme externes, volontaires ou non.

 Le niveau de résilience se mesure  avec des critères tels que la structure de l’organisation mise en place, les ressources humaines consacrées au fonctionnement du système, la redondance et le durcissement des systèmes et des équipements, les procédures en place, des compétences acquises à travers une formation et un entrainement dédiés, la connaissance fine de l’état de fonctionnement du système et la capacité à diagnostiquer une défaillance potentielle.
 
Sur le cyber-espace, la cyber-résilience implique donc de se préparer et de prendre  les mesures adaptées pour assurer le rétablissement d’un système. Par  ailleurs, dans le monde cyber dans l'entreprise l’incertitude peut régner dans l'usage :
\begin{itemize}
  \item  sur la sécurité des systèmes (virus déployés dont les effets ne sont pas maîtrisés), 
   \item l’intégrité du système n’est plus garantie, 
  \item l’activité du système peut être dégradée (données corrompues ou altérées) ou inopérante (communications inactives), 
  \item les risques de propagation des menaces sont augmentés si les interconnexions entre systèmes restent ouvertes, 
    
\end{itemize}

 Cette résilience peut  nécessiter des bascules vers des modes dégradés, avec l'isolation ou l'arrêt de certaines sous-systèmes. Ce type de décision nécessite de circuit de décision rapide et court.

C'est dans un cadre de cette continuité d'activité que se situent la majorité des référentiels de management de l'incident de de sécurité.

\begin{notebox}{Résilience et continuité d'activité}
La capacité d’un système  à résister à une panne ou une cyber-attaque et à revenir à son état initial après l’incident sera indifféremment appelé dans \ecours résilience ou continuité d'activité. 
\end{notebox}


\subsection{L'intégration dans la gestion des incidents ITIL}

ITIL (\g{Information Technology Infrastructure Library }pour \g{Bibliothèque pour l'infrastructure des technologies de l'information}) est un ensemble d'ouvrages recensant les bonnes pratiques (\g{best practices}) du management du système d'information. 

La Gestion des incidents vue du côté d'ITIL  inclut tout événement qui perturbe, ou pourrait perturber, un service. Ceci inclut les événements communiqués directement par les utilisateurs, via le Centre de services, une interface web ou autrement.
Ce processus appartient au sens ITIL à l'étape Service Operation(SO) du cycle de vie d'un SI.

Même si les incidents et les demandes de service sont rapportés au Centre de services, cela ne veut pas dire qu'ils sont de même type. Les demandes de service ne représentent pas une perturbation de service comme le sont les incidents. Voir le processus Exécution des requêtes pour plus d'information sur le processus qui gère le cycle de vie 

Les objectifs du processus de Gestion des incidents sont :

\begin{itemize}
  \item Veiller à ce que des méthodes et des procédures normalisées soient utilisées pour répondre, analyser, documenter, gérer et suivre efficacement les incidents.
  \item  Augmenter la visibilité et la communication des incidents à l'entreprise et aux groupes de soutien du SI.
  \item  Améliorer la perception des utilisateurs par rapport aux TI via une approche professionnelle dans la communication et la résolution rapide des incidents lorsqu'ils se produisent.
  \item Harmoniser les activités et les priorités de gestion des incidents avec ceux de l'entreprise.
  \item Maintenir la satisfaction de l'utilisateur avec la qualité des services du SI.
\end{itemize}

Généralement, cette gestion d'incident s'inscrit dans une chaine d'outillage avec des processus permettant de définir l'état ou le statut de l'incident.

\begin{itemize}
  \item \textbf{Nouveau} : un incident est soumis, mais n'a pas été assigné à un groupe ou une ressource pour résolution.
  \item \textbf{Assigné} : un incident est assigné à un groupe ou une ressource pour résolution.
  \item \textbf{En traitement }: l'incident est en cours d'investigation pour résolution.
  \item \textbf{Résolu} : une résolution a été mise en place.
  \item \textbf{Fermé} : la résolution a été confirmée par l'utilisateur comme quoi le service normal est rétabli.
\end{itemize}

On ne peut toutefois pas oublier, que la gestion de la sécurité dans une entreprise mature, doit s'intégrer aux processus IT de l'entreprise et de remarquer que certaines activités de sécurité peuvent aussi s'intégrer dans un respect du référentiel ITIL.

\begin{itemize}
  \item Le centre de services (service desk) cf le niveau 1 d'un \g{Security Operation Center};
  \item La gestion des incidents (incident management) ;
  \item La gestion des problèmes (problem management) ;
  \item La gestion des changements (change management) voir  les mécanismes de couverture de vulnérabilités (patch management par exemple);
  \item La gestion des mises en production (release management) ;
  \item La gestion des configurations (configuration management).
\end{itemize}

Dans ces processus le cycle de vie de l'incident suis un cycle connu et reconnu : 

\begin{itemize}
  \item \textbf{Identification} : détecter ou rendre compte d’un incident ;
  \item \textbf{Enregistrement} : les incidents sont enregistrés dans le système de gestion des incidents ;
  \item \textbf{Classement}  : les incidents sont classés par priorité ;
  \item \textbf{Priorisation} : l’incident est classé par ordre de priorité, sur la base de son impact et de son urgence, pour une meilleure utilisation des ressources et du temps disponible par l’équipe de support ;
  \item \textbf{Escalade}  : l’équipe de support doit-elle obtenir de l’aide de la part d’un autre service ? Si oui, on engage une procédure de demande de service sinon, la résolution de l'incident s’effectue au niveau du support initial.
  \item \textbf{Diagnostic}  : révélation du symptôme complet de l’incident ;
  \item \textbf{Résolution et rétablissement } : une fois que la solution est trouvée et que la correction est apportée alors l’incident est résolu ; La solution peut alors être ajoutée à la base des erreurs connues dans l'optique de résoudre plus rapidement un incident similaire dans le futur.
  \item \textbf{Clôture de l’incident}  : l’enregistrement de l’incident dans le système de gestion du management est clôturé en appliquant le statut « terminé » à celui-ci.
\end{itemize}


Les standards de gestion d’incidents (NIST 800-61 et ISO/IEC 27035) recommandent d’isoler les systèmes infectés directement après leur détection ce qui n'est pas toujours facile en contexte opérationnel. Cependant,  avec des APT, il doit être supposé que de nombreux systèmes informatiques peuvent avoir été compromis.  Ainsi, un confinement trop précoce sans analyse concrète de cette présence aura pour conséquence que d’informer le cyber-criminel qu'il est potentiellement découvert. Le cyber-attaquant pourra donc réagir et prendre des contre-mesures telles qu’installer de nouveaux logiciels malveillants, détruire les traces numériques liées à son attaque (méthodes anti-forensiques) ou encore endommager l’environnement. Il faut donc dans ma mesure du possible identifier intégralement a menace avant d’isoler les systèmes infectés et d’éradiquer les souches malveillantes.

 \subsection{La gestion des incidents avec l'ISO 27035}
 
 %https://pecb.com/en/education-and-certification-for-individuals/iso-iec-27035
 
 La mise en place d'un processus de gestion d’incidents, qu'il soit totalement intégré à la. DSI via ITIL, ou des processus ISO9001 est complexe en entreprise mais les enjeux sont toujours identiques :
\begin{itemize}
  \item  Améliorer la sécurité de l’information;
  \item  Réduire les impacts sur le business;
  \item  Renforcer la prévention d’incident;
  \item  Assurer le recevabilité des preuves;
  \item  Mettre à jour l’appréciation des risques;
  \item  Prévention et sensibilisation.
\end{itemize}

  C'est ce que l'on retrouve dans l'ISO 27035, une norme de l'ISO qui structure  une organisation sur la réponse à incident autour d'une \textbf{politique} de gestion des incidents de sécurité

Ce document de politique définit les éléments structurants. Il doit être pragmatique et adapté aux enjeux et à la taille de l’entité. Une bonne appropriation de cette  politique par les salariés est indispensable. 

La norme donne un guide  des incontournables de ce document :

\begin{itemize}
  \item Organisation générale (rôles et responsabilités, processus, équipes internes et externes, service de l'état, régulateurs ou agences nationales,... );
  \item Grandes définitions  (en particulier événement, incident, alerte et vulnérabilité);
  \item Sources (techniques, informationnelles et humaines) de remontée d’événements;
  \item Catégorisation et priorisation des incidents suivant des critères à préciser;
  \item Analyse post-mortem et analyse des retours d’expérience;
  \item Activation et fonctionnement de la cellule de réponse aux incidents (CSIRT - Cybersecurity Incident Response Team) comprenant les modalités de notification des incidents majeurs et d’activation de la cellule de crise.
  \item Sensibilisation des collaborateurs et formations dédiées.
\end{itemize}

Les phases de qualification et de décision reposent sur des expertises techniques très diverses en fonction des incidents.

Se lancer dans la construction d'un processus de gestion des incidents en interne ou pour un client en mode service, il est important d'être conscient des expertises nécessaire pour opérer :

Les \textbf{expertises techniques} liées à la \textbf{\g{surveillance détection}}qui s’appuient sur les solutions et équipements disponibles (IPS/IDS : intrusion detection / Prevention System, Security Information and Event Management : SIEM, logs locaux, analyseurs réseaux, supervision…), les clients,  partenaires et fournisseurs (CERT : Computer Emergency Response Team privés ou étatiques, opérateurs...).

Les expertises dépendent des caractéristiques techniques et fonctionnelles \textbf{des systèmes d’information} (technologies, logiciels, architectures, services cloud...). Le niveau de compétence des équipes internes ou externes qui assurent la maintenance conditionne la qualification  d’un événement dit incident. 

Des \textbf{expertises  plus orientées vers la réaction }pour traiter la réaction nécessitent des expertises pointues en sécurité (analyse du mode de propagation d’un code malveillant, analyse « forensic » d’un poste de travail  compromis…).

Le \textbf{niveau de capacité de support technique }ou de gestion de crise (Helpdesk) détermine le temps et la qualité de la réaction à l’incident et sa capacité de limiter les impacts. Les mauvaises décisions prises dans l’urgence pendant un incident sont souvent dues à un manque d’expertise ou d’organisation dans la phase d'organisation de ces plans de réaction à incidents. Ces mauvaises décisions peuvent amplifier les impacts de l’incident voire faire basculer l’entité en crise.

Le  champ d’application de la norme est une approche planifiée et structurée :
\begin{itemize}
  \item De la détection, de l'analyse et du reporting des incidents de sécurité,
  \item De la réponse et du management des incidents de sécurité,
  \item De la détection, de l’analyse et du management des vulnérabilités de la sécurité de l’information,
  \item De l’amélioration continue de la sécurité de l’information et de la gestion d’incident, dans le cadre plus global du management de « l’incidentologie » et des vulnérabilités.
\end{itemize}


\subsection{Et avec la NIST 800-61}


Ce référentiel du NIST dénommé \g{Computer Security Incident Handling Guide} donne des recommendations interessante. Contrairement au normes ISO, l'intérêt des documents de NIST c'est qu'ils sont accessibles et concrets. \link{https://csrc.nist.gov/publications/detail/sp/800-61/rev-2/final}{GUIDE 800-61}

Entre ISO 27035 et NIST 800-61, Les deux normes fondent la gestion des incidents sur une approche cyclique assez comparable : 

Cycle de gestion des incidents dans NIST SP 800-61:

\begin{itemize}
  \item Préparation
  \item Détection et analyse
  \item Confinement, éradication et récupération
  \item Activité post-incident
\end{itemize}

Cycle de gestion des incidents dans ISO / IEC 27035:

\begin{itemize}
  \item Planifier et préparer
  \item Détection et signalement
  \item Évaluation et décision
  \item Réponses
  \item Leçons apprises
\end{itemize}

Les deux normes fournissent des recommandations détaillées sur l'équipe d'intervention en cas d'incident et les politiques et procédures de gestion des incidents. À mon avis, ces deux éléments sont essentiels pour une gestion efficace des incidents - pas les outils techniques. Donc, si vous commencez à développer des capacités de gestion des incidents dans votre organisation, concentrez-vous d'abord sur ces deux. Des procédures opérationnelles standard constamment améliorées aideront votre équipe à être efficace. Ils peuvent également aider à automatiser une partie des tâches.
Le NIST 800-61 propose une liste de contrôle de gestion des incidents, avec 9 phases simple à appréhender. Cette liste de contrôle est le premier stade de ce qui doit être maitrisé, et les personnes en charge de réponses aux incidents doit savoir comment exécuter les étapes contenues dans cette liste.
L'ISO / CEI 27035  se concentre davantage sur une organisation elle-même que les bonnes pratiques et le partage. 
C'est la combinaison des deux qui permet de se préparer correctement.
Je conseille aussi la lecture d'un document assez ancien (décembre 2006) mais intéressant pour organiser une petite équipe issu de l'ENISA \link{https://www.enisa.europa.eu/publications/csirt-setting-up-guide-in-french}{Guide de création d'un CSIRT pas à pas}

\subsection{Continuité d'activité  avec l'ISO 22301}

Autour de la gestion de la résilience, il existe un cadre normatif qui permet d'organiser les plans de continuité d'activité qui sont le pendant opérationnel  de la DSI de la gestion de l'incident de cybersécurité.
 La vocation des plans de continuité d’activité (Business Continuity Plans) est donc de répondre à des situations critiques, souvent  rares mais pouvant avoir des impacts graves pour l'entreprise.  Ces plans prennent en compte (inondation, incendie, accident industriel), on  intègre de plus en plus souvent des risques de conflit social, des attaques cyber de grande ampleur, des ruptures de service d’un prestataire  ou sous-traitant. La démarche pour concevoir son système de management de la continuité d’activité est l’objet de la norme ISO 22301. Une étape initiale consiste à analyser les impacts métiers (Business Impact Analysis) pour identifier les activités critiques et les besoins de reprise. La norme ISO 22317 fournit un cadre et des bonnes pratiques pour réaliser cette analyse.
 La norme ISO 22301 est intitulée « Sécurité sociétale — Systèmes de management de la continuité d'activité — Exigences ». Elle constitue un outil des organisations « pour anticiper et gérer la continuité de leurs activités » et « délivre des lignes directrices pour la mise en place d’un système de management spécifique et efficace  ». Elle a été publiée dans sa première version en 2012, puis révisée en 2019. Cette norme remplace des standards qui étaient jusque-là nationaux, comme celui par exemple britannique (BS-25999).
 
 
 En effet, lorsque toutes les stratégies de défense ont échoué et que la crise survient il est important de définir une cadre de résilience :    Comment l'entreprise peut-elle continuer à fonctionner, rétablir ses activités le plus rapidement possible et essayer de minimiser son impact ?
 
 C’est pour répondre à ces questions que  cette norme ISO 22301 a été construite.  Aujourd’hui encore, de nombreuses PME qui subissent une cyberattaque incapacifiante ne survivent pas . C’est  souvent  par un manque de en place d'une gestion de la continuité d’activité. La cyber résilience  encore un parent pauvre de la cybersécurité, non pas le manque travaux, mais simplement pas la non prise de conscience du risque de rupture totale de l'activité par des attaques informatiques.

% TODO  relecture

 La norme ISO 22301 se fixe comme objectif  de spécifier les exigences pour planifier, établir, mettre en place et en œuvre, contrôler, réviser, maintenir et améliorer de manière continue un système de management documenté afin de se protéger des incidents perturbateurs, réduire leur probabilité de survenance, s'y préparer, y répondre et de s'en rétablir lorsqu'ils surviennent. 

En résumé, elle aide les organisations « à se montrer mieux préparées et plus solides face à des interruptions de toutes sortes »  grâce notamment à la création d’un système de management de la continuité d’activité. Ce système de management de la continuité permettra également de s’assurer que les objectifs de la continuité soient alignés avec ceux de l’entreprise et de la direction .

Cette norme a été rédigée de façon générique pour englober le plus de situations possibles et pouvoir être appliquée dans des organisations de tous types et de toutes tailles . Les exigences spécifiées dans la norme le sont de manière « relativement brève et concise »  afin de pouvoir servir de base pour la certification. Pour avoir un peu plus de détaille, vous pouvez consulter la norme ISO 22313 donne les bonnes pratiques de la Continuité d’activité.


