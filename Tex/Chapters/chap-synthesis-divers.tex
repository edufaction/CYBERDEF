%********************************************
% $Chapitre : SYNTHESE
% $theme :  synthèse
% $File : chap-synthesis-divers .tex
%
%********************************************
\uchap{Synthèse du cours}
%********************************************


%************************************************
\section{Une synthèse}
%************************************************
Comme nous l'avons vu à la sécurité opérationnel à Cyberdefense est un ensemble d'actions  élémentaires dont l'objectif est  de maintenir le niveau de continuité d'activité de l'entreprise.

A titre de synthèse des éléments que nous avons vu, je vous propose une synthèse très macroscopique des éléments à retenir.

Les actions des sécurités opérationnelles sont structurées autour de axes  fondamentaux  qui se concrétisent par des métiers et des compétences différentes. Dans une entreprise de faible maturité, la réaction aux incidents est le premier de maturité en cyberdéfense

% BEGIN PRZ .........................................................
\begin{frame}
\frametitle<presentation>{Sécurité opérationnelle}
\framesubtitle<presentation>{les 3 grandes fonctions }
%.............................................................................
\begin{itemize}
  \item \textbf{Répondre} au plus tôt aux incidents de sécurité afin de limiter l'impact des attaques (Equipe et compétences en réponse à incident, inforensic, analyse post-mortem, qualification PRIS de l'ANSSI).
  \item \textbf{Détecter} au plus tôt les tentatives d'attaques et les attaques en cours afin d'y répondre de manière adaptée en corrigeant si nécessaire les fragilités ayant été utilisées ; (Analystes en cybersécurité, SOC, outillage Logs, SIEM, Référentiel PDIS de l'ANSSI)
  \item \textbf{Rechercher} des fragilités connues, et  détecter des vulnérabilités intrinsèques  et les corriger avant qu'un attaquant ne les utilisent. (Auditeurs, Pentesteurs, base de vulnérabilités, référentiel PASSI de l'ANSSI)
\end{itemize}
%.............................................................................
\end{frame}
% END PRZ..............................................................


% BEGIN PRZ .........................................................
\begin{frame}
\frametitle<presentation>{3 fonctions = 3 processus SECOPS}
%.............................................................................


\upicture{../Tex/Pictures/img-chapvar}{Eléments Secops}{01}{lb:chapvar}


%.............................................................................
\end{frame}
% END PRZ..............................................................


%************************************************
\section{Réponse aux incidents}
%************************************************

La réponse à incident est souvent la première activité qui est interpellé dans une entreprise. C'est aux équipes informatiques ou le cas échéant à un prestataire de service qu'incombe dans un premier temps le traitement de l'attaque. Généralement cette attaque est activité, et a concrètement déjà impacté le fonctionnement d'une partie de l'entreprise, ou si cela n'est pas directement le cas, le risque est élevé pour un impact non négligeable. 

Les équipes informatiques doivent dont au plus tôt circonscrire l'attaque, déterminer les niveaux d'impact, les risques de propagation, et communiquer à l'environnement managerial les premiers impacts connus ou fortement probables.  Par la suite le processus de réponse à incident pourra dérouler ses mécanismes.

A l'image d'un sinistre par le feux, les premières actions sauver ce qui doit l'être ou peu l'être dans le contexte de l'action ( sécuriser les actifs) , sécuriser l'environnement pour le feu de se propage pas (limiter l'impact), éteindre le feu (circonscrire et bloquer l'attaque). Par la suite, il sera peut être nécessaire d'analyser les causes de l'incendie et d'identifier si le sinistre est intentionnel ou accidentel. Ves analyses sont plutôt du métier de l'enquêteur, sui se trouve le plus souvent du côté des forces de polices que des pompiers.

% BEGIN PRZ .........................................................
\begin{frame}
\frametitle<presentation>{Réponse à incidents}
\framesubtitle<presentation>{Etapes}
%.............................................................................
Les étapes du cycle de vie de la gestion d'incidents sont :

\begin{itemize}
  \item \textbf{Caractériser} rapidement pour identifier les impacts (tout en continuant les investigations);
  \item \textbf{Répondre au plus tôt} pour limiter l'impact (tout en suivant les actions de remédiation et leurs effets);
  \item \textbf{Apprendre} de l'attaque (Analyse Malware, Forensic, corriger les failles, corriger les postures et mécanismes de réaction) ;
  \item Mettre en place de nouvelles \g{contre-mesures}, et rapidement adapter les processus de détection;
  \item \textbf{Orienter ses capteurs } vers les menaces pour identifier si possible l'attaquant, et se préparer à d'autres actions de sa part;
  \item \textbf{Neutraliser} les sources menaces avec les services spécialisés de l'état.
\end{itemize}
%.............................................................................
\end{frame}
% END PRZ..............................................................

Il est important de pouvoir répondre aux incidents rapidement, avec efficacité avec une certaine forme de mode réflexe. Toutefois mettre en péril sont entreprise à chaque attaque ou incident de sécurité et fonctionner en mode pompier n'est peut être pas la meilleur solution pour maintenir les équipes IT dans un fonctionnement normal et nominal. En effet tous les incidents, évènements ne sont pas du même niveau de gravité, et certain nécessite un analyse préalable afin de déterminer les forces à mettre en oeuvre pour couvrir l'incident. Il est peut être intéressant de regarder comment mettre. en oeuvre des \g{vigiles} équipés de capteurs qui leur permettront de détecter et caractériser des alarmes et par la suite de préparer des interventions qui pourront aller jusqu'à l'activation d'équipes spécialisées.


%************************************************
\section{Détection des attaques}
%************************************************

La détection de l'attaque est au coeur du quotidien d'un ingénieur SECOPS, toutefois la veille sur la menace dont l'attaque est la concrétisation doit rester au coeur des préoccupation de la gestion des risques.  L'anticipation grace au renseignement acquis par des outils de veille associé à des mécanismes de remontées d'alertes de menace peut permettre de mettre les équipes SOC en étant de vigilance (Etat d'alerte). l'ensemble de ces processus  est dénommé \ubg{Threat Management}.

Dans une activité normale, il y a généralement de nombreux évènements qui permettent de déterminer les écarts  de fonctionnement de l'IT de l'entreprise (Nouvelles applications, déviance comportementales, nouveaux utilisateurs). Des équipes de surveillance et de détection d'incident sont à organiser pour assurer cette tache de veille continu, de caractérisation des alarmes et de déclenchement des alertes\ubg{Threat Detection} Elles doivent :

% BEGIN PRZ .........................................................
\begin{frame}
\frametitle<presentation>{Détection d'attaques}
\framesubtitle<presentation>{les actions}
%.............................................................................
\begin{itemize}
  \item Disposer des outils permettant de \textbf{voir ce qui se passe} dans l'environnement numérique de l'entreprise (Interne sur son SI, externe sur ses partenaires, clients et fournisseurs), mais aussi surveiller l'écosystème technologique et l'environnement de menaces. ( Log Management pour son SI, et Veille sur l'externe); Ces outils doivent être alimenter d'information, renseignement provenant de sources de \ubg{Threat Intelligence}
  \item Disposer des moyens  pour \textbf{détecter} dans les flots de données, d'informations, d'évènements les corrélations qui permettent de détecter la concrétisation d'une menace : une attaque (SIEM);
  \item Mettre en oeuvre les \textbf{mesures} d'analyse des évènements et de \textbf{remontée} des alertes au bon niveau de décision;
  \item Disposer d'une équipe apte à \textbf{décider} ce qui doit passer mettre l'entreprise en alerte et engager une réponse à incident;
  \item Disposer d'un ensemble de \textbf{compétences}, pour assurer la mise en place de nouveaux mécanismes, de \textbf{nouvelles règles }de détection face aux nouvelles menaces ou aux menaces spécifiques (SOC, expertises menaces).
\end{itemize}
%.............................................................................
\end{frame}
% END PRZ..............................................................

%************************************************
\section{la couverture des fragilités}
%************************************************

Couvrir ou corriger ses vulnérabilités est certainement l'activité la plus visible de la gestion de sécurité opérationnelle. Elle fait appellent à toutes la panoplie des tests de sécurité dont l'objectif est de découvrir des fragilités Humaines, Organisationnelle, ou Technique qui permettent d'attaquer des actifs dans l'entreprise.
Cette couverture des vulnérabilités s'organise autour d'un processus appelé \ubg{Vulnerability Management}

% BEGIN PRZ .........................................................
\begin{frame}
\frametitle<presentation>{Vulnerability Management}
\framesubtitle<presentation>{3 activités de la couverture des fragilités}
%.............................................................................
\begin{itemize}
  \item Pentest, Bug Bounty, Fuzzing et autres techniques offrent un panel de métier dans le domaine de recherche et l'analyse des failles. La maturité des chaines de développement dans le domaine du logiciel est encore suffisamment faible pour que l'on continue à trouver des défauts de programmation connues conduisant à des  vulnérabilités  logicielles.
  \item La complexité des systèmes d'information induit aussi une complexité à maitriser le déploiement de politiques de sécurité sur l'ensemble du périmètre induisant des défauts de configuration laissant ouvertes des portes pour des attaques. 
  \item La pression du DEVOPS devant rendre opérationnel des codes dont la conception et la vérification ne sont pas optimums, ne facilite pas le déploiement de systèmes robustes.
\end{itemize}
%.............................................................................
\end{frame}
% END PRZ..............................................................

Le processus de gestion des vulnérabilités   \ubg{Vulnerability Detection} s'organise donc autour de 2 axes`

% BEGIN PRZ .........................................................
\begin{frame}
\frametitle<presentation>{Fragilités}
\framesubtitle<presentation>{2 axes}
%.............................................................................
\begin{itemize}
  \item La \textbf{détection} de vulnérabilités dans ses actifs basée :
	\begin{itemize}
		  \item sur des catalogues de vulnérabilités connues sur des actifs  utilisants des codes externes (Codes Open-source, Progiciels ... ); 
 		 \item  sur la mauvaise configuration de ses actifs dans le contexte de l'entreprise et sur des catalogues de mauvaises configurations.;
		  \item  sur la non conformité aux politiques de sécurité de l'entreprise induisant des failles systémiques.
		\end{itemize}
 \item La \textbf{recherche} des vulnérabilités utilisant :
\begin{itemize}
		\item	des techniques de rétro-conception pour rechercher des failles d'implémentation;
		\item	des techniques d'analyse de code (Basé ou non sur des outils d'analyse de code statique) pour rechercher des erreurs de conception ou de programmation.
		\item	des services de veille \ubg{Vulnerability Intelligence} pour accéder à ce que d'autres font en matière de recherche de vulnérabilités (CERT en particulier)
\end{itemize}
  	
\end{itemize}


%.............................................................................
\end{frame}
% END PRZ..............................................................


%************************************************
\section{La veille}
%************************************************

La veille reste centrale dans la sécurité opérationnelle et permet d'alimenter les processus de SECOPS :

% BEGIN PRZ .........................................................
\begin{frame}
\frametitle<presentation>{Veiller et Surveiller}
\framesubtitle<presentation>{Menaces et vulnérabilités}
%.............................................................................
\begin{itemize}
  \item\textbf{ La veille sur les vulnérabilités} permet de connaitre les vulnérabilités apparaissant dans les logiciels ou codes connus que l'entreprise utilise (en ses murs, dans le cloud, ou chez des partenaires, fournisseurs ...) pour peu bien entendu que l'entreprise possède un cartographie exhaustive de ces logiciels. Sinon elle aura à effectuer des audits ponctuels ou continus pour cartographier puis corriger ces failles (en mettant à jour les logiciels ou en trouvant un mécanisme de couverture)
  \item  \textbf{La veille sur les menaces} permet de disposer d'éléments pour alimenter les mécanismes de détection, il peut s'agir :
\begin{itemize}
  \item  d'adresses mail, d'adresses IP, de nom de domaines malveillants;
  \item  d'IOC indice de compromission sorte de signature comportemental d'un code malveillant;
  \item de scénario complexe de nouvelle attaques;
  \item de vulnérabilités \g{ZERODAY} c'est à dire n'ayant pas encore de \g{correctifs} disponibles.
\end{itemize}
\end{itemize}
%.............................................................................
\end{frame}
% END PRZ..............................................................

La mise en place de mécanismes de veille ciblée sur l'entreprise se rapproche des techniques de renseignement dans l'espace militaires.

% BEGIN PRZ .........................................................
\begin{frame}
\frametitle<presentation>{Veiller et Surveiller}
\framesubtitle<presentation>{l'environnement de l'entreprise (darkweb)}
%.............................................................................
\begin{itemize}
  \item On y trouve la\textbf{ détection de compromission} ou de fuites de données en particulier la détection de couple Utilisateurs/Mots de passe sur la base d'adresse mail de l'entreprise, des bases de données clients piratées ....;
  \item  Le\textbf{ \g{targeting}}, c'est à dire la détection d'éléments ou d'information permettant d'alerter l'entreprise qu'une attaque se prépare contre elle ou contre les entreprises du secteur. On y trouve en particulier la lutte AntiDDOS, ou il es possible avec un renseignement suffisamment actifs de détecter avec un certains temps d'avance que des adresses IP, ou des noms de domaines particuliers vont être ciblées par des \g{BOTs}.
\end{itemize}
%.............................................................................
\end{frame}
% END PRZ..............................................................


% BEGIN PRZ .........................................................
\begin{frame}
\frametitle<presentation>{En quelques mots }
%.............................................................................
\upicture{../Tex/Pictures/img-synthesis}{4 	axes à retenir}{0.8}{lb:synthesis}
%.............................................................................
\end{frame}
% END PRZ..............................................................

