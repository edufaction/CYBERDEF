%====================
%. CONFIGURATION   
%. NOTES DE COURS
%====================
% Detect & Alerte et 
%.gestion de crise
%====================

%******************************
% Chargement des variables 
%.du modèle
%******************************


%===========================
% COURS "CYBERDEF"
% Configuration générale des articles
%===========================


\newcommand{\ubooktitleBefore}{Notes de synthèse}
\newcommand{\ubooktitleMain}{Cours SEC 101 Cnam Bretagne}
\newcommand{\ubooktitle}{UBOOKTITTLE}
\newcommand{\ubooksubtitle}{\includegraphics[width=0.04\paperwidth]{../Tex/Pictures/shield-20.pdf} }
 %------------------------------------------
\newcommand{\edoc}{e document\xspace}
\newcommand{\ecours}{e cours} 
\newcommand{\uJournalInfo}{Orange et CNAM Bretagne, Cybersécurité SEC101, eduf@ction}

%
\ifdefined\PRZMODE 
\newcommand{\umainload}{%===========================
%               COURS "CYBERDEF"
% Modèle Template des Planches Beamer
%===========================
%             Dé	cembre 2019
% (c) DUPUIS Eric eric.dupuis@orange.fr
%==========================


\documentclass[ignorenonframetext, 10pt]{beamer}

\usepackage{xcolor}

\usepackage{ae}
\usepackage[french]{babel}
\usepackage[utf8]{inputenc}
\usepackage[T1]{fontenc}


\usepackage{graphicx,hyperref,url}


\usepackage{../Tex/template.inc/edxbeamer}




%===========================
% EDX PACKAGE
%===========================

\usepackage[utf8]{inputenc}
\usepackage[french]{babel}
\usepackage[T1]{fontenc}


\usepackage{tikz} % Required for drawing custom shapes
\usetikzlibrary{mindmap,trees, backgrounds}
\usepackage{verbatim}
\usepackage{wrapfig}
\usepackage{xpatch}

%\usepackage{environ}

%\usepackage{amsmath}

\usepackage{graphicx} % Required for including pictures
\graphicspath{{Pictures/}} % Specifies the directory where pictures are stored

%\usepackage{lipsum} % Inserts dummy text



\usepackage{fontawesome} 



%\setlist{nolistsep} % Reduce spacing between bullet points and numbered lists

\usepackage{booktabs} % Required for nicer horizontal rules in tables


\usepackage{setspace}
\setstretch{1,1}

\usepackage{csquotes}


\usepackage{fancyhdr}

\usepackage{makeidx} 

\usepackage{xspace} 


%%===========================
% EDX PACKAGE ART
%===========================



%\usepackage{fontawesome}             % plusieurs icônes
%\usepackage{awesomebox}   

%\usepackage{comment}

%\newenvironment{frameB}{\begin{frame}}{\end{frame}}

% pour ne pas afficher les solutions

%\newcommand{\exclure}[1]{\renewenvironment{#1}{\begingroup\comment}{\endcomment\endgroup\ignorespaces}}


%===========================
% EDX COMMAND
%===========================

\newcommand\UKword[1]{\emph{#1}}
\newcommand\Ucenter[1]{\begin{center}{#1}\end{center}}

\newcommand\head[1]{\textbf{#1}}
\newcommand\tb[1]{\textbf{#1}}
\newcommand{\g}[1]{\og #1 \fg}
\newcommand{\uload}[1] {\input{../Tex/#1}}
\newcommand{\ucurcolor}{black}


%-----------------------------------------------------
%										 	\upspicture 
%---------------------------------------------------
\newcommand{\upspicture}[3]
{
\renewcommand{\ucurcolor}{#3}
\resizebox{#2\textwidth}{!}{\input{#1}}
}

%-----------------------------------------------------
%											\upicture 
%---------------------------------------------------

\newcommand{\upicture}[4]{
\begin{figure}[] %h
  \begin{center}
	 \includegraphics[scale=#3]{#1.pdf}
  \end{center}
\caption{#2 \label{#4}}
\end{figure}
}

%-----------------------------------------------------
%											\updfimage 
%---------------------------------------------------

\newcommand{\updfimage}[3]
{
\begin{wrapfigure}{R}{#3\textwidth}
  \begin{center}
	 \resizebox{#3\textwidth}{!}{\includegraphics{#1.pdf}}
  \end{center}
\caption{#2}
\end{wrapfigure}
}



%************************PACKAGE***

\usepackage{tikz}% Required for drawing custom shapes
\usetikzlibrary{shadows}


%-----------------------------------------------------
%											 		\rem 
% insertion d'une remarque avec R
%---------------------------------------------------
\newcommand\rem[1]{%
   \marginpar{%
   \tikzpicture[baseline={(title.base)}]
      \node[inner sep=5pt,text width=4cm,drop shadow={shadow yshift=-5pt,shadow xshift=5pt,ocre},fill=white] (box) {\vskip5pt \nointerlineskip #1};
      \node[right=10pt,inner sep=0pt,outer sep=10pt] (title) at (box.north west) {\bfseries\color{ocre}Remarque};
      \draw[draw=ocre,very thick](title.west)--(box.north west)--(box.south west)--(box.south east)--(box.north east)--(title.east);
      \fill[ocre]([yshift=-10pt]box.north west)--+(-5pt,-5pt)--+(0pt,-10pt);
   \endtikzpicture}%
}
%-----------------------------------------------------
%											 \utikzimage 
% insertion d'une image sur fichier tkz.tex
%---------------------------------------------------

\newcommand{\utikzimage}[3]
{
\begin{wrapfigure}{R}{#3\textwidth}
  \begin{center}
	 \resizebox{#3\textwidth}{!}{\input{#1.tkz.tex}}
  \end{center}
\caption{#2}
\end{wrapfigure}
}


%************************PACKAGE***
 \usepackage{fontawesome}
  \usepackage{awesomebox}   
%-----------------------------------------------------
%											 \videobox 
% Video BOX lien
%----------------------------------------------------
  
%\newcommand{\videobox}[1]{%
%\awesomebox{\faYoutubePlay}{\aweboxrulewidth}{url}{#1}}

%-----------------------------------------------------
%														\link 
% Url et footnote du lien
%-----------------------------------------------------
  
  \newcommand{\link}[2]
  {
  \href{#1}
  {
  #2~\raisebox{-0.2ex}{\faExternalLink}\footnote
  	{
 #1
  	}
  }
  }
  
  
  
  

%%===========================
% EDX COMMAND Pour article
%===========================

%************************PACKAGE***
\usepackage{fontawesome}
%\usepackage{awesomebox}   
%-----------------------------------------------------
%											 \videobox 
% Video BOX lien
%----------------------------------------------------
  
%\newcommand{\videobox}[1]{%
%\awesomebox{\faYoutubePlay}{\aweboxrulewidth}{url}{#1}}

%-----------------------------------------------------
%														\link 
% Url et footnote du lien
%-----------------------------------------------------
  
  \renewcommand{\link}[2]
  {
  \href{#1}
  {
  #2~\raisebox{-0.2ex}{\faExternalLink}\footnote
  	{
 #1
  	}
  }
  }
  
  
%\newcommand{\frametitle}{}
%  
%  \newboolean{BBeam}\setboolean{BBeam}{false}
%2 \newcommand{\AvecBeam}{\setboolean{BBeam}{true}}
%3 \newcommand{\SansBeam}{\setboolean{BBeam}{false}}
%4 \newcommand{\frametitle}[1]{%
%5 \ifthenelse{\boolean{BBeam}}{}{}}
%
%6 \newcommand{\PiedNotes}[1]{%
%7 \ifthenelse{\boolean{BBeam}}{#1}{}}
%  

  
  





%\newcommand{\labelitemi}{\Large{$\bullet$}}

%\setbeamercolor{itemize item}{fg=red}

%-------------------------------------------------------
% The title of the presentation:
%  - first a short version which is visible at the bottom of each slide;
%  - second the full title shown on the title slide;
\title[SEC101]{\utitle}

%-------------------------------------------------------
% Optional: a subtitle to be dispalyed on the title slide
\subtitle{Cours du Cnam}



%-------------------------------------------------------
% The author(s) of the presentation:
%  - again first a short version to be displayed at the bottom;
%  - next the full list of authors, which may include contact information;
\author[eduf@ction]{
Eric DUPUIS \\ \medskip
  {\small \url{eric.dupuis@cnam.fr} \and \url{eric.dupuis@orange.com}\\ 
  {\small \url{http://www.cnam.fr}}}}

% The institute:
%  - to start the name of the university as displayed on the top of each slide
%    this can be adjusted such that you can also create a Dutch version
%  - next the institute information as displayed on the title slide
\institute[Conservatoire National des Arts et Métiers]{
  Conservatoire National des Arts et Métiers \\ Chaire de Cybersécurité}

% Add a date and possibly the name of the event to the slides
%  - again first a short version to be shown at the bottom of each slide
%  - second the full date and event name for the title slide
\date[V2019]{
  Version initiale du cours  \\
  \today}

\begin{document}

\begin{frame}[plain]
  \titlepage
\end{frame}

\begin{frame}
  \frametitle{Sommaire}
  \tableofcontents
\end{frame}


\mode<all>{\ubody}





\end{document}
% Section titles are shown in at the top of the slides with the current section 
% highlighted. Note that the number of sections determines the size of the top 
% bar, and hence the university name and logo. If you do not add any sections 
% they will not be visible.
\section{Introduction}

\begin{frame}
  \frametitle{Introduction}

  \begin{itemize}
    \item This is just a short example
    \item The comments in the \LaTeX\ file are most important
    \item This is just the result after running pdflatex
    \item The style is based on the webpage \url{http://www.ru.nl/}
  \end{itemize}
\end{frame}

\begin{frame}
  \frametitle{Introduction 2}

  \begin{itemize}
    \item qdsqsd
    \item     \item Tqsdqsd
    \item Tsqsqdqsd
    \item exemple 
  \end{itemize}
\end{frame}


\section{Background information}

\begin{frame}
  \frametitle{Background information}

  \begin{block}{Slides with \LaTeX}
    Beamer offers a lot of functions to create nice slides using \LaTeX.
  \end{block}

  \begin{block}{The basis}
    This style uses the following default styles:
    \begin{itemize}
      \item split
      \item whale
      \item rounded
      \item orchid
    \end{itemize}
  \end{block}
\end{frame}

\section{The important things}

\begin{frame}
  \frametitle{The important things}

  \begin{enumerate}
    \item This just shows the effect of the style
    \item It is not a Beamer tutorial
    \item Read the Beamer manual for more help
    \item Contact me only concerning the style file
  \end{enumerate}
\end{frame}

\section{Analysis of the work}

\begin{frame}
  \frametitle{Analysis of the work}

  This style file gives your slides some nice Radboud branding.
  When you know how to work with the Beamer package it is easy to use.
  Just add:\\ ~~~$\backslash$usepackage$\{$ru$\}$ \\ at the top of your file.
\end{frame}

\section{Conclusion}

\begin{frame}
  \frametitle{Conclusion}

  \begin{itemize}
    \item Easy to use
    \item Good results
  \end{itemize}
\end{frame}


}
\else
\newcommand{\umainload}{%-------------------------------------------------------------
%               FR CYBERDEF SECOPS COURSE
%                          ARTICLE MAIN FILE
%                             2020 eduf@ction
%-------------------------------------------------------------

\documentclass[10pt,fleqn,twoside]{../Tex/template.inc/ArticleModel/edxstyle} 

\usepackage{beamerarticle}

\usepackage{ae}
\usepackage[french]{babel}
\usepackage[utf8]{inputenc}
\usepackage[T1]{fontenc}
\usepackage{lmodern}


\usepackage{titletoc} % Required for manipulating the table of contents

%===========================
% EDX PACKAGE
%===========================

\usepackage[utf8]{inputenc}
\usepackage[french]{babel}
\usepackage[T1]{fontenc}



\usepackage{tikz} % Required for drawing custom shapes
\usetikzlibrary{tikzmark} 
\usetikzlibrary{mindmap,trees, backgrounds}
\usepackage{verbatim}
\usepackage{wrapfig}
\usepackage{xpatch}

\usepackage{hologo}

\usepackage{fontawesome}

\usepackage[skins]{tcolorbox} % don't forget SKINS option

\usepackage{listings}
\usepackage{upquote}

\lstset{
upquote=true,columns=flexible, basicstyle=\ttfamily,
language=HTML, 
frameround=tttt,
commentstyle=\color{gray},
identifierstyle=\color{blue},
keywordstyle=\color{ocre}\bfseries,
xleftmargin=2em,
xrightmargin=2em,
aboveskip=\topsep,
belowskip=\topsep, 
frame=single,
rulecolor=\color{ocre},
backgroundcolor=\color{ocre!5},
breaklines,
breakindent=1.5em,
showspaces=false,
showstringspaces=false,
showtabs=false,
}


\usepackage{graphicx} % Required for including pictures
\graphicspath{{Pictures/}} % Specifies the directory where pictures are stored

\usepackage{booktabs} % Required for nicer horizontal rules in tables

\usepackage{setspace}
\setstretch{1,1}

\usepackage{csquotes}

\usepackage{fancyhdr}

\usepackage{makeidx} 

\usepackage{xspace} 

\usepackage{titletoc} % Required for manipulating the table of contents

\usepackage{hyperref}

\hypersetup{pdftitle={\utitle},pdfauthor={\uauthor},hidelinks,backref=true,pagebackref=true,hyperindex=true,colorlinks=false,breaklinks=true,urlcolor=ocre,bookmarks=true,bookmarksopen=false}

%\pdfstringdefDisableCommands{%
%  \def\\{}%
%  \def\texttt#1{<#1>}%
%}

\usepackage[style=numeric,citestyle=numeric,sorting=nyt,sortcites=true,autopunct=true,babel=hyphen,hyperref=true,abbreviate=false,backref=true,backend=biber]{biblatex}

\usepackage{cleveref}



%===========================
% EDX COMMAND STANDART
%===========================

\newcommand\UKword[1]{\emph{#1}}
\newcommand\Ucenter[1]{\begin{center}{#1}\end{center}}

\newcommand\head[1]{\textbf{#1}}
\newcommand\tb[1]{\textbf{#1}}
\newcommand{\g}[1]{\og #1 \fg{}}
\newcommand{\uload}[1] {\input{../Tex/#1}}
\newcommand{\ucurcolor}{black}

\newcommand{\mak}[1]{\faicon{\aArrowCircleORight} #1}
%-----------------------------------------------------
%										 	\upspicture 
%---------------------------------------------------
\newcommand{\upspicture}[3]
{
\renewcommand{\ucurcolor}{#3}
\resizebox{#2\textwidth}{!}{\input{#1}}
}

%-----------------------------------------------------
%										 	\includer 
%---------------------------------------------------

\newcommand{\includer}[1]
{
\input{../Tex/Chapters/#1}
}


%-----------------------------------------------------
%  command DEF : InTextx
%-----------------------------------------------------
\newcommand\InTexta{no text}
\newcommand\InTextb{no text}
\newcommand\InTextc{no text}
\newcommand\rnc{\renewcommand}


%-----------------------------------------------------
%										 	ENV warningbox
%---------------------------------------------------

\newtcolorbox{warningbox}[2][]
{
  colframe = ocre!25,
  colback  = black!05,
  coltitle = ocre!20!black,
  fonttitle = \bfseries, 
  title    = #2,
  #1,
}

\newtcolorbox{notebox2}[2][]
{
sharp corners, 
colback = ocre!5!white, 
colframe = ocre!75!black,
fonttitle = \bfseries, 
colbacktitle= ocre!85!black,
title=#2,#1
}

\newtcolorbox{notebox}[2][]{colback=ocre!5!white,
colframe=ocre!75!black,fonttitle=\bfseries,
colbacktitle=ocre!85!black,enhanced,
attach boxed title to top right={yshift=+1mm},
title=#2,#1}

%-----------------------------------------------------
%										 	\uchap
%---------------------------------------------------
\ifdefined\INTERTITLE
\newcommand{\uchap}[1]
	{
	\begin{LARGE}
	\textbf{\Ucenter{#1 \\ -oOo-}}	
	\end{LARGE}
	}
\else
	\newcommand{\uchap}[1]{}
\fi

%-----------------------------------------------------
%											\uexpand
%---------------------------------------------------
\newcommand {\uexpand}[1]{
\mode<all>\input{../Tex/Chapters/#1}
}

%-----------------------------------------------------
%											\upicture 
%---------------------------------------------------


\ifdefined\PRZMODE

\newcommand{\upicture}[4]{
\framesubtitle{#2}
%\begin{figure}[!h] %h
  \begin{center}
	 \includegraphics[height=0.95\textheight, width=0.95\textwidth, keepaspectratio ]{#1.pdf}
  \end{center}
%  \end{figure}
}

\else


\newcommand{\upicture}[4]{
\begin{figure}[!h] %h
  \begin{center}
	 \includegraphics[width=#3\textwidth]{#1.pdf}
  \end{center}
\caption{#2 \label{#4}}
\end{figure}
}
	
	
\fi


%-----------------------------------------------------
%											\uref 
%---------------------------------------------------
\newcommand{\uref}[1]{(Voir~\cref{#1} page~\pageref{#1})}



%-----------------------------------------------------
%											\updfimage 
%---------------------------------------------------

\newcommand{\updfimage}[3]
{
\begin{wrapfigure}{R}{#3\textwidth}
  \begin{center}
	 \resizebox{#3\textwidth}{!}{\includegraphics{#1.pdf}}
  \end{center}
\caption{#2}
\end{wrapfigure}
}



%************************PACKAGE***

%\usepackage{tikz}% Required for drawing custom shapes
\usetikzlibrary{shadows}


%-----------------------------------------------------
%											 		\rem 
% insertion d'une remarque avec R
%---------------------------------------------------
\newcommand\rem[1]{%
   \marginpar{%
   \tikzpicture[baseline={(title.base)}]
      \node[inner sep=5pt,text width=4cm,drop shadow={shadow yshift=-5pt,shadow xshift=5pt,ocre},fill=white] (box) {\vskip5pt \nointerlineskip #1};
      \node[right=10pt,inner sep=0pt,outer sep=10pt] (title) at (box.north west) {\bfseries\color{ocre}Remarque};
      \draw[draw=ocre,very thick](title.west)--(box.north west)--(box.south west)--(box.south east)--(box.north east)--(title.east);
      \fill[ocre]([yshift=-10pt]box.north west)--+(-5pt,-5pt)--+(0pt,-10pt);
   \endtikzpicture}%
}
%-----------------------------------------------------
%											 \utikzimage 
% insertion d'une image sur fichier tkz.tex
%---------------------------------------------------

\newcommand{\utikzimage}[3]
{
\begin{wrapfigure}{R}{#3\textwidth}
  \begin{center}
	 \resizebox{#3\textwidth}{!}{\input{#1.tkz.tex}}
  \end{center}
\caption{#2}
\end{wrapfigure}
}

%-----------------------------------------------------
%														\link 
% Url et footnote du lien
%-----------------------------------------------------
  
%  \newcommand{\link}[2]
%  {
%  \href{#1}
%  {
%  #2~\raisebox{-0.2ex}{\footnote
%  	{
% #1
%  	}
%  }
%  }
%  }
  
  
    \newcommand{\link}[2]
  {
  \href{#1}
  {
  #2~\raisebox{-0.2ex}{\faExternalLink}\footnote
  	{
 #1
  	}
  }
  }
  

  %-----------------------------------------------------
%														\Pframe 
% Beamer Frame
%-----------------------------------------------------




%\newenvironment{Pframe}[1]{\mode<all>{\begin{frame}<presentation>\frametitle{#1}}{\end{frame}}}
%
%\newenvironment{Tframe}[1]{{\begin{frame}<presentation>\frametitle{#1}}{\end{frame}}
%
%
%\newenvironment{Aframe}{\begin{frame}}{\end{frame}}
%

%\newcommand{\Aframetitle}[1]
%{
%\mode<presentation>{\frametittle{#1}}
%}
%













%-------------------------------------------------------------
%               FR CYBERDEF SECOPS COURSE
%
%                                    EDX STYLE
%
%                              2020 eduf@ction
%-------------------------------------------------------------
\usepackage{parskip}
\setlength{\parindent}{0em}
\setlength{\parskip}{5pt} % 1ex plus 0.5ex minus 0.2ex}

%---------------------------------------------------------------
%	MARGINS
%---------------------------------------------------------------

\usepackage{geometry}

%\geometry{
%	paper=a4paper, 
%	top=4cm, 
%	bottom=4cm, 
%	left=3cm, 
%	right=3cm, 
%	headheight=20pt, 
%	footskip=1.4cm, 
%	headsep=10pt, 
%%	showframe, 
%	showcrop 
%}

%---------------------------------------------------------------
%	FOOTNOTE (format)
%---------------------------------------------------------------

\makeatletter
\long\def\@makefntextFB#1{%
    \ifx\thefootnote\ftnISsymbol
        \@makefntextORI{#1}%
    \else
        \rule\z@\footnotesep
        \setbox\@tempboxa\hbox{\@thefnmark}%
            \ifdim\wd\@tempboxa>\z@
                \kern2em\llap{\@thefnmark.\kern0.5em}%
            \fi
        \hangindent2em\hangafter\@ne#1
    \fi}
\makeatother

%-----------------------------------------------------
%	COLOR
%-----------------------------------------------------
%\usepackage{xcolor} % Required for specifying colors by name
\definecolor{ocre}{RGB}{160,0,0} 
\definecolor{edxcolorcover}{RGB}{160,0,0} 
\definecolor{grey}{RGB}{50,80,80}
\definecolor{cnam}{RGB}{128, 0, 32} 
                 

  \definecolor{comments}{rgb}{0.7,0,0}    % rouge foncé
  \definecolor{link}{rgb}{0,0.4,0.6}      % ~RoyalBlue de dvips
  \definecolor{url}{rgb}{0.6,0,0}         % rouge-brun
  \definecolor{citation}{rgb}{0,0.5,0}    % vert foncé
  \definecolor{ULlinkcolor}{rgb}{0,0,0.3} % de ulthese.cls
  \definecolor{rouge}{rgb}{0.85,0,0.07}   % rouge bandeau identitaire
  \definecolor{or}{rgb}{1,0.8,0}          % or bandeau identitaire

%-----------------------------------------------------
%	FONTS
%-----------------------------------------------------

%%\usepackage{avant} % Use the Avantgarde font for headings
%\usepackage{times} % Use the Times font for headings
%\usepackage{mathptmx} % Use the Adobe Times Rgman as the default text font together with math symbols from the Sym­bol, Chancery and Com­puter Modern fonts

%%\usepackage{microtype} % Slightly tweak font spacing for aesthetics
%%\usepackage[utf8]{inputenc} % Required for including letters with accents
%\usepackage[T1]{fontenc} % Use 8-bit encoding that has 256 glyphs

% Using AVANT GARDE Family

%\renewcommand{\familydefault}{\sfdefault}

%\fontfamily{pag}\selectfont
%\renewcommand{\familydefault}{pag} % sfdefault pag lmss

%bch         Charter
%lmr         Latin Modern Roman
%lmss        Latin Modern Sans Serif
%lmssq       Latin Modern Sans Serif extended
%lmtt        Latin Modern Typewriter
%lmvtt       Latin Modern Typewriter proportional
%pag         Avant Garde
%pbk         Bookman
%pcr         Courier
%phv         Helvetica
%pnc         New Century Schoolbook
%ppl         Palatino
%ptm         Times
%put         Utopia
 
%\renewcommand{\rmdefault}{pag} % text 
%\renewcommand{\sfdefault}{pag} % titre


\renewcommand{\rmdefault}{pag} % text 
\renewcommand{\sfdefault}{phv} % titre


%-----------------------------------------------------
%	BIBLIOGRAPHY AND INDEX
%-----------------------------------------------------


\usepackage{calc} % For simpler calculation - used for spacing the index letter headings correctly

\usepackage{imakeidx}
%\makeindex % Tells LaTeX to create the files required for indexing

\makeindex[columns=2,intoc=true,options={-s \upath/indexstyle.ist}]

%-----------------------------------------------------
%	HEADERS AND FOOTERS
%-----------------------------------------------------


\usepackage{fancyhdr} % Required for header and footer configuration
\pagestyle{fancy} % Enable the custom headers and footers

\ifcsname chapter \endcsname   % verify if command name exists in the CLASS (Book, report, article ...)
% BEGIN IF
\renewcommand{\chaptermark}[1]{\markboth{\sffamily\normalsize\bfseries\chaptername\ \thechapter.\ #1}{}} % Styling for the current chapter in the header
  \else
%nothing
\fi
% END IF

\renewcommand{\sectionmark}[1]{\markright{\sffamily\normalsize\thesection\hspace{5pt}#1}{}} % Styling for the current section in the header

\fancyhf{} % Clear default headers and footers
\fancyhead[LE,RO]{\sffamily\normalsize\thepage} % Styling for the page number in the header
\fancyhead[LO]{\sffamily\normalsize\rightmark} % Print the nearest section name on the left side of odd pages
\fancyhead[RE]{\sffamily\normalsize\leftmark} % Print the current chapter name on the right side of even pages
\fancyfoot[RE,LO]{\sffamily\normalsize\uCoursetittle} % Uncomment to include a footer
\fancyfoot[LE,RO]{\hyperlink{toc}{\sffamily\normalsize\uCoursetheme}} % Uncomment to include a footer
% \hyperlink{toc}
\renewcommand{\headrulewidth}{1.0pt} % Thickness of the rule under the header

\fancypagestyle{plain}{% Style for when a plain pagestyle is specified
	\fancyhead{}\renewcommand{\headrulewidth}{0pt}} 

% Removes the header from odd empty pages at the end of chapters
\makeatletter
\renewcommand{\cleardoublepage}{
\clearpage\ifodd\c@page\else
\hbox{}
\vspace*{\fill}
\thispagestyle{empty}
\newpage
\fi}


%----------------------------------------------------------------------------------------
%	MAIN TABLE OF CONTENTS
%----------------------------------------------------------------------------------------
%\usepackage{titletoc} % Required for manipulating the table of contents

\contentsmargin{0cm} % Removes the default margin

% Part text styling (this is mostly taken care of in the PART HEADINGS section of this file)
\titlecontents{part}
	[0cm] % Left indentation
	{\addvspace{20pt}\bfseries} % Spacing and font options for parts
	{}
	{}
	{}

% Chapter text styling
\titlecontents{chapter}
	[1.25cm] % Left indentation
	{\addvspace{12pt}\large\sffamily\bfseries} % Spacing and font options for chapters
	{\color{ocre!60}\contentslabel[\Large\thecontentslabel]{1.25cm}\color{ocre}} % Formatting of numbered sections of this type
	{\color{ocre}} % Formatting of numberless sections of this type
	{\color{ocre!60}\normalsize\;\titlerule*[.5pc]{.}\;\thecontentspage} % Formatting of the filler to the right of the heading and the page number

% Section text styling
\titlecontents{section}
	[1.25cm] % Left indentation
	{\addvspace{3pt}\sffamily\bfseries} % Spacing and font options for sections
	{\contentslabel[\thecontentslabel]{1.25cm}} % Formatting of numbered sections of this type
	{} % Formatting of numberless sections of this type
	{\hfill\color{black}\thecontentspage} % Formatting of the filler to the right of the heading and the page number

% Subsection text styling
\titlecontents{subsection}
	[1.25cm] % Left indentation
	{\addvspace{1pt}\sffamily\small} % Spacing and font options for subsections
	{\contentslabel[\thecontentslabel]{1.25cm}} % Formatting of numbered sections of this type
	{} % Formatting of numberless sections of this type
	{\ \titlerule*[.5pc]{.}\;\thecontentspage} % Formatting of the filler to the right of the heading and the page number

%\titlecontents{subsubsection}
%	[1.25cm] % Left indentation
%	{\addvspace{1pt}\sffamily\tiny} % Spacing and font options for subsections
%	{\contentslabel[\thecontentslabel]{1.25cm}} % Formatting of numbered sections of this type
%	{} % Formatting of numberless sections of this type
%	{\ \titlerule*[.5pc]{.}\;\thecontentspage} % Formatting of the filler to the right of the heading and the page number

\titlecontents{subsubsection}
[1.25cm]
  {\addvspace{1pt}\sffamily\tiny}
 {\contentslabel[\thecontentslabel]{1.25cm}}
  {}
  {}

% Figure text styling
\titlecontents{figure}
	[1.25cm] % Left indentation
	{\addvspace{1pt}\sffamily\small} % Spacing and font options for figures
	{\thecontentslabel\hspace*{1em}} % Formatting of numbered sections of this type
	{} % Formatting of numberless sections of this type
	{\ \titlerule*[.5pc]{.}\;\thecontentspage} % Formatting of the filler to the right of the heading and the page number

% Table text styling
\titlecontents{table}
	[1.25cm] % Left indentation
	{\addvspace{1pt}\sffamily\small} % Spacing and font options for tables
	{\thecontentslabel\hspace*{1em}} % Formatting of numbered sections of this type
	{} % Formatting of numberless sections of this type
	{\ \titlerule*[.5pc]{.}\;\thecontentspage} % Formatting of the filler to the right of the heading and the page number

%----------------------------------------------------------------------------------------
%	MINI TABLE OF CONTENTS IN PART HEADS
%----------------------------------------------------------------------------------------

% Chapter text styling
\titlecontents{lchapter}
	[0em] % Left indentation
	{\addvspace{15pt}\large\sffamily\bfseries} % Spacing and font options for chapters
	{\color{ocre}\contentslabel[\Large\thecontentslabel]{1.25cm}\color{ocre}} % Chapter number
	{}  
	{\color{ocre}\normalsize\sffamily\bfseries\;\titlerule*[.5pc]{.}\;\thecontentspage} % Page number

% Section text styling
\titlecontents{lsection}
	[0em] % Left indentation
	{\sffamily\small} % Spacing and font options for sections
	{\contentslabel[\thecontentslabel]{1.25cm}} % Section number
	{}
	{}

% Subsection text styling (note these aren't shown by default, display them by searchings this file for tgcdepth and reading the commented text)
%\titlecontents{lsubsection}
%	[.5em] % Left indentation
%	{\sffamily\footnotesize} % Spacing and font options for subsections
%	{\contentslabel[\thecontentslabel]{1.25cm}}
%	{}
%	{}


\usepackage[]{ccicons}


%----------------------------------------------------------------------------------------
%	THEOREM STYLES
%----------------------------------------------------------------------------------------

\usepackage{amsmath,amsfonts,amssymb,amsthm} % For math equations, theorems, symbols, etc

\newcommand{\intoo}[2]{\mathopen{]}#1\,;#2\mathclose{[}}
\newcommand{\ud}{\mathop{\mathrm{{}d}}\mathopen{}}
\newcommand{\intff}[2]{\mathopen{[}#1\,;#2\mathclose{]}}
\renewcommand{\qedsymbol}{$\blacksquare$}

\ifcsname chapter \endcsname   % verify if command name exists in the CLASS (Book, report, article ...)
	\newtheorem{notation}{Notation}[chapter]
 \else
	\newtheorem{notation}{Notation}[section]
\fi


% Boxed/framed environments
\newtheoremstyle{ocrenumbox}% Theorem style name
{0pt}% Space above
{0pt}% Space below
{\normalfont}% Body font
{}% Indent amount
{\small\bf\sffamily\color{ocre}}% Theorem head font
{\;}% Punctuation after theorem head
{0.25em}% Space after theorem head
{\small\sffamily\color{ocre}\thmname{#1}\nobreakspace\thmnumber{\@ifnotempty{#1}{}\@upn{#2}}% Theorem text (e.g. Theorem 2.1)
\thmnote{\nobreakspace\the\thm@notefont\sffamily\bfseries\color{black}---\nobreakspace#3.}} % Optional theorem note

\newtheoremstyle{ocrenumbox}% Theorem style name
{0pt}% Space above
{0pt}% Space below
{\normalfont}% Body font
{}% Indent amount
{\small\bf\sffamily\color{ocre}}% Theorem head font
{\;}% Punctuation after theorem head
{0.25em}% Space after theorem head
{\small\sffamily\color{ocre}\thmname{#1}\nobreakspace\thmnumber{\@ifnotempty{#1}{}\@upn{#2}}% Theorem text (e.g. Theorem 2.1)
\thmnote{\nobreakspace\the\thm@notefont\sffamily\bfseries\color{black}---\nobreakspace#3.}} % Optional theorem note

\newtheoremstyle{blacknumex}% Theorem style name
{5pt}% Space above
{5pt}% Space below
{\normalfont}% Body font
{} % Indent amount
{\small\bf\sffamily}% Theorem head font
{\;}% Punctuation after theorem head
{0.25em}% Space after theorem head
{\small\sffamily{\tiny\ensuremath{\blacksquare}}\nobreakspace\thmname{#1}\nobreakspace\thmnumber{\@ifnotempty{#1}{}\@upn{#2}}% Theorem text (e.g. Theorem 2.1)
\thmnote{\nobreakspace\the\thm@notefont\sffamily\bfseries---\nobreakspace#3.}}% Optional theorem note


\newtheoremstyle{blacknumboxS}% name of the style to be used
{5pt}% measure of space to leave above the theorem. E.g.: 3pt
{5pt}% measure of space to leave below the theorem. E.g.: 3pt
{\normalfont}% name of font to use in the body of the theorem
{}% measure of space to indent
{\small\bf\sffamily}% name of head font
{}% punctuation between head and body
{ }% space after theorem head; " " = normal interword space
{\color{black}\faEye~  \color{ocre}\small\sffamily\thmnote{#3 : }}

% Non-boxed/non-framed environments
\newtheoremstyle{ocrenum}% Theorem style name
{5pt}% Space above
{5pt}% Space below
{\normalfont}% Body font
{}% Indent amount
{\small\bf\sffamily\color{ocre}}% Theorem head font
{\;}% Punctuation after theorem head
{0.25em}% Space after theorem head
{\small\sffamily\color{ocre}\thmname{#1}\nobreakspace\thmnumber{\@ifnotempty{#1}{}\@upn{#2}}% Theorem text (e.g. Theorem 2.1)
\thmnote{\nobreakspace\the\thm@notefont\sffamily\bfseries\color{black}---\nobreakspace#3.}} % Optional theorem note
\makeatother

% Defines the theorem text style for each type of theorem to one of the three styles above



\ifcsname chapter \endcsname

\newcounter{dummy} 

\numberwithin{dummy}{section}
	\theoremstyle{ocrenumbox}
		\newtheorem{theoremeT}[dummy]{Proposition}
		\newtheorem{problem}{Problem}[chapter]
		\newtheorem{exerciseT}{Outillage}[chapter]
	\theoremstyle{blacknumex}
		\newtheorem{exampleT}{Example}[chapter]
	\theoremstyle{blacknumbox}
		\newtheorem{vocabulary}{Vocabulary}[chapter]
		\newtheorem{definitionT}{Definition}[section]
		\newtheorem{corollaryT}[dummy]{Concept}
	\theoremstyle{blacknumboxS}
		\newtheorem{remarqueST}[]{}
		\newtheorem{remarqueT}[dummy]{Remarque}
	\theoremstyle{ocrenum}
	\newtheorem{proposition}[dummy]{Proposition}
  
 \else
 
\newcounter{dummy} 
\numberwithin{dummy}{section}

	\theoremstyle{ocrenumbox}
		\newtheorem{theoremeT}[dummy]{Proposition}
		%\newtheorem{problem}{Problem}[section]
		\newtheorem{exerciseT}{Outillage}[section]
	\theoremstyle{blacknumex}
		\newtheorem{exampleT}{Example}[section]
	\theoremstyle{blacknumbox}
		\newtheorem{vocabulary}{Vocabulary}[section]
		\newtheorem{definitionT}{Definition}[section]
		\newtheorem{corollaryT}[dummy]{concept}
	\theoremstyle{blacknumboxS}
		\newtheorem{remarqueST}[]{}
		\newtheorem{remarqueT}[dummy]{Remarque}
	\theoremstyle{ocrenum}
		\newtheorem{proposition}[dummy]{Proposition}
\fi



%----------------------------------------------------------------------------------------
%	DEFINITION OF COLORED BOXES
%----------------------------------------------------------------------------------------

\RequirePackage[framemethod=default]{mdframed} % Required for creating the theorem, definition, exercise and corollary boxes

% Theorem box
\newmdenv[skipabove=7pt,
skipbelow=7pt,
backgroundcolor=black!5,
linecolor=ocre,
innerleftmargin=5pt,
innerrightmargin=5pt,
innertopmargin=5pt,
leftmargin=0cm,
rightmargin=0cm,
innerbottommargin=5pt]{tBox}

% Exercise box	  
\newmdenv[skipabove=7pt,
skipbelow=7pt,
rightline=false,
leftline=true,
topline=false,
bottomline=false,
backgroundcolor=ocre!10,
linecolor=ocre,
innerleftmargin=5pt,
innerrightmargin=5pt,
innertopmargin=5pt,
innerbottommargin=5pt,
leftmargin=0cm,
rightmargin=0cm,
linewidth=4pt]{eBox}	

% Definition box
\newmdenv[skipabove=7pt,
skipbelow=7pt,
rightline=false,
leftline=true,
topline=false,
bottomline=false,
linecolor=ocre,
innerleftmargin=5pt,
innerrightmargin=5pt,
innertopmargin=0pt,
leftmargin=0cm,
rightmargin=0cm,
linewidth=4pt,
innerbottommargin=0pt]{dBox}	

% Corollary box
\newmdenv[skipabove=7pt,
skipbelow=7pt,
rightline=false,
leftline=true,
topline=false,
bottomline=false,
linecolor=gray,
backgroundcolor=black!5,
innerleftmargin=5pt,
innerrightmargin=5pt,
innertopmargin=5pt,
leftmargin=0cm,
rightmargin=0cm,
linewidth=4pt,
innerbottommargin=5pt]{cBox}

% Creates an environment for each type of theorem and assigns it a theorem text style from the "Theorem Styles" section above and a colored box from above
%\newenvironment{theorem}{\begin{tBox}\begin{theoremeT}}{\end{theoremeT}\end{tBox}}
\newenvironment{exercise}{\begin{eBox}\begin{exerciseT}}{\hfill{\color{ocre}\tiny\ensuremath{\blacksquare}}\end{exerciseT}\end{eBox}}				  
%\newenvironment{definition}{\begin{dBox}\begin{definitionT}}{\end{definitionT}\end{dBox}}	
%\newenvironment{example}{\begin{exampleT}}{\hfill{\tiny\ensuremath{\blacksquare}}\end{exampleT}}		
%\newenvironment{corollary}{\begin{cBox}\begin{corollaryT}}{\end{corollaryT}\end{cBox}}	
\newenvironment{nota}{\begin{cBox}\begin{remarqueT}}{\end{remarqueT}\end{cBox}}	

%----------------------------------------------------------------------------------------
%	REMARK ENVIRONMENT
%----------------------------------------------------------------------------------------

\newenvironment{remark}{\par\vspace{10pt}\small % Vertical white space above the remark and smaller font size
\begin{list}{}{
\leftmargin=35pt % Indentation on the left
\rightmargin=25pt}\item\ignorespaces % Indentation on the right
\makebox[-2.5pt]{\begin{tikzpicture}[overlay]
\node[draw=ocre!60,line width=1pt,circle,fill=ocre!25,font=\sffamily\bfseries,inner sep=2pt,outer sep=0pt] at (-15pt,0pt){\textcolor{ocre}{i}};\end{tikzpicture}} % Orange R in a circle
\advance\baselineskip -1pt}{\end{list}\vskip5pt} % Tighter line spacing and white space after remark

%----------------------------------------------------------------------------------------
%	SECTION NUMBERING IN THE MARGIN
%----------------------------------------------------------------------------------------

\makeatletter
% \renewcommand{\@seccntformat}[1]{\textcolor{ocre}{\csname the#1\endcsname}\quad}   

%\makeatother

%----------------------------------------------------------------------------------------
%	PART HEADINGS
%----------------------------------------------------------------------------------------

% Numbered part in the table of contents
\newcommand{\@mypartnumtocformat}[2]{%
	\setlength\fboxsep{0pt}%
	\noindent\colorbox{ocre!20}{\strut\parbox[c][.7cm]{\ecart}{\color{ocre!70}\Large\sffamily\bfseries\centering#1}}\hskip\esp\colorbox{ocre!40}{\strut\parbox[c][.7cm]{\linewidth-\ecart-\esp}{\Large\sffamily\centering#2}}%
}

% Unnumbered part in the table of contents
\newcommand{\@myparttocformat}[1]{%
	\setlength\fboxsep{0pt}%
	\noindent\colorbox{ocre!40}{\strut\parbox[c][.7cm]{\linewidth}{\Large\sffamily\centering#1}}%
}

\newlength\esp
\setlength\esp{4pt}
\newlength\ecart
\setlength\ecart{1.2cm-\esp}
\newcommand{\thepartimage}{}%
\newcommand{\partimage}[1]{\renewcommand{\thepartimage}{#1}}%
\def\@part[#1]#2{%
\ifnum \c@secnumdepth >-2\relax%
\refstepcounter{part}%
\addcontentsline{toc}{part}{\texorpdfstring{\protect\@mypartnumtocformat{\thepart}{#1}}{\partname~\thepart\ ---\ #1}}
\else%
\addcontentsline{toc}{part}{\texorpdfstring{\protect\@myparttocformat{#1}}{#1}}%
\fi%
\startcontents%
\markboth{}{}%
{\thispagestyle{empty}%
\begin{tikzpicture}[remember picture,overlay]%
\node at (current page.north west){\begin{tikzpicture}[remember picture,overlay]%	
\fill[ocre!20](0cm,0cm) rectangle (\paperwidth,-\paperheight);
\node[anchor=north] at (4cm,-3.25cm){\color{ocre!40}\fontsize{220}{100}\sffamily\bfseries\thepart}; 
\node[anchor=south east] at (\paperwidth-1cm,-\paperheight+1cm){\parbox[t][][t]{8.5cm}{
\printcontents{l}{0}{\setcounter{tocdepth}{1}}% The depth to which the Part mini table of contents displays headings; 0 for chapters only, 1 for chapters and sections and 2 for chapters, sections and subsections
}};
\node[anchor=north east] at (\paperwidth-1.5cm,-3.25cm){\parbox[t][][t]{15cm}{\strut\raggedleft\color{white}\fontsize{30}{30}\sffamily\bfseries#2}};
\end{tikzpicture}};
\end{tikzpicture}}%
\@endpart}
\def\@spart#1{%
\startcontents%
\phantomsection
{\thispagestyle{empty}%
\begin{tikzpicture}[remember picture,overlay]%
\node at (current page.north west){\begin{tikzpicture}[remember picture,overlay]%	
\fill[ocre!20](0cm,0cm) rectangle (\paperwidth,-\paperheight);
\node[anchor=north east] at (\paperwidth-1.5cm,-3.25cm){\parbox[t][][t]{15cm}{\strut\raggedleft\color{white}\fontsize{30}{30}\sffamily\bfseries#1}};
\end{tikzpicture}};
\end{tikzpicture}}
\addcontentsline{toc}{part}{\texorpdfstring{%
\setlength\fboxsep{0pt}%
\noindent\protect\colorbox{ocre!40}{\strut\protect\parbox[c][.7cm]{\linewidth}{\Large\sffamily\protect\centering #1\quad\mbox{}}}}{#1}}%
\@endpart}
\def\@endpart{\vfil\newpage
\if@twoside
\if@openright
\null
\thispagestyle{empty}%
\newpage
\fi
\fi
\if@tempswa
\twocolumn
\fi}

%----------------------------------------------------------------------------------------
%	CHAPTER HEADINGS
%-------------------------------------------------%--------------------------------------

%
%
%% A switch to conditionally include a picture, implemented by Christian Hupfer
\newif\ifusechapterimage
\usechapterimagetrue
\newcommand{\thechapterimage}{}%
\newcommand{\chapterimage}[1]{\ifusechapterimage\renewcommand{\thechapterimage}{#1}\fi}%
\newcommand{\autodot}{.}

\def\@makechapterhead#1{%
{\parindent \z@ \raggedright \normalfont
\ifnum \c@secnumdepth >\m@ne
\if@mainmatter
\begin{tikzpicture}[remember picture,overlay]
	
\node at (current page.north west)
			{\begin{tikzpicture}[remember picture,overlay]
			\node[anchor=north west,inner sep=0pt] at (0,0) {\ifusechapterimage\includegraphics[width=\paperwidth]{\thechapterimage}\fi};
			\draw[anchor=west] (\Gm@lmargin,-9cm) node [line width=2pt,rounded corners=15pt,draw=ocre,fill=white,fill opacity=1,inner sep=15pt]{\strut\makebox[22cm]{}};
			
			\draw[anchor=west] (\Gm@lmargin+.3cm,-9cm) node {\fontsize{20}{30}\selectfont\sffamily\bfseries\color{ocre}~#1\strut};
			%{\Huge\sffamily\bfseries\color{ocre}\thechapter\autodot~#1\strut};
			\node at (current page.north east)  [xshift=-3.95cm, yshift=-4.05cm, text opacity=1]  {{\color{white}\centering\fontsize{200}{30}\selectfont \bfseries\sffamily\thechapter\strut} };
			\node at (current page.north east)  [xshift=-4cm, yshift=-4cm, text opacity=1]  {{\color{ocre}\centering\fontsize{200}{30}\selectfont \bfseries\sffamily\thechapter\strut} };
			
			\end{tikzpicture}};
	\end{tikzpicture}

\else

\begin{tikzpicture}[remember picture,overlay]
\node at (current page.north west)
{\begin{tikzpicture}[remember picture,overlay]
\node[anchor=north west,inner sep=0pt] at (0,0) {\ifusechapterimage\includegraphics[width=\paperwidth]{\thechapterimage}\fi};
\draw[anchor=west] (\Gm@lmargin,-9cm) node [line width=2pt,rounded corners=15pt,draw=ocre,fill=white,fill opacity=1,inner sep=15pt]{\strut\makebox[22cm]{}};

\draw[anchor=east] (\Gm@lmargin,-9cm) node [line width=2pt,rounded corners=15pt,draw=ocre,fill=white,fill opacity=1,inner sep=15pt]{\strut\makebox[22cm]{}};
\draw[anchor=west] (\Gm@lmargin+.3cm,-9cm) node {\huge\sffamily\bfseries\color{ocre}#1\strut};

\end{tikzpicture}};
 
\end{tikzpicture}
\fi\fi\par\vspace*{270\p@}}}

\makeatother
%-------------------------------------------

%\def\@makeschapterhead#1{%
%\begin{tikzpicture}[remember picture,overlay]
%\node at (current page.north west)
%{\begin{tikzpicture}[remember picture,overlay]
%\node[anchor=north west,inner sep=0pt] at (0,0) {\ifusechapterimage\includegraphics[width=\paperwidth]{\thechapterimage}\fi};
%\draw[anchor=west] (\Gm@lmargin,-9cm) node [line width=2pt,rounded corners=15pt,draw=ocre,fill=white, fill opacity=0.8,inner sep=15pt]{\strut\makebox[22cm]{}};
%\draw[anchor=west] (\Gm@lmargin+.3cm,-9cm) node {\huge\sffamily\bfseries\color{ocre}#1\strut};
%\end{tikzpicture}};
%\end{tikzpicture}
%\par\vspace*{270\p@}}




%----------------------------------------------------------------------------------------
%	LINKS
%----------------------------------------------------------------------------------------


\usepackage{bookmark}
\bookmarksetup{
open,
numbered,
addtohook={%
\ifnum\bookmarkget{level}=0 % chapter
\bookmarksetup{bold}%
\fi
\ifnum\bookmarkget{level}=-1 % part
\bookmarksetup{color=ocre,bold}%
\fi
}
}

%----------------------------------------------------------------------------------------
%	FANCY CHAPTER Header
%----------------------------------------------------------------------------------------





\definecolor{ocre}{RGB}{160,0,0}   
%-------------------------------------------------
%	BIBLIOGRAPHY
%-------------------------------------------------

%\usepackage{enumitem}
\addbibresource{../tex/bibliography.bib} % BibTeX bibliography file
\defbibheading{bibempty}{}

%-------------------------------------------------
%	COLORS & BORDERS
%-------------------------------------------------
\setlength{\fboxrule}{0.75pt} % Width of the border around the abstract
\definecolor{color1}{RGB}{160,0,0} % Color of the article title and section
\definecolor{color2}{RGB}{220,220,220} % Color of the boxes behind the abstract and headings

%-------------------------------------------------
% ITEMS DEFINITION
%-------------------------------------------------

\setlist [itemize,1]{label=\color{color1}\faCaretRight }


\begin{document}

%-------------------------------------------------
%	ARTICLE INFORMATION
%-------------------------------------------------

\Abstract {\uabstract%===========================
% COURS "INTRO CYBERDEF"
% Abstract général des articles
%===========================


Il fait partie du cours introductif aux fondamentaux de la sécurité des systèmes d'information vue sous deux prismes quelques fois opposés dans la littérature : la gouvernance et la gestion opérationnelle de la sécurité.
Le cours est constitué d'un ensemble de notes de synthèse compilé en un document unique.\\
Ce document ne constitue pas à lui seul le référentiel du cours. Ce sont des notes de synthèse mises à disposition comme support pédagogique. }
\JournalInfo{\uJournalInfo} % Journal information
\Archive{Notes de cours éditées le  \DTMnow} % Additional notes (e.g. copyright, DOI, review/research article)
\PaperTitle{\utitle} % Article title
\Authors{\uauthor\textsuperscript{1,}\textsuperscript{2}*} % Authors
\affiliation{\textsuperscript{1}\textit{\uproa}} % Author affiliation
\affiliation{\textsuperscript{2}\textit{\uprob}} % Author affiliation
\affiliation{*\textbf{email}: \umaila\ -- \umailb} % Corresponding author
\Keywords{\ukeywords} % Keywords 
\newcommand{\keywordname}{Mots clefs} % Defines the keywords heading name

%-------------------------------------------------
%	MAKE TITLE
%-------------------------------------------------

\maketitle

\begin{tikzpicture}[remember picture,overlay]\node at (current page.north west)[xshift=4cm, yshift=-4cm, text opacity=1.0]{\includegraphics[width=0.1\paperwidth]{../Tex/template.inc/Commons/CommonsPictures/shield-20.pdf}}; 
\end{tikzpicture} 

%-------------------------------------------------	
% Download
\newcommand{\safeqrcode}[2][]{%
  \qrcode[#1]{\detokenize{#2}} }
\newcommand{\GITfilename}{https://github.com/edufaction/CYBERDEF/raw/master/Builder/\jobname.pdf}
\begin{center}
\setstretch{2}
 {\link{\GITfilename}{\textbf{\jobname.pdf }sur GITHUB CYBERDEF} }   \\ 
{{\huge\ccbyncndeu}}  \\  
{2020 eduf@ction Publication en Creative Common BY-NC-ND }   \newline % Copyright notice
{\safeqrcode[padding]{\GITfilename} }  
\end{center}
%-------------------------------------------------
	\newpage

%-------------------------------------------------
\ubody % MAIN BODY defined external
%-------------------------------------------------

	\printbibliography
	\newpage
	
%-------------------------------------------------	
\Ucontribute % Contribution TEXT
%-------------------------------------------------
	
	\newpage
	\tableofcontents 
	
\iftotalfigures
  \listoffigures
\fi

\end{document}
}
\fi





%====================
\newcommand {\ukeywords}
%====================
{%----------------------------------------------------
anticipation, veille, alerte, réponse
}%

%******************************
% Chargement des variables dédiées à l'article
%******************************

%====================
\newcommand {\utitle}
{%----------------------------------------------------
SECOPS : Des évènements de sécurité gérés \\à une cybercrise maitrisée
}%----------------------------------------------------


%====================
\newcommand {\uabstract}
{%----------------------------------------------------
C\edoc introduit le triptype de la partie cyberdéfense de la sécurité opérationnelle : Anticiper, Détecter, Réagir  et ceci sur les trois grands invariants des risques numériques : les vulnérabilités, les menaces et l'impact. Il donne les grandes lignes des trois chapitres qui suivent.
}%----------------------------------------------------


%******************************
%  variable définissant  le corps de l'article
%******************************

%====================
\newcommand {\ubody}
{%----------------------------------------------------

% Chapitre introduction de la partie 3 du cours VAR

%\textbf{\Ucenter{--- Chapitre introductif de la Partie 3 du cours SEC 101 ---}}

\section{Sécurité opérationnelle}


Après avoir vu dans le cycle de vie de sécurisation de système d'information les deux démarche que nous pourrions classer suivant les termes anglo-saxons des projets :

\begin{itemize}
 \item \textbf{THINK/DESIGN} : Des risques évalués à la politique sécurité pensée ;
 \item \textbf{BUILD}: De la politique de sécurité déployée à la construction d’une sécurité implémentée;
\end{itemize}

Il nous reste à aborder la dernière phase du cycle de vie : les activités d'exploitation de la sécurité. \textbf{RUN} : Des évènements de sécurité gérés à la cybercrise maitrisée, que que certains appellent \textbf{SECOPS} : Sécurité Opérationnelle.

% begin PRZ==========================
\begin{frame}
\frametitle<presentation>{Sécurité opérationnelle}
\framesubtitle<presentation>{Une définition}
% end header PRZ======================
 Ce terme de \g{sécurité opérationnelle}, est relativement jeune dans l’histoire de la sécurité des technologies de l’information. Le terme de SSI sécurité des systèmes d’information était né pour distinguer des disciplines qui s’attachaient à protéger l’information qui circule dans les systèmes de l’entreprise (cf protection et classification de l’information). La sécurité des réseaux et la sécurité informatique ont été les précurseurs de la cybersécurité, le cyber recouvrant en un seul terme les enjeux de sécurité liés au réseau et à l’informatique, mais plus largement la sécurité de l'économie numérique.
\end{frame}
% end PRZ ===========================

Comme nous l’avons abordé, dans les chapitres précédents, la cybersécurité est un domaine vaste qui regroupe de nombreuses disciplines pouvant intervenir dans des cycles projets pour construire des systèmes sûrs et pour assurer la continuité d’activité dans l'entreprise.
C’est dans ce dernier contexte que l’on parle plutôt de sécurité opérationnelle. Ces activités opérationnelles supportent en particulier le maintien en condition de sécurité au quotidien de l’entreprise. En France, au sein des armées, on parle de lutte informatique dite défensive permettant de différentier les activités des Cyberdéfense des activités de Cyber-protection. Ces activités sont à opposer à la lutte informatique\g{offensive} qui ne sera pas abordé dans c\ecours car elle relève de prérogative des états et non des entreprises.
La sécurité opérationnelle ajoute par ailleurs à son périmètre d'action, la surveillance de l'intérieur des périmètres de responsabilités de l'entreprise : \g{Les murs sont épais et solides, les douaniers sont aux portes de la cité, la police doit toutefois veiller à la sécurité des biens et des citoyens dans la ville, car certains sont néanmoins des brigands. Quand à l'armée, elle veille aux frontières du pays informée pas nos agents à l'étranger}.

On traitera donc cette partie avec une équivalence dans les terminologies suivantes :

% begin PRZ==========================
\begin{frame}
\frametitle<presentation>{Plusieurs terminologies, une dynamique}
% end header PRZ======================
	\begin{itemize}
		\item Maintien en condition de sécurité (MCS);
		\item Sécurité opérationnelle (SECOPS);
		\item Lutte informatique défensive (LID);
		\item Cyberdéfense au sens de la cyberdéfense d'entreprise (CYBERDEFENSE).
	\end{itemize}
\end{frame}
% end PRZ ===========================

Le but de cette sécurité opérationnelle est d’être au coeur de l’action de la sécurité de l’entreprise. En effet, la sécurité de l’entreprise est une propriété multiforme. Elle est d’abord statique dans la mesure où elle correspond à un niveau de confiance dans l’environnement pour conserver la disponibilité, la confidentialité et l’intégrité de l’entreprise. Cette forme statique est souvent liée à la conformité de l’entreprise aux différents référentiels sécuritaires (ISO27000, GDPR, LPM, NIS ...), mais surtout aux objectifs sécurité de l'entreprise face à ses risques et au exigences sécurité des clients souvent inscrites dans des plans d’assurance sécurité (Cf. PAS). 
Elle est aussi dynamique car c’est aussi une propriété systémique qui mesure la capacité à anticiper les menaces, identifier les fragilités , détecter en temps réel les attaques et réagir à temps ou au pire disposer des capacités de revenir dans un état de fonctionnement compatible avec la survie de l’entreprise (Modes dégradés par exemple).
Le système évolue, faisant apparaitre ici et là de nouvelles fragilités, l’entreprise se transforme, vit, suscitant de nouveaux potentiels d’attaques. L’entreprise doit s’organiser pour disposer de fonctions opérationnelles adaptées et dédiées à cette activité. Ces fonctions nécessitent des savoirs, des savoirs-faire et de l’outillage. C’est l’ensemble de ces techniques que nous allons tenter d’aborder dans c\edoc.

Globalement, on peut remarquer que le cycle de vie est a prendre dans le sens inverse de notre présentation. Dans les entreprises moins matures en gouvernance de la sécurité, la dynamique de cette sécurité opérationnelle est la première visible et opérée. Dans l'entreprise va réagir le,plus souvent dans une dynamique de réponse immédiates aux problèmes de sécurité sans pour autant investiguer plus avant dans les fragilités globales. Les mécanismes de cybersécurité sont donc construits dans une entreprise peu mature dans le sens suivant :

	\begin{itemize}
		\item Répondre aux incidents de sécurité, tenter de répondre à la question : \g{qui nous attaque et pourquoi}; 
		\item Améliorer les filtrages;
		\item Couvrir les vulnérabilités découvertes;
		\item Rechercher les vulnérabilités existantes dans le périmètre de responsabilité ;
		\item Anticiper les attaques;
		\item Anticiper les risques informatiques;
		\item Anticiper les risques sur l'information;
		\item Anticiper la menaces.
	\end{itemize}


\section{Lutte contre la menace}

La finalité de cette défense d’entreprise est de lutter contre ces attaques qui ne sont pas qu’informatique. L’attaquant peut utiliser des scenarii utilisant de nombreux vecteurs qui peuvent utiliser des fragilités organisationnelles ou humaines. On peut dire qu’une attaque est une fonction complexe, qui peut viser ou utiliser de nombreux facteurs internes et externes à l’entreprise. Ces facteurs constituent ce que certains nomment l’environnement numérique ou digital de l’entreprise. Cet environnement est globalement constitué de l’ensemble des outils, services, moyens informatiques ou réseaux utilisés par l’entreprise.
Mai 2017 a été un tournant dans la prise de conscience de la menace de la part des entreprises. Le Rançon-logiciel WannaCry a plus fait trembler les médias que les entreprises, mais a permis de faire comprendre au grand public les enjeux des menaces informatiques.


\begin{nota}[Paramètres d’une attaque]
\begin{equation}
Attaque = Fonction \left[ Fragilit\acute{e}s\,HOT\, Entreprise\otimes Gains\,Escompt\acute{e}s\,PF \right]
\end{equation}
\end{nota}

\begin{itemize}
	\item Fragilités HOT : Humaines, Organisationnelles, Techniques 
	\item Gains pour l'attaquant Idéologiques Politiques, Financiers, ...
\end{itemize}

On peut noter quatre grandes classes d’attaques informatiques \footnote{La majorité des attaques élémentaires peut être rangée dans ces classes}:

% begin PRZ==========================
\begin{frame}
\frametitle<presentation>{Grandes typologies des attaques numériques}
% end header PRZ======================
\begin{itemize}
\item Attaques \textbf{d’interception} d’information, vols par écoutes passives ou actives dans les flux transitant entre un émetteur et un récepteur;
\item Attaques par \textbf{déni de services}, généralement sur le réseaux : Ce type d’attaque est un atteinte à la DISPONIBILITE du système, basé souvent sur la saturation d’une capacité de traitement. Le système saturé dans l’exécution de certaines de ses fonctions, ne peut plus répondre aux demandes légitimes, car il est occupé à traiter d’autres sollicitations;
\item Attaques par \textbf{exploitation de failles }logiciels : Ce type d’attaque va utiliser une vulnérabilité, d’un système d’exploitation ou d’un logiciel pour exécuter du code malveillant. Ce code réalisera alors sa mission;
\item Attaques par \textbf{exploitation de défauts} de configuration : Ce type d’attaque utilise simplement un ou des défauts de configuration pour que légitimement l’agresseur puisse dérouler un scénario, qui pourra lui donner par exemple des droits particuliers pour conduire des attaques.
\end{itemize}
\end{frame}
% end PRZ==========================

Nous pourrions remarquer que ce nombre est relativement faible. Toutefois, la vrai difficulté réside dans la multiplicité des vulnérabilités, et des défauts de configuration. Les développeurs réalisent des logiciels possédant des failles (vulnérabilités), les utilisateurs ou les administrateurs déploient des systèmes en faisant des erreurs de configuration, ou ne les configurent que très rarement en pensant à la malveillance.

Les motivations des attaquants sont nombreuses, et leurs objectifs variés :

\begin{itemize}
\item obtenir un accès au système pour s’y maintenir en attendant un opportunité ;
\item récupérer de l’information, secrets, données personnelles exploitables (en gros toutes information ayant de la valeurs)
\item récupérer des données bancaires ;
\item s'informer sur l'organisation (entreprise de l'utilisateur, etc.) ;
\item troubler, couper, bloquer le fonctionnement d'un service (les rançongiciels entre dans cette catégories) ;
\item utiliser le système d’un utilisateur, pour rebondir vers un autre système ;
\item détourner les ressources du système d’un utilisateur (utiliser de la bande passante, utiliser de la capacité de calcul) ;
\end{itemize}

Bien entendu, il n’y a que très rarement un seul objectif, c’est la combinaison des méthodes d’attaques, des objectifs unitaires qui définissent globalement une mission ou un objectif final. L’exploitation de vulnérabilités au sein de l’entreprise va permettre le déploiement par l’attaquant d’un scénario.

\subsection{Politiques et Stratégies}

\upicture{\upath/Pictures/img-cyclevie-pol-strat}{Positionnement de la sécurité opérationnelle}{0.4}{lbl_pol-strat}

A partir des risques identifiés, l’entreprise a posé des politiques de sécurité qui ont permis de mettre en place des mesures de sécurité. Ces mesures sont d’ordre techniques avec des systèmes de sécurité, ou des SI avec des architectures particulières, mais aussi d’ordre organisationnel avec des procédures et des mécanismes à respecter.
L’ensemble de cette dynamique construit un niveau de sécurité qu’il va être nécessaire de maintenir dans le temps. Toutefois ce niveau de sécurité n’est pas suffisant pour une simple et bonne raison : la menace évolue, les vulnérabilités apparaissent (découvertes, ou créées), la valeur\g{marchande} des actifs d’une entreprise change aussi. 
Les occurrences de ces éléments de vie sont considérés comme des évènements qu’il convient de détecter avec suffisamment d’avance sur l’attaquant pour pouvoir le plus rapidement les prendre en compte.

La gestion des événements qui peuvent être un source de mesure de l’évolution du niveau de sécurité de l’entreprise est au coeur des stratégies de cyberdéfense. Ces évènements sont corrélés avec des sources provenant de deux processus particuliers qui seront décrit dans ce c\edoc.

\begin{itemize}
\item Recherche des vulnérabilités : Processus qui permet de rechercher, découvrir, couvrir les vulnérabilités ou fragilités de l’entreprise ou ayant un impact sur l’entreprise que celles-ci soient techniques, humaines ou organisationnelles ;
\item Prévention de la menace : Processus qui permet de connaitre les menaces directes sur l’entreprise ou potentielles afin d’anticiper et/ou se préparer à un type d’attaque.
\end{itemize}

C’est la confrontation entre les vulnérabilités, les menaces et la détection de l’activité de l’entreprise qui va permettre d’être efficace dans le processus de réponse. Il y a de nombreuses manière d’aborder la cyberdéfense d’entreprise

C\edoc présente donc une dynamique de cyberdéfense en trois\g{volets }

\begin{itemize}
\item Gestion des vulnérabilités (\textit{Vulnerability Management and CERT}) : maitriser ses vulnérabilités mais aussi surveiller l’environnement technologique. 
\item Surveillance, Détection de la menace (\textit{Event and Threat Management}) : Analyser en temps réel l’environnement protégé mais aussi surveiller l’écosystème lié à la menace pour anticiper 
\item Gestion des incidents et réponse aux incidents (\textit{Incident Response – CSIRT}) : Réagir en cas d’incident et assurer la remédiation
\end{itemize}

\upicture{\upath/Pictures/img-triangle}{Des 3 des volets de la sécurité opérationnelle}{0.6}{lbl_triptyque}

Ces trois volets ne sont pas les seuls qui concourent à la cyberdéfense d’entreprise, mais ils en restent les trois faces principales. Il est à noter que ces trois volets correspondent aussi en France à trois référentiels de qualification de l’ANSSI des prestataires de services de cybersécurité au profit des entreprises. Ces labels sont obtenus par les entreprises qui respectent un cahier des charges rigoureux sur le plan de l'éthique, du professionnalisme, et de la compétence des experts intervenants. Il y trois cadres principaux de certifications sont :

\begin{itemize}
\item PASSI : Prestataire d’Audit de la sécurité des systèmes d’information ;
\item PDIS : Prestataire de détection d’incident de sécurité ;
\item PRIS : Prestataire de réponse à incident.
\end{itemize}

Ces trois référentiels définissent l’ensemble des exigences d’assurance pour\g{qualifier} des prestataires de services en cybersécurité sur ces trois thématiques. En effet, il serait en effet important de confier la recherche de ses vulnérabilités, leurs remédiations à des sociétés de confiance. 

A ces 3 volets il ne faut pas oublier, le volet administration des briques informatique et de télécommunications de l’environnement de l’entreprise. C’est un volet que nous traiterons pas directement dans c\edoc pour se concentrer. sur les mécanismes de maintien en continue le niveau de sécurité de l’entreprise avec des mécanismes de veille, d’alerte et de réaction.


%TODO Leure et Hony Posts
%Leurres et Pots de Miel 
%Renseignement


\subsection{Stratégies d'action}

% Begin PRZ ===========================
\begin{frame}
\frametitle<presentation>{Stratégies de l'action}

% end header PRZ =======================
La cyberdéfense est un ensemble de mécanismes liés à une stratégie de l'action. Les outils de cyberdéfense sont construits pour aider à surveiller l'environnement, détecter des menaces et/ou des attaques mais surtout agir et réagir pour limiter les impacts. Si les outils de protection sont configurés à partir d'éléments de politique de sécurité (droits, accès, filtrage ...), les outils de défense sont basés sur les stratégies des attaquants.
On distinguera donc ici trois grands mécanismes de Cyberdefense que les anglo-saxons appellent : 

\begin{itemize}
	\item Predictive Cyberdefense
	\item Active and Proactive Cyberdefense
	\item Reactive Cyberdefense 
\end{itemize}
\end{frame}

Nous aborderons, en particulier ces concepts quand nous évoquerons la notion de SOC (Security Operational Center) activité qui opère ce volet de cyberdéfense.
% End PRZ ===========================

Il ne faut pas, par ailleurs, oublier le renseignement (\textit{Intelligence}). qui reste une des grandes étapes de la cyberdéfense domaine que nous explorerons sous son volet cyber avec les sources de \g{threat intelligence}, mais aussi avec le Renseignement d'Origine Cyber que les anglo-saxons nomme \g{intelligence cyber}

Dans les grandes organisations une autre stratégie globale de la cyberdéfense est de penser l'anticipation et la détection de manière globale à l'environnement digital de l'entreprise mais de structurer, la réaction de manière locale. 

%TODO https://en.wikipedia.org/wiki/Proactive_cyber_defence

Nous avons positionné l'audit technique comme une des activités fondamentale de la gestion des vulnérabilités.
En effet les techniques d'audit font partie des méthodes de référence pour disposer d'un état des fragilités de l'entreprise. On y trouvera donc les grands basiques des audits techniques que sont les tests d'intrusion, la sécurité applicative, l'audit de configuration, et le fuzzing.

Par ailleurs nous explorerons rapidement, les techniques de déception et de leurre qui font partie cette défense proactive avec les honeypots qui peuvent être couplés avec le \UKword{cyber-hunting}, technique de chasse aux codes malveillants dans l'entreprise.
 

\subsection{les modèles Cybersécurité}

Il existe de nombreux modèles de description de l'activité de Cyberdefense dans un contexte de cybersécurité.
Certains sont totalement intégrés au modèle de cybersécurité comme l'ISO 27K, ou le Cybersecurity FrameWork du NIST \uref{lbl_nist} avec les activités \textbf{DETECT, RESPOND et RECOVER};

\upicture{\upath/Pictures/img-nist}{modèle NIST}{0.4}{lbl_nist}

Ce que l'on peut reprocher au modèle du NIST, c'est qu'il ne possède pas explicitement la gestion des fragilités / vulnérabilités, mais il apporte toutefois un modèle très détaillé, que nous utiliserons pour partie.

Dans l'environnement ISO 27000, le modèle est piloté par les risques  \uref{lbl_risk27}.
\upicture{\upath/Pictures/img-risk27}{modèle ISO27 et risques}{0.4}{lbl_risk27}

Nous avons fait le choix de positionner la présentation du volet sécurité opérationnelle en nous éloignant un peu des modèles pour présenter les 3 grands moteurs de la sécurité opérationnelle. En effet les modèles cités sont orientée sur un axe de cycle de vie.
En sécurité opérationnelle ou cyberdéfense, l'objectif est de conduire en continue et de frond des processus de maitrise des risques cyber opérationnels.
\begin{itemize}
  \item Les systèmes d'information évoluent en continue et des vulnérabilités peuvent s'insérer et/ou découvertes chaque jour au grès des modifications et évolutions,
  \item Des menaces se concrétisent quotidiennement par des attaques ciblées ou non, nécessitant de réagir vite et en cohérence avec des enjeux de l'entreprise
  \item Avoir la capacité de réagir, et d'assurer la continuité d'activité face à des attaques d'ampleur, ou à fort impact techniques ou médiatique.
\end{itemize}

\newpage
\section{structure du cours}

Notre propos sera donc centré sur ces trois axes  qui nous déclinerons dans trois chapitres. Le travail de fond d'une équipe de sécurité opérationnelle, ou simplement de l'activité SECOPS est de pouvoir gérer de front trois grandes tâches : 

% Begin PRZ ===========================
\begin{frame}
\frametitle<presentation>{SECOPS en 3 thématiques}
% end header PRZ =======================
\begin{itemize}
 \item maitriser les fragilités numériques de l'entreprise (\UKword{Vulnerability Management)} quelles soient au sein du SI mais aussi dans l'environnement dit digital de cette entreprise (réseaux sociaux, partenaires, ...);
 \item anticiper les menaces et les scénarios associés (\UKword{Threat Management)}, gérer au quotidien les événements de sécurité ;
 \item réagir vite et en cohérence avec l'activité de l'entreprise en cas d'incident (\UKword{Incident Management)}.
\end{itemize}
\end{frame}
% end PRZ ===========================

Nous aborderons aussi quelques compléments à ces processus SECOPS, comme la détection des fuites de données (Leak Detection), qui peut s'entendre comme un incident de sécurité externe, ou une détection d'évènement hors de périmètre du SI, mais dans le périmètre de surveillance.

\upicture{\upath/Pictures/img-chapvar}{Synthèse des meta-processus SECOPS}{0.6}{lbl_prosecops}

Des activités qui nécessitent, pour être efficace, une symbiose parfaite entre les équipes qui gèrent l'activité digitale (Systèmes d'informations, réseaux sociaux, ...) et les équipes de sécurité opérationnelle.






%-------------------------------------------------------------
%               FR CYBERDEF SECOPS COURSE
%                                        SECOPS
%.                    Vulnerability/Threat/Incident
%
%                           Introduction Cyberdefense
%                            chap-Cyberdef-intro.tex
%
%                              2020 eduf@ction
%-------------------------------------------------------------
\uchap{Chapitre introductif de la Partie 3 du cours SEC 101 - CYBERDEF}

% ***** CHAPTER CYBERDEF


\section{SECOPS et cyberdéfense}

Mettre en place des stratégies de cyberdéfense, c'est partir du principe que l'entreprise sera attaquée et que l'enjeu des équipes est de se préparer à des attaques pouvant violemment impacter l'entreprise. Pour cela, l'entreprise doit anticiper, détecter à temps et réagir vite pour réduire l'impact.

	\subsection{Des opérations de cyberdéfense d'entreprise}
	
C'est dans ce contexte de sécurité  que doit s'organiser cette cyberdéfense opérationnelle et se structurer autours des axes qui caractérise une posture de cyberdéfense d'entreprise : 
	
\begin{itemize}
  \item Le renseignement : 
  				\begin{itemize}
  					\item sur les menaces  : les attaquant et leurs intentions, leurs techniques et outils, les sources compromises,
  					\item sur les vulnérabilités : des logiciels, et des structures organisationnelles,
			\end{itemize}
  \item La détection d'attaques ou de menaces dormantes ou cachées;
  \item La mise en alerte, ou la réponse à incidents pour aller à la gestion de crise;
  \item La neutralisation de sources malveillantes.
\end{itemize}
	
\subsection{Veille et renseignement}

Au coeur des opérations de cyberdéfense, le renseignement reste le moteur de l'anticipation.  Une grande partie des attaques exploitent des failles ou des vulnérabilités. Être au plus tôt au courant de l'existence d'une vulnérabilité sur un système utilisé par l'entreprise est le premier stade d'une veille pour ANTICIPER une attaque potentielle basée sur cette faille.

Toutes les attaques ne sont pas liées à l'utilisation d'une faille technique, il existe d'autres marquants ou indicateurs qui peuvent être surveiller pour évaluer les risques. Nous verrons dans le chapitre sur la détection, que les marquants des menaces sont souvent des données techniques qui peuvent caractériser l'attaque. l'attaquant, les techniques utilisées...
Par ailleurs le renseignement permet aussi de détecter des fuites de données, ou des informations sensibles compromises en surveillant les sites spécialisées.
On peut citer le célèbre \textbf{\link{https://haveibeenpwned.com}{';--have i been pwned?}} qui indique si une adresse électronique utilisée comme identifiant de compte a été compromise lors d'un vol de données sur un site. 

L'ensemble des informations liées aux menaces s'appelle de la \UKword{Threat Intelligence ou Cyber-Threat Intelligence (CTI)}, et celles liées aux vulnérabilités  classées dans la dynamique \UKword{Vulnerability Intelligence}.

\upicture{Tex/Pictures/img-intelligence}{Les axes du renseignement cyber}{0.5}{lbl-intelligence}

\subsection{VTI}

\upicture{Tex/Pictures/img-VTIsimple}{Les axes du renseignement cyber}{0.7}{lbl-VTIsimple}

Le triptyque VTI (Vulnérabilités/Failles, Menaces/Attaques, Incidents/Alertes), est souvent présenté comme le coeur des activités de sécurité opérationnelle. 

La sécurité opérationnelle c’est l’ensemble des processus opérationnels qu’il faut mettre en place et évaluer au quotidien afin réduire la surface d’exposition du système d’information aux risques, mais aussi réduire l'impact en cas d'attaque. 

On y trouve en particulier : 

\begin{itemize}
  \item Les audits techniques, pentest, scan : pour identifier, mais aussi la cartographie des actifs, des acteurs ...,
  \item Les systèmes de détection comme le SIEM, de chasse comme le Threat Hunting ...,
  \item Les mécanismes de remédiation, les techniques et outils de forensique.
\end{itemize}

Tous ces actions génèrent au sein de l'entreprise de l'information, souvent à destination d'acteurs différents mais participant globalement à l'objectif de cyberdéfense.

\subsection{Fusion Center}

Les préoccupations des responsables de cette sécurité opérationnelle restent encore l'accès et le partage de l'information car dans le cycle de gestion du risque et de réduction de la surface d'attaque. Le terme \UKword{Fusion Center} est historiquement liée au attentats au 11 septembre 2001, qui avaient montré les lacunes de partage de l'information entre les services de police et renseignements américains face à cette attaque. Repris par les experts de la cybersécurité, pour gérer les menaces 
La sécurité opérationnelle est gérée généralement par deux principales structures : le SOC, en charge de la détection, de la qualification et de la gestion des incidents ; et le CSIRT, responsable de la gestion de crise, de l’investigation numérique, de la veille et de la Threat Intelligence.


\upicture{Tex/Pictures/img-fusioncenter}{Fusion Center et CSOC}{0.7}{lbl-fusioncenter}

\begin{itemize}
  \item Automatiser la \textbf{création de règles} basées sur des menaces avérées  et détecter de manière avancée avec le \UKword{Machine Learning}
  \item Automatiser le \textbf{processus de management }des incidents avec en  particulier les SOAR (Security Orchestration, Automation et Response) outils d’aide et d’automatisation de la réaction aux incidents de sécurité.
  \item Adapter \textbf{l’organisation} à cette automatisation
  \item et surtout automatiser la \textbf{collecte et la distribution} de l'information dans les différents services grace à un \g{Centre de Fusion de l'information et du renseignement}.
\end{itemize}


\subsection{Entrainement à la cyberdéfense}

En matière de cybersécurité, les organisations sont confrontées à la persistance des cyberattaques, combinées à la montée en puissance de nouvelles menaces, ne sont pas suffisamment préparées à anticiper les incidents et à y apporter les réponses adéquates. De nombreuses études continuent à montrer  que la formation du personnel demeure le défi premier, se traduisant par un investissement conséquent de la part des organisations.

L’entrainement répond à différents objectifs de l'entreprise  :
\begin{itemize}
  \item Sensibilisation des acteurs (salariés, décideurs, managers) aux risques « cyber ». Ces entrainements, relativement brefs et regroupant qu'une partie des acteurs  permettent d’illustrer concrètement les menaces. Il permet en particulier de l'illustrer concrètement les impacts des cyberattaques sur une entreprise notamment pour les niveaux stratégiques;
  \item   Evaluation du dispositif de sécurité et de gestion de crise en particulier pour les évaluations autour des travaux de PCA/PRA avec l'ISO 22301. L’objectif est d'évaluer toute la chaîne opérationnelle, des équipes techniques informatiques et sécurité, aux top management sur l'ensemble de structures de la résilience;
  \item Entraînement des équipes techniques, par exemple au sein des SOC (Security Operation Center). Cela suppose de s’appuyer sur des environnements de simulation informatique représentatif de la réalité pour s'adapter à l'évolution des attaques (Outils de Range en particulier);
\end{itemize}


Pour répondre à ces besoins, trois niveaux d’entraînement doivent donc être distingués :

\begin{itemize}
  \item Des entraînements techniques au niveau des opérations. Ces entraînement peuvent intervenir dans le cadre de formations des équipes;
  \item Des entraînements managériales ou niveau des managers de proximité ou des COMEX. La durée de ces entraînements dépend des objectifs de monté en compétence. Pour une  sensibilisation de COMEX,  la durée doit être adaptée et courte (de l'ordre d'une demi journée). Les entraînements  ou exercices peuvent doivent être réguliers, une fréquence d'exercice annuelle est recommandée.
  \item Des entraînements mixtes associant des populations techniques et non techniques, afin de traiter l'inter-dépendance des différentes structures engagées.
\end{itemize}



\subsection{Gestion de crise}

Les opérations de sécurité opérationnelle doivent embarquer le processus de gestion de crise de l'entreprise, on trouvera dans le chapitre de gestion des incidents la manière donc les opérations de cybersécurité peuvent s'intégrer à des processus de gestion de crise existants ou comment créer des processus spécifiques à la cyber-crise. En effet, une des particularités des cyber-crises c'est que le systèmes d'information et de communication peut lui aussi être compromis, et que la confiance dans les systèmes informatiques de gestion peut être altérée, ou ces systèmes totalement inopérants.  

\section{Range et Simulation}

Les cyber-ranges, environnements de simulation informatique, dédiés non seulement à l’expérimentation des technologies, mais également à la formation par la pratique et à l’entraînement du personnel, constituent une réponse à l'enjeu de vision globale.  Initialement développés dans un cadre militaire,  les  cyber-ranges intéressent aujourd'hui l’ensemble de l’écosystème de cybersécurité. Ils offrent des conditions d’entrainement proches du réel, tant dans les topologies réseau reconstituées que dans les technologies de sécurité déployées. Ils fournissent un environnement d’affrontement informatique permettant une nouvelle approche de la formation axée sur l’opérationnel.

Il existe de nombreux outils d'entraînements. La majorité sont construits sur des technologies de virtualisation.
L’environnement de simulation technique est donc basé sur un « bac à sable » numérique (une sorte de paillasse numérique) offrant notamment des capacités de :
\begin{itemize}
  \item Virtualisation de postes de travail et de serveurs  et de tout autres composants informatique ou réseaux. l'environnement de simulation informatique peut être  « hybride » lorsqu’il permet également de connecter des équipements physiques (routeur, sonde, automate industriel...) ou des appliances ;
  \item Virtualisation des différents couches de transports, de stockage et de traitement ( couche réseau pour les liens et les équipements, couches base de données et infrastructures de base : Annuaire, Infrastructure de gestion de clefs, ...) ; 
  \item  Création de scénarios pédagogiques ou de situation tactiques pour fournir des contextes d'entraînement et de formation ;
  \item  Simulation et Génération de trafic pour donner de la vie au système.

\end{itemize}

L'entraînement est une solution pour dynamiser les activités de cyberdéfense.


\toolsbox{CyberRange-Airbus}{range}
\toolsbox{Hynesim}{range}
\toolsbox{EDUYesWeHack}{range}


\upicture{Tex/Pictures/img-range}{Entrainement}{0.8}{lbl-range}

Sur le plan humain,  l'utilisation d'un cyber-range se construit autour de deux équipes que nous explorerons dans les chapitres sur la gestion des vulnérabilités et la gestion des menaces :

\begin{itemize}
  \item La « Red-Team », composée de hackers éthiques professionnels (les PENTESTEURS). Ceux-ci reproduisent des attaques ciblées et d’ampleur et de complexité croissante, de nature à challenger la défense tout au long de l’entraînement ;
  \item  La « Blue-Team », chargée de la défense des réseaux et systèmes d’information (le SOC),, qui est donc constituée des apprenants participant au programme d’entraînement ou de formation.
\end{itemize}


%Vous pouvez trouver une document intéressant sur \link{https://www.defense.gouv.fr/content/download/502443/8528007/file/OBS_Monde cybernétique_201703.pdf}{le site du ministère des armées} 

% https://www.cloudrangecyber.com/ (autre Range)






}%----------------------------------------------------

%===========================
\newcommand {\Ucontribute}
{
%contrinbite

\section{Contributions}

\subsection{Comment contribuer}

Les fichiers sources de ce document sont publiés sur GITHUB \link{https://github.com/edufaction/CYBERDEF}{(edufaction/CYBERDEF)} . Vous pouvez contribuer au projet des notes de cours Cyerbsécurité SEC101 (CYBERDEF101). Le fichier Tex/Contribute/Contribs.tex contient la liste des personnes ayant contribué à ces notes de cours.
Le guide de contribution est disponible sur le GITHUB


%\input{Tex/Contribute/CYBERDEF-30-var-intro.contribs.tex}
%===========================
% Contribution globale
%===========================

\subsection{Les contributeurs/auteurs du cours}

Les auteurs des contributions sont :

\subsubsection{Années 2019}

\begin{itemize}
  \item \head{François REGIS} (Orange) : CyberHunting
\end{itemize}


\subsubsection{Années 2018}

\begin{itemize}
  \item \head{Julia HEINZ} (Tyvazoo.com) : ISO dans la gouvernance de la cybersécurité
\end{itemize}


}

%******************************
% Chargement  du MODELE
%******************************

\umainload
