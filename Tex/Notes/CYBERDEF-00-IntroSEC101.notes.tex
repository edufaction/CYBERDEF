%========================================
%. CONFIGURATION ARTICLE  COURS  SEC 101
%========================================
% 				CONTRIBUTIONS
%========================================

%************************************************************
% Chargement des variables du modèle
%************************************************************

\title{Contributions SEC101}

%===========================
% COURS "CYBERDEF"
% Configuration générale des articles
%===========================


\newcommand{\ubooktitleBefore}{Notes de synthèse}
\newcommand{\ubooktitleMain}{Cours SEC 101 Cnam Bretagne}
\newcommand{\ubooktitle}{UBOOKTITTLE}
\newcommand{\ubooksubtitle}{\includegraphics[width=0.04\paperwidth]{../Tex/Pictures/shield-20.pdf} }
 %------------------------------------------
\newcommand{\edoc}{e document\xspace}
\newcommand{\ecours}{e cours} 
\newcommand{\uJournalInfo}{Orange et CNAM Bretagne, Cybersécurité SEC101, eduf@ction}

%
\ifdefined\PRZMODE 
\newcommand{\umainload}{%===========================
%               COURS "CYBERDEF"
% Modèle Template des Planches Beamer
%===========================
%             Dé	cembre 2019
% (c) DUPUIS Eric eric.dupuis@orange.fr
%==========================


\documentclass[ignorenonframetext, 10pt]{beamer}

\usepackage{xcolor}

\usepackage{ae}
\usepackage[french]{babel}
\usepackage[utf8]{inputenc}
\usepackage[T1]{fontenc}


\usepackage{graphicx,hyperref,url}


\usepackage{../Tex/template.inc/edxbeamer}




%===========================
% EDX PACKAGE
%===========================

\usepackage[utf8]{inputenc}
\usepackage[french]{babel}
\usepackage[T1]{fontenc}


\usepackage{tikz} % Required for drawing custom shapes
\usetikzlibrary{mindmap,trees, backgrounds}
\usepackage{verbatim}
\usepackage{wrapfig}
\usepackage{xpatch}

%\usepackage{environ}

%\usepackage{amsmath}

\usepackage{graphicx} % Required for including pictures
\graphicspath{{Pictures/}} % Specifies the directory where pictures are stored

%\usepackage{lipsum} % Inserts dummy text



\usepackage{fontawesome} 



%\setlist{nolistsep} % Reduce spacing between bullet points and numbered lists

\usepackage{booktabs} % Required for nicer horizontal rules in tables


\usepackage{setspace}
\setstretch{1,1}

\usepackage{csquotes}


\usepackage{fancyhdr}

\usepackage{makeidx} 

\usepackage{xspace} 


%%===========================
% EDX PACKAGE ART
%===========================



%\usepackage{fontawesome}             % plusieurs icônes
%\usepackage{awesomebox}   

%\usepackage{comment}

%\newenvironment{frameB}{\begin{frame}}{\end{frame}}

% pour ne pas afficher les solutions

%\newcommand{\exclure}[1]{\renewenvironment{#1}{\begingroup\comment}{\endcomment\endgroup\ignorespaces}}


%===========================
% EDX COMMAND
%===========================

\newcommand\UKword[1]{\emph{#1}}
\newcommand\Ucenter[1]{\begin{center}{#1}\end{center}}

\newcommand\head[1]{\textbf{#1}}
\newcommand\tb[1]{\textbf{#1}}
\newcommand{\g}[1]{\og #1 \fg}
\newcommand{\uload}[1] {\input{../Tex/#1}}
\newcommand{\ucurcolor}{black}


%-----------------------------------------------------
%										 	\upspicture 
%---------------------------------------------------
\newcommand{\upspicture}[3]
{
\renewcommand{\ucurcolor}{#3}
\resizebox{#2\textwidth}{!}{\input{#1}}
}

%-----------------------------------------------------
%											\upicture 
%---------------------------------------------------

\newcommand{\upicture}[4]{
\begin{figure}[] %h
  \begin{center}
	 \includegraphics[scale=#3]{#1.pdf}
  \end{center}
\caption{#2 \label{#4}}
\end{figure}
}

%-----------------------------------------------------
%											\updfimage 
%---------------------------------------------------

\newcommand{\updfimage}[3]
{
\begin{wrapfigure}{R}{#3\textwidth}
  \begin{center}
	 \resizebox{#3\textwidth}{!}{\includegraphics{#1.pdf}}
  \end{center}
\caption{#2}
\end{wrapfigure}
}



%************************PACKAGE***

\usepackage{tikz}% Required for drawing custom shapes
\usetikzlibrary{shadows}


%-----------------------------------------------------
%											 		\rem 
% insertion d'une remarque avec R
%---------------------------------------------------
\newcommand\rem[1]{%
   \marginpar{%
   \tikzpicture[baseline={(title.base)}]
      \node[inner sep=5pt,text width=4cm,drop shadow={shadow yshift=-5pt,shadow xshift=5pt,ocre},fill=white] (box) {\vskip5pt \nointerlineskip #1};
      \node[right=10pt,inner sep=0pt,outer sep=10pt] (title) at (box.north west) {\bfseries\color{ocre}Remarque};
      \draw[draw=ocre,very thick](title.west)--(box.north west)--(box.south west)--(box.south east)--(box.north east)--(title.east);
      \fill[ocre]([yshift=-10pt]box.north west)--+(-5pt,-5pt)--+(0pt,-10pt);
   \endtikzpicture}%
}
%-----------------------------------------------------
%											 \utikzimage 
% insertion d'une image sur fichier tkz.tex
%---------------------------------------------------

\newcommand{\utikzimage}[3]
{
\begin{wrapfigure}{R}{#3\textwidth}
  \begin{center}
	 \resizebox{#3\textwidth}{!}{\input{#1.tkz.tex}}
  \end{center}
\caption{#2}
\end{wrapfigure}
}


%************************PACKAGE***
 \usepackage{fontawesome}
  \usepackage{awesomebox}   
%-----------------------------------------------------
%											 \videobox 
% Video BOX lien
%----------------------------------------------------
  
%\newcommand{\videobox}[1]{%
%\awesomebox{\faYoutubePlay}{\aweboxrulewidth}{url}{#1}}

%-----------------------------------------------------
%														\link 
% Url et footnote du lien
%-----------------------------------------------------
  
  \newcommand{\link}[2]
  {
  \href{#1}
  {
  #2~\raisebox{-0.2ex}{\faExternalLink}\footnote
  	{
 #1
  	}
  }
  }
  
  
  
  

%%===========================
% EDX COMMAND Pour article
%===========================

%************************PACKAGE***
\usepackage{fontawesome}
%\usepackage{awesomebox}   
%-----------------------------------------------------
%											 \videobox 
% Video BOX lien
%----------------------------------------------------
  
%\newcommand{\videobox}[1]{%
%\awesomebox{\faYoutubePlay}{\aweboxrulewidth}{url}{#1}}

%-----------------------------------------------------
%														\link 
% Url et footnote du lien
%-----------------------------------------------------
  
  \renewcommand{\link}[2]
  {
  \href{#1}
  {
  #2~\raisebox{-0.2ex}{\faExternalLink}\footnote
  	{
 #1
  	}
  }
  }
  
  
%\newcommand{\frametitle}{}
%  
%  \newboolean{BBeam}\setboolean{BBeam}{false}
%2 \newcommand{\AvecBeam}{\setboolean{BBeam}{true}}
%3 \newcommand{\SansBeam}{\setboolean{BBeam}{false}}
%4 \newcommand{\frametitle}[1]{%
%5 \ifthenelse{\boolean{BBeam}}{}{}}
%
%6 \newcommand{\PiedNotes}[1]{%
%7 \ifthenelse{\boolean{BBeam}}{#1}{}}
%  

  
  





%\newcommand{\labelitemi}{\Large{$\bullet$}}

%\setbeamercolor{itemize item}{fg=red}

%-------------------------------------------------------
% The title of the presentation:
%  - first a short version which is visible at the bottom of each slide;
%  - second the full title shown on the title slide;
\title[SEC101]{\utitle}

%-------------------------------------------------------
% Optional: a subtitle to be dispalyed on the title slide
\subtitle{Cours du Cnam}



%-------------------------------------------------------
% The author(s) of the presentation:
%  - again first a short version to be displayed at the bottom;
%  - next the full list of authors, which may include contact information;
\author[eduf@ction]{
Eric DUPUIS \\ \medskip
  {\small \url{eric.dupuis@cnam.fr} \and \url{eric.dupuis@orange.com}\\ 
  {\small \url{http://www.cnam.fr}}}}

% The institute:
%  - to start the name of the university as displayed on the top of each slide
%    this can be adjusted such that you can also create a Dutch version
%  - next the institute information as displayed on the title slide
\institute[Conservatoire National des Arts et Métiers]{
  Conservatoire National des Arts et Métiers \\ Chaire de Cybersécurité}

% Add a date and possibly the name of the event to the slides
%  - again first a short version to be shown at the bottom of each slide
%  - second the full date and event name for the title slide
\date[V2019]{
  Version initiale du cours  \\
  \today}

\begin{document}

\begin{frame}[plain]
  \titlepage
\end{frame}

\begin{frame}
  \frametitle{Sommaire}
  \tableofcontents
\end{frame}


\mode<all>{\ubody}





\end{document}
% Section titles are shown in at the top of the slides with the current section 
% highlighted. Note that the number of sections determines the size of the top 
% bar, and hence the university name and logo. If you do not add any sections 
% they will not be visible.
\section{Introduction}

\begin{frame}
  \frametitle{Introduction}

  \begin{itemize}
    \item This is just a short example
    \item The comments in the \LaTeX\ file are most important
    \item This is just the result after running pdflatex
    \item The style is based on the webpage \url{http://www.ru.nl/}
  \end{itemize}
\end{frame}

\begin{frame}
  \frametitle{Introduction 2}

  \begin{itemize}
    \item qdsqsd
    \item     \item Tqsdqsd
    \item Tsqsqdqsd
    \item exemple 
  \end{itemize}
\end{frame}


\section{Background information}

\begin{frame}
  \frametitle{Background information}

  \begin{block}{Slides with \LaTeX}
    Beamer offers a lot of functions to create nice slides using \LaTeX.
  \end{block}

  \begin{block}{The basis}
    This style uses the following default styles:
    \begin{itemize}
      \item split
      \item whale
      \item rounded
      \item orchid
    \end{itemize}
  \end{block}
\end{frame}

\section{The important things}

\begin{frame}
  \frametitle{The important things}

  \begin{enumerate}
    \item This just shows the effect of the style
    \item It is not a Beamer tutorial
    \item Read the Beamer manual for more help
    \item Contact me only concerning the style file
  \end{enumerate}
\end{frame}

\section{Analysis of the work}

\begin{frame}
  \frametitle{Analysis of the work}

  This style file gives your slides some nice Radboud branding.
  When you know how to work with the Beamer package it is easy to use.
  Just add:\\ ~~~$\backslash$usepackage$\{$ru$\}$ \\ at the top of your file.
\end{frame}

\section{Conclusion}

\begin{frame}
  \frametitle{Conclusion}

  \begin{itemize}
    \item Easy to use
    \item Good results
  \end{itemize}
\end{frame}


}
\else
\newcommand{\umainload}{%-------------------------------------------------------------
%               FR CYBERDEF SECOPS COURSE
%                          ARTICLE MAIN FILE
%                             2020 eduf@ction
%-------------------------------------------------------------

\documentclass[10pt,fleqn,twoside]{../Tex/template.inc/ArticleModel/edxstyle} 

\usepackage{beamerarticle}

\usepackage{ae}
\usepackage[french]{babel}
\usepackage[utf8]{inputenc}
\usepackage[T1]{fontenc}
\usepackage{lmodern}


\usepackage{titletoc} % Required for manipulating the table of contents

%===========================
% EDX PACKAGE
%===========================

\usepackage[utf8]{inputenc}
\usepackage[french]{babel}
\usepackage[T1]{fontenc}



\usepackage{tikz} % Required for drawing custom shapes
\usetikzlibrary{tikzmark} 
\usetikzlibrary{mindmap,trees, backgrounds}
\usepackage{verbatim}
\usepackage{wrapfig}
\usepackage{xpatch}

\usepackage{hologo}

\usepackage{fontawesome}

\usepackage[skins]{tcolorbox} % don't forget SKINS option

\usepackage{listings}
\usepackage{upquote}

\lstset{
upquote=true,columns=flexible, basicstyle=\ttfamily,
language=HTML, 
frameround=tttt,
commentstyle=\color{gray},
identifierstyle=\color{blue},
keywordstyle=\color{ocre}\bfseries,
xleftmargin=2em,
xrightmargin=2em,
aboveskip=\topsep,
belowskip=\topsep, 
frame=single,
rulecolor=\color{ocre},
backgroundcolor=\color{ocre!5},
breaklines,
breakindent=1.5em,
showspaces=false,
showstringspaces=false,
showtabs=false,
}


\usepackage{graphicx} % Required for including pictures
\graphicspath{{Pictures/}} % Specifies the directory where pictures are stored

\usepackage{booktabs} % Required for nicer horizontal rules in tables

\usepackage{setspace}
\setstretch{1,1}

\usepackage{csquotes}

\usepackage{fancyhdr}

\usepackage{makeidx} 

\usepackage{xspace} 

\usepackage{titletoc} % Required for manipulating the table of contents

\usepackage{hyperref}

\hypersetup{pdftitle={\utitle},pdfauthor={\uauthor},hidelinks,backref=true,pagebackref=true,hyperindex=true,colorlinks=false,breaklinks=true,urlcolor=ocre,bookmarks=true,bookmarksopen=false}

%\pdfstringdefDisableCommands{%
%  \def\\{}%
%  \def\texttt#1{<#1>}%
%}

\usepackage[style=numeric,citestyle=numeric,sorting=nyt,sortcites=true,autopunct=true,babel=hyphen,hyperref=true,abbreviate=false,backref=true,backend=biber]{biblatex}

\usepackage{cleveref}



%===========================
% EDX COMMAND STANDART
%===========================

\newcommand\UKword[1]{\emph{#1}}
\newcommand\Ucenter[1]{\begin{center}{#1}\end{center}}

\newcommand\head[1]{\textbf{#1}}
\newcommand\tb[1]{\textbf{#1}}
\newcommand{\g}[1]{\og #1 \fg{}}
\newcommand{\uload}[1] {\input{../Tex/#1}}
\newcommand{\ucurcolor}{black}

\newcommand{\mak}[1]{\faicon{\aArrowCircleORight} #1}
%-----------------------------------------------------
%										 	\upspicture 
%---------------------------------------------------
\newcommand{\upspicture}[3]
{
\renewcommand{\ucurcolor}{#3}
\resizebox{#2\textwidth}{!}{\input{#1}}
}

%-----------------------------------------------------
%										 	\includer 
%---------------------------------------------------

\newcommand{\includer}[1]
{
\input{../Tex/Chapters/#1}
}


%-----------------------------------------------------
%  command DEF : InTextx
%-----------------------------------------------------
\newcommand\InTexta{no text}
\newcommand\InTextb{no text}
\newcommand\InTextc{no text}
\newcommand\rnc{\renewcommand}


%-----------------------------------------------------
%										 	ENV warningbox
%---------------------------------------------------

\newtcolorbox{warningbox}[2][]
{
  colframe = ocre!25,
  colback  = black!05,
  coltitle = ocre!20!black,
  fonttitle = \bfseries, 
  title    = #2,
  #1,
}

\newtcolorbox{notebox2}[2][]
{
sharp corners, 
colback = ocre!5!white, 
colframe = ocre!75!black,
fonttitle = \bfseries, 
colbacktitle= ocre!85!black,
title=#2,#1
}

\newtcolorbox{notebox}[2][]{colback=ocre!5!white,
colframe=ocre!75!black,fonttitle=\bfseries,
colbacktitle=ocre!85!black,enhanced,
attach boxed title to top right={yshift=+1mm},
title=#2,#1}

%-----------------------------------------------------
%										 	\uchap
%---------------------------------------------------
\ifdefined\INTERTITLE
\newcommand{\uchap}[1]
	{
	\begin{LARGE}
	\textbf{\Ucenter{#1 \\ -oOo-}}	
	\end{LARGE}
	}
\else
	\newcommand{\uchap}[1]{}
\fi

%-----------------------------------------------------
%											\uexpand
%---------------------------------------------------
\newcommand {\uexpand}[1]{
\mode<all>\input{../Tex/Chapters/#1}
}

%-----------------------------------------------------
%											\upicture 
%---------------------------------------------------


\ifdefined\PRZMODE

\newcommand{\upicture}[4]{
\framesubtitle{#2}
%\begin{figure}[!h] %h
  \begin{center}
	 \includegraphics[height=0.95\textheight, width=0.95\textwidth, keepaspectratio ]{#1.pdf}
  \end{center}
%  \end{figure}
}

\else


\newcommand{\upicture}[4]{
\begin{figure}[!h] %h
  \begin{center}
	 \includegraphics[width=#3\textwidth]{#1.pdf}
  \end{center}
\caption{#2 \label{#4}}
\end{figure}
}
	
	
\fi


%-----------------------------------------------------
%											\uref 
%---------------------------------------------------
\newcommand{\uref}[1]{(Voir~\cref{#1} page~\pageref{#1})}



%-----------------------------------------------------
%											\updfimage 
%---------------------------------------------------

\newcommand{\updfimage}[3]
{
\begin{wrapfigure}{R}{#3\textwidth}
  \begin{center}
	 \resizebox{#3\textwidth}{!}{\includegraphics{#1.pdf}}
  \end{center}
\caption{#2}
\end{wrapfigure}
}



%************************PACKAGE***

%\usepackage{tikz}% Required for drawing custom shapes
\usetikzlibrary{shadows}


%-----------------------------------------------------
%											 		\rem 
% insertion d'une remarque avec R
%---------------------------------------------------
\newcommand\rem[1]{%
   \marginpar{%
   \tikzpicture[baseline={(title.base)}]
      \node[inner sep=5pt,text width=4cm,drop shadow={shadow yshift=-5pt,shadow xshift=5pt,ocre},fill=white] (box) {\vskip5pt \nointerlineskip #1};
      \node[right=10pt,inner sep=0pt,outer sep=10pt] (title) at (box.north west) {\bfseries\color{ocre}Remarque};
      \draw[draw=ocre,very thick](title.west)--(box.north west)--(box.south west)--(box.south east)--(box.north east)--(title.east);
      \fill[ocre]([yshift=-10pt]box.north west)--+(-5pt,-5pt)--+(0pt,-10pt);
   \endtikzpicture}%
}
%-----------------------------------------------------
%											 \utikzimage 
% insertion d'une image sur fichier tkz.tex
%---------------------------------------------------

\newcommand{\utikzimage}[3]
{
\begin{wrapfigure}{R}{#3\textwidth}
  \begin{center}
	 \resizebox{#3\textwidth}{!}{\input{#1.tkz.tex}}
  \end{center}
\caption{#2}
\end{wrapfigure}
}

%-----------------------------------------------------
%														\link 
% Url et footnote du lien
%-----------------------------------------------------
  
%  \newcommand{\link}[2]
%  {
%  \href{#1}
%  {
%  #2~\raisebox{-0.2ex}{\footnote
%  	{
% #1
%  	}
%  }
%  }
%  }
  
  
    \newcommand{\link}[2]
  {
  \href{#1}
  {
  #2~\raisebox{-0.2ex}{\faExternalLink}\footnote
  	{
 #1
  	}
  }
  }
  

  %-----------------------------------------------------
%														\Pframe 
% Beamer Frame
%-----------------------------------------------------




%\newenvironment{Pframe}[1]{\mode<all>{\begin{frame}<presentation>\frametitle{#1}}{\end{frame}}}
%
%\newenvironment{Tframe}[1]{{\begin{frame}<presentation>\frametitle{#1}}{\end{frame}}
%
%
%\newenvironment{Aframe}{\begin{frame}}{\end{frame}}
%

%\newcommand{\Aframetitle}[1]
%{
%\mode<presentation>{\frametittle{#1}}
%}
%













%-------------------------------------------------------------
%               FR CYBERDEF SECOPS COURSE
%
%                                    EDX STYLE
%
%                              2020 eduf@ction
%-------------------------------------------------------------
\usepackage{parskip}
\setlength{\parindent}{0em}
\setlength{\parskip}{5pt} % 1ex plus 0.5ex minus 0.2ex}

%---------------------------------------------------------------
%	MARGINS
%---------------------------------------------------------------

\usepackage{geometry}

%\geometry{
%	paper=a4paper, 
%	top=4cm, 
%	bottom=4cm, 
%	left=3cm, 
%	right=3cm, 
%	headheight=20pt, 
%	footskip=1.4cm, 
%	headsep=10pt, 
%%	showframe, 
%	showcrop 
%}

%---------------------------------------------------------------
%	FOOTNOTE (format)
%---------------------------------------------------------------

\makeatletter
\long\def\@makefntextFB#1{%
    \ifx\thefootnote\ftnISsymbol
        \@makefntextORI{#1}%
    \else
        \rule\z@\footnotesep
        \setbox\@tempboxa\hbox{\@thefnmark}%
            \ifdim\wd\@tempboxa>\z@
                \kern2em\llap{\@thefnmark.\kern0.5em}%
            \fi
        \hangindent2em\hangafter\@ne#1
    \fi}
\makeatother

%-----------------------------------------------------
%	COLOR
%-----------------------------------------------------
%\usepackage{xcolor} % Required for specifying colors by name
\definecolor{ocre}{RGB}{160,0,0} 
\definecolor{edxcolorcover}{RGB}{160,0,0} 
\definecolor{grey}{RGB}{50,80,80}
\definecolor{cnam}{RGB}{128, 0, 32} 
                 

  \definecolor{comments}{rgb}{0.7,0,0}    % rouge foncé
  \definecolor{link}{rgb}{0,0.4,0.6}      % ~RoyalBlue de dvips
  \definecolor{url}{rgb}{0.6,0,0}         % rouge-brun
  \definecolor{citation}{rgb}{0,0.5,0}    % vert foncé
  \definecolor{ULlinkcolor}{rgb}{0,0,0.3} % de ulthese.cls
  \definecolor{rouge}{rgb}{0.85,0,0.07}   % rouge bandeau identitaire
  \definecolor{or}{rgb}{1,0.8,0}          % or bandeau identitaire

%-----------------------------------------------------
%	FONTS
%-----------------------------------------------------

%%\usepackage{avant} % Use the Avantgarde font for headings
%\usepackage{times} % Use the Times font for headings
%\usepackage{mathptmx} % Use the Adobe Times Rgman as the default text font together with math symbols from the Sym­bol, Chancery and Com­puter Modern fonts

%%\usepackage{microtype} % Slightly tweak font spacing for aesthetics
%%\usepackage[utf8]{inputenc} % Required for including letters with accents
%\usepackage[T1]{fontenc} % Use 8-bit encoding that has 256 glyphs

% Using AVANT GARDE Family

%\renewcommand{\familydefault}{\sfdefault}

%\fontfamily{pag}\selectfont
%\renewcommand{\familydefault}{pag} % sfdefault pag lmss

%bch         Charter
%lmr         Latin Modern Roman
%lmss        Latin Modern Sans Serif
%lmssq       Latin Modern Sans Serif extended
%lmtt        Latin Modern Typewriter
%lmvtt       Latin Modern Typewriter proportional
%pag         Avant Garde
%pbk         Bookman
%pcr         Courier
%phv         Helvetica
%pnc         New Century Schoolbook
%ppl         Palatino
%ptm         Times
%put         Utopia
 
%\renewcommand{\rmdefault}{pag} % text 
%\renewcommand{\sfdefault}{pag} % titre


\renewcommand{\rmdefault}{pag} % text 
\renewcommand{\sfdefault}{phv} % titre


%-----------------------------------------------------
%	BIBLIOGRAPHY AND INDEX
%-----------------------------------------------------


\usepackage{calc} % For simpler calculation - used for spacing the index letter headings correctly

\usepackage{imakeidx}
%\makeindex % Tells LaTeX to create the files required for indexing

\makeindex[columns=2,intoc=true,options={-s \upath/indexstyle.ist}]

%-----------------------------------------------------
%	HEADERS AND FOOTERS
%-----------------------------------------------------


\usepackage{fancyhdr} % Required for header and footer configuration
\pagestyle{fancy} % Enable the custom headers and footers

\ifcsname chapter \endcsname   % verify if command name exists in the CLASS (Book, report, article ...)
% BEGIN IF
\renewcommand{\chaptermark}[1]{\markboth{\sffamily\normalsize\bfseries\chaptername\ \thechapter.\ #1}{}} % Styling for the current chapter in the header
  \else
%nothing
\fi
% END IF

\renewcommand{\sectionmark}[1]{\markright{\sffamily\normalsize\thesection\hspace{5pt}#1}{}} % Styling for the current section in the header

\fancyhf{} % Clear default headers and footers
\fancyhead[LE,RO]{\sffamily\normalsize\thepage} % Styling for the page number in the header
\fancyhead[LO]{\sffamily\normalsize\rightmark} % Print the nearest section name on the left side of odd pages
\fancyhead[RE]{\sffamily\normalsize\leftmark} % Print the current chapter name on the right side of even pages
\fancyfoot[RE,LO]{\sffamily\normalsize\uCoursetittle} % Uncomment to include a footer
\fancyfoot[LE,RO]{\hyperlink{toc}{\sffamily\normalsize\uCoursetheme}} % Uncomment to include a footer
% \hyperlink{toc}
\renewcommand{\headrulewidth}{1.0pt} % Thickness of the rule under the header

\fancypagestyle{plain}{% Style for when a plain pagestyle is specified
	\fancyhead{}\renewcommand{\headrulewidth}{0pt}} 

% Removes the header from odd empty pages at the end of chapters
\makeatletter
\renewcommand{\cleardoublepage}{
\clearpage\ifodd\c@page\else
\hbox{}
\vspace*{\fill}
\thispagestyle{empty}
\newpage
\fi}


%----------------------------------------------------------------------------------------
%	MAIN TABLE OF CONTENTS
%----------------------------------------------------------------------------------------
%\usepackage{titletoc} % Required for manipulating the table of contents

\contentsmargin{0cm} % Removes the default margin

% Part text styling (this is mostly taken care of in the PART HEADINGS section of this file)
\titlecontents{part}
	[0cm] % Left indentation
	{\addvspace{20pt}\bfseries} % Spacing and font options for parts
	{}
	{}
	{}

% Chapter text styling
\titlecontents{chapter}
	[1.25cm] % Left indentation
	{\addvspace{12pt}\large\sffamily\bfseries} % Spacing and font options for chapters
	{\color{ocre!60}\contentslabel[\Large\thecontentslabel]{1.25cm}\color{ocre}} % Formatting of numbered sections of this type
	{\color{ocre}} % Formatting of numberless sections of this type
	{\color{ocre!60}\normalsize\;\titlerule*[.5pc]{.}\;\thecontentspage} % Formatting of the filler to the right of the heading and the page number

% Section text styling
\titlecontents{section}
	[1.25cm] % Left indentation
	{\addvspace{3pt}\sffamily\bfseries} % Spacing and font options for sections
	{\contentslabel[\thecontentslabel]{1.25cm}} % Formatting of numbered sections of this type
	{} % Formatting of numberless sections of this type
	{\hfill\color{black}\thecontentspage} % Formatting of the filler to the right of the heading and the page number

% Subsection text styling
\titlecontents{subsection}
	[1.25cm] % Left indentation
	{\addvspace{1pt}\sffamily\small} % Spacing and font options for subsections
	{\contentslabel[\thecontentslabel]{1.25cm}} % Formatting of numbered sections of this type
	{} % Formatting of numberless sections of this type
	{\ \titlerule*[.5pc]{.}\;\thecontentspage} % Formatting of the filler to the right of the heading and the page number

%\titlecontents{subsubsection}
%	[1.25cm] % Left indentation
%	{\addvspace{1pt}\sffamily\tiny} % Spacing and font options for subsections
%	{\contentslabel[\thecontentslabel]{1.25cm}} % Formatting of numbered sections of this type
%	{} % Formatting of numberless sections of this type
%	{\ \titlerule*[.5pc]{.}\;\thecontentspage} % Formatting of the filler to the right of the heading and the page number

\titlecontents{subsubsection}
[1.25cm]
  {\addvspace{1pt}\sffamily\tiny}
 {\contentslabel[\thecontentslabel]{1.25cm}}
  {}
  {}

% Figure text styling
\titlecontents{figure}
	[1.25cm] % Left indentation
	{\addvspace{1pt}\sffamily\small} % Spacing and font options for figures
	{\thecontentslabel\hspace*{1em}} % Formatting of numbered sections of this type
	{} % Formatting of numberless sections of this type
	{\ \titlerule*[.5pc]{.}\;\thecontentspage} % Formatting of the filler to the right of the heading and the page number

% Table text styling
\titlecontents{table}
	[1.25cm] % Left indentation
	{\addvspace{1pt}\sffamily\small} % Spacing and font options for tables
	{\thecontentslabel\hspace*{1em}} % Formatting of numbered sections of this type
	{} % Formatting of numberless sections of this type
	{\ \titlerule*[.5pc]{.}\;\thecontentspage} % Formatting of the filler to the right of the heading and the page number

%----------------------------------------------------------------------------------------
%	MINI TABLE OF CONTENTS IN PART HEADS
%----------------------------------------------------------------------------------------

% Chapter text styling
\titlecontents{lchapter}
	[0em] % Left indentation
	{\addvspace{15pt}\large\sffamily\bfseries} % Spacing and font options for chapters
	{\color{ocre}\contentslabel[\Large\thecontentslabel]{1.25cm}\color{ocre}} % Chapter number
	{}  
	{\color{ocre}\normalsize\sffamily\bfseries\;\titlerule*[.5pc]{.}\;\thecontentspage} % Page number

% Section text styling
\titlecontents{lsection}
	[0em] % Left indentation
	{\sffamily\small} % Spacing and font options for sections
	{\contentslabel[\thecontentslabel]{1.25cm}} % Section number
	{}
	{}

% Subsection text styling (note these aren't shown by default, display them by searchings this file for tgcdepth and reading the commented text)
%\titlecontents{lsubsection}
%	[.5em] % Left indentation
%	{\sffamily\footnotesize} % Spacing and font options for subsections
%	{\contentslabel[\thecontentslabel]{1.25cm}}
%	{}
%	{}


\usepackage[]{ccicons}


%----------------------------------------------------------------------------------------
%	THEOREM STYLES
%----------------------------------------------------------------------------------------

\usepackage{amsmath,amsfonts,amssymb,amsthm} % For math equations, theorems, symbols, etc

\newcommand{\intoo}[2]{\mathopen{]}#1\,;#2\mathclose{[}}
\newcommand{\ud}{\mathop{\mathrm{{}d}}\mathopen{}}
\newcommand{\intff}[2]{\mathopen{[}#1\,;#2\mathclose{]}}
\renewcommand{\qedsymbol}{$\blacksquare$}

\ifcsname chapter \endcsname   % verify if command name exists in the CLASS (Book, report, article ...)
	\newtheorem{notation}{Notation}[chapter]
 \else
	\newtheorem{notation}{Notation}[section]
\fi


% Boxed/framed environments
\newtheoremstyle{ocrenumbox}% Theorem style name
{0pt}% Space above
{0pt}% Space below
{\normalfont}% Body font
{}% Indent amount
{\small\bf\sffamily\color{ocre}}% Theorem head font
{\;}% Punctuation after theorem head
{0.25em}% Space after theorem head
{\small\sffamily\color{ocre}\thmname{#1}\nobreakspace\thmnumber{\@ifnotempty{#1}{}\@upn{#2}}% Theorem text (e.g. Theorem 2.1)
\thmnote{\nobreakspace\the\thm@notefont\sffamily\bfseries\color{black}---\nobreakspace#3.}} % Optional theorem note

\newtheoremstyle{ocrenumbox}% Theorem style name
{0pt}% Space above
{0pt}% Space below
{\normalfont}% Body font
{}% Indent amount
{\small\bf\sffamily\color{ocre}}% Theorem head font
{\;}% Punctuation after theorem head
{0.25em}% Space after theorem head
{\small\sffamily\color{ocre}\thmname{#1}\nobreakspace\thmnumber{\@ifnotempty{#1}{}\@upn{#2}}% Theorem text (e.g. Theorem 2.1)
\thmnote{\nobreakspace\the\thm@notefont\sffamily\bfseries\color{black}---\nobreakspace#3.}} % Optional theorem note

\newtheoremstyle{blacknumex}% Theorem style name
{5pt}% Space above
{5pt}% Space below
{\normalfont}% Body font
{} % Indent amount
{\small\bf\sffamily}% Theorem head font
{\;}% Punctuation after theorem head
{0.25em}% Space after theorem head
{\small\sffamily{\tiny\ensuremath{\blacksquare}}\nobreakspace\thmname{#1}\nobreakspace\thmnumber{\@ifnotempty{#1}{}\@upn{#2}}% Theorem text (e.g. Theorem 2.1)
\thmnote{\nobreakspace\the\thm@notefont\sffamily\bfseries---\nobreakspace#3.}}% Optional theorem note


\newtheoremstyle{blacknumboxS}% name of the style to be used
{5pt}% measure of space to leave above the theorem. E.g.: 3pt
{5pt}% measure of space to leave below the theorem. E.g.: 3pt
{\normalfont}% name of font to use in the body of the theorem
{}% measure of space to indent
{\small\bf\sffamily}% name of head font
{}% punctuation between head and body
{ }% space after theorem head; " " = normal interword space
{\color{black}\faEye~  \color{ocre}\small\sffamily\thmnote{#3 : }}

% Non-boxed/non-framed environments
\newtheoremstyle{ocrenum}% Theorem style name
{5pt}% Space above
{5pt}% Space below
{\normalfont}% Body font
{}% Indent amount
{\small\bf\sffamily\color{ocre}}% Theorem head font
{\;}% Punctuation after theorem head
{0.25em}% Space after theorem head
{\small\sffamily\color{ocre}\thmname{#1}\nobreakspace\thmnumber{\@ifnotempty{#1}{}\@upn{#2}}% Theorem text (e.g. Theorem 2.1)
\thmnote{\nobreakspace\the\thm@notefont\sffamily\bfseries\color{black}---\nobreakspace#3.}} % Optional theorem note
\makeatother

% Defines the theorem text style for each type of theorem to one of the three styles above



\ifcsname chapter \endcsname

\newcounter{dummy} 

\numberwithin{dummy}{section}
	\theoremstyle{ocrenumbox}
		\newtheorem{theoremeT}[dummy]{Proposition}
		\newtheorem{problem}{Problem}[chapter]
		\newtheorem{exerciseT}{Outillage}[chapter]
	\theoremstyle{blacknumex}
		\newtheorem{exampleT}{Example}[chapter]
	\theoremstyle{blacknumbox}
		\newtheorem{vocabulary}{Vocabulary}[chapter]
		\newtheorem{definitionT}{Definition}[section]
		\newtheorem{corollaryT}[dummy]{Concept}
	\theoremstyle{blacknumboxS}
		\newtheorem{remarqueST}[]{}
		\newtheorem{remarqueT}[dummy]{Remarque}
	\theoremstyle{ocrenum}
	\newtheorem{proposition}[dummy]{Proposition}
  
 \else
 
\newcounter{dummy} 
\numberwithin{dummy}{section}

	\theoremstyle{ocrenumbox}
		\newtheorem{theoremeT}[dummy]{Proposition}
		%\newtheorem{problem}{Problem}[section]
		\newtheorem{exerciseT}{Outillage}[section]
	\theoremstyle{blacknumex}
		\newtheorem{exampleT}{Example}[section]
	\theoremstyle{blacknumbox}
		\newtheorem{vocabulary}{Vocabulary}[section]
		\newtheorem{definitionT}{Definition}[section]
		\newtheorem{corollaryT}[dummy]{concept}
	\theoremstyle{blacknumboxS}
		\newtheorem{remarqueST}[]{}
		\newtheorem{remarqueT}[dummy]{Remarque}
	\theoremstyle{ocrenum}
		\newtheorem{proposition}[dummy]{Proposition}
\fi



%----------------------------------------------------------------------------------------
%	DEFINITION OF COLORED BOXES
%----------------------------------------------------------------------------------------

\RequirePackage[framemethod=default]{mdframed} % Required for creating the theorem, definition, exercise and corollary boxes

% Theorem box
\newmdenv[skipabove=7pt,
skipbelow=7pt,
backgroundcolor=black!5,
linecolor=ocre,
innerleftmargin=5pt,
innerrightmargin=5pt,
innertopmargin=5pt,
leftmargin=0cm,
rightmargin=0cm,
innerbottommargin=5pt]{tBox}

% Exercise box	  
\newmdenv[skipabove=7pt,
skipbelow=7pt,
rightline=false,
leftline=true,
topline=false,
bottomline=false,
backgroundcolor=ocre!10,
linecolor=ocre,
innerleftmargin=5pt,
innerrightmargin=5pt,
innertopmargin=5pt,
innerbottommargin=5pt,
leftmargin=0cm,
rightmargin=0cm,
linewidth=4pt]{eBox}	

% Definition box
\newmdenv[skipabove=7pt,
skipbelow=7pt,
rightline=false,
leftline=true,
topline=false,
bottomline=false,
linecolor=ocre,
innerleftmargin=5pt,
innerrightmargin=5pt,
innertopmargin=0pt,
leftmargin=0cm,
rightmargin=0cm,
linewidth=4pt,
innerbottommargin=0pt]{dBox}	

% Corollary box
\newmdenv[skipabove=7pt,
skipbelow=7pt,
rightline=false,
leftline=true,
topline=false,
bottomline=false,
linecolor=gray,
backgroundcolor=black!5,
innerleftmargin=5pt,
innerrightmargin=5pt,
innertopmargin=5pt,
leftmargin=0cm,
rightmargin=0cm,
linewidth=4pt,
innerbottommargin=5pt]{cBox}

% Creates an environment for each type of theorem and assigns it a theorem text style from the "Theorem Styles" section above and a colored box from above
%\newenvironment{theorem}{\begin{tBox}\begin{theoremeT}}{\end{theoremeT}\end{tBox}}
\newenvironment{exercise}{\begin{eBox}\begin{exerciseT}}{\hfill{\color{ocre}\tiny\ensuremath{\blacksquare}}\end{exerciseT}\end{eBox}}				  
%\newenvironment{definition}{\begin{dBox}\begin{definitionT}}{\end{definitionT}\end{dBox}}	
%\newenvironment{example}{\begin{exampleT}}{\hfill{\tiny\ensuremath{\blacksquare}}\end{exampleT}}		
%\newenvironment{corollary}{\begin{cBox}\begin{corollaryT}}{\end{corollaryT}\end{cBox}}	
\newenvironment{nota}{\begin{cBox}\begin{remarqueT}}{\end{remarqueT}\end{cBox}}	

%----------------------------------------------------------------------------------------
%	REMARK ENVIRONMENT
%----------------------------------------------------------------------------------------

\newenvironment{remark}{\par\vspace{10pt}\small % Vertical white space above the remark and smaller font size
\begin{list}{}{
\leftmargin=35pt % Indentation on the left
\rightmargin=25pt}\item\ignorespaces % Indentation on the right
\makebox[-2.5pt]{\begin{tikzpicture}[overlay]
\node[draw=ocre!60,line width=1pt,circle,fill=ocre!25,font=\sffamily\bfseries,inner sep=2pt,outer sep=0pt] at (-15pt,0pt){\textcolor{ocre}{i}};\end{tikzpicture}} % Orange R in a circle
\advance\baselineskip -1pt}{\end{list}\vskip5pt} % Tighter line spacing and white space after remark

%----------------------------------------------------------------------------------------
%	SECTION NUMBERING IN THE MARGIN
%----------------------------------------------------------------------------------------

\makeatletter
% \renewcommand{\@seccntformat}[1]{\textcolor{ocre}{\csname the#1\endcsname}\quad}   

%\makeatother

%----------------------------------------------------------------------------------------
%	PART HEADINGS
%----------------------------------------------------------------------------------------

% Numbered part in the table of contents
\newcommand{\@mypartnumtocformat}[2]{%
	\setlength\fboxsep{0pt}%
	\noindent\colorbox{ocre!20}{\strut\parbox[c][.7cm]{\ecart}{\color{ocre!70}\Large\sffamily\bfseries\centering#1}}\hskip\esp\colorbox{ocre!40}{\strut\parbox[c][.7cm]{\linewidth-\ecart-\esp}{\Large\sffamily\centering#2}}%
}

% Unnumbered part in the table of contents
\newcommand{\@myparttocformat}[1]{%
	\setlength\fboxsep{0pt}%
	\noindent\colorbox{ocre!40}{\strut\parbox[c][.7cm]{\linewidth}{\Large\sffamily\centering#1}}%
}

\newlength\esp
\setlength\esp{4pt}
\newlength\ecart
\setlength\ecart{1.2cm-\esp}
\newcommand{\thepartimage}{}%
\newcommand{\partimage}[1]{\renewcommand{\thepartimage}{#1}}%
\def\@part[#1]#2{%
\ifnum \c@secnumdepth >-2\relax%
\refstepcounter{part}%
\addcontentsline{toc}{part}{\texorpdfstring{\protect\@mypartnumtocformat{\thepart}{#1}}{\partname~\thepart\ ---\ #1}}
\else%
\addcontentsline{toc}{part}{\texorpdfstring{\protect\@myparttocformat{#1}}{#1}}%
\fi%
\startcontents%
\markboth{}{}%
{\thispagestyle{empty}%
\begin{tikzpicture}[remember picture,overlay]%
\node at (current page.north west){\begin{tikzpicture}[remember picture,overlay]%	
\fill[ocre!20](0cm,0cm) rectangle (\paperwidth,-\paperheight);
\node[anchor=north] at (4cm,-3.25cm){\color{ocre!40}\fontsize{220}{100}\sffamily\bfseries\thepart}; 
\node[anchor=south east] at (\paperwidth-1cm,-\paperheight+1cm){\parbox[t][][t]{8.5cm}{
\printcontents{l}{0}{\setcounter{tocdepth}{1}}% The depth to which the Part mini table of contents displays headings; 0 for chapters only, 1 for chapters and sections and 2 for chapters, sections and subsections
}};
\node[anchor=north east] at (\paperwidth-1.5cm,-3.25cm){\parbox[t][][t]{15cm}{\strut\raggedleft\color{white}\fontsize{30}{30}\sffamily\bfseries#2}};
\end{tikzpicture}};
\end{tikzpicture}}%
\@endpart}
\def\@spart#1{%
\startcontents%
\phantomsection
{\thispagestyle{empty}%
\begin{tikzpicture}[remember picture,overlay]%
\node at (current page.north west){\begin{tikzpicture}[remember picture,overlay]%	
\fill[ocre!20](0cm,0cm) rectangle (\paperwidth,-\paperheight);
\node[anchor=north east] at (\paperwidth-1.5cm,-3.25cm){\parbox[t][][t]{15cm}{\strut\raggedleft\color{white}\fontsize{30}{30}\sffamily\bfseries#1}};
\end{tikzpicture}};
\end{tikzpicture}}
\addcontentsline{toc}{part}{\texorpdfstring{%
\setlength\fboxsep{0pt}%
\noindent\protect\colorbox{ocre!40}{\strut\protect\parbox[c][.7cm]{\linewidth}{\Large\sffamily\protect\centering #1\quad\mbox{}}}}{#1}}%
\@endpart}
\def\@endpart{\vfil\newpage
\if@twoside
\if@openright
\null
\thispagestyle{empty}%
\newpage
\fi
\fi
\if@tempswa
\twocolumn
\fi}

%----------------------------------------------------------------------------------------
%	CHAPTER HEADINGS
%-------------------------------------------------%--------------------------------------

%
%
%% A switch to conditionally include a picture, implemented by Christian Hupfer
\newif\ifusechapterimage
\usechapterimagetrue
\newcommand{\thechapterimage}{}%
\newcommand{\chapterimage}[1]{\ifusechapterimage\renewcommand{\thechapterimage}{#1}\fi}%
\newcommand{\autodot}{.}

\def\@makechapterhead#1{%
{\parindent \z@ \raggedright \normalfont
\ifnum \c@secnumdepth >\m@ne
\if@mainmatter
\begin{tikzpicture}[remember picture,overlay]
	
\node at (current page.north west)
			{\begin{tikzpicture}[remember picture,overlay]
			\node[anchor=north west,inner sep=0pt] at (0,0) {\ifusechapterimage\includegraphics[width=\paperwidth]{\thechapterimage}\fi};
			\draw[anchor=west] (\Gm@lmargin,-9cm) node [line width=2pt,rounded corners=15pt,draw=ocre,fill=white,fill opacity=1,inner sep=15pt]{\strut\makebox[22cm]{}};
			
			\draw[anchor=west] (\Gm@lmargin+.3cm,-9cm) node {\fontsize{20}{30}\selectfont\sffamily\bfseries\color{ocre}~#1\strut};
			%{\Huge\sffamily\bfseries\color{ocre}\thechapter\autodot~#1\strut};
			\node at (current page.north east)  [xshift=-3.95cm, yshift=-4.05cm, text opacity=1]  {{\color{white}\centering\fontsize{200}{30}\selectfont \bfseries\sffamily\thechapter\strut} };
			\node at (current page.north east)  [xshift=-4cm, yshift=-4cm, text opacity=1]  {{\color{ocre}\centering\fontsize{200}{30}\selectfont \bfseries\sffamily\thechapter\strut} };
			
			\end{tikzpicture}};
	\end{tikzpicture}

\else

\begin{tikzpicture}[remember picture,overlay]
\node at (current page.north west)
{\begin{tikzpicture}[remember picture,overlay]
\node[anchor=north west,inner sep=0pt] at (0,0) {\ifusechapterimage\includegraphics[width=\paperwidth]{\thechapterimage}\fi};
\draw[anchor=west] (\Gm@lmargin,-9cm) node [line width=2pt,rounded corners=15pt,draw=ocre,fill=white,fill opacity=1,inner sep=15pt]{\strut\makebox[22cm]{}};

\draw[anchor=east] (\Gm@lmargin,-9cm) node [line width=2pt,rounded corners=15pt,draw=ocre,fill=white,fill opacity=1,inner sep=15pt]{\strut\makebox[22cm]{}};
\draw[anchor=west] (\Gm@lmargin+.3cm,-9cm) node {\huge\sffamily\bfseries\color{ocre}#1\strut};

\end{tikzpicture}};
 
\end{tikzpicture}
\fi\fi\par\vspace*{270\p@}}}

\makeatother
%-------------------------------------------

%\def\@makeschapterhead#1{%
%\begin{tikzpicture}[remember picture,overlay]
%\node at (current page.north west)
%{\begin{tikzpicture}[remember picture,overlay]
%\node[anchor=north west,inner sep=0pt] at (0,0) {\ifusechapterimage\includegraphics[width=\paperwidth]{\thechapterimage}\fi};
%\draw[anchor=west] (\Gm@lmargin,-9cm) node [line width=2pt,rounded corners=15pt,draw=ocre,fill=white, fill opacity=0.8,inner sep=15pt]{\strut\makebox[22cm]{}};
%\draw[anchor=west] (\Gm@lmargin+.3cm,-9cm) node {\huge\sffamily\bfseries\color{ocre}#1\strut};
%\end{tikzpicture}};
%\end{tikzpicture}
%\par\vspace*{270\p@}}




%----------------------------------------------------------------------------------------
%	LINKS
%----------------------------------------------------------------------------------------


\usepackage{bookmark}
\bookmarksetup{
open,
numbered,
addtohook={%
\ifnum\bookmarkget{level}=0 % chapter
\bookmarksetup{bold}%
\fi
\ifnum\bookmarkget{level}=-1 % part
\bookmarksetup{color=ocre,bold}%
\fi
}
}

%----------------------------------------------------------------------------------------
%	FANCY CHAPTER Header
%----------------------------------------------------------------------------------------





\definecolor{ocre}{RGB}{160,0,0}   
%-------------------------------------------------
%	BIBLIOGRAPHY
%-------------------------------------------------

%\usepackage{enumitem}
\addbibresource{../tex/bibliography.bib} % BibTeX bibliography file
\defbibheading{bibempty}{}

%-------------------------------------------------
%	COLORS & BORDERS
%-------------------------------------------------
\setlength{\fboxrule}{0.75pt} % Width of the border around the abstract
\definecolor{color1}{RGB}{160,0,0} % Color of the article title and section
\definecolor{color2}{RGB}{220,220,220} % Color of the boxes behind the abstract and headings

%-------------------------------------------------
% ITEMS DEFINITION
%-------------------------------------------------

\setlist [itemize,1]{label=\color{color1}\faCaretRight }


\begin{document}

%-------------------------------------------------
%	ARTICLE INFORMATION
%-------------------------------------------------

\Abstract {\uabstract%===========================
% COURS "INTRO CYBERDEF"
% Abstract général des articles
%===========================


Il fait partie du cours introductif aux fondamentaux de la sécurité des systèmes d'information vue sous deux prismes quelques fois opposés dans la littérature : la gouvernance et la gestion opérationnelle de la sécurité.
Le cours est constitué d'un ensemble de notes de synthèse compilé en un document unique.\\
Ce document ne constitue pas à lui seul le référentiel du cours. Ce sont des notes de synthèse mises à disposition comme support pédagogique. }
\JournalInfo{\uJournalInfo} % Journal information
\Archive{Notes de cours éditées le  \DTMnow} % Additional notes (e.g. copyright, DOI, review/research article)
\PaperTitle{\utitle} % Article title
\Authors{\uauthor\textsuperscript{1,}\textsuperscript{2}*} % Authors
\affiliation{\textsuperscript{1}\textit{\uproa}} % Author affiliation
\affiliation{\textsuperscript{2}\textit{\uprob}} % Author affiliation
\affiliation{*\textbf{email}: \umaila\ -- \umailb} % Corresponding author
\Keywords{\ukeywords} % Keywords 
\newcommand{\keywordname}{Mots clefs} % Defines the keywords heading name

%-------------------------------------------------
%	MAKE TITLE
%-------------------------------------------------

\maketitle

\begin{tikzpicture}[remember picture,overlay]\node at (current page.north west)[xshift=4cm, yshift=-4cm, text opacity=1.0]{\includegraphics[width=0.1\paperwidth]{../Tex/template.inc/Commons/CommonsPictures/shield-20.pdf}}; 
\end{tikzpicture} 

%-------------------------------------------------	
% Download
\newcommand{\safeqrcode}[2][]{%
  \qrcode[#1]{\detokenize{#2}} }
\newcommand{\GITfilename}{https://github.com/edufaction/CYBERDEF/raw/master/Builder/\jobname.pdf}
\begin{center}
\setstretch{2}
 {\link{\GITfilename}{\textbf{\jobname.pdf }sur GITHUB CYBERDEF} }   \\ 
{{\huge\ccbyncndeu}}  \\  
{2020 eduf@ction Publication en Creative Common BY-NC-ND }   \newline % Copyright notice
{\safeqrcode[padding]{\GITfilename} }  
\end{center}
%-------------------------------------------------
	\newpage

%-------------------------------------------------
\ubody % MAIN BODY defined external
%-------------------------------------------------

	\printbibliography
	\newpage
	
%-------------------------------------------------	
\Ucontribute % Contribution TEXT
%-------------------------------------------------
	
	\newpage
	\tableofcontents 
	
\iftotalfigures
  \listoffigures
\fi

\end{document}
}
\fi



 


%************************************************************
% Chargement des variables dédiées à l'article
%************************************************************

%========================================
\newcommand {\ukeywords}
%========================================
{%---------------------------------------------------------------------
SEC101, Cybersécurité, Cyberdéfense, PSSI, ISO27001, Analyse de risques
}%

%========================================
\newcommand {\utitle}
{%---------------------------------------------------------------------
 Introduction à la cyberdéfense d'entreprise SEC101
}%---------------------------------------------------------------------

%========================================
\newcommand {\uabstract}
{%---------------------------------------------------------------------
C\edoc  fournit les éléments d'introduction au cours SEC101 du CNAM et d'Orange Campus Cyber.
}
%---------------------------------------------------------------------


%************************************************************
%  variable définissant  le corps de l'article
%************************************************************

%===========================
\newcommand {\Ucontribute}
{
%contrinbite

\section{Contributions}

\subsection{Comment contribuer}

Les fichiers sources de ce document sont publiés sur GITHUB \link{https://github.com/edufaction/CYBERDEF}{(edufaction/CYBERDEF)} . Vous pouvez contribuer au projet des notes de cours Cyerbsécurité SEC101 (CYBERDEF101). Le fichier Tex/Contribute/Contribs.tex contient la liste des personnes ayant contribué à ces notes de cours.
Le guide de contribution est disponible sur le GITHUB


}


%========================================
\newcommand {\ubody}
{%---------------------------------------------------------------------
%$BEGIN_UBODY
% Chap-Intro-gen.tex
% Introduction générale au cours SEC 101

\section{Avant propos }

Chaque jour, la presse se fait l'écho d'attaques et de piratages informatiques, de fragilités découvertes dans les produits et services incluant des codes logiciels, de vols de données. ou de divulgations d'informations sensibles.
Derrière ces incidents, nous découvrons des menaces de tout ordre, actions criminelles, étatiques, hacktivismes. Construire des systèmes surs, les protéger et les défendre, dans une société en ou d'accélérer la digitalisation est devenu un challenge quotidien pour des équipes spécialisées qui luttent contre ces menaces. 
La cybersécurité est un domaine de mythes et de légendes. Ses activités plongent au plus profond de notre histoire avec des notions comme la course entre le méchant et le gentil, le gendarme et le voleur jusqu'au corsaire et au pirate, en n'oubliant pas les luttes secrètes entre les espions et le contre-espionnage. Une thématique qui résonne, donc comme un domaine de romans, qui se traduit toutefois par une réalité souvent moins réjouissante pour les équipes chargés de la cybersécurité dans les entreprises. Les métiers de la cybersécurité sont nombreux, pour certains très techniques, d'autres plus fonctionnels, juridiques, ou managériaux. 

La cybersécurité est une discipline transverse et interdisciplinaire à plusieurs titres :
\begin{itemize}
  \item nécessité de maîtriser les nombreuses techniques et technologies des systèmes d'information ainsi que leurs zones de fragilités;
  \item nécessité de maitriser de nombreuses solutions de sécurité permettant de couvrir, en n'oubliant qu'elles aussi peuvent être fragiles (Cf. Certification et Qualification de produits de sécurité et Critères communs);
% TODO : référe 
  \item nécessité de faire coopérer des métiers et des cultures différentes ;
  \item nécessité de gérer l'entreprise dans des cadres de conformité souvent complexes et coûteux;
  \item nécessité d'intégrer ces démarches en tenant compte des cultures et des pratiques des nombreux métiers de l'entreprise.
\end{itemize}


\mode<all>{\picframe{Tex/Pictures/img-4model}{Cybersécurité : un domaine hollistique}{0.9}{lb_4model}}

Ces métiers concourent tous à une seule et même mission : \g{assurer la continuité de la mission ou du service en préservant le patrimoine de l'entreprise contre toute menace dans l'environnement numérique}.

%\section{les grands cyber-risques}

% les grzands typologies de menaces liées au politique,, au vols, à la guerre écononiques" : Chnatage, Ilmages, Vole, Continuité d'activité ... manipulation, esionnage ...

% Les cyberrisques dépenade de la cibles, et souvent la cibles ets liée à l'attaquant

%\section{les vecteurs cyber}

\section{Aborder la cybersécurité}

% ajouter le joke sur les corsaires et une hallebarde

La cybersécurité ou la sécurité du numérique \footnote{Historiquement d'autres termes sont utilisés comme  \gls{aSSI} ou  Sécurité Informatique}, peut être découverte par de nombreuses voies.

La plus courante est certainement pour les technophiles, l'aventure passionnante de découvrir ce domaine par la technique.  Longtemps abordé par le triptyque académique cryptologie,   sécurité  protocolaire des réseaux, et informatique fondamentale (compilation et théorie des langage, architecture système et bases de données), le domaine s'est vulgarisé avec une forme de gamification de l'apprentissage.

On y trouve en particulier :

\begin{itemize}
  \item Les challenges comme les \UKword{\gls{aCTF}}, ou les \UKword{\gls{aDTF}} qui permettent de mettre le pied dans les techniques et stratégies d'intrusion pour les pentesteurs et auditeurs techniques en herbe,;
  \item Les bug-bounty qui permette de se confronter à ses propres limites avec la recherche de failles dans les logiciels avec pour partie des rémunérations au niveau des difficultés;
\end{itemize}

Toutefois, un volet peu enseigné, qui ne mobilise pas spécialement les jeunes apprenants du domaine concerne la gouvernance de cette sécurité numérique de l'entreprise.

J'ai souhaité présenter une approche globale de la sécurité du numérique pas le biais de quelques processus, en particulier ceux de la sécurité opérationnelle. L'enjeu est de fournir une trame de connaissance pour déployer des actions de cybersécurité en entreprise.
Cette trame a pour intention fournir des points d'accroche et des modèles de compréhension des différentes compétences, actions, et outils du large domaine de la cybersécurité.

Destiné à un public large, cet ouvrage tente d'offrir un niveau de lecture permettant à un expert technique de repositionner sa technicité dans un ensemble plus large, et  à un débutant de découvrir de nombreuses facettes du domaine avec quelques éclairages techniques.

La cybersécurité dans une entreprise est une co-activité d'hommes de l'art.  C'est aussi un domaine en perpétuelle évolution, soutenu et contraint par des lois, des règlements , des normes, des méthodologies,  des technologies spécialisées et en particuliers des expertises.  Il nécessite pour être efficace d'être orchestré pour maintenir en condition de sécurité une organisation dont le périmètre peut être complexe face à des menaces elles aussi en perpétuelles évolutions.

Il y a de nombreuses manières d’aborder le pilotage de la cybersécurité au sein de l’entreprise, et nombreux ouvrages spécialisés en détaillent les concepts et les méthodologies. Nous avons toutefois délibérément choisi ici de confronter, si ce n'est corréler, dans un seul support, trois domaines qui apparaissent souvent dans la littérature comme des domaines d'expertise différents : la gestion des risques , la gouvernance de la cybersécurité, et la cybersécurité opérationnelle. 

Nous avons donc fait ce choix de structurer notre approche suivant le prisme de la cyberdéfense d'entreprise avec une analyse en trois axes majeurs qui résument les difficultés dont relève cette discipline holistique \cite{sch13}. 




% begin PRZ===================
\begin{frame}<presentation>{Les 3 axes de la cybersécurité}
	\begin{itemize}
 \item \tb{l'analyse des risques} informatiques sur les actifs les plus sensibles de l'entreprise avec les difficultés d'identifier la sensibilité de ces actifs et les menaces qui pèsent sur l'environnement ;
\item la structuration des \tb{politiques de sécurité} des systèmes d'information pour des architectures de sécurité de confiance, dans des systèmes d'informations complexes, intégrant des services dans le cloud, des technologies obsolètes et des politiques de sécurité sédimentées ;
\item la construction et l’organisation d'une \tb{sécurité opérationnelle} vue sous un angle d'anticipation et de veille, de détection, et enfin d'alerte et de réponse aux attaques, nécessitant une activité continue avec des ressources de plus en plus expertes et avec des outils plus \g{pointus}.
\end{itemize}
\end{frame}
% end PRZ====================

%\rem{Politiques : le monde de la reflexion, Stratégies le monde l'action}
\subsection {Politiques versus stratégies}

\mode<all>{
\picframe{Tex/Pictures/img-cycle}{Processus Cyber d'entreprise}{0.8}{lbl-cycle}
}

La figure \ref{lbl-cycle} présente la dynamique avec laquelle nous avons structuré dans c\edoc.


De  \head{l’analyse de risque}, nous pouvons déduire et/ou modifier des politiques de sécurité adaptées. 
Sur la base de l’existant, il est alors possible d’adapter ou simplement de mieux utiliser ou configurer les architecture techniques et organisationnelles pour définir un niveau de protection attendu.
Il y a malheureusement toujours un écart entre les mesures de sécurité souhaitées et la réalité des mesures déployées. Que ce soit des défauts de configuration, des délais de mise en place plus longue prévu, le système n’est que très rarement au niveau décrit dans les éléments de spécification ou les documents d'assurance sécurité.
Mesurer ce niveau, analyser les écarts et remédier relève d’un des grands thèmes de la gouvernance sécurité. Il reste à lui seul un consommateur à plus de 30\% des charges d'activité de cette gouvernance.

Après avoir définit des \head{politiques de sécurité} et mesurer leur déploiement dans l’environnement de l’entreprise, il n’en demeure pas moins que l’ennemi est toujours à ses portes et de plus en plus souvent. Il arrive à pénétrer le périmètre de sécurité.
Non pas que les barrières et filtre de l'entreprise ne sont plus efficaces mais simplement parce que l'attaquant change plus souvent de stratégie que l'entreprise de politique. La sécurité se doit d'être plus dynamique.
L’entreprise doit faire face à des attaquants qui ne raisonnent pas sous forme de politiques d’attaque, mais en stratégie d'action. L’entreprise doit raisonner aussi de la même manière pour se défendre. C'est à ce titre que l'on parle de stratégie de Cyberdefense.
C'est avec des stratégies de \g{cyberdéfense} que nous aborderons les moyens organisationnels et techniques à mettre en place.

Vous trouverez  dans ce document une terminologie qui peut être certaines fois éloigné des expressions classiques de la sécurité informatique. \\
J'ai choisi de d'utiliser et mixer sans trop de complexes des termes et concepts issus du monde militaire (renseignement, tenir une position, infiltration …) et de nombreux autres issus de l'univers médical (infection, épidémie, comportement).\\ Ces incursions dans les analogies d'autres champs professionnels, bien que présents pour illustrer certains concepts, n'en demeurent pas moins justifiés par leurs usages de plus en plus répandus dans le monde de la cybersécurité. Par ailleurs, les termes sécurité, et cybersécurité pourront être utilisés indifféremment dans le corps de ce document.

\subsection {Transformation numérique}

% Les causes de la complexité

La cybersécurité est devenue en quelques années un axe fondamental dans la prise en compte de ces nouveaux risques sociétaux qu'apporte l'informatique au cœur de chaque activité sociale, économique ou politique. \\

%Quelque que soit le niveau d'impact  --- personnel, entreprise, Etatique, 
Les  transformations digitales d'une grande partie des acteurs économiques apportent de nouveaux risques. Les modifications des conditions d'utilisation des technologies dans les crises comme celle du COVID-19 engendrent aussi de risques globaux réduisant les frontières entre les espaces professionnels et les espaces privés.

Le législateur s'en est saisi depuis bien des années avec de nombreuses réglementations et lois permettant de protéger en particulier, le citoyen et l'Etat.

On notera en particulier dans cette évolution du cadre réglementaire,  la protection des données personnelles, mais aussi la protection des systèmes sensibles stratégiques \footnote{\gls{aOIV}  et \gls{aOSE} } en lien avec la protection de la nation avec la dynamique de Cyberdéfense soutenue par les différentes lois de programmation militaire depuis 2008. L'entreprise se trouve quant à elle prise en sandwich entre les exigences de l'état et les désirs de liberté que défend le citoyen. Il faut aussi noter que le salarié est souvent un citoyen et son rôle dans la cybersécurité de l'entreprise peut soulever des problématiques complexes\footnote{Lanceurs d'alertes, comportements déviants des utilisateurs légitimes}.
 
Se sentir en sécurité dans un monde de transformation digitale c'est bien entendu disposer des moyens de se protéger et protéger son patrimoine, que ce dernier soit ou non informationnel, mais aussi de le défendre en continue. Il est moins de moins en moins accepté de le protéger, en érigeant des murs épais, solides et supposés infranchissables. L'entreprise a besoin de faire circuler rapidement les savoirs, de partager largement des informations entre les salariés, les clients, les citoyens, les fournisseurs...


Il est donc nécessaire de correctement définir les biens vitaux ou essentiels pour y mettre les meilleurs moyens pour les défendre. Par ailleurs comme toute activité protégée et défendue qui peut subir des dommages, il est important de structurer l'activité numérique d'une entreprise ou d'une organisation pour pouvoir fonctionner en mode dégradé, et revenir à la normale en moins de temps possible.


Entre une maitrise des risques cyber et une capacité de se défendre et réagir, il est nécessaire de disposer déjà d'un bon niveau de protection adaptée aux enjeux du numérique. Il existe de nombreuses définitions de cette cybersécurité.

Pour ma part je vous propose de poser pour la suite de mon propos, une définition simple, qui fait consensus et résume en une pseudo équation la manière dont nous traiterons ce domaine dans c\ecours. \\
\begin{nota}[Une définition de la cybersécurité]
\begin{align}
Cybers\acute{e}curit\acute{e} \cong Cyberprotection\oplus Cyberd\acute{e}fense \oplus Cyberr\acute{e}silience
\end{align}
\end{nota}


% begin PRZ====================
\begin{frame}<presentation>{Une définition de la cybersécurité}
\begin{align}
Cybers\acute{e}curit\acute{e} \cong Cyberprotection\oplus Cyberd\acute{e}fense \oplus Cyberr\acute{e}silience
\end{align}
\end{frame}
% end PRZ====================

% begin PRZ====================
\begin{frame}

La cybersécurité est l'enchainement opéré, organisé, documenté, piloté, optimisé de trois environnements d'actions :
\begin{itemize}
 \item \tb{Protéger} l'environnement par les mesures et solutions technologies adaptées au niveau de risque que l'entreprise est prêt à prendre; 
 \item \tb{Défendre} les actifs les plus sensibles de l'entreprise en surveillant et combattant la menace (y compris l'image de l'entreprise);
 \item assurer \tb{la continuité et la reprise d'activité} de l'entreprise face à tout incident rendant indisponible tout ou partie d'une fonction essentielle de celle-ci.
\end{itemize}

\end{frame}
% end PRZ====================

La Cybersécurité, est donc, avant-tout, le déploiement de mécanismes de protection des biens et des processus numériques sensibles. C’est avec cette première dynamique que l’entreprise déploie en premier lieu des solutions de sécurité. 

Toutefois, malgré ce niveau de protection et souvent les lourds investissements réalisés dans des composants de sécurité périmètrique, l’entreprise peut se faire surprendre avec des attaques contournant ces mesures. Face à ces attaques, l’entreprise découvre que la solidité de l’entreprise n’est pas directement lié aux investissements sur les systèmes de protections. Il lui faut anticiper les menaces, les détecter non seulement sur son périmètre mais aussi dans l’écosystème de l’environnement les menaces potentielles. Ces menaces exploitent des vulnérabilités qu’il convient de détecter en amont.

Malheureusement, malgré ces mesures de protection et de défense qui permet de réagir vite et efficacement, il arrive que des attaques informatiques arrivent à leurs fins. La capacité de l’entreprise à revenir à une situation normale, avec un contexte assaini est un critère dont un chef d’entreprise appréciera la valeur\textbf{ qu’après un incident}.

% ajouter l’equation du risque dans le texte du RISQUE

Il fut une époque pas si lointaine, ou dans l'évaluation de la probabilité d'une attaque, l'analyste consacrait du temps. Aujourd'hui bien que ce paramètre continue quand même à être pris en compte, l'analyste positionne cette probabilité ou vraisemblance à 100\%.  (cf. figure \ref{lbl-riskassess})

% I M A G E -----------> 


\mode<all>{\picframe{Tex/Pictures/img-risk-assess}{le cyber-risque}{0.9}{lbl-riskassess}}

L'ensemble des experts du domaine est globalement en accord sur la posture que doivent prendre les entreprises et les organisations : \g{Le temps n'est plus de savoir si on sera attaqué ou pas, mais plutôt de savoir quand et comment on le sera}, qui concrètement se résume à la certitude que tout incident de sécurité peut se produire.
 
Dans les modèles d'analyse de risque et d'évaluation de la cybersécurité, l'analyste se positionne aujourd'hui du point de vu de l'attaquant. Ce regard lui permet de mieux comprendre la menace comme équation duale du risque, mais vu de l'énergie dépenser par l'attaquant et du risque qu'il prend. (cf. figure \ref{lbl-riskjail})



\mode<all>{\picframe{Tex/Pictures/img-risk-prison}{La menace : une vision de l'attaquant}{0.9}{lbl-riskjail}}


\section{Sécurité du système d'information}

Le système d'information est au coeur de ce \g{monde digital }, et il est le lieu d'activités humaines très denses, permettant à des utilisateurs de réaliser leurs activités, professionnelles ou privées, à l'aide de processus informatiques et de services. 

Ces activités doivent de plus en plus faire face à tout un système d'agression orchestré par des attaquants non seulement humains mais aussi\g{automatiques}. On observe aujourd'hui une multitude de situations critiques, incertaines dont l'occurrence quasi quotidienne provient de phénomènes variés, humains (isolés, en réseau,…), physiques et/ou technologiques. Parmi ces difficultés qui profitent aux pirates informatiques il y a de nombreuses failles ou fragilités que nous découvrirons, provenant des systèmes du SI sans lesquelles ils ne pourraient exploiter leurs attaques. Ces phénomènes sont une menace pour les conditions de sécurité du système d'information.

\subsection{Les fonctions SSI de gouvernance}

Au sein des grandes entreprises, il existe de nombreuses fonctions ou missions pour gouverner, piloter cette sécurité numérique.

%----begin FRAME-------
\mode<presentation>{\texframe
{SSI : Responsabilités}
{les métiers}
{
\begin{itemize}
 \item \head{Le gestionnaire de risque} ou \UKword{Risk Manager} qui porte l'animation de la gestion des risques dans les projets ou dans l'entreprise;
\item \head{Le responsable sureté / sécurité } généralement responsable de la sécurité physique ou sein de l'entreprise (vol, intrusion physique, contrôle d'accès). Il endosse le plus souvent la responsabilité des biens et des personnes;
\item \head{L'audit et le contrôle} : Au sein des grandes organisations, il peut exister un service \g{indépendant} dont la mission est d'auditer et de contrôler les activités des services;
\item \head{Les RSSI} : Responsables de la sécurité des Systèmes d'Information;
\item \head{Les DSSI} : Au sein des grandes entreprises, les RSSI globaux ne dépendent plus trop de DSI, et possèdent le rang de directeur;
\item \head{Le DPO} : la dernière responsabilité apparue dans l'environnement de la sécurité (En France successeur du CIL , Correspondant Informatique et Liberté) (\UKword{Data Protection Officer}).
\end{itemize}
}}%----end FRAME-------


Nous ne présentons rapidement ici que ceux qui seront utilisés directement dans ce document et qui sont fortement en liaison avec la sécurité des systèmes d'information.

\subsubsection{Les DSSI et RSSI}

Au sein de l'entreprise, il est important que quelqu'un porte la charge de suivre ces conditions de sécurité. C'est le rôle du RSSI (Responsable de la sécurité des systèmes d'information), ou DSSI (Directeur de la sécurité des systèmes d'information).
La mission de ce RSSI d'entreprise est de protéger son Système d'Information (SI), de le mettre dans une posture d'amélioration continue tant de son système de protection que de son système de défense. Le RSSI n'est pas seul pour assumer ces missions, à tous les niveaux de l'entreprise, s'organise des fonctions de sécurité tant au sein de la DSI (auquel est souvent rattaché le RSSI), qu'au sein d'autres activités de l'entreprise. 

\subsubsection{Le DPO}

A partir de mai 2018, une responsabilité plus juridique liée à la protection des données a été rendue plus visible avec la nécessité de disposer d'un \gls{aDPO} en entreprise (Data Protection Officer) héritier en France  \textit{\gls{aCIL}}.
Nous n'aborderons pas la fonction, les missions et la dynamique de responsabilité du \gls{aDPO} ici. Il faut toutefois que cette fonction possède de nombreux recouvrements dans la chaine de gouvernance du risque \g{informatique} auprès des directions d'entreprise. Orienté vers la protection des données à caractère personnel, le tropisme de la fonction \gls{aDPO} peut conduire certaines structures à oublier des pans importants des risques numériques comme :

\begin{itemize}
	\item la protection du patrimoine informationnel. (Espionage industriel).
	\item la protection des systèmes d'information contre risques de ruptures de services (Continuité d'activité)
\end{itemize}

\subsubsection{L'officier de sécurité de défense}
Pour les entreprises traitant des informations classifiées de défense ou liées aux contraintes de la classification de l'état, il est indispensable de se doter d'une fonction \gls{aOS} de défense. Son rôle est de s'assurer de la conformité à l'Instruction Générale Interministérielle IGI 1300 pour le \g{Confidentiel Défense} et l'instruction Interministérielle II901 pour le \g{Diffusion Restreinte}.

Nous resterons donc dans le cadre fonctionnel de la Cybersécurité dans son volet protection des Systèmes d'information et gestion des risques numériques.
Quand nous aborderont des sujets en forte adhérence avec les dynamiques de la \gls{aGDPR} nous donnerons les liens et les indications adaptés pour les \gls{aDPO}.
Par exemple, nous aborderons l'usage des données nominatives collectées et traitées dans les \gls{aSIEM}, ou celle recueillis sur le DarkWeb etc ...
\begin{remark}
Consulter le site \url{www.cnil.fr} pour parfaire ses connaissances en matière de réglementation européenne sur la protection des données personnelles. 
\end{remark}

\subsubsection{Responsabilités SSI}

Le maintien des conditions de sécurité du système d'information des grandes entreprises nécessite un \gls{aRSSI} Central ou une fonction semblable rattachée à un niveau plus global de l'entreprise.
On découvre ainsi des  \gls{aRSSI} rattachés à la direction des risques, la direction générale, au contrôle interne …\\
 Il n'y pas de rattachement bien définit. La couverture de responsabilité dépend grandement de la taille et de l'activité de l'entreprise, mais aussi de la maturité de celle-ci en matière de gestion de risque et de gouvernance. Il peut y avoir des RSSI par entité, par projet à l’intérieur d'une entreprise. Leur mandat est fixé en fonction des enjeux sécurité de ces entités ou ces projets. 

Le fin mot de l'histoire est le \g{R} de  \gls{aRSSI}. Son domaine de responsabilité dépendra de son mandat pour assumer ce rôle de garant d'un environnement \g{possédant } des bonnes conditions de sécurité.
La gouvernance de la sécurité, est au coeur du métier du  \gls{aRSSI}. Cette discipline que ce dernier pilote dans l’entreprise se nomme \gls{aGRC}.

La notion de système d'information a profondément évoluée ces dernières années. Le périmètre des risques digitaux inclus maintenant des systèmes et services externes à l'entreprise. Beaucoup d'entre eux sous la forme de réseaux sociaux, de services cloud ouvrant par ailleurs le domaine de supervision à la téléphonie avec les smartphones et leurs applications professionnelles ou non.

Bien entendu en fonction de la taille de l'entreprise et de ses enjeux, on peut disposer au sein de l'entreprise de nombreuses personnes ayant une fonction de \gls{aRSSI}. 

Le métier est riche et dispose d'un spectre de responsabilité et d'activité très large en terme de poste on y trouve par exemple :

% FRAME beamer PRZ ------------------------------------
\mode<all>{\texframe{Fonctions RSSI}{différents métiers} 
{
\begin{itemize}
\item \textbf{RSSI d'entreprise} : Responsable de la sécurité de sa structure.
\item  \textbf{RSSI d'un département, d'une organisation intermédiaire} : A l'image d'un RSSI d'entreprise, il assure toute les taches de gouvernance, il applique et fait appliquer les directives et politique de sécurité aux équipes du département / division / structure intermédiaire, il déploie les actions décidée dans la chaîne fonctionnelle sécurité
\item   \textbf{RSSI d'un contrat, d'un projet contractualisé (Security Manager)} : Responsable de la sécurité du  déroulement d'un contrat. Souvent lié à un plan d'assurance sécurité, le RSSI contrat se doit d'assurer pour le client ou pour le fournisseur le suivi des exigences de sécurité du contrat.
\end{itemize}
}} % end FRAME.........................................................


\mode<all>{\texframe{Fonctions RSSI}{suite}
%. . . . . . . . . . . . . . . . . . . . . . . . . . . . . . . . . . . . . . . . . . . .
{\begin{itemize}
	\item \textbf{RSSI Projet} : La responsabilité sécurité couvre le projet. Le RSSI on parle souvent de \g{security by design}. La responsabilité dans ce type de poste recouvre l'intégration de la sécurité dans le système, le suivi des indicateurs définis (contractuels, ou réglementaires), la remonté des indicateurs de suivi de sécurité à la MOA (Maitrise d'ouvrage), la prise de décision autour des choix de sécurité ....
	\item \textbf{RSSI Produit / Service} : Au delà de ce qui est fait pour un projet, le RSSI produit a en charge de gérer la sécurité opérationnel c'est à dire Maintenir la sécurité de son produit ou de son service.
 \item \textbf{RSOP} : Le responsable sécurité opérationnelle, est souvent une RSSI dépendant d'une DSI, il est généralement et dans beaucoup de d'entreprise de taille moyenne le RSSI technique. Il assure opérationnellement la mise en place technique des politiques de sécurité et maintien en condition de sécurité l'ensemble de l'environnement informatique. Il est aujourd'hui au coeur de la sécurité opérationnelle face aux attaques et aux crises cyber.
\end{itemize}
}} % end FRAME.........................................................


\subsection{Maintien en condition de sécurité}

Les conditions de sécurité représentent les propriétés fondamentales du SI, appelées : \gls{aDICT} , qui favoriseront le fonctionnement optimisé du SI et éviteront l'avènement d'incidents de sécurité irréversibles ou même gênants pour son fonctionnement. D'un certain point de vue, les conditions de sécurité représentent le paramétrage du SI pour lequel le système fonctionne bien dans des conditions de sécurité \g{connues et approuvées}.

Ces fameux critères \textbf{\gls{aDICT}} ou propriétés de sécurité des systèmes d'information vise les objectifs suivants :
%-----------------------------------------
\begin{nota}[DISPONIBILITE]
le système doit fonctionner sans faille (arrêt, ou dégradation) durant les plages d'utilisation prévues et garantir l'accès aux services et ressources définies et installées avec le temps de réponse attendu.
\end{nota}
%-----------------------------------------
\begin{nota}[INTEGRITE]
Les données doivent être celles que l'on attend, et ne doivent pas être altérées de façon fortuite, illicite ou malveillante. En clair, les éléments considérés doivent être exacts et complets.
\end{nota}
%-----------------------------------------
\begin{nota}[CONFIDENTIALITE] Seules les personnes autorisées peuvent avoir accès aux informations qui leur sont destinées. Tout accès indésirable doit être empêché.
\end{nota}
%-----------------------------------------
\begin{nota}[TRACABILITE] (ou preuve ) : garantie que les accès et tentatives d'accès aux éléments considérés sont tracés et que ces traces sont conservées et exploitables.
\end{nota}
%-----------------------------------------
D'autres aspects peuvent aussi être considérés comme des objectifs de la sécurité tels que :
%-----------------------------------------
\begin{nota}[AUTHENTICITE]
l'identification des utilisateurs est fondamentale pour gérer les accès aux espaces de travail pertinents et maintenir la confiance dans les relations d'échange. On voit aussi dans la littérature la terminologie \g{critères \gls{aACID} (Authentification, Confidentialité, Intégrité, Disponibilité)}.
\end{nota}
%-----------------------------------------
\begin{nota}[NON-REPUDIATION]
La non-répudiation et l'imputation : aucun utilisateur ne doit pouvoir contester les opérations qu'il a réalisées dans le cadre de ses actions autorisées et aucun tiers ne doit pouvoir s'attribuer les actions d'un autre utilisateur.
\end{nota}

Dans un contexte d'activité économique dense et en perpétuel renouvellement, les conditions de sécurité sont aussi en perpétuelle évolution, c’est pourquoi nous parlons d'un cycle de vie vertueux au cours duquel les nouveaux paramètres tirent profit des expériences passées.  Ainsi, l’amélioration continue également appelée \g{ lean management} dans d’autres domaines (industrie, …) travaille-t-elle sur le cycle de vie des conditions de sécurité souvent appelé \gls{aPDCA}.\\
Ce cycle de vie doit néanmoins être maitrisé par le RSSI en place avec ses équipes, il faut co-produire ces conditions de sécurité, cette maîtrise est complexe, fortement dépendante du contexte de l'entreprise, c’est pourquoi elle doit être accompagnée d'une méthodologie rigoureuse et partagée qui constitue le savoir-faire de base du RSSI et de son équipe. Par ailleurs, parmi ces conditions, certaines sont universelles et d’autres propres à chaque entreprise. \\ Comme le montre le diagramme \ref{lbl-cycledevie}, il est possible aussi d’utiliser un cycle de vie sécurité de type projet, qui se rapproche par ailleurs de la manière dont nous avons structurer c\edoc. 

\mode<all>{\picframe{Tex/Pictures/img-cyclevie}{Cycle de vie sécurité dans les projets}{0.9}{lbl-cycledevie}}


Dans cette optique, ce cadre méthodologie a été défini par le sous-comité 27 de l'ISO, par l'ensemble de normes ISO 27x. Il s’agit également d'un ensemble de bonnes pratiques, qu'un RSSI peut suivre au travers de trois volets fondamentaux qui constituent les référentiels utilisés pour ce cours sur la cybersécurité. La norme 27001 est en particulier un cadre pour organiser la dynamique de la mise en condition de sécurité de l’entreprise et son maintien dans le temps. Cet environnement que le \g{RSSI} doit bâtir est le système de management de la sécurité (\g{SMSI}).\\
Notre dynamique méthodologique est soutenue dans  c\edoc, par trois cadres normatifs : 
 
% FRAME beamer PRZ ------------------------------------
\mode<all>{\texframe{Cadres normatifs}{3 modèles du cadre}
%. . . . . . . . . . . . . . . . . . . . . . . . . . . . . . . . . . . . . . . . . . . .
{
\begin{itemize}
\item Identifier ses cyber-risques sur la base de méthodologies que l’on retrouve dans l’environnement ISO/CEI 27001/27005 mais aussi sur la méthodologie EBIOS de l’ANSSI (Méthode EBIOS RM en particulier); 
\item Elaborer une politique de cybersécurité sur la base des cadres ISO/CEI  27001 et 27002, en n’oubliant pas les architectures de sécurité et la sécurité des architectures associées ; 
\item Détecter en amont des attaques et savoir réagir à ses cyber-incidents en se basant sur ISO 27035 et sur la continuité d’activité avec l’ISO 22301 et 27031.
\end{itemize}
}} % end FRAME.........................................................


\begin{nota}[Pourquoi des normes dans c\edoc?]
L’objectif de c\edoc, n’est pas de présenter en détail un cadre normatif, mais bien de les utiliser pour ce qu’elles sont : des langages communs permettant d’appréhender une terminologie, des méthodologies, des outils. L’ISO 27001 comporte un grand nombre de normes (plus de 50…) qu’il convient de connaitre comme outils terminologiques et de référence. Leur maitrise nécessite une spécialisation le plus souvent demandée pour des métiers de conseil ou d’implémentation pour une certification.
\end{nota}

Ces documents définissent un cadre méthodologique et normatif pour définir, créer, élaborer maintenir, améliorer les conditions ou les critères de sécurité pour le fonctionnement du système protégé et surveillé. 
Ils permettent aux acteurs de l'entreprise évoluant autour du métier RSSI un cadre méthodologique ainsi qu’un \g{how to} du maintien en conditions de sécurité. C’est en particulier au travers de ces trois axes que la mission de RSSI repose. 
Le nombre d’entreprises prêtes à accueillir des spécialistes de ce savoir-faire est en forte augmentation car les PME/PMI ont pris conscience que la sécurisation de l’entreprise est devenu primordiale pour \g{survivre} dans l’écosystème digital de nos sociétés modernes. Les contraintes légales issues de la Loi de Programmation Militaire (\gls{gLPM}), de la Règlementation pour la Protection des Données Personnelles (RGPD), de la directive NIS nécessitent de disposer d’une vision globale et transverse tant technique, qu'organisationnelle ou humaine de la cybersécurité.

Nous tenterons donc dans le suite du cours, de vous donner des contextes d'usages de ces cadres normatifs indispensables pour aborder la cyberdéfense d’entreprise. 

\section{Enjeux légaux}

Beaucoup d'environnements normatifs sont issus de la pression des différents cadres législatifs sur le marché. Que ce soit avec la pression du grand public ou avec les enjeux stratégiques et économiques des pays, ces lois organisent profondément les modes de gouvernance de la sécurité en entreprise.

\subsection{Quelques cadres législatifs d'influence}
Parmi les grandes lois qui ont influencé  le monde de la sécurité des entreprises ces dix dernières années :

\begin{itemize}
 \item En France, le cyberdéfense est largement orientée par les différentes \g{Lois de programmation militaire} avec des directives nationale de sécurité par grands domaines d'infrastructures vitales.
 \item En Europe deux grandes directives ont donné plus de responsabilité aux entreprises dans l'engagement sécurité avec GDPR et NIS qui sont déclinés en droits français via la CNIL, et l'ANSSI. On notera par ailleurs la montée en puissance dans la confiance numérique avec le cadre de certification européen.
 \item Aux Etats Unis, le \UKword{Cloud security Act}, a bouleversé la vision des risques numériques des états avec les potentielles nuisances liées à l'extraterritorialité de lois américaines
 \item En Russie et en Chine, plusieurs lois autour de l'usage d'internet interpellent les entreprises et en particulier celles du numérique sur la protections des données de leurs clients ou utilisateurs de leurs services.
\end{itemize}


\subsection{Le cadre de certification européen}

Le règlement établit un \link{https://www.ssi.gouv.fr/entreprise/reglementation/cybersecurity-act-2/le-cadre-de-certification-europeen}{cadre européen de certification}
 de cybersécurité pour harmoniser à l’échelle européenne les méthodes d’évaluation et les différents niveaux d’assurance de la certification, au sein duquel l’ENISA trouve toute sa place. Les certificats délivrés bénéficieront d’une reconnaissance mutuelle au sein de l’Union européenne (UE). 

\subsection{Cyberdefense et loi de programmation militaire}
Pour ceux intéressés par les contraintes et cadre généraux de la cyberdéfense au sein des lois de programmation successives (2008, 2013, 2019 ...) il est conseiller d'aller voir sur le site de l'ANSSI. Les différentes LPM ont fait évoluer le cadre réglementaire pour assurer à la France une capacité de défendre la continuité de l'état et des infrastructures vitales du pays \link{https://www.ssi.gouv.fr/entreprise/protection-des-oiv/protection-des-oiv-en-france/}{(Cf. Opérateurs d'infrastructures vitales)}.

%-------------------------------------------------------------
%               FR CYBERDEF SECOPS COURSE
%                                        SECOPS
%                                            Intro
%
%                           Introduction Cyberdefense
%                            % Chap-Intro-Ref.tex
%
%                              2020 eduf@ction
%-------------------------------------------------------------



\section{Quelques organismes de référence}

Pour l'entreprise la cybersécurité est un domaine  de nombreux cadres normatifs et réglementaires soutenus bien souvent par contraintes legislatives propres à chaque pays.

Cette normalisation et ces réglementations est riche mais certaine fois complexe.
Le plus simple pour s'enrichir de ces savoirs et surtout pour disposer des meilleurs informations à la sources autant \g{fréquenter} les sites internet institutionnels des organismes qui sont et continuent à être les points de  référence dans le domaine de cybersécurité.

De nombreux services étatiques et de normalisation possèdent des activités dites Cyber dans leur structures :

\begin{itemize}
    \item  Organismes français : AFNOR, Cert FR, CNIL, HADOPI, ANSSI, DGSE, DGSI, DGA/MI, Commandement de la cyberdéfense, C3N, OCLCTIC, BEFTI ...
  \item  Organismes internationaux : ISO, ETSI, CERT, Europo, lnterpol , ENISA , FIRST ...
  \item  Organismes étrangers : FBI, CIA, NSA, GCHQ, Unité 8200, Fapsi, The SANS institute, CISA ...
\end{itemize}

Je vous propose de donner quelques pointeurs par portée sur des organismes de référence du point de vue occidental.

\subsection{International et Etats-Unis}

Au niveau international, on ne peut éviter les Etats-Unis, un pays qui oeuvre fortement dans le domaine des standards.

\subsubsection{Le NIST}
Le National Institute of Standards and Technology, ou NIST est une agence du département du Commerce des États-Unis. Son but est de promouvoir l'économie en développant des technologies, la métrologie et des standards avec l'industrie. 

\begin{itemize}
  \item \link{https://csrc.nist.gov/}{NIST COMPUTER SECURITY RESOURCE CENTER }
  \item \link{https://www.nist.gov/itl/fips-general-information}{NIST INFORMATION TECHNOLOGY LABORATORY}
\end{itemize}

On notera en particulier les référentiels cryptographiques du NIST et ceux liées à la cyberdéfense en particulier avec le \UKword{CyberSecurity FrameWork}

\subsubsection{SEI : Université de Carnegie Mellon}

Le Software Engineering Institute (SEI) est un centre de recherche-développement financé par des fonds fédéraux et placé sous le parrainage du département de la Défense des États-Unis ; son fonctionnement incombe à Carnegie Mellon University. Le SEI travaille avec des organisations pour apporter des améliorations significatives à leurs capacités d’ingénierie logicielle en leur fournissant le leadership technique afin de faire progresser la pratique de l’ingénierie logicielle. Le CERT Division du SEI est l’entité qui fait autorité et cherche à améliorer la sécurité et la résilience des systèmes et réseaux en particulier dans le domaine du logiciel(\link{https://www.sei.cmu.edu/research-capabilities/cybersecurity/}{Carnegie Mellon University - Cybersecurity research}).

\subsubsection{l'ISO : International Organization for Standardization }


L’ISO est une Organisation Internationale participant à l’élaboration de Standards. En ce sens la conformité à une norme a l’avantage d’être reconnu internationalement.
 
Les normes de la famille ISO 27000 permettent d’organiser et structurer la démarche de la gestion de la sécurité des systèmes d’information, une grande famille de normes avec des  positionnement sur l'ensemble du spectre  de la sécurité des systèmes d'information : 

\begin{itemize}
  \item ISO~27001 décrit les processus permettant le management de la sécurité de l’information (SMSI);
  \item ISO 27002 présente un catalogue de bonnes pratiques de sécurité;
  \item ISO 27003 décrit les différentes phases initiales à accomplir afin d’aboutir à un système de Management tel que décrit dans la norme ISO~27001;
  \item ISO 27004 permet de définir les contrôles de fonctionnement du SMSI;
  \item ISO 27005 décrit les processus de la gestion des risques;
  \item ISO 27006 décrit les exigences relatives aux organismes qui auditent et certifient les SMSI des sociétés.
\end{itemize}

Nous aborderons dans le chapitre sur les politiques de sécurité, l'usage de ce cadre normatif dans la gouvernance globale de la cybersécurité au sein de l'entreprise

%TODO : Pourquoi se faire certifier ISO27K, les points durs.

\subsection{Europe}

Au niveau européen, le règlement (CE) 460/2004 du Parlement européen et du Conseil du 10 mars 2004 à institué l'Agence européenne chargée de la sécurité des réseaux et de l'information  Agence européenne chargée de la sécurité des réseaux et de l'information \link{https://www.enisa.europa.eu}{ENISA}. Sont rôle est de : 

%BEGIN wikpedia
% TODO
\begin{itemize}
  \item Conseiller et assister la Commission et les États membres en matière de sécurité de l'information et les aider, en concertation avec le secteur, à faire face aux problèmes de sécurité matérielle et logicielle.
  \item Recueillir et analyser les données relatives aux incidents liés à la sécurité en Europe et aux risques émergents.
  \item Promouvoir des méthodes d'évaluation et de gestion des risques afin d'améliorer notre capacité de faire face aux menaces pesant sur la sécurité de l'information.
  \item Favoriser l'échange de bonnes pratiques en matière de sensibilisation et de coopération avec les différents acteurs du domaine de la sécurité de l'information, notamment en créant des partenariats entre le secteur public et le secteur privé avec des entreprises spécialisées.
  \item Suivre l'élaboration des normes pour les produits et services en matière de sécurité des réseaux et de l'information.
\end{itemize}
%END Wikipeida

\subsection{France}

En France, la Cybersécurité est pilotée par un organisme dépendant des services du 1er Ministre,  l'Agence National des Systèmes d'information (\gls{aANSSI}).
L'\gls{aANSSI} possède plusieurs rôles de fait. C'est un \g{régulateur} c'est à dire qu'elle définit des cadres réglementaires pour les entreprises mais c'est aussi une agence qui édicte des préconisations et des guides.

 \link{https://www.ssi.gouv.fr/agence/cybersécurité/ssi-en-france/}{Le site de l'agence} est riche en information et guide sur la cybersécurité.

Dépendant aussi de l'état, la \link{https://www.cnil.fr/}{CNIL} (Commission National Informatique et Liberté) est une autorité dont la mission est de protéger le citoyen. Avec l'avènement du règlement de protection des données personnelles, la \gls{aCNIL} a vu son pouvoir étendu.  

Il faut aussi citer l'AFNOR \gls{aAFNOR}, qui relaie en France la normalisation internationale dont l'ISO au de la de ses actions de normalisation purement française.

\section{Quelques associations et groupements professionnelles} 

A titre d'information, vous trouverez avec ces associations des points d'entrées sur 

\begin{itemize}
  \item \head{Club des Experts de la sécurité de l’Information et du Numérique}.
le \link{https://www.cesin.fr}{CESIN} est une association regroupant les RSSI d'entreprises, l'adhésion à cette association nécessite un parrainage et vous devez être RSSI.

  \item\head{Club de la sécurité de l'information Français}
\link{https://clusif.fr}{CLUSIF}, association qui propose de nombreux échanges sur la cybersécurité.

  \item\head{Club CyberEdu}
\link{https://www.cyberedu.fr}{CyberEdu}, issu des travaux sur la formation des enseignants en cybersécurité de l'ANSSI, l'association regroupe les écoles et les utilisateurs des travaux de CyberEdu.

  \item\head{Club HexaTrust}
\link{https://www.hexatrust.com/le-club/}{HexaTrust}, regroupe les éditeurs et fournisseurs de services français en cybersécurité.
\end{itemize}


 
%----------------------------------------------------------------------
% 		S E C T IO N   PEDAGO
%----------------------------------------------------------------------
\section{Objectifs pédagogiques}
Il me semblait important d'apporter au lecteur un peu d'information autour des éléments pédagogiques de c\ecours. Vous trouverez donc dans ce chapitre quelques éléments sur les compétences, les métiers, le positionnement des activités de la cybersécurité.
En effet, c\ecours tente d'être une introduction à la \etitle permettant à des acteurs du digital n'ayant pas ou peu de connaissance du domaine de repérer dans ce domaine à large spectre d'activités et de métiers.

Nous y abordons aussi les limites de c\ecours ainsi que des recommandations pour aborder le contenu avec plus de facilité pour ceux moins familiers du monde de l'informatique et des réseaux.

\subsection{Les compétences à acquérir}
A l'issue de c\ecours, vous devriez être en mesure de comprendre les mécanismes qui contribuent à la mise en place d'une organisation de cyberdéfense d'entreprise avec les grandes capacités nécessaires.  Pour les réaliser avec efficacité, il est nécessaire de positionner ces activités au sein des fonctions sécurité plus large. Les compétences acquises sont de diverses natures, mais globalement vous devriez être en mesure à un niveau de gouvernance et de pilotage de :  

\begin{itemize}
	\item Analyser les risques numériques pesant sur l'entreprise ou l'organisation ;
	\item Mesurer le niveau de sécurité de de l'environnement ;
	\item Auditer, conseiller, accompagner le changement ;
	\item Mettre en place une gouvernance efficace dans le domaine de la cybersécurité ;
	\item Déployer une politique de sécurité informatique et de cybersécurité et appliquer des méthodologies efficaces de renforcement et d'aguerrissement ;
	\item Comprendre l'intégration des solutions de sécurité suite à l'analyse de risque ;
	\item Gérer des situations d'incident pouvant aller à la crise cyber.
\end{itemize}
La complexité de l'entreprise, sa taille, sa dynamique de prise en compte des enjeux sécurité, sa culture, l'adhérence ou non aux technologies de l'information nécessitent le plus souvent des projets spécifiques adaptés et très contextualisés. Des sociétés de services assistent les entreprises pour auditer, construire, maintenir la sécurité de l'entreprise. Ce document a aussi pour objectif de fournir au lecteur des clefs de lecture pour encadrer et piloter de telles prestations dans le contexte de l'organisation. 


\subsection{Métiers et compétences } 
Il est complexe d'identifier les métiers de la cybersécurité vers lesquels ces compétences peuvent conduire. Il existe plusieurs modèles permettant de classer les métiers de la cybersécurité, et les compétences associées. Pour ma part,  j'ai retenu un modèle que j'ai proposé dans le cadre d'une GPEC (Gestion des emplois et compétence) dans chez un opérateur de services de cybersécurité. Ce modèle est centré sur une classification des outils technologiques utilisés par l'expertise. Issue plutôt de l'expérience, il ne reflète pas les dénominations des différents métiers ou fiche de poste que l'on trouve dans le domaine mais se centre sur les technologies de sécurité vu du côté des opérationnels. Ceci permet de décliner 5 grands domaines d'activité.

\upicture{../Tex/Pictures/img-metiers}{les  grands domaines de métiers}{0.9}{lbl-metiers}

Il y a en effet une grande différence de métier, de compétences entre un spécialiste de la gestion des accès  qui conduira l'intégration de système d'IAM  \footnote{Identity et Access Management} et un ethical hacker qui devra recherche des scenarii d'attaques potentielles sur un système. 

\upicture{../Tex/Pictures/img-metiers-risk2crisis}{les quelques grandes zones de métiers}{0.9}{lbl-risk2crisis}

Au delà de ces grands métiers du service, il est possible de positionner dans le cycle de vie des systèmes différents métiers de la cybersécurité. Les cultures, les objectifs, les technologies utilisées sont différentes mais concourent à la même finalité de protection de l'entreprise.

\upicture{../Tex/Pictures/img-metierslist}{les métiers dans le cycle de vie}{0.9}{lbl-metiersall}


%TODO : Métiers de l'intégration, des opérations, ...

% De l'importance de la TAXONOMIE, POSTE, METIER et COMPETENCE


Si vous souhaitez connaitre avec plus de détails les compétences nécessaires pour les métiers de la sécurité vous pouvez consulter deux grands sites de référence comme celui de l'\head{ANSSI} des 
 \link{https://www.ssi.gouv.fr/particulier/formations/profils-metiers-de-la-cybersécurité/}{métiers de la cybersécurité} ou celui du NIST sur le référentiel  \link{https://www.nist.gov/itl/applied-cybersecurity/nice/resources/nice-cybersecurity-workforce-framework}{NICE Cybersecurity Workforce Framework}


\subsection{Compétences et certifications}

Se former en cybersécurité, c'est pour celui qui travaille avec vous une certaine garantie de compétences. Dans le domaine de la Cybersécurité, la confiance dans les compétences d'un acteur du domaine se base dans le domaine des services en particulier sur la certification. Dans ces certifications, formes de perfectionnement dans un métier, on trouve généralement des  certifications EDITEURS (liés à des produits de sécurité), et des certifications d'associations professionnelles.

Cette dynamique de certification est une manière de compléter les formations initiales et sont  assez différentiante sur un CV dans le monde de l'entreprise en particulier celles qui travaillent dans un environnement international.



\subsection{Certifications éditeurs}

Nous verrons dans le chapitre sur les architectures de sécurité, les produits et services technologiques de sécurité. Une grande partie des fonctions de sécurité techniques est opérée par des produits (Logiciels, Appliances, Services Saas ...). La complexité de ces produits nécessite une formation spécifique pour en exploiter toute la richesse fonctionnelle.
Ces certifications sont par ailleurs souvent obligatoires pour travailler dans les métiers de l'intégration car elle permettent d'accéder au support des éditeurs. A titre d'exemple, nous pouvons citer deux acteurs connus qui disposent de mécanismes et programme de certifications à leurs produits. Ces certifications peuvent par ailleurs être délivrées par des tiers.


\head{Pour CISCO} \link{https://www.cisco.com}{Certifications de carrière CCNA, CCDA} 
%https://www.cisco.com/c/fr_ca/training-events/career-certification


\head{Pour Microsoft} \link{https://www.microsoft.com/fr-fr/learning/certification-overview.aspx}{Certifications }  pour  développeurs, Administrateurs, Architectes Solutions, Consultants.

\subsection{Certifications professionelles} 

La validation d'expertise par des certifications professionnelles est assez répandue dans le milieu de cybersécurité et en particulier dans les pays anglo-saxons. De nombreuses certifications existent, portées par des associations professionnelles, des groupes d'experts ou des entreprises de référence. Ces certifications nécessitent le plus souvent en plus de l'examen des années d'expérience et de pratiques prouvées.


% parler du fait qu'il faut des années d'experience et maintenir par des preuves

\head{\link{https://www.isc2.org/Certifications}{ISC}}  \UKword{the International Information System Security Certification Consortium} délivre des certifications reconnues et d'excellent niveau de reconnaissance.
Les deux principales sont :
\begin{itemize}
  \item CISSP : Certified Information Systems Security Professional
  \item SSCP : Systems Security Certified Practitioner
\end{itemize}


\head{\link{https://www.isaca.org/}{ISICA} } IT Audit, Security, Governance and Risk 

sous le nom de \UKword{Information Systems Audit and Control Association} cette association professionnelle existe depuis 1967, connue pour sont support à COBIT elle propose plusieurs certifications réclamée par les clients. 


\begin{itemize}
  \item CISA: Certified Information Systems Auditor
  \item CISM: Certified Information Security Manager
 \item CGEIT: Certified in the Governance of Enterprise IT
  \item CRISC: Certified in Risk and Information Systems Control
\end{itemize}

\subsection{Certifications Hacking} 

Il nous faut citer deux certifications très utilisées cans les métiers techniques de la cybersécurité et accessible sans expérience professionnelle à prouver.

\head{SANS Institute} (SysAdmin, Audit, Network, Security) et le GIAC \link{https://www.giac.org}{(Global Information Assurance Certification) }

\begin{itemize}
  \item Cyber Defense
  \item Penetration Testing
  \item Incident Response and Forensics
  \item Management, Audit, Legal
  \item Developer
  \item Industrial Control Systems

\end{itemize}


\includer{inc-certifs-hacking.tex}


%----------------------------------------------------------------------
% 				S E C T IO N   Structure 
%----------------------------------------------------------------------
\section {Structure pédagogique du cours}
Nous avons abordé le cours sur cheminement basé sur trois pivots :

\begin{itemize}
\item Pivot \head{RISQUES} : Pour défendre son espace  cyber  c'est-à-dire l'ensemble des produits, services, matériels, données utilisateurs utilisés par l'activité  économique de l'entreprise il faut non seulement que celui-ci soit identifié mais que les risques portant sur les éléments le constituant soient aussi clairement et consciemment pris en compte. C'est sur la base d'analyses des risques que sont construits les objectifs de sécurité d'un système. Il est bien entendu que de nombreux systèmes préexistent à une analyse de risque et que les objectifs de sécurité ayant conduit à la construction sont issus de  la sédimentation dans le temps de choix technologiques qui ne sont, par ailleurs rarement formalisés. Ainsi on remarque, que l'activité du l'évaluation des risques, ce que appelle en anglais  risk management  est porté plutôt par le domaine d'activités dénommé  information Security management  ou INFOSEC dans les pays anglo-saxons, mais que nous pouvons traduire  management de la sécurité de l'information.
\item Pivot  \head{ARCHITECTURE du SI}. Architecture de sécurité, défense en profondeur, politique de sécurité, usage du SI, IAM. L'analyse sera faite à partie des Politiques de sécurité pour construire ou améliorer la cybersécurité de l'entreprise. Définir des objectifs de sécurité relatifs aux risques,  positionner les politiques de contrôle, decfiltrage, et de gestion sur l'environnement  informationnel  de  l'organisation pour garantir la protection et la confiance sur les actifs sensibles. 
\item  Pivot  \head{MAINTIEN EN CONDITION DE sécurité}. Malgré toutes les précautions pour mettre en confiance un système d'information, il est illusoire d'une part de vouloir tout protéger, mais aussi de penser que les mécanismes de  protection résisteront à toutes les agressions. C'est donc en continu qu'il est nécessaire de veiller à la menace, de vérifier que de nouvelles fragilités n'apparaissent pas, de réagir au plus vite en cas de suspicion d'attaque ou de compromission.  Ce sécurité continue, dite dynamique est à la base du maintien en condition de sécurité de l'environnement digital de l'entreprise. 
A titre indicatif, on peut rapidement donner une matrice des classes de métiers associées à chaque pivot. Ceci permettra au lecteur de se focaliser peut être sur un chapitre qui le concerne un peu plus dans son quotidien.
\end{itemize}



%TODO  ajouter une description des classes d'activités 
% Image sur le cycle de vie

\begin{nota}[Les limites de l'exercice]
C\ecours est essentiellement une introduction à la cybersécurité sur son volet gouvernance (politiques et stratégies). Il permet de mettre en perspective les choix techniques, tant de protection et de défense face à une réalité économique, qui nécessite d'adapter protection et défense au niveau de risque. La décomposition sur ces 3 axes est un parti-pris qui évidement ne couvre pas dans le détail, l'ensemble des processus et actions du domaine de la cybersécurité. 
\end{nota}

Vu du côté du responsable sécurité, et donc des compétences acquises : Le RSSI se doit de  maitriser les risques de son SI vis à vis des conditions de sécurité, il est un auditeur en mesure : 

\begin{itemize}
  \item d'analyser les risques à partir de l'analyse des enjeux de l'entreprise, de ses actifs, de son existant, de la menace inhérente ou non à son entreprise ;
  \item de les traiter, les accepter ou pas, 
  \item de proposer les objectifs de sécurité à déployer pour construire les mesures de sécurité.
  \item Ceci conduit à l'objectif professionnel de cette partie :  Savoir comment démarrer la prise en compte de la sécurité des systèmes d'information dans une entreprise .  Il trouvera donc de bons outils théoriques et pratiques dans l'ISO 27005.
\end{itemize}


\begin{nota}[Dynamicité des risques]
Un RSSI ou son équipe conduit les analyses vis à vis de la menace. Il peut être conduit à lancer des audits.  Les mesures issus de ces audits permettent de définir sur les mesures en cours sont faibles, inutiles, vulnérables vis-à-vis des objectifs de sécurité. C'est ainsi qu'il est possible de conduire des analyses de risques sur des systèmes existants et de vérifier si les mesures actives sont compatibles avec les objectifs. On peut aussi constater qu'à ce titre une analyse de risque n'est pas figée dans le temps car les menaces ainsi que la sensibilité des actifs évoluent.
\end{nota}


Le RSSI se doit de maitriser les politiques de sécurité des systèmes d'information,  la PSSI étant le modèle de référence de façon à :
\begin{itemize}
\item planifier et produire ces conditions de sécurité ;
\item les adapter à l'entreprise ;
\item les mettre en œuvre au travers d'une architecture de sécurité propre à l'entreprise ;
\end{itemize}

Le lecteur trouvera un référentiel global dans l'environnement de l'ISO 27001 pour travailler autour du système de management de la sécurité.

Au delà de la gouvernance classique que l'on dit \g{de protection} de la cybersécurité d'entreprise qui se veut un moyen de deployer des mesures de sécurité (préventives, de formation, d'architecture), la sécurité opérationnelle apporte un nouveau lot de mesures et d'outillages liés à l'anticipation, la détection et la réponse aux attaques.

\begin{nota}[sécurité Opérationnelle] Lutte informatique défensive, sécurité dynamique, 
Cyberdéfense : plusieurs terminologies se côtoient pour évoquer des concepts, techniques, mesures, et méthodes souvent proches. 
\end{nota}


\subsection{Structure du cours}

L\ecours est donc organisé en 3 temps. Chaque temps est un module qui structure l'ensemble des éléments présentés dans le programme de l'unité d'enseignement dans une dynamique associée à la forme d'enseignement à distance et structurée autour de 3 cours issus des retours d'expérience d'experts du domaine de la Cybersécurité.  
\begin{itemize}
\item \head{Temps 1} : De l'analyse des risques à la déclinaison des objectifs de sécurité sur les essentiels de l'entreprise;
\item \head{Temps 2} : Des objectifs de sécurité à une politique de sécurité guidant et mesurant une sécurité implémentée (architectures et systèmes  de sécurité et sécurité des architectures et de systèmes d'information) ;
\item \head{Temps 3} : D'un système d'information \textbf{outillé, protégé et défendu} en matière de sécurité à une sécurité opérationnelle \textbf{maintenue, vigilante et  réactive}.
\end{itemize}

C\edoc regroupe de manière plus détaillée les éléments la 3ième partie de l'unité d'enseignement que je nommerai pour la suite dans ce texte   VAR : Veille / Alerte / Réponse ,  les deux premières parties sont toutefois résumés dans deux chapitres préliminaires, permettant de positionner la démarche VAR dans un contexte global.

\subsection{Pour s'engager plus rapidement}

Du point de vue pédagogique, il est important de noter que vous pouvez aller vous initier au domaine de la sécurité des systèmes d'information avec les travaux de l'ANSSI de la \link{https://www.ssi.gouv.fr/administration/formations/cyberedu/contenu-pedagogique-cyberedu}{Mallette CyberEDU}. Cette mallette de cours contient les basics pour aborder la cyberdéfense d'entreprise.

Ces travaux sont issus d'un marché public de réalisation avec l’Université européenne de Bretagne  (qui regroupe 28 établissements d’enseignement supérieur et de recherche) et Orange pour la réalisation de livrables à destination des responsables de formation et/ou des enseignants en informatique.

l’ANSSI met à disposition cette mallette pédagogique qui contient : un guide pédagogique, un cours préparé d’environ 24 heures sur l’enseignement des bases de la sécurité informatique, ainsi que des éléments de cours pour les masters en informatique (réseaux, systèmes d’exploitation et développement).
Ces documents, réalisés par le consortium et l’ANSSI, sont disponibles sur le site de l'ANSSI.
 

\subsubsection {Pour le niveau BAC +3}

Pour ce niveau la mallette contient un syllabus pour le cours de sensibilisation et initiation à la Cybersécurité ainsi que 4 modules de support de cours.

\begin{itemize}
	\item module 1 :  notions de base
	\item module 2  :  hygiène informatique
	\item module 3 :  réseau et applications
	\item module 4 :  gestion de la cybersécurité au sein d’une organisation
\end{itemize}
Un quizz est également à disposition pour permettre d’évaluer les compétences acquises au fur et à mesure de l’avancé des enseignements.

\subsubsection{Pour le niveau Bac + 5} 

Pour ce niveau, des fiches pédagogiques par domaine permettent de découvrir :
\begin{itemize}
	\item la sécurité des réseaux
	\item la sécurité des logiciels
	\item sécurité des systèmes
	\item l’authentification
	\item la cybersécurité au sein des composants électroniques
\end{itemize}



%$END_UBODY
}%---------------------------------------------------------------------

%************************************************************
% Chargement  du MODELE
%************************************************************

\umainload



