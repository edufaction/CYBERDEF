%========================================
%. CONFIGURATION ARTICLE  COURS  SEC 101
%  utilise main-article et UStructure, UArticle
%========================================

% 				Gestion des Risques

%========================================

%************************************************************
% Chargement des variables du modèle
%************************************************************

%===========================
% COURS "CYBERDEF"
% Configuration générale des articles
%===========================


\newcommand{\ubooktitleBefore}{Notes de synthèse}
\newcommand{\ubooktitleMain}{Cours SEC 101 Cnam Bretagne}
\newcommand{\ubooktitle}{UBOOKTITTLE}
\newcommand{\ubooksubtitle}{\includegraphics[width=0.04\paperwidth]{../Tex/Pictures/shield-20.pdf} }
 %------------------------------------------
\newcommand{\edoc}{e document\xspace}
\newcommand{\ecours}{e cours} 
\newcommand{\uJournalInfo}{Orange et CNAM Bretagne, Cybersécurité SEC101, eduf@ction}

%
\ifdefined\PRZMODE 
\newcommand{\umainload}{%===========================
%               COURS "CYBERDEF"
% Modèle Template des Planches Beamer
%===========================
%             Dé	cembre 2019
% (c) DUPUIS Eric eric.dupuis@orange.fr
%==========================


\documentclass[ignorenonframetext, 10pt]{beamer}

\usepackage{xcolor}

\usepackage{ae}
\usepackage[french]{babel}
\usepackage[utf8]{inputenc}
\usepackage[T1]{fontenc}


\usepackage{graphicx,hyperref,url}


\usepackage{../Tex/template.inc/edxbeamer}




%===========================
% EDX PACKAGE
%===========================

\usepackage[utf8]{inputenc}
\usepackage[french]{babel}
\usepackage[T1]{fontenc}


\usepackage{tikz} % Required for drawing custom shapes
\usetikzlibrary{mindmap,trees, backgrounds}
\usepackage{verbatim}
\usepackage{wrapfig}
\usepackage{xpatch}

%\usepackage{environ}

%\usepackage{amsmath}

\usepackage{graphicx} % Required for including pictures
\graphicspath{{Pictures/}} % Specifies the directory where pictures are stored

%\usepackage{lipsum} % Inserts dummy text



\usepackage{fontawesome} 



%\setlist{nolistsep} % Reduce spacing between bullet points and numbered lists

\usepackage{booktabs} % Required for nicer horizontal rules in tables


\usepackage{setspace}
\setstretch{1,1}

\usepackage{csquotes}


\usepackage{fancyhdr}

\usepackage{makeidx} 

\usepackage{xspace} 


%%===========================
% EDX PACKAGE ART
%===========================



%\usepackage{fontawesome}             % plusieurs icônes
%\usepackage{awesomebox}   

%\usepackage{comment}

%\newenvironment{frameB}{\begin{frame}}{\end{frame}}

% pour ne pas afficher les solutions

%\newcommand{\exclure}[1]{\renewenvironment{#1}{\begingroup\comment}{\endcomment\endgroup\ignorespaces}}


%===========================
% EDX COMMAND
%===========================

\newcommand\UKword[1]{\emph{#1}}
\newcommand\Ucenter[1]{\begin{center}{#1}\end{center}}

\newcommand\head[1]{\textbf{#1}}
\newcommand\tb[1]{\textbf{#1}}
\newcommand{\g}[1]{\og #1 \fg}
\newcommand{\uload}[1] {\input{../Tex/#1}}
\newcommand{\ucurcolor}{black}


%-----------------------------------------------------
%										 	\upspicture 
%---------------------------------------------------
\newcommand{\upspicture}[3]
{
\renewcommand{\ucurcolor}{#3}
\resizebox{#2\textwidth}{!}{\input{#1}}
}

%-----------------------------------------------------
%											\upicture 
%---------------------------------------------------

\newcommand{\upicture}[4]{
\begin{figure}[] %h
  \begin{center}
	 \includegraphics[scale=#3]{#1.pdf}
  \end{center}
\caption{#2 \label{#4}}
\end{figure}
}

%-----------------------------------------------------
%											\updfimage 
%---------------------------------------------------

\newcommand{\updfimage}[3]
{
\begin{wrapfigure}{R}{#3\textwidth}
  \begin{center}
	 \resizebox{#3\textwidth}{!}{\includegraphics{#1.pdf}}
  \end{center}
\caption{#2}
\end{wrapfigure}
}



%************************PACKAGE***

\usepackage{tikz}% Required for drawing custom shapes
\usetikzlibrary{shadows}


%-----------------------------------------------------
%											 		\rem 
% insertion d'une remarque avec R
%---------------------------------------------------
\newcommand\rem[1]{%
   \marginpar{%
   \tikzpicture[baseline={(title.base)}]
      \node[inner sep=5pt,text width=4cm,drop shadow={shadow yshift=-5pt,shadow xshift=5pt,ocre},fill=white] (box) {\vskip5pt \nointerlineskip #1};
      \node[right=10pt,inner sep=0pt,outer sep=10pt] (title) at (box.north west) {\bfseries\color{ocre}Remarque};
      \draw[draw=ocre,very thick](title.west)--(box.north west)--(box.south west)--(box.south east)--(box.north east)--(title.east);
      \fill[ocre]([yshift=-10pt]box.north west)--+(-5pt,-5pt)--+(0pt,-10pt);
   \endtikzpicture}%
}
%-----------------------------------------------------
%											 \utikzimage 
% insertion d'une image sur fichier tkz.tex
%---------------------------------------------------

\newcommand{\utikzimage}[3]
{
\begin{wrapfigure}{R}{#3\textwidth}
  \begin{center}
	 \resizebox{#3\textwidth}{!}{\input{#1.tkz.tex}}
  \end{center}
\caption{#2}
\end{wrapfigure}
}


%************************PACKAGE***
 \usepackage{fontawesome}
  \usepackage{awesomebox}   
%-----------------------------------------------------
%											 \videobox 
% Video BOX lien
%----------------------------------------------------
  
%\newcommand{\videobox}[1]{%
%\awesomebox{\faYoutubePlay}{\aweboxrulewidth}{url}{#1}}

%-----------------------------------------------------
%														\link 
% Url et footnote du lien
%-----------------------------------------------------
  
  \newcommand{\link}[2]
  {
  \href{#1}
  {
  #2~\raisebox{-0.2ex}{\faExternalLink}\footnote
  	{
 #1
  	}
  }
  }
  
  
  
  

%%===========================
% EDX COMMAND Pour article
%===========================

%************************PACKAGE***
\usepackage{fontawesome}
%\usepackage{awesomebox}   
%-----------------------------------------------------
%											 \videobox 
% Video BOX lien
%----------------------------------------------------
  
%\newcommand{\videobox}[1]{%
%\awesomebox{\faYoutubePlay}{\aweboxrulewidth}{url}{#1}}

%-----------------------------------------------------
%														\link 
% Url et footnote du lien
%-----------------------------------------------------
  
  \renewcommand{\link}[2]
  {
  \href{#1}
  {
  #2~\raisebox{-0.2ex}{\faExternalLink}\footnote
  	{
 #1
  	}
  }
  }
  
  
%\newcommand{\frametitle}{}
%  
%  \newboolean{BBeam}\setboolean{BBeam}{false}
%2 \newcommand{\AvecBeam}{\setboolean{BBeam}{true}}
%3 \newcommand{\SansBeam}{\setboolean{BBeam}{false}}
%4 \newcommand{\frametitle}[1]{%
%5 \ifthenelse{\boolean{BBeam}}{}{}}
%
%6 \newcommand{\PiedNotes}[1]{%
%7 \ifthenelse{\boolean{BBeam}}{#1}{}}
%  

  
  





%\newcommand{\labelitemi}{\Large{$\bullet$}}

%\setbeamercolor{itemize item}{fg=red}

%-------------------------------------------------------
% The title of the presentation:
%  - first a short version which is visible at the bottom of each slide;
%  - second the full title shown on the title slide;
\title[SEC101]{\utitle}

%-------------------------------------------------------
% Optional: a subtitle to be dispalyed on the title slide
\subtitle{Cours du Cnam}



%-------------------------------------------------------
% The author(s) of the presentation:
%  - again first a short version to be displayed at the bottom;
%  - next the full list of authors, which may include contact information;
\author[eduf@ction]{
Eric DUPUIS \\ \medskip
  {\small \url{eric.dupuis@cnam.fr} \and \url{eric.dupuis@orange.com}\\ 
  {\small \url{http://www.cnam.fr}}}}

% The institute:
%  - to start the name of the university as displayed on the top of each slide
%    this can be adjusted such that you can also create a Dutch version
%  - next the institute information as displayed on the title slide
\institute[Conservatoire National des Arts et Métiers]{
  Conservatoire National des Arts et Métiers \\ Chaire de Cybersécurité}

% Add a date and possibly the name of the event to the slides
%  - again first a short version to be shown at the bottom of each slide
%  - second the full date and event name for the title slide
\date[V2019]{
  Version initiale du cours  \\
  \today}

\begin{document}

\begin{frame}[plain]
  \titlepage
\end{frame}

\begin{frame}
  \frametitle{Sommaire}
  \tableofcontents
\end{frame}


\mode<all>{\ubody}





\end{document}
% Section titles are shown in at the top of the slides with the current section 
% highlighted. Note that the number of sections determines the size of the top 
% bar, and hence the university name and logo. If you do not add any sections 
% they will not be visible.
\section{Introduction}

\begin{frame}
  \frametitle{Introduction}

  \begin{itemize}
    \item This is just a short example
    \item The comments in the \LaTeX\ file are most important
    \item This is just the result after running pdflatex
    \item The style is based on the webpage \url{http://www.ru.nl/}
  \end{itemize}
\end{frame}

\begin{frame}
  \frametitle{Introduction 2}

  \begin{itemize}
    \item qdsqsd
    \item     \item Tqsdqsd
    \item Tsqsqdqsd
    \item exemple 
  \end{itemize}
\end{frame}


\section{Background information}

\begin{frame}
  \frametitle{Background information}

  \begin{block}{Slides with \LaTeX}
    Beamer offers a lot of functions to create nice slides using \LaTeX.
  \end{block}

  \begin{block}{The basis}
    This style uses the following default styles:
    \begin{itemize}
      \item split
      \item whale
      \item rounded
      \item orchid
    \end{itemize}
  \end{block}
\end{frame}

\section{The important things}

\begin{frame}
  \frametitle{The important things}

  \begin{enumerate}
    \item This just shows the effect of the style
    \item It is not a Beamer tutorial
    \item Read the Beamer manual for more help
    \item Contact me only concerning the style file
  \end{enumerate}
\end{frame}

\section{Analysis of the work}

\begin{frame}
  \frametitle{Analysis of the work}

  This style file gives your slides some nice Radboud branding.
  When you know how to work with the Beamer package it is easy to use.
  Just add:\\ ~~~$\backslash$usepackage$\{$ru$\}$ \\ at the top of your file.
\end{frame}

\section{Conclusion}

\begin{frame}
  \frametitle{Conclusion}

  \begin{itemize}
    \item Easy to use
    \item Good results
  \end{itemize}
\end{frame}


}
\else
\newcommand{\umainload}{%-------------------------------------------------------------
%               FR CYBERDEF SECOPS COURSE
%                          ARTICLE MAIN FILE
%                             2020 eduf@ction
%-------------------------------------------------------------

\documentclass[10pt,fleqn,twoside]{../Tex/template.inc/ArticleModel/edxstyle} 

\usepackage{beamerarticle}

\usepackage{ae}
\usepackage[french]{babel}
\usepackage[utf8]{inputenc}
\usepackage[T1]{fontenc}
\usepackage{lmodern}


\usepackage{titletoc} % Required for manipulating the table of contents

%===========================
% EDX PACKAGE
%===========================

\usepackage[utf8]{inputenc}
\usepackage[french]{babel}
\usepackage[T1]{fontenc}



\usepackage{tikz} % Required for drawing custom shapes
\usetikzlibrary{tikzmark} 
\usetikzlibrary{mindmap,trees, backgrounds}
\usepackage{verbatim}
\usepackage{wrapfig}
\usepackage{xpatch}

\usepackage{hologo}

\usepackage{fontawesome}

\usepackage[skins]{tcolorbox} % don't forget SKINS option

\usepackage{listings}
\usepackage{upquote}

\lstset{
upquote=true,columns=flexible, basicstyle=\ttfamily,
language=HTML, 
frameround=tttt,
commentstyle=\color{gray},
identifierstyle=\color{blue},
keywordstyle=\color{ocre}\bfseries,
xleftmargin=2em,
xrightmargin=2em,
aboveskip=\topsep,
belowskip=\topsep, 
frame=single,
rulecolor=\color{ocre},
backgroundcolor=\color{ocre!5},
breaklines,
breakindent=1.5em,
showspaces=false,
showstringspaces=false,
showtabs=false,
}


\usepackage{graphicx} % Required for including pictures
\graphicspath{{Pictures/}} % Specifies the directory where pictures are stored

\usepackage{booktabs} % Required for nicer horizontal rules in tables

\usepackage{setspace}
\setstretch{1,1}

\usepackage{csquotes}

\usepackage{fancyhdr}

\usepackage{makeidx} 

\usepackage{xspace} 

\usepackage{titletoc} % Required for manipulating the table of contents

\usepackage{hyperref}

\hypersetup{pdftitle={\utitle},pdfauthor={\uauthor},hidelinks,backref=true,pagebackref=true,hyperindex=true,colorlinks=false,breaklinks=true,urlcolor=ocre,bookmarks=true,bookmarksopen=false}

%\pdfstringdefDisableCommands{%
%  \def\\{}%
%  \def\texttt#1{<#1>}%
%}

\usepackage[style=numeric,citestyle=numeric,sorting=nyt,sortcites=true,autopunct=true,babel=hyphen,hyperref=true,abbreviate=false,backref=true,backend=biber]{biblatex}

\usepackage{cleveref}



%===========================
% EDX COMMAND STANDART
%===========================

\newcommand\UKword[1]{\emph{#1}}
\newcommand\Ucenter[1]{\begin{center}{#1}\end{center}}

\newcommand\head[1]{\textbf{#1}}
\newcommand\tb[1]{\textbf{#1}}
\newcommand{\g}[1]{\og #1 \fg{}}
\newcommand{\uload}[1] {\input{../Tex/#1}}
\newcommand{\ucurcolor}{black}

\newcommand{\mak}[1]{\faicon{\aArrowCircleORight} #1}
%-----------------------------------------------------
%										 	\upspicture 
%---------------------------------------------------
\newcommand{\upspicture}[3]
{
\renewcommand{\ucurcolor}{#3}
\resizebox{#2\textwidth}{!}{\input{#1}}
}

%-----------------------------------------------------
%										 	\includer 
%---------------------------------------------------

\newcommand{\includer}[1]
{
\input{../Tex/Chapters/#1}
}


%-----------------------------------------------------
%  command DEF : InTextx
%-----------------------------------------------------
\newcommand\InTexta{no text}
\newcommand\InTextb{no text}
\newcommand\InTextc{no text}
\newcommand\rnc{\renewcommand}


%-----------------------------------------------------
%										 	ENV warningbox
%---------------------------------------------------

\newtcolorbox{warningbox}[2][]
{
  colframe = ocre!25,
  colback  = black!05,
  coltitle = ocre!20!black,
  fonttitle = \bfseries, 
  title    = #2,
  #1,
}

\newtcolorbox{notebox2}[2][]
{
sharp corners, 
colback = ocre!5!white, 
colframe = ocre!75!black,
fonttitle = \bfseries, 
colbacktitle= ocre!85!black,
title=#2,#1
}

\newtcolorbox{notebox}[2][]{colback=ocre!5!white,
colframe=ocre!75!black,fonttitle=\bfseries,
colbacktitle=ocre!85!black,enhanced,
attach boxed title to top right={yshift=+1mm},
title=#2,#1}

%-----------------------------------------------------
%										 	\uchap
%---------------------------------------------------
\ifdefined\INTERTITLE
\newcommand{\uchap}[1]
	{
	\begin{LARGE}
	\textbf{\Ucenter{#1 \\ -oOo-}}	
	\end{LARGE}
	}
\else
	\newcommand{\uchap}[1]{}
\fi

%-----------------------------------------------------
%											\uexpand
%---------------------------------------------------
\newcommand {\uexpand}[1]{
\mode<all>\input{../Tex/Chapters/#1}
}

%-----------------------------------------------------
%											\upicture 
%---------------------------------------------------


\ifdefined\PRZMODE

\newcommand{\upicture}[4]{
\framesubtitle{#2}
%\begin{figure}[!h] %h
  \begin{center}
	 \includegraphics[height=0.95\textheight, width=0.95\textwidth, keepaspectratio ]{#1.pdf}
  \end{center}
%  \end{figure}
}

\else


\newcommand{\upicture}[4]{
\begin{figure}[!h] %h
  \begin{center}
	 \includegraphics[width=#3\textwidth]{#1.pdf}
  \end{center}
\caption{#2 \label{#4}}
\end{figure}
}
	
	
\fi


%-----------------------------------------------------
%											\uref 
%---------------------------------------------------
\newcommand{\uref}[1]{(Voir~\cref{#1} page~\pageref{#1})}



%-----------------------------------------------------
%											\updfimage 
%---------------------------------------------------

\newcommand{\updfimage}[3]
{
\begin{wrapfigure}{R}{#3\textwidth}
  \begin{center}
	 \resizebox{#3\textwidth}{!}{\includegraphics{#1.pdf}}
  \end{center}
\caption{#2}
\end{wrapfigure}
}



%************************PACKAGE***

%\usepackage{tikz}% Required for drawing custom shapes
\usetikzlibrary{shadows}


%-----------------------------------------------------
%											 		\rem 
% insertion d'une remarque avec R
%---------------------------------------------------
\newcommand\rem[1]{%
   \marginpar{%
   \tikzpicture[baseline={(title.base)}]
      \node[inner sep=5pt,text width=4cm,drop shadow={shadow yshift=-5pt,shadow xshift=5pt,ocre},fill=white] (box) {\vskip5pt \nointerlineskip #1};
      \node[right=10pt,inner sep=0pt,outer sep=10pt] (title) at (box.north west) {\bfseries\color{ocre}Remarque};
      \draw[draw=ocre,very thick](title.west)--(box.north west)--(box.south west)--(box.south east)--(box.north east)--(title.east);
      \fill[ocre]([yshift=-10pt]box.north west)--+(-5pt,-5pt)--+(0pt,-10pt);
   \endtikzpicture}%
}
%-----------------------------------------------------
%											 \utikzimage 
% insertion d'une image sur fichier tkz.tex
%---------------------------------------------------

\newcommand{\utikzimage}[3]
{
\begin{wrapfigure}{R}{#3\textwidth}
  \begin{center}
	 \resizebox{#3\textwidth}{!}{\input{#1.tkz.tex}}
  \end{center}
\caption{#2}
\end{wrapfigure}
}

%-----------------------------------------------------
%														\link 
% Url et footnote du lien
%-----------------------------------------------------
  
%  \newcommand{\link}[2]
%  {
%  \href{#1}
%  {
%  #2~\raisebox{-0.2ex}{\footnote
%  	{
% #1
%  	}
%  }
%  }
%  }
  
  
    \newcommand{\link}[2]
  {
  \href{#1}
  {
  #2~\raisebox{-0.2ex}{\faExternalLink}\footnote
  	{
 #1
  	}
  }
  }
  

  %-----------------------------------------------------
%														\Pframe 
% Beamer Frame
%-----------------------------------------------------




%\newenvironment{Pframe}[1]{\mode<all>{\begin{frame}<presentation>\frametitle{#1}}{\end{frame}}}
%
%\newenvironment{Tframe}[1]{{\begin{frame}<presentation>\frametitle{#1}}{\end{frame}}
%
%
%\newenvironment{Aframe}{\begin{frame}}{\end{frame}}
%

%\newcommand{\Aframetitle}[1]
%{
%\mode<presentation>{\frametittle{#1}}
%}
%













%-------------------------------------------------------------
%               FR CYBERDEF SECOPS COURSE
%
%                                    EDX STYLE
%
%                              2020 eduf@ction
%-------------------------------------------------------------
\usepackage{parskip}
\setlength{\parindent}{0em}
\setlength{\parskip}{5pt} % 1ex plus 0.5ex minus 0.2ex}

%---------------------------------------------------------------
%	MARGINS
%---------------------------------------------------------------

\usepackage{geometry}

%\geometry{
%	paper=a4paper, 
%	top=4cm, 
%	bottom=4cm, 
%	left=3cm, 
%	right=3cm, 
%	headheight=20pt, 
%	footskip=1.4cm, 
%	headsep=10pt, 
%%	showframe, 
%	showcrop 
%}

%---------------------------------------------------------------
%	FOOTNOTE (format)
%---------------------------------------------------------------

\makeatletter
\long\def\@makefntextFB#1{%
    \ifx\thefootnote\ftnISsymbol
        \@makefntextORI{#1}%
    \else
        \rule\z@\footnotesep
        \setbox\@tempboxa\hbox{\@thefnmark}%
            \ifdim\wd\@tempboxa>\z@
                \kern2em\llap{\@thefnmark.\kern0.5em}%
            \fi
        \hangindent2em\hangafter\@ne#1
    \fi}
\makeatother

%-----------------------------------------------------
%	COLOR
%-----------------------------------------------------
%\usepackage{xcolor} % Required for specifying colors by name
\definecolor{ocre}{RGB}{160,0,0} 
\definecolor{edxcolorcover}{RGB}{160,0,0} 
\definecolor{grey}{RGB}{50,80,80}
\definecolor{cnam}{RGB}{128, 0, 32} 
                 

  \definecolor{comments}{rgb}{0.7,0,0}    % rouge foncé
  \definecolor{link}{rgb}{0,0.4,0.6}      % ~RoyalBlue de dvips
  \definecolor{url}{rgb}{0.6,0,0}         % rouge-brun
  \definecolor{citation}{rgb}{0,0.5,0}    % vert foncé
  \definecolor{ULlinkcolor}{rgb}{0,0,0.3} % de ulthese.cls
  \definecolor{rouge}{rgb}{0.85,0,0.07}   % rouge bandeau identitaire
  \definecolor{or}{rgb}{1,0.8,0}          % or bandeau identitaire

%-----------------------------------------------------
%	FONTS
%-----------------------------------------------------

%%\usepackage{avant} % Use the Avantgarde font for headings
%\usepackage{times} % Use the Times font for headings
%\usepackage{mathptmx} % Use the Adobe Times Rgman as the default text font together with math symbols from the Sym­bol, Chancery and Com­puter Modern fonts

%%\usepackage{microtype} % Slightly tweak font spacing for aesthetics
%%\usepackage[utf8]{inputenc} % Required for including letters with accents
%\usepackage[T1]{fontenc} % Use 8-bit encoding that has 256 glyphs

% Using AVANT GARDE Family

%\renewcommand{\familydefault}{\sfdefault}

%\fontfamily{pag}\selectfont
%\renewcommand{\familydefault}{pag} % sfdefault pag lmss

%bch         Charter
%lmr         Latin Modern Roman
%lmss        Latin Modern Sans Serif
%lmssq       Latin Modern Sans Serif extended
%lmtt        Latin Modern Typewriter
%lmvtt       Latin Modern Typewriter proportional
%pag         Avant Garde
%pbk         Bookman
%pcr         Courier
%phv         Helvetica
%pnc         New Century Schoolbook
%ppl         Palatino
%ptm         Times
%put         Utopia
 
%\renewcommand{\rmdefault}{pag} % text 
%\renewcommand{\sfdefault}{pag} % titre


\renewcommand{\rmdefault}{pag} % text 
\renewcommand{\sfdefault}{phv} % titre


%-----------------------------------------------------
%	BIBLIOGRAPHY AND INDEX
%-----------------------------------------------------


\usepackage{calc} % For simpler calculation - used for spacing the index letter headings correctly

\usepackage{imakeidx}
%\makeindex % Tells LaTeX to create the files required for indexing

\makeindex[columns=2,intoc=true,options={-s \upath/indexstyle.ist}]

%-----------------------------------------------------
%	HEADERS AND FOOTERS
%-----------------------------------------------------


\usepackage{fancyhdr} % Required for header and footer configuration
\pagestyle{fancy} % Enable the custom headers and footers

\ifcsname chapter \endcsname   % verify if command name exists in the CLASS (Book, report, article ...)
% BEGIN IF
\renewcommand{\chaptermark}[1]{\markboth{\sffamily\normalsize\bfseries\chaptername\ \thechapter.\ #1}{}} % Styling for the current chapter in the header
  \else
%nothing
\fi
% END IF

\renewcommand{\sectionmark}[1]{\markright{\sffamily\normalsize\thesection\hspace{5pt}#1}{}} % Styling for the current section in the header

\fancyhf{} % Clear default headers and footers
\fancyhead[LE,RO]{\sffamily\normalsize\thepage} % Styling for the page number in the header
\fancyhead[LO]{\sffamily\normalsize\rightmark} % Print the nearest section name on the left side of odd pages
\fancyhead[RE]{\sffamily\normalsize\leftmark} % Print the current chapter name on the right side of even pages
\fancyfoot[RE,LO]{\sffamily\normalsize\uCoursetittle} % Uncomment to include a footer
\fancyfoot[LE,RO]{\hyperlink{toc}{\sffamily\normalsize\uCoursetheme}} % Uncomment to include a footer
% \hyperlink{toc}
\renewcommand{\headrulewidth}{1.0pt} % Thickness of the rule under the header

\fancypagestyle{plain}{% Style for when a plain pagestyle is specified
	\fancyhead{}\renewcommand{\headrulewidth}{0pt}} 

% Removes the header from odd empty pages at the end of chapters
\makeatletter
\renewcommand{\cleardoublepage}{
\clearpage\ifodd\c@page\else
\hbox{}
\vspace*{\fill}
\thispagestyle{empty}
\newpage
\fi}


%----------------------------------------------------------------------------------------
%	MAIN TABLE OF CONTENTS
%----------------------------------------------------------------------------------------
%\usepackage{titletoc} % Required for manipulating the table of contents

\contentsmargin{0cm} % Removes the default margin

% Part text styling (this is mostly taken care of in the PART HEADINGS section of this file)
\titlecontents{part}
	[0cm] % Left indentation
	{\addvspace{20pt}\bfseries} % Spacing and font options for parts
	{}
	{}
	{}

% Chapter text styling
\titlecontents{chapter}
	[1.25cm] % Left indentation
	{\addvspace{12pt}\large\sffamily\bfseries} % Spacing and font options for chapters
	{\color{ocre!60}\contentslabel[\Large\thecontentslabel]{1.25cm}\color{ocre}} % Formatting of numbered sections of this type
	{\color{ocre}} % Formatting of numberless sections of this type
	{\color{ocre!60}\normalsize\;\titlerule*[.5pc]{.}\;\thecontentspage} % Formatting of the filler to the right of the heading and the page number

% Section text styling
\titlecontents{section}
	[1.25cm] % Left indentation
	{\addvspace{3pt}\sffamily\bfseries} % Spacing and font options for sections
	{\contentslabel[\thecontentslabel]{1.25cm}} % Formatting of numbered sections of this type
	{} % Formatting of numberless sections of this type
	{\hfill\color{black}\thecontentspage} % Formatting of the filler to the right of the heading and the page number

% Subsection text styling
\titlecontents{subsection}
	[1.25cm] % Left indentation
	{\addvspace{1pt}\sffamily\small} % Spacing and font options for subsections
	{\contentslabel[\thecontentslabel]{1.25cm}} % Formatting of numbered sections of this type
	{} % Formatting of numberless sections of this type
	{\ \titlerule*[.5pc]{.}\;\thecontentspage} % Formatting of the filler to the right of the heading and the page number

%\titlecontents{subsubsection}
%	[1.25cm] % Left indentation
%	{\addvspace{1pt}\sffamily\tiny} % Spacing and font options for subsections
%	{\contentslabel[\thecontentslabel]{1.25cm}} % Formatting of numbered sections of this type
%	{} % Formatting of numberless sections of this type
%	{\ \titlerule*[.5pc]{.}\;\thecontentspage} % Formatting of the filler to the right of the heading and the page number

\titlecontents{subsubsection}
[1.25cm]
  {\addvspace{1pt}\sffamily\tiny}
 {\contentslabel[\thecontentslabel]{1.25cm}}
  {}
  {}

% Figure text styling
\titlecontents{figure}
	[1.25cm] % Left indentation
	{\addvspace{1pt}\sffamily\small} % Spacing and font options for figures
	{\thecontentslabel\hspace*{1em}} % Formatting of numbered sections of this type
	{} % Formatting of numberless sections of this type
	{\ \titlerule*[.5pc]{.}\;\thecontentspage} % Formatting of the filler to the right of the heading and the page number

% Table text styling
\titlecontents{table}
	[1.25cm] % Left indentation
	{\addvspace{1pt}\sffamily\small} % Spacing and font options for tables
	{\thecontentslabel\hspace*{1em}} % Formatting of numbered sections of this type
	{} % Formatting of numberless sections of this type
	{\ \titlerule*[.5pc]{.}\;\thecontentspage} % Formatting of the filler to the right of the heading and the page number

%----------------------------------------------------------------------------------------
%	MINI TABLE OF CONTENTS IN PART HEADS
%----------------------------------------------------------------------------------------

% Chapter text styling
\titlecontents{lchapter}
	[0em] % Left indentation
	{\addvspace{15pt}\large\sffamily\bfseries} % Spacing and font options for chapters
	{\color{ocre}\contentslabel[\Large\thecontentslabel]{1.25cm}\color{ocre}} % Chapter number
	{}  
	{\color{ocre}\normalsize\sffamily\bfseries\;\titlerule*[.5pc]{.}\;\thecontentspage} % Page number

% Section text styling
\titlecontents{lsection}
	[0em] % Left indentation
	{\sffamily\small} % Spacing and font options for sections
	{\contentslabel[\thecontentslabel]{1.25cm}} % Section number
	{}
	{}

% Subsection text styling (note these aren't shown by default, display them by searchings this file for tgcdepth and reading the commented text)
%\titlecontents{lsubsection}
%	[.5em] % Left indentation
%	{\sffamily\footnotesize} % Spacing and font options for subsections
%	{\contentslabel[\thecontentslabel]{1.25cm}}
%	{}
%	{}


\usepackage[]{ccicons}


%----------------------------------------------------------------------------------------
%	THEOREM STYLES
%----------------------------------------------------------------------------------------

\usepackage{amsmath,amsfonts,amssymb,amsthm} % For math equations, theorems, symbols, etc

\newcommand{\intoo}[2]{\mathopen{]}#1\,;#2\mathclose{[}}
\newcommand{\ud}{\mathop{\mathrm{{}d}}\mathopen{}}
\newcommand{\intff}[2]{\mathopen{[}#1\,;#2\mathclose{]}}
\renewcommand{\qedsymbol}{$\blacksquare$}

\ifcsname chapter \endcsname   % verify if command name exists in the CLASS (Book, report, article ...)
	\newtheorem{notation}{Notation}[chapter]
 \else
	\newtheorem{notation}{Notation}[section]
\fi


% Boxed/framed environments
\newtheoremstyle{ocrenumbox}% Theorem style name
{0pt}% Space above
{0pt}% Space below
{\normalfont}% Body font
{}% Indent amount
{\small\bf\sffamily\color{ocre}}% Theorem head font
{\;}% Punctuation after theorem head
{0.25em}% Space after theorem head
{\small\sffamily\color{ocre}\thmname{#1}\nobreakspace\thmnumber{\@ifnotempty{#1}{}\@upn{#2}}% Theorem text (e.g. Theorem 2.1)
\thmnote{\nobreakspace\the\thm@notefont\sffamily\bfseries\color{black}---\nobreakspace#3.}} % Optional theorem note

\newtheoremstyle{ocrenumbox}% Theorem style name
{0pt}% Space above
{0pt}% Space below
{\normalfont}% Body font
{}% Indent amount
{\small\bf\sffamily\color{ocre}}% Theorem head font
{\;}% Punctuation after theorem head
{0.25em}% Space after theorem head
{\small\sffamily\color{ocre}\thmname{#1}\nobreakspace\thmnumber{\@ifnotempty{#1}{}\@upn{#2}}% Theorem text (e.g. Theorem 2.1)
\thmnote{\nobreakspace\the\thm@notefont\sffamily\bfseries\color{black}---\nobreakspace#3.}} % Optional theorem note

\newtheoremstyle{blacknumex}% Theorem style name
{5pt}% Space above
{5pt}% Space below
{\normalfont}% Body font
{} % Indent amount
{\small\bf\sffamily}% Theorem head font
{\;}% Punctuation after theorem head
{0.25em}% Space after theorem head
{\small\sffamily{\tiny\ensuremath{\blacksquare}}\nobreakspace\thmname{#1}\nobreakspace\thmnumber{\@ifnotempty{#1}{}\@upn{#2}}% Theorem text (e.g. Theorem 2.1)
\thmnote{\nobreakspace\the\thm@notefont\sffamily\bfseries---\nobreakspace#3.}}% Optional theorem note


\newtheoremstyle{blacknumboxS}% name of the style to be used
{5pt}% measure of space to leave above the theorem. E.g.: 3pt
{5pt}% measure of space to leave below the theorem. E.g.: 3pt
{\normalfont}% name of font to use in the body of the theorem
{}% measure of space to indent
{\small\bf\sffamily}% name of head font
{}% punctuation between head and body
{ }% space after theorem head; " " = normal interword space
{\color{black}\faEye~  \color{ocre}\small\sffamily\thmnote{#3 : }}

% Non-boxed/non-framed environments
\newtheoremstyle{ocrenum}% Theorem style name
{5pt}% Space above
{5pt}% Space below
{\normalfont}% Body font
{}% Indent amount
{\small\bf\sffamily\color{ocre}}% Theorem head font
{\;}% Punctuation after theorem head
{0.25em}% Space after theorem head
{\small\sffamily\color{ocre}\thmname{#1}\nobreakspace\thmnumber{\@ifnotempty{#1}{}\@upn{#2}}% Theorem text (e.g. Theorem 2.1)
\thmnote{\nobreakspace\the\thm@notefont\sffamily\bfseries\color{black}---\nobreakspace#3.}} % Optional theorem note
\makeatother

% Defines the theorem text style for each type of theorem to one of the three styles above



\ifcsname chapter \endcsname

\newcounter{dummy} 

\numberwithin{dummy}{section}
	\theoremstyle{ocrenumbox}
		\newtheorem{theoremeT}[dummy]{Proposition}
		\newtheorem{problem}{Problem}[chapter]
		\newtheorem{exerciseT}{Outillage}[chapter]
	\theoremstyle{blacknumex}
		\newtheorem{exampleT}{Example}[chapter]
	\theoremstyle{blacknumbox}
		\newtheorem{vocabulary}{Vocabulary}[chapter]
		\newtheorem{definitionT}{Definition}[section]
		\newtheorem{corollaryT}[dummy]{Concept}
	\theoremstyle{blacknumboxS}
		\newtheorem{remarqueST}[]{}
		\newtheorem{remarqueT}[dummy]{Remarque}
	\theoremstyle{ocrenum}
	\newtheorem{proposition}[dummy]{Proposition}
  
 \else
 
\newcounter{dummy} 
\numberwithin{dummy}{section}

	\theoremstyle{ocrenumbox}
		\newtheorem{theoremeT}[dummy]{Proposition}
		%\newtheorem{problem}{Problem}[section]
		\newtheorem{exerciseT}{Outillage}[section]
	\theoremstyle{blacknumex}
		\newtheorem{exampleT}{Example}[section]
	\theoremstyle{blacknumbox}
		\newtheorem{vocabulary}{Vocabulary}[section]
		\newtheorem{definitionT}{Definition}[section]
		\newtheorem{corollaryT}[dummy]{concept}
	\theoremstyle{blacknumboxS}
		\newtheorem{remarqueST}[]{}
		\newtheorem{remarqueT}[dummy]{Remarque}
	\theoremstyle{ocrenum}
		\newtheorem{proposition}[dummy]{Proposition}
\fi



%----------------------------------------------------------------------------------------
%	DEFINITION OF COLORED BOXES
%----------------------------------------------------------------------------------------

\RequirePackage[framemethod=default]{mdframed} % Required for creating the theorem, definition, exercise and corollary boxes

% Theorem box
\newmdenv[skipabove=7pt,
skipbelow=7pt,
backgroundcolor=black!5,
linecolor=ocre,
innerleftmargin=5pt,
innerrightmargin=5pt,
innertopmargin=5pt,
leftmargin=0cm,
rightmargin=0cm,
innerbottommargin=5pt]{tBox}

% Exercise box	  
\newmdenv[skipabove=7pt,
skipbelow=7pt,
rightline=false,
leftline=true,
topline=false,
bottomline=false,
backgroundcolor=ocre!10,
linecolor=ocre,
innerleftmargin=5pt,
innerrightmargin=5pt,
innertopmargin=5pt,
innerbottommargin=5pt,
leftmargin=0cm,
rightmargin=0cm,
linewidth=4pt]{eBox}	

% Definition box
\newmdenv[skipabove=7pt,
skipbelow=7pt,
rightline=false,
leftline=true,
topline=false,
bottomline=false,
linecolor=ocre,
innerleftmargin=5pt,
innerrightmargin=5pt,
innertopmargin=0pt,
leftmargin=0cm,
rightmargin=0cm,
linewidth=4pt,
innerbottommargin=0pt]{dBox}	

% Corollary box
\newmdenv[skipabove=7pt,
skipbelow=7pt,
rightline=false,
leftline=true,
topline=false,
bottomline=false,
linecolor=gray,
backgroundcolor=black!5,
innerleftmargin=5pt,
innerrightmargin=5pt,
innertopmargin=5pt,
leftmargin=0cm,
rightmargin=0cm,
linewidth=4pt,
innerbottommargin=5pt]{cBox}

% Creates an environment for each type of theorem and assigns it a theorem text style from the "Theorem Styles" section above and a colored box from above
%\newenvironment{theorem}{\begin{tBox}\begin{theoremeT}}{\end{theoremeT}\end{tBox}}
\newenvironment{exercise}{\begin{eBox}\begin{exerciseT}}{\hfill{\color{ocre}\tiny\ensuremath{\blacksquare}}\end{exerciseT}\end{eBox}}				  
%\newenvironment{definition}{\begin{dBox}\begin{definitionT}}{\end{definitionT}\end{dBox}}	
%\newenvironment{example}{\begin{exampleT}}{\hfill{\tiny\ensuremath{\blacksquare}}\end{exampleT}}		
%\newenvironment{corollary}{\begin{cBox}\begin{corollaryT}}{\end{corollaryT}\end{cBox}}	
\newenvironment{nota}{\begin{cBox}\begin{remarqueT}}{\end{remarqueT}\end{cBox}}	

%----------------------------------------------------------------------------------------
%	REMARK ENVIRONMENT
%----------------------------------------------------------------------------------------

\newenvironment{remark}{\par\vspace{10pt}\small % Vertical white space above the remark and smaller font size
\begin{list}{}{
\leftmargin=35pt % Indentation on the left
\rightmargin=25pt}\item\ignorespaces % Indentation on the right
\makebox[-2.5pt]{\begin{tikzpicture}[overlay]
\node[draw=ocre!60,line width=1pt,circle,fill=ocre!25,font=\sffamily\bfseries,inner sep=2pt,outer sep=0pt] at (-15pt,0pt){\textcolor{ocre}{i}};\end{tikzpicture}} % Orange R in a circle
\advance\baselineskip -1pt}{\end{list}\vskip5pt} % Tighter line spacing and white space after remark

%----------------------------------------------------------------------------------------
%	SECTION NUMBERING IN THE MARGIN
%----------------------------------------------------------------------------------------

\makeatletter
% \renewcommand{\@seccntformat}[1]{\textcolor{ocre}{\csname the#1\endcsname}\quad}   

%\makeatother

%----------------------------------------------------------------------------------------
%	PART HEADINGS
%----------------------------------------------------------------------------------------

% Numbered part in the table of contents
\newcommand{\@mypartnumtocformat}[2]{%
	\setlength\fboxsep{0pt}%
	\noindent\colorbox{ocre!20}{\strut\parbox[c][.7cm]{\ecart}{\color{ocre!70}\Large\sffamily\bfseries\centering#1}}\hskip\esp\colorbox{ocre!40}{\strut\parbox[c][.7cm]{\linewidth-\ecart-\esp}{\Large\sffamily\centering#2}}%
}

% Unnumbered part in the table of contents
\newcommand{\@myparttocformat}[1]{%
	\setlength\fboxsep{0pt}%
	\noindent\colorbox{ocre!40}{\strut\parbox[c][.7cm]{\linewidth}{\Large\sffamily\centering#1}}%
}

\newlength\esp
\setlength\esp{4pt}
\newlength\ecart
\setlength\ecart{1.2cm-\esp}
\newcommand{\thepartimage}{}%
\newcommand{\partimage}[1]{\renewcommand{\thepartimage}{#1}}%
\def\@part[#1]#2{%
\ifnum \c@secnumdepth >-2\relax%
\refstepcounter{part}%
\addcontentsline{toc}{part}{\texorpdfstring{\protect\@mypartnumtocformat{\thepart}{#1}}{\partname~\thepart\ ---\ #1}}
\else%
\addcontentsline{toc}{part}{\texorpdfstring{\protect\@myparttocformat{#1}}{#1}}%
\fi%
\startcontents%
\markboth{}{}%
{\thispagestyle{empty}%
\begin{tikzpicture}[remember picture,overlay]%
\node at (current page.north west){\begin{tikzpicture}[remember picture,overlay]%	
\fill[ocre!20](0cm,0cm) rectangle (\paperwidth,-\paperheight);
\node[anchor=north] at (4cm,-3.25cm){\color{ocre!40}\fontsize{220}{100}\sffamily\bfseries\thepart}; 
\node[anchor=south east] at (\paperwidth-1cm,-\paperheight+1cm){\parbox[t][][t]{8.5cm}{
\printcontents{l}{0}{\setcounter{tocdepth}{1}}% The depth to which the Part mini table of contents displays headings; 0 for chapters only, 1 for chapters and sections and 2 for chapters, sections and subsections
}};
\node[anchor=north east] at (\paperwidth-1.5cm,-3.25cm){\parbox[t][][t]{15cm}{\strut\raggedleft\color{white}\fontsize{30}{30}\sffamily\bfseries#2}};
\end{tikzpicture}};
\end{tikzpicture}}%
\@endpart}
\def\@spart#1{%
\startcontents%
\phantomsection
{\thispagestyle{empty}%
\begin{tikzpicture}[remember picture,overlay]%
\node at (current page.north west){\begin{tikzpicture}[remember picture,overlay]%	
\fill[ocre!20](0cm,0cm) rectangle (\paperwidth,-\paperheight);
\node[anchor=north east] at (\paperwidth-1.5cm,-3.25cm){\parbox[t][][t]{15cm}{\strut\raggedleft\color{white}\fontsize{30}{30}\sffamily\bfseries#1}};
\end{tikzpicture}};
\end{tikzpicture}}
\addcontentsline{toc}{part}{\texorpdfstring{%
\setlength\fboxsep{0pt}%
\noindent\protect\colorbox{ocre!40}{\strut\protect\parbox[c][.7cm]{\linewidth}{\Large\sffamily\protect\centering #1\quad\mbox{}}}}{#1}}%
\@endpart}
\def\@endpart{\vfil\newpage
\if@twoside
\if@openright
\null
\thispagestyle{empty}%
\newpage
\fi
\fi
\if@tempswa
\twocolumn
\fi}

%----------------------------------------------------------------------------------------
%	CHAPTER HEADINGS
%-------------------------------------------------%--------------------------------------

%
%
%% A switch to conditionally include a picture, implemented by Christian Hupfer
\newif\ifusechapterimage
\usechapterimagetrue
\newcommand{\thechapterimage}{}%
\newcommand{\chapterimage}[1]{\ifusechapterimage\renewcommand{\thechapterimage}{#1}\fi}%
\newcommand{\autodot}{.}

\def\@makechapterhead#1{%
{\parindent \z@ \raggedright \normalfont
\ifnum \c@secnumdepth >\m@ne
\if@mainmatter
\begin{tikzpicture}[remember picture,overlay]
	
\node at (current page.north west)
			{\begin{tikzpicture}[remember picture,overlay]
			\node[anchor=north west,inner sep=0pt] at (0,0) {\ifusechapterimage\includegraphics[width=\paperwidth]{\thechapterimage}\fi};
			\draw[anchor=west] (\Gm@lmargin,-9cm) node [line width=2pt,rounded corners=15pt,draw=ocre,fill=white,fill opacity=1,inner sep=15pt]{\strut\makebox[22cm]{}};
			
			\draw[anchor=west] (\Gm@lmargin+.3cm,-9cm) node {\fontsize{20}{30}\selectfont\sffamily\bfseries\color{ocre}~#1\strut};
			%{\Huge\sffamily\bfseries\color{ocre}\thechapter\autodot~#1\strut};
			\node at (current page.north east)  [xshift=-3.95cm, yshift=-4.05cm, text opacity=1]  {{\color{white}\centering\fontsize{200}{30}\selectfont \bfseries\sffamily\thechapter\strut} };
			\node at (current page.north east)  [xshift=-4cm, yshift=-4cm, text opacity=1]  {{\color{ocre}\centering\fontsize{200}{30}\selectfont \bfseries\sffamily\thechapter\strut} };
			
			\end{tikzpicture}};
	\end{tikzpicture}

\else

\begin{tikzpicture}[remember picture,overlay]
\node at (current page.north west)
{\begin{tikzpicture}[remember picture,overlay]
\node[anchor=north west,inner sep=0pt] at (0,0) {\ifusechapterimage\includegraphics[width=\paperwidth]{\thechapterimage}\fi};
\draw[anchor=west] (\Gm@lmargin,-9cm) node [line width=2pt,rounded corners=15pt,draw=ocre,fill=white,fill opacity=1,inner sep=15pt]{\strut\makebox[22cm]{}};

\draw[anchor=east] (\Gm@lmargin,-9cm) node [line width=2pt,rounded corners=15pt,draw=ocre,fill=white,fill opacity=1,inner sep=15pt]{\strut\makebox[22cm]{}};
\draw[anchor=west] (\Gm@lmargin+.3cm,-9cm) node {\huge\sffamily\bfseries\color{ocre}#1\strut};

\end{tikzpicture}};
 
\end{tikzpicture}
\fi\fi\par\vspace*{270\p@}}}

\makeatother
%-------------------------------------------

%\def\@makeschapterhead#1{%
%\begin{tikzpicture}[remember picture,overlay]
%\node at (current page.north west)
%{\begin{tikzpicture}[remember picture,overlay]
%\node[anchor=north west,inner sep=0pt] at (0,0) {\ifusechapterimage\includegraphics[width=\paperwidth]{\thechapterimage}\fi};
%\draw[anchor=west] (\Gm@lmargin,-9cm) node [line width=2pt,rounded corners=15pt,draw=ocre,fill=white, fill opacity=0.8,inner sep=15pt]{\strut\makebox[22cm]{}};
%\draw[anchor=west] (\Gm@lmargin+.3cm,-9cm) node {\huge\sffamily\bfseries\color{ocre}#1\strut};
%\end{tikzpicture}};
%\end{tikzpicture}
%\par\vspace*{270\p@}}




%----------------------------------------------------------------------------------------
%	LINKS
%----------------------------------------------------------------------------------------


\usepackage{bookmark}
\bookmarksetup{
open,
numbered,
addtohook={%
\ifnum\bookmarkget{level}=0 % chapter
\bookmarksetup{bold}%
\fi
\ifnum\bookmarkget{level}=-1 % part
\bookmarksetup{color=ocre,bold}%
\fi
}
}

%----------------------------------------------------------------------------------------
%	FANCY CHAPTER Header
%----------------------------------------------------------------------------------------





\definecolor{ocre}{RGB}{160,0,0}   
%-------------------------------------------------
%	BIBLIOGRAPHY
%-------------------------------------------------

%\usepackage{enumitem}
\addbibresource{../tex/bibliography.bib} % BibTeX bibliography file
\defbibheading{bibempty}{}

%-------------------------------------------------
%	COLORS & BORDERS
%-------------------------------------------------
\setlength{\fboxrule}{0.75pt} % Width of the border around the abstract
\definecolor{color1}{RGB}{160,0,0} % Color of the article title and section
\definecolor{color2}{RGB}{220,220,220} % Color of the boxes behind the abstract and headings

%-------------------------------------------------
% ITEMS DEFINITION
%-------------------------------------------------

\setlist [itemize,1]{label=\color{color1}\faCaretRight }


\begin{document}

%-------------------------------------------------
%	ARTICLE INFORMATION
%-------------------------------------------------

\Abstract {\uabstract%===========================
% COURS "INTRO CYBERDEF"
% Abstract général des articles
%===========================


Il fait partie du cours introductif aux fondamentaux de la sécurité des systèmes d'information vue sous deux prismes quelques fois opposés dans la littérature : la gouvernance et la gestion opérationnelle de la sécurité.
Le cours est constitué d'un ensemble de notes de synthèse compilé en un document unique.\\
Ce document ne constitue pas à lui seul le référentiel du cours. Ce sont des notes de synthèse mises à disposition comme support pédagogique. }
\JournalInfo{\uJournalInfo} % Journal information
\Archive{Notes de cours éditées le  \DTMnow} % Additional notes (e.g. copyright, DOI, review/research article)
\PaperTitle{\utitle} % Article title
\Authors{\uauthor\textsuperscript{1,}\textsuperscript{2}*} % Authors
\affiliation{\textsuperscript{1}\textit{\uproa}} % Author affiliation
\affiliation{\textsuperscript{2}\textit{\uprob}} % Author affiliation
\affiliation{*\textbf{email}: \umaila\ -- \umailb} % Corresponding author
\Keywords{\ukeywords} % Keywords 
\newcommand{\keywordname}{Mots clefs} % Defines the keywords heading name

%-------------------------------------------------
%	MAKE TITLE
%-------------------------------------------------

\maketitle

\begin{tikzpicture}[remember picture,overlay]\node at (current page.north west)[xshift=4cm, yshift=-4cm, text opacity=1.0]{\includegraphics[width=0.1\paperwidth]{../Tex/template.inc/Commons/CommonsPictures/shield-20.pdf}}; 
\end{tikzpicture} 

%-------------------------------------------------	
% Download
\newcommand{\safeqrcode}[2][]{%
  \qrcode[#1]{\detokenize{#2}} }
\newcommand{\GITfilename}{https://github.com/edufaction/CYBERDEF/raw/master/Builder/\jobname.pdf}
\begin{center}
\setstretch{2}
 {\link{\GITfilename}{\textbf{\jobname.pdf }sur GITHUB CYBERDEF} }   \\ 
{{\huge\ccbyncndeu}}  \\  
{2020 eduf@ction Publication en Creative Common BY-NC-ND }   \newline % Copyright notice
{\safeqrcode[padding]{\GITfilename} }  
\end{center}
%-------------------------------------------------
	\newpage

%-------------------------------------------------
\ubody % MAIN BODY defined external
%-------------------------------------------------

	\printbibliography
	\newpage
	
%-------------------------------------------------	
\Ucontribute % Contribution TEXT
%-------------------------------------------------
	
	\newpage
	\tableofcontents 
	
\iftotalfigures
  \listoffigures
\fi

\end{document}
}
\fi





%========================================
\newcommand {\ukeywords}
%========================================
{%---------------------------------------------------------------------
Stratégie, Cyberdéfense, anticiper
}%

%************************************************************
% Chargement des variables dédiées à l'article
%************************************************************

%========================================
\newcommand {\utitle}
{%---------------------------------------------------------------------
Stratégies de cyberdéfense
}%---------------------------------------------------------------------

%========================================
\newcommand {\uabstract}
{%---------------------------------------------------------------------
C\edoc donne les grands supports d'une stratégie de cyberdéfense d'entreprise.
}%---------------------------------------------------------------------


%************************************************************
%  variable définissant  le corps de l'article
%************************************************************

%===========================
\newcommand {\Ucontribute}
{
%contrinbite

\section{Contributions}

\subsection{Comment contribuer}

Les fichiers sources de ce document sont publiés sur GITHUB \link{https://github.com/edufaction/CYBERDEF}{(edufaction/CYBERDEF)} . Vous pouvez contribuer au projet des notes de cours Cyerbsécurité SEC101 (CYBERDEF101). Le fichier Tex/Contribute/Contribs.tex contient la liste des personnes ayant contribué à ces notes de cours.
Le guide de contribution est disponible sur le GITHUB


%\input{Tex/Contribute/CYBERDEF-      contribs.tex}
%===========================
% Contribution globale
%===========================

\subsection{Les contributeurs/auteurs du cours}

Les auteurs des contributions sont :

\subsubsection{Années 2019}

\begin{itemize}
  \item \head{François REGIS} (Orange) : CyberHunting
\end{itemize}


\subsubsection{Années 2018}

\begin{itemize}
  \item \head{Julia HEINZ} (Tyvazoo.com) : ISO dans la gouvernance de la cybersécurité
\end{itemize}


}

%========================================
\newcommand {\ubody}
{%---------------------------------------------------------------------
%-------------------------------------------
% Chapitre
% Stratégie Cyber
% Intro
% File : chap-stratcyber-intro.tex
%-------------------------------------------




Pour être prêt à faire face aux cyber-incidents inévitables, il ne suffit pas de se préparer à simplement réagir pour neutraliser une attaque isolée. Cela nécessite la capacité d’intervenir efficacement et de manière répétitive pour planifier pro-activement, défendre énergiquement vos systèmes et vos actifs informationnels vitaux, devancer l’évolution des menaces, et assurer une reprise complète des activités après
les attaques.
Dans un contexte où les cyber-attaques grèvent de plus en plus les résultats financiers et ternissent la réputation des grandes sociétés, la mise sur pied d’une solide capacité d’intervention en cas de cyber-incident (ICC) devient impérieuse pour les entreprises qui tiennent à sauvegarder leur sécurité, leur vigilance et leur résilience. Une solide capacité d’intervention en cas de cyberincident peut aider votre entreprise à faire ce qui suit :
Comprendre rapidement la nature d’une attaque pour mieux faire face aux questions quoi, où, comment et combien, et y répondre
Réduire le plus possible les coûts – en temps, en ressources et en perte de confiance des clients – associés à la perte de données
Instaurer un niveau accru de gestion et de contrôle pour renforcer les TI et les processus opérationnels et, ainsi, pouvoir vous concentrer sur vos activités de base génératrices de valeur

\section{Orchestration}

CyberDefenseMatrix  \cite{dutta2019cyber}

\section{Cyberrange}



\subsection {les grandes fragilités des infrastructures}

La résilience est dépendante de composantes 
\begin{itemize}
  \item Sur l'identité : Les annuaires : (AD ...)
  \item Sur l'infrastructure de routage (DNS, AS ..)
\end{itemize}

\subsection {Dceeptive sécruity}
Mitre Acttack



}%---------------------------------------------------------------------

%************************************************************
% Chargement  du MODELE
%************************************************************

\umainload

